Ett stort problem för datorsystemdesignern är förekomsten av osäkerhet, vilken
är på grund av sådana fenomen som variationer relaterade till tillverkning,
arbetsprestation och arbetsbelastning. Denna osäkerhet är i många fall naturlig
och oundviklig, och den kan leda till försämring av servicekvaliteten i bästa
fall och till allvarliga fel eller bränd kisel i värsta fall. Därför är det
viktigt att analysera och kvantifiera denna osäkerhet och lindra dess skadliga
följder genom att designa datorsystem så att de är väl medvetna om osäkerhet och
väl utrustade för att på ett effektivt och ändamålsenligt sätt ta hänsyn till
denna.

Vi börjar med att överväga tekniker för deterministisk systemnivåanalys och
systemnivådesign av vissa aspekter av datorsystem. Dessa tekniker tar inte
hänsyn till osäkerhet; de fungerar dock som en solid grund för de som gör det.
Vi fokuserar främst på faktorer som kraft och temperatur eftersom de är av
central betydelse för att uppnå robusthet och energieffektivitet. Vi utvecklar
ett nytt tillvägagångssätt för dynamisk stabiliserad temperaturanalys av
datorsystem och tillämpar det inom ramen för tillförlitlighetsoptimering.

Vi fortsätter sedan med att utveckla tekniker som tar hänsyn till osäkerhet. Den
första tekniken av detta slag är att kvantifiera föränderligheten hos
processparametrar, som framkallas av processvariation, över kiselplattor baserat
på indirekta och dessutom ofullständiga och bullriga mätningar. Den andra
tekniken går ut på att studera olika systemnivåegenskaper med avseende på
variabiliteten som härrör från processvariation. I synnerhet tillåter den att
gradvis analysera övergående kraft- och temperaturprofiler hos datorsystem samt
att analysera dynamiska stabiliserade kraft- och temperaturprofiler. Detta
illustreras genom att överväga problemet med utforskning av ett designutrymme
med probabilistiska begränsningar relaterade till tillförlitlighet. Syftet med
den tredje tekniken som vi utvecklar är att effektivt ta itu med mindre
regelbundna osäkerhetsfaktorer än processvariation som till exempel variationer
i arbetsbelastning. Denna teknik exemplifieras genom att analysera den effekt
som arbetsbelastningsenheter med okända behandlingstider har på det aktuella
systemets tids-, kraft- och temperaturrelaterade egenskaper.

Vi tar även hänsyn till frågan om körtidshantering av datorsystem under
osäkerhet. I det här sammanhanget utför vi en tidig undersökning av användningen
av avancerade prediktiva tekniker för att få en förfinad prognos för långsiktig
användning av resursanvändningen i stora datorsystem.

Alla de föreslagna teknikerna åtföljs av omfattande experimentella utvärderingar
som påvisar den överlägsna prestationsförmågan av våra metoder för analys och
design av datorsystem med avseende på befintliga tekniker.

% vim: set spelllang=sv:

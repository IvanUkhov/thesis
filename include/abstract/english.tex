Computer systems are omnipresent and omniscient. They penetrate deep into
everyday life and are unsettlingly indispensable and increasingly intelligent at
the tasks entrusted to them. It is then understandable that analysis and design
of computer systems are acutely difficult and vastly far-reaching endeavors.

One major concern of the designer of a computer system is the presence of
uncertainty due to such phenomena as process, performance, and workload
variation. In many cases, uncertainty is inherent and inevitable, and it leads
to degradation of the quality of service in the best case and to severe faults
or burnt silicon in the worst-case scenario. Thus, it is crucial to acknowledge
and analyze uncertainty and to mitigate its deteriorating consequences by
designing computer systems in such a way that they are well aware of uncertainty
and well equipped in order to effectively and efficiently take it into account.

We begin by considering techniques for deterministic system-level analysis of
certain aspects of computer systems. These techniques do not take uncertainty
into account; however, they serve as a solid foundation for those that do. Our
attention revolves primarily around power and temperature as they are of central
importance for attaining robustness and energy efficiency. We develop an
accurate and fast approach to dynamic steady-state temperature analysis of
multiprocessor systems and apply it in the context of reliability optimization.

We then proceed to developing techniques that address uncertainty. The first
technique of this kind is to infer the variability of process parameters that is
induced by process variation, across silicon wafers based on indirect
measurements such as readings from thermal sensors. The second technique is
applicable to studying diverse quantities with respect to the variability
originated from process variation. In particular, it allows for progressively
analyzing transient power and temperature profiles of multiprocessor systems. In
addition, dynamic steady-state temperature profiles can be quantified, and it is
illustrated by addressing a problem of design-space exploration with
probabilistic constraints related to reliability. The third technique that we
develop is to efficiently tackle less regular sources of uncertainty than
process variation such as workload variation. This technique is exemplified by
quantifying the effect that workload units with uncertain processing times have
on the timing-, power-, and temperature-related characteristics of
multiprocessor systems.

We finish by focusing on runtime management of computer systems under
uncertainty. In this context, we investigate the utility of advanced prediction
techniques for the purpose of fine-grained long-range forecasting of the
resource usage in cluster clusters. We also develop an infrastructure for data
synthesis that makes it tractable to experiment with the highly promising but
data-demanding state-of-the-art techniques for resource-usage prediction.

\vspace{1em}
\noindent
\emph{The research presented in this thesis has been partially funded by the
National Graduate School in Computer Science (\up{CUGS}) in Sweden.}

One major problem for the designer of electronic systems is the presence of
uncertainty, which is due to phenomena such as process and workload variation.
Very often, uncertainty is inherent and inevitable. If ignored, it can lead to
degradation of the quality of service in the best case and to severe faults or
burnt silicon in the worst case. Thus, it is crucial to analyze uncertainty and
to mitigate its damaging consequences by designing electronic systems in such a
way that uncertainty is effectively and efficiently taken into account.

We begin by considering techniques for deterministic system-level analysis and
design of certain aspects of electronic systems. These techniques do not take
uncertainty into account, but they serve as a solid foundation for those that
do. Our attention revolves primarily around power and temperature, as they are
of central importance for attaining robustness and energy efficiency. We develop
a novel approach to dynamic steady-state temperature analysis of electronic
systems and apply it in the context of reliability optimization.

We then proceed to develop techniques that address uncertainty. The first
technique is designed to quantify the variability in process parameters, which
is induced by process variation, across silicon wafers based on indirect and
potentially incomplete and noisy measurements. The second technique is designed
to study diverse system-level characteristics with respect to the variability
originating from process variation. In particular, it allows for analyzing
transient temperature profiles as well as dynamic steady-state temperature
profiles of electronic systems. This is illustrated by considering a problem of
design-space exploration with probabilistic constraints related to reliability.
The third technique that we develop is designed to efficiently tackle the case
of sources of uncertainty that are less regular than process variation, such as
workload variation. This technique is exemplified by analyzing the effect that
workload units with uncertain processing times have on the timing-, power-, and
temperature-related characteristics of the system under consideration.

We also address the issue of runtime management of electronic systems that are
subject to uncertainty. In this context, we perform an early investigation into
the utility of advanced prediction techniques for the purpose of fine-grained
long-range forecasting of resource usage in large computer systems.

All the proposed techniques are assessed by extensive experimental evaluations,
which demonstrate the superior performance of our approaches to analysis and
design of electronic systems compared to existing techniques.

\vspace{1em}
\noindent
\emph{
  The research presented in this thesis has been partially funded by the
  National Computer Science Graduate School (\up{CUGS}) in Sweden.
}

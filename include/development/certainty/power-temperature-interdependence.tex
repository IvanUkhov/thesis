There is one important aspect about power and temperature that we have not
discussed yet. The power consumption of an electrical circuit consists of two
main components: dynamic and static. The dynamic power---which is the power that
we have implicitly considered so far---is due to the actual useful work done by
the circuit. The static power is due to various parasitic currents that cannot
be entirely eliminated. The major component of the static power is the leakage
power, which is especially prominent in modern \up{CMOS} transistors. The
leakage power and, hence, the total power have a positive dependence on the
operating temperature \cite{liu2007}: a higher heat dissipation leads to a
higher power consumption, and this dependence is strong. On the other hand, the
more power is consumed, the more heat is dissipated. Thus, power and temperature
instigate each other, which cannot be neglected since it leads to a higher power
(energy) consumption than the one expected, to a higher temperature than the one
expected, and, in the worst case, to a thermal runaway.

In order to take the aforementioned power-temperature interplay into account in
the computation of the \up{DSSTP} described in
\sref{dynamic-steady-state-present}, two techniques can be applied as follows.
\[
  \vp(t) = \vp_\dynamic(t) + \vp_\static(t).
\]

First of all, the power and temperature profiles can be calculated several times
in turns until they stop changing. With each new temperature profile we update
the power profile by computing the leakage power and adding it to the dynamic
power: $\mathbb{P}_i = \mathbb{P}_\text{dyn} +
\mathbb{P}_\text{leak}(\mathbb{T}_i)$. The process continues until the
temperature converges, i.e., the difference between two successive temperature
profiles is below a predefined bound. In our experiments we used $0.5^\circ C$
as the maximal acceptable difference and observed that the number of required
iterations to converge is 4--7.

A linear approximation of the leakage power has the following matrix form:
$\v{P}_\text{leak}(\v{T}) = \m{A} \: \v{T}(t) + \v{B}$ where $\m{A}$ is a $N_n
\times N_n$ diagonal matrix of the proportionality and $\v{B}$ is a vector with
$N_n$ elements of the intercept. Both characterize the leakage power for each of
the $N_n$ thermal nodes in the system. It can be seen that the approximation
keeps \eref{temperature-model-original} untouched: $\m{C} \:
\frac{d\v{T}(t)}{dt} + \bar{\m{G}} \: (\v{T}(t) - \v{T}_\ambient) = \bar{\v{P}}$
where $\bar{\m{G}} = \m{G} - \m{A}$ and $\bar{\v{P}} = \v{P}_\text{dyn} + \m{A}
\: \v{T}_\ambient + \v{B}$. Therefore, all solutions proposed in this paper are
perfectly valid with the linearized model. Moreover, in spite of its simplicity,
the model provides a good estimation, as shown in \cite{liu2007}.

In order to evaluate the linearization, we have constructed a number of
hypothetical platforms with 2--32 cores (other parameters are given in
\tref{parameters}) and compared temperature profiles obtained with the
linearization and the exponential model (\sref{power-model}), respectively. For
the later, we use the iterative approach described in
\sref{power-temperature-interdependence}. For the linearization, the power curve
fitting with the least squares regression \cite{press2007} has been employed,
targeted at the range between 40 and $80^\circ C$. From the experiments we have
observed that the NRMSE is bounded by 1--2\%, indicating a good accuracy of the
linear approximation.

For the experiments in \sref{dynamic-steady-state-result}, the leakage power has
not been considered. If considered according to the linearized model, execution
times remain unchanged; if considered according to the iterative model,
execution times increase proportionally for all the methods, which does not
affect any of the conclusions.

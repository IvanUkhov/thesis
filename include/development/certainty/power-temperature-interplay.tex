There is one important aspect about the power consumption that we have not
discussed yet. The total power consumption of an electrical circuit consists of
two main components: dynamic and static. The dynamic power---which is the power
that has been implicitly considered so far---is due to the actual useful work
done by the circuit. It is modeled as follows:
\[
  \p_\dynamic = C f V^2
\]
where $C$ is the effective switched capacitance, $f$ is the frequency, and $V$
is the supply voltage of the circuit. In contrast, the static power is due to
various parasitic currents that cannot be entirely eliminated. One such current
is the leakage current, which is by far dominant, especially in modern \up{CMOS}
transistors. The leakage and, by extension, static power are modeled as follows
\cite{liao2005}:
\begin{equation} \elab{static-power}
  \p_\static = \numberof{g} V I_0 \left(a \q^2 e^{\frac{\alpha V + \beta}{\q}} + b e^{\gamma V + \delta}\right)
\end{equation}
where $\q$ is the current temperature, $V$ is the current supply voltage, $I_0$
is the average leakage current at the reference temperature and reference supply
voltage, and $\numberof{g}$ is the number of gates in the circuit. Quantities
$a$, $b$, $\alpha$, $\beta$, $\gamma$, and $\delta$ are technology-dependent
constants, which can found in \cite{liao2005}.

It can be seen in the above equations that the dynamic power does not depend on
the operating temperature whereas the static power does. The dependence is
positive and strong: a higher heat dissipation leads to a considerably higher
power consumption. On the other hand, the more power is consumed, the more heat
is dissipated. Therefore, power and temperature constantly instigate each other.
This phenomenon cannot be neglected since it leads to a higher power (energy)
consumption and higher a heat dissipation than what is expected and, in the
worst case, to a thermal runaway.

In order to take the aforementioned power-temperature interplay into account
when performing temperature analysis, two approaches can be employed. To begin
with, in \eref{temperature-model-original} and thereafter, let
\[
  \vp(t) = \vp_\dynamic(t) + \vp_\static(t)
\]
where $\vp_\dynamic \in \real^\np$ and $\vp_\static \in \real^\np$ are the
dynamic and static components of the total dissipation of power, respectively.
In addition, the power profile that is given as an input to temperature analysis
contains samples of $\vp_\dynamic$ and, therefore, is better denoted by
$\mp_\dynamic$. It should be noted that, in the above equation, $\vp$ and
$\vp_\static$ are implicitly functions of $\vq$ due to the leakage power.

The first approach is to calculate the power and temperature profiles several
times in turns until the latter converges. In this case, we obtain a series of
pairs of a power profile and a temperature profile
\[
  \{ (\mp_i, \mq_i): i = 1, 2, \dots \}.
\]
With each new temperature profile $\mq_i$---which is computed by performing
transient, static steady-state, or dynamic steady-state analysis as usual---a
new power profile $\mp_i$ is obtained by recalculating the leakage power and
adding it to the dynamic power, which we write as follows:
\[
  \mp_i = \mp_\dynamic + \mp_\static(\mq_i).
\]
The process continues until the difference between two successive temperature
profiles is below a predefined threshold. According to our experience, it
usually takes 4--7 iterations until the uniform norm of the difference between
two profiles (the maximum absolute difference) drops below 0.5${}^\circ$C.

The second approach is to use a linear approximation of the leakage power, which
can be formalized as follows:
\[
  \vp_\static(t) = \hm{A} \vq(t) + \hv{B}
\]
where $\hm{A} \in \real^{\nn \times \nn}$ and $\hv{B} \in \real^\nn$ are a
diagonal matrix and a vector to be fitted. It can be shown that the linear
approximation keeps the original system of differential equations in
\eref{temperature-differential-original} almost intact. More precisely, it
becomes
\[
  \m{C} \frac{d\tvs(t)}{dt} + \hm{G} \tvs(t) = \tm{B} \hv{p}(t)
\]
where
\begin{align*}
  & \hm{G} = \m{G} - \tm{B} \hm{A} \transpose{\tm{B}} \text{ and} \\
  & \hv{p}(t) = \vp_\dynamic(t) + \hm{A} \vq_\ambient + \hv{B},
\end{align*}
and $\hm{G}$ has the same properties as $\m{G}$. Therefore, all the solutions
that have been derived in this chapter remain perfectly valid. Moreover, as
shown in \cite{liu2007}, despite of its simplicity, the approximation is
relatively accurate, and, if it is needed, it can be improved by considering
multiple linear segments.

In order to evaluate the linearization, we consider a number of hypothetical
platforms with 2--32 processors and undertake dynamic steady-state temperature
analysis. The calculations are performed with both the linearization and the
exponential model in \eref{static-power}. For the former, the model is fitted
using least squares \cite{press2007} targeted at range 40--80${}^\circ$C. For
the latter, we use the iterative approach described earlier. The experiment
shows that the \up{NRMSE} is bounded by 2\%, indicating a good accuracy of the
linear approximation.

Lastly, let us note that the leakage power is not considered in the experiments
given in \sref{dynamic-steady-state-results}. However, if it was taken into
account using the linearization, the computation times would remain unchanged,
and, if the iterative model was utilized, the computation times would increase
proportionally for all the techniques, which would not affect any of the
conclusions.

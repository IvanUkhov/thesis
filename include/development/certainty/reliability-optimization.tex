We show how our approach makes it possible to efficiently perform reliability
optimization, based on the thermal cycling (TC) failure mechanism. More exactly,
we propose a temperature-aware task mapping and scheduling technique that
addresses the TC aging effect. Experiments demonstrate the superiority of the
proposed techniques, compared to the state of the art.

The proposed calculation of the SSDTA can be used in a wide range of system
optimizations. One of them is reliability optimization that we discuss in this
section. We perform a temperature-aware task mapping and scheduling in order to
address the thermal cycling fatigue.

\subsection{Platform Model}
\slab{platform-model}

We consider a heterogeneous multicore architecture with a set of processing
elements $\Pi$ defined as the following:
\[
  \Pi = \{ \pi_i = (V_i, f_i, N_{\text{gate} \: i}): \; i = \range{0}{N_p - 1} \}
\]
where $V_i$, $f_i$, and $N_{\text{gate} \: i}$ are the supply voltage,
frequency, and number of gates \cite{liao2005} of the $i$th core, respectively.

\subsection{Application Model}
\slab{application-model}

The periodic application is modeled as a task graph $G = (V, \: E, \: \tau)$
where $V$ is a set of $N_t$ tasks (vertices of the graph), $E$ is a set of data
dependencies between tasks (edges), and $\tau$ is the period of the application,
which we assume to be equal to the deadline. Each pair of a task $v_i \in V$ and
processing element $\pi_j \in \Pi$ is characterized by a tuple $(N_{\text{clock}
\: ij}, C_{\text{eff} \; ij})$, where $N_{\text{clock} \: ij}$ is the number of
clock cycles and $C_{\text{eff} \; ij}$ is the effective switched capacitance.

\subsection{Power Model}
\slab{power-model}

The total power dissipation of a processing element is defined as the sum of the
dynamic and leakage power: $P = P_\text{dyn} + P_\text{leak}$. The dynamic part
is modeled as $P_\text{dyn} = C_\text{eff} f V^2$ where $C_\text{eff}$ is the
effective switched capacitance, $V$ and $f$ are the supply voltage and
frequency, respectively. The leakage part of the power dissipation is defined as
\cite{liao2005}:
\begin{equation} \elab{total-power}
  P_\text{leak}(T) = N_\text{gate} V I_0 \left[ A T^2 e^{\frac{\alpha V + \beta}{T}} + B e^{(\gamma V + \delta)} \right]
\end{equation}
where $T$ and $V$ are the current temperature and supply voltage, respectively,
$N_\text{gate}$ is the number of gates in the circuit, $I_0$ is the average
leakage current at the reference temperature and supply voltage. $A$, $B$,
$\alpha$, $\beta$, $\gamma$, and $\delta$ are the technology-dependent constants
found in \cite{liao2005}.

\subsection{Reliability Model}
\slab{reliability-model}

We address temperature-driven failure mechanisms with the reliability model
presented in \cite{huang2009, xiang2010}. In this paper, our particular focus is
on the thermal cycling (TC) fatigue, which is directly connected to the
temperature variations. The derivation of the model is given in the appendix
(\sref{reliability-optimization}).

Assuming the TC fatigue, the parameters affecting reliability are the amplitude
and number of thermal cycles as well as the maximal temperature. A thermal cycle
is a time interval in which the temperature starts from a certain value and,
after reaching an extremum, returns back.

The mean time to failure (MTTF) of one processing element in the system can be
estimated as the following:
\begin{align} \elab{one-mttf}
  \theta = \frac{\tau}{\sum_{i=0}^{N_m - 1} \frac{1}{N_{c \: i}}}
\end{align}
where $N_m$ is the number of thermal cycles during the application period
$\tau$. $N_{c \: i}$ characterizes the $i$th thermal cycle and is calculated
according to the following expression:
\begin{equation} \elab{cycles-to-failure}
  N_c = A (\Delta T - \Delta T_0)^{-b} e^{\frac{E_a}{k T_\text{max}}}
\end{equation}
where $\Delta T$ is the thermal cycle excursion (the distance between the
minimal and maximal temperatures) and $T_\text{max}$ is the maximal temperature
during the thermal cycle (more details in \sref{reliability-optimization}).

It can be seen that the computation requires the identification of the thermal
cycles with their amplitudes and maximal temperatures. All these are captured by
the SSDTP, which is needed as an input to the reliability optimization.

\subsection{Motivational Example}

Consider an application with six tasks, denoted ``T0''--``T5'', and a
heterogeneous architecture with two cores, labeled ``PE0'' and ``PE1''. The task
graph of the application is given in \fref{task-graph} along with the execution
times for both cores. The period of the application is 0.06 $s$. A first
alternative mapping and schedule, and the resulting SSDTP are shown at the top
of \fref{motivation} (where the height of a task represents its relative dynamic
power consumption). It can be observed that initially PE0 is experiencing three
thermal cycles. If we change the mapping of T5 and move it to PE1, we achieve
two thermal cycles of PE0 instead of three. Finally, if we vary the schedule as
well and change the order of T1 and T3, the number of cycles of PE0 becomes one.
Using the reliability model from \sref{reliability-model}, we observe
improvements in the lifetime of 44.69\% and 54.53\%, respectively, relative to
the initial configuration.

\subsection{Problem Formulation}

The problem formulation is the following:

Given:
\begin{itemize}

\item A multiprocessor system $\Pi$ (\sref{platform-model}).

\item A periodic application $G$ (\sref{application-model}).

\item The floorplan of the chip at the desired level of details, configuration
of the thermal package, and thermal parameters.

\item The parameters of the reliability model (\sref{reliability-model}), i.e.,
the constants $A$, $\Delta T_0$, $b$, $E_a$ (see \elab{cycles-to-failure}).

\end{itemize}

Maximize:
\begin{equation} \elab{fitness-function}
  \mathcal{F} = \min_{i = 0}^{N_p - 1} \theta_i
\end{equation}
such that
\begin{align}
  & t_{\text{end} \: i} \leq \tau, \: \forall i \elab{deadline} \\
  & T_{ij} \leq T_\text{max}, \: \forall i, j \elab{t-max}
\end{align}
where $\theta_i$ is the MTTF of the $i$th processing element given by
\eref{one-mttf}, $t_{\text{end} \: i}$ denotes the end time of the $i$th task,
$\tau$ is the period of the application, and $T_{ij}$ are temperature values in
the SSDTP. \eref{deadline} imposes the application deadline, which we assume to
be equal to the period. \eref{t-max} enforces the constraint on the maximal
temperature in the temperature profile $\mathbb{T} = \{ T_{ij} \}$.

The optimization procedure is based on a genetic algorithm (GA)
\cite{schmitz2004} with the fitness function $\mathcal{F}$ given by
\eref{fitness-function}. The algorithm is outlined in \sref{genetic-algorithm}.

This section contains the derivation of the reliability model discussed in
\sref{reliability-optimization} and the description of the actual optimization
procedure.

\subsubsection{Temperature-Aware Reliability Model}

In our analysis, we use the reliability model presented in \cite{huang2009,
xiang2010}. The model is based on the assumption that the time to failure
$\mathcal{T}$ has a Weibull distribution, i.e., $\mathcal{T} \sim Weibull(\eta,
\beta)$ where $\eta$ and $\beta$ are the scaling and shape parameters,
respectively. The expectation of the distribution is the following:
\begin{equation} \elab{general-mttf}
  \mathbb{E}\left[\mathcal{T}\right] = \eta \; \Gamma(1 + \frac{1}{\beta})
\end{equation}
where $\Gamma$ is the gamma function. $\mathbb{E}\left[\mathcal{T}\right]$ is
the mean time to failure (MTTF) that we denote by $\theta$.

The shape parameter $\beta$ is independent of the temperature variation
\cite{chang2006}, which, however, is not the case with the scaling parameter
$\eta$. Therefore, the distribution varies with the temperature. We can split
the overall period of the application $\tau$ into $N_m$ time intervals $\Delta
t_i$, so that during each time interval $\Delta t_i$ the corresponding $\eta_i$
is a constant:
\begin{equation} \elab{eta-one}
  \eta_i = \frac{\theta_i}{\Gamma(1 + \frac{1}{\beta})}
\end{equation}
where $\theta_i$ is the MTTF in the $i$th time interval as if we had the failure
distribution of this interval all the time. For now the values $\theta_i$ are
unknown and depend on the particular failure mechanism. As it is shown in
\cite{xiang2010}, the reliability function $R(t)$, i.e., the probability of
survival until an arbitrary time $t \geq 0$, can be approximated as the
following:
\[
  R(t) = e^{-(\frac{t}{\tau} \sum_{i=0}^{N_m - 1} \frac{\Delta t_i}{\eta_i})^\beta}
\]
The formula keeps the form of the Weibull distribution with the scaling
parameter equal to:
\begin{equation} \elab{eta-many}
  \eta = \frac{\tau}{\sum_{i=0}^{N_m - 1} \frac{\Delta t_i}{\eta_i}}
\end{equation}
The MTTF with respect to the whole application period can be obtained by
combining \eref{general-mttf}, \eref{eta-one}, and \eref{eta-many}.

As mentioned previously, in order to compute the MTTF, we need to consider the
particular failure mechanism and determine the values $\theta_i$ needed in
\eref{eta-one}. We focus on the thermal cycling fatigue
(\sref{reliability-model}). Assuming this concrete failure model, the duration
$\Delta t_i$, during which the corresponding scaling parameter $\eta_i$ is
constant \eref{eta-one}, is exactly a thermal cycle.

When the system is exposed to identical thermal cycles, the number of such
cycles to failure can be estimated using a modified version of the well-known
Coffin-Manson equation with the Arrhenius term \cite{xiang2010, jedec2010}:
\[
  N_c = A (\Delta T - \Delta T_0)^{-b} e^{\frac{E_a}{k T_\text{max}}}
\]
where $A$ is an empirically determined constant, $\Delta T$ is the thermal cycle
excursion, $\Delta T_0$ is the portion of the temperature range in the elastic
region which does not cause damage, $b$ is the Coffin-Manson exponent, which is
also empirically determined, $E_{a}$ is the activation energy, $k$ is the
Boltzmann constant, and $T_\text{max}$ is the maximal temperature during the
thermal cycle. Over the application period, the system undergoes a number of
different thermal cycles each with its own duration $\Delta t_i$ and each cycle
causes its own damage. Therefore, having $N_m$ thermal cycles characterized by
the number of cycles to failure $N_{c\:i}$ and duration $\Delta t_i$, we can
compute $\theta_i$:
\begin{equation} \elab{mttf-cycle}
  \theta_i = N_{c \: i} \; \Delta t_i
\end{equation}
Taking equations \eref{general-mttf}, \eref{eta-one}, \eref{eta-many}, and
\eref{mttf-cycle} together, we obtain the following expression to estimate the
MTTF of one component in the system:
\begin{align}
  \theta = \frac{\tau}{\sum_{i=0}^{N_m - 1} \frac{1}{N_{c \: i}}}
\end{align}
In order to identify thermal cycles in the temperature curve, we follow the
approach given in \cite{xiang2010} where the rainflow counting method is
employed.

\subsubsection{Optimization Procedure}
\slab{genetic-algorithm}

The optimization procedure is based on a genetic algorithm \cite{schmitz2004}
with the fitness function $\mathcal{F}$ given by \eref{fitness-function}. Each
chromosome is a vector of $2 \times N_t$ elements, where the first half encodes
priorities of the tasks and the second represents a mapping. The population
contains $4 \times N_t$ individuals that are initialized partially randomly and
partially based on the initial temperature-aware solution \cite{xie2006}. In
each generation, a number of individuals, called parents, are chosen for
breeding by the tournament selection with the number of competitors proportional
to the population size. The parents undergo the 2-point crossover with $0.8$
probability and uniform mutation with $0.01$ probability. The evolution
mechanism follows the elitism model where the best individual always survives.
The stopping condition is an absence of improvement within 200 successive
generations.

The fitness of a chromosome, \eref{fitness-function}, is evaluated in a number
of steps. First, the decoded priorities and mapping are given to a list
scheduler that produces schedules for each of the cores. If the application
schedule does not satisfy the deadline, the solution is penalized proportionally
to the delay and is not further evaluated; otherwise, based on the parameters of
the architecture and tasks, a power profile is obtained and the corresponding
SSDTP is computed by our proposed method. If the SSDTP violates the temperature
constraint given by \eref{t-max}, the solution is penalized proportionally to
the amount of violation and not further processed; otherwise, the MTTF of each
core is estimated according to \eref{one-mttf} and the fitness function
$\mathcal{F}$ is computed.

The goal of static steady-state analysis is to calculate the temperature of the
thermal equilibrium that the system reaches when the power consumption stays at
a certain constant level $\vp$ for a sufficiently long time. In this case, there
are no dynamics, which means that the derivative in
\eref{temperature-differential} is zero. Then the steady-state temperature $\vq$
can be calculated simply as
\begin{equation} \elab{static-steady-state}
  \vs = -\m{A}^{-1} \m{B} \vp,
\end{equation}
followed by \eref{temperature-algebraic}, where $\m{A}^{-1}$ is computed using
\eref{matrix-inverse}. In this case, \vp and \vq can be viewed as a degenerate
power profile \mp and a degenerate temperature profile \mq, respectively, in the
sense that their number of samples \ns is one.

The static steady-state temperature produced by static steady-state analysis is
frequently used as a computationally cheap approximation of the system's thermal
behavior. For instance, given a full-fledged power profile \mp---that is, a
power profile with $\ns > 1$---one might simply compute the average power vector
\vp, perform the static analysis, and use the resulting static \vq as a guide
inside a temperature-aware design-space-exploration procedure.

However, this approximation is of limited applicability. It assumes that the
system functions at one constant temperature at each spatial location, which is
very often not the case in reality since the power consumption changes.

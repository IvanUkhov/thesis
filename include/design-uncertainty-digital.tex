In this chapter, we continue elaborating on designing with uncertainty. However,
the source of uncertainty is different here. Specifically, the source is the
workload that the system under consideration is supposed to process.

\section{Introduction}

Probabilistic analysis of computer systems is an extensive and diverse area,
which is expanding with an accelerating pace. The rapid growth is instigated by
the fact that computer systems naturally become more sophisticated and refined,
and that they penetrate deeper into everyday life. Hence, the impact of
uncertainty inevitably becomes more prominent and entails more severe
consequences, necessitating an adequate treatment. Therefore, the designer of a
computer system is obliged to account for the presence of uncertainty.

In order to account for uncertainty, one has to quantify it first. As discussed
in \cref{design-uncertainty-analog}, in this setting, the designer is typically
interested in evaluating a certain quantity whose complete knowledge would be
highly profitable for the design at hand but cannot be attained since the
quantity depends a number of parameters that are inherently uncertain at design
time. For instance, such a quantity could be the maximum temperature of a system
whose thermal behavior depends on the runtime workload, which is unknown at
design time.

Uncertainty quantification \cite{maitre2010} is a broad area. The techniques for
uncertainty quantification can deliver radically different pieces of information
about the quantity under consideration. It can be seen in the previous chapters
that we are primarily interested in probability distributions rather than, for
instance, corner cases, and we follow this trend in this chapter as well.
Designing for the worst case leads often to a poor solution as the system under
consideration might easily end up being too conservative, overdesigned
\cite{quinton2012}.

As we note in \sref{chaos-prior}, when probability distributions and uncertainty
quantification in general are concerned, sampling methods are of great use. The
classical \acf{MC} sampling, quasi-\ac{MC} sampling, and Latin hypercube
sampling are examples of such methods \cite{asmussen2007}. Compared to other
techniques for probabilistic analysis, they are straightforward to apply. The
system at hand is treated as an opaque object, and one only has to evaluate this
object a number of times in order to start to draw conclusions about the
system's behavior. The major problem with sampling, however, is in sampling: one
should be able to obtain sufficient many realizations of the quantity of
interest in order to accurately estimate the needed statistics about this
quantity \cite{diaz-emparanza2002}. When the subject under analysis is
computationally expensive to evaluate, sampling methods are rendered slow and
often unfeasible.

Similar to \cref{design-uncertainty-analog}, here we propose a design-time
system-level framework for the analysis of computer systems that are dependent
on uncertain parameters. At it is the case with sampling methods, our technique
treats the system at hand as a ``black box'' and, therefore, is straightforward
to apply since no handcrafting is required, and existing codes need no change.
Hence, the quantities that the framework is able to tackle are diverse. Examples
include those concerned with timing-, power-, and temperature-related
characteristics of elaborate applications running on heterogeneous platforms.

In contrast to \cref{design-uncertainty-analog} and sampling methods, the
framework presented in this chapter explores and exploits the nature of the
problem---that is, the way the quantity of interest depends on the uncertain
parameters---by exercising the aforementioned ``black box'' at a set of points
chosen adaptively. The adaptivity that we leverage is hybrid \cite{jakeman2012}:
it tries to pick up both global (that is, on the level of individual dimensions
\cite{klimke2006}) and, more importantly, local (that is, on the level on
individual points \cite{ma2009}) variations. This means that the framework is
able to benefit from any particularities that might be present in the stochastic
space, that is, in the space of the uncertain parameters.

The remainder of the chapter is organized as follows. The adaptivity is the
capital feature of our technique, and we motivate and illustrate it in the next
section, \sref{interpolant-example}. The problem formulation is given in
\sref{interpolant-problem}. In \sref{interpolant-prior}, the prior work is
discussed. Our solution is outlined in \sref{interpolant-solution}. The
experimental results are given in \sref{interpolant-results}. Finally,
\sref{interpolant-conclusion} concludes the chapter.

\section{Motivational Example}
\slab{interpolant-example}

The uncertainties present in computer systems originate from both the physical
world and the virtual one. An example of a physical source of uncertainty is
process variation \cite{srivastava2010}, which the topic of
\cref{analysis-uncertainty-analog} and \cref{design-uncertainty-analog} and is
central to many lines of research. An example of a virtual source of uncertainty
is workload. To elaborate, the characteristics of the codes running on modern
devices change from one activation to another depending on the environment and
input data. This source of uncertainty has not been deprived of attention
either, especially in the real-time community \cite{quinton2012, diaz2002,
santinelli2011, tanasa2015}. Regardless of the origin, such phenomena as the
ones mentioned above render the behavior of a computer system nondeterministic
to the designer.

Due to its nature, the variability coming from the physical world is typically
smooth, well behaved. In such cases, uncertainty quantification based on
\acf{PC} expansions \cite{xiu2010} and other approximation techniques making use
of global polynomials generally work well, which is demonstrated in
\cref{design-uncertainty-analog}. On the other hand, the variability coming from
the virtual world has often steep gradients and favors nondifferentiability and
even discontinuity. In such cases, \ac{PC} expansions and similar techniques
fail: they require extremely many evaluations of the studied quantity in order
to deliver an acceptable level of accuracy and, consequently, are not worth it.

\inputfigure{interpolant-example}
In order to illustrate this concern, let us consider an example. Suppose that
our system has only one processing element, and it is running an application
with only one task. Suppose further that the task has two branches and takes
either one depending on the input data. Assume that one branch takes 0.1~s to
execute and has probability 0.6, and the other branch takes 1~s and has
probability 0.4. Our goal is to find the distribution of the end-to-end delay of
the application. In this example, the quantity of interest is the end-to-end
delay, and it coincides with the execution time of the task; hence, we already
know the answer. Let us pretend we do not and try to obtain it by other means.

Suppose the above scenario is modeled by a uniformly distributed random variable
$\u: \Omega \to [0, 1]$: the execution time of the task (the end-to-end delay of
the application) is 0.1~s if $\u \in [0, 0.6]$, and it is 1~s if $\u \in (0.6,
1]$. The response in this case is a step function, which is illustrated by the
yellow line in \fref{interpolant-example}.

First, we try to quantify the end-to-end delay by constructing and subsequently
sampling a \ac{PC} expansion founded on the Legendre polynomial basis
\cite{xiu2010}; see \sref{polynomial-chaos}. The orange line in
\fref{interpolant-example} shows a \ac{PC} expansion of level nine ($\lc = \lq =
9$), which uses 10 points ($\nq = 10$). It can be seen that the approximation is
poor---not to mention negative execution times---which means that the follow-up
sampling will also yield a poor approximation of the true distribution. The
observed oscillating behavior is the well-known Gibbs phenomenon stemming from
the discontinuity of the response. Regardless of the number of points spent, the
oscillations will never go away completely.

Let us now investigate how the framework developed in this chapter solves the
same problem. For the purpose of the experiment, our technique is constrained to
make use of the same number of points as the \ac{PC} expansion does. The result
is the blue curve in \fref{interpolant-example}, and the adaptively chosen
points are plotted on the horizontal axis. It can be seen that the approximation
is good, and, in fact, it would be indistinguishable from the true response with
a few additional points. One can note that the adaptive procedure started to
concentrate interpolation points at the jump and left the insipid regions on
both sides of the jump with no particular attention. Having constructed such a
representation, one can proceed to the calculation of the probability
distribution of the quantity of interest, which, in general, is done via
sampling followed by such techniques as kernel density estimation. The crucial
point to note is that this follow-up sampling does not involve the original
system in any way, which implies that it costs practically nothing in terms of
the computation time.

The example discussed above illustrates the fact that the proposed framework is
well suited for nonsmooth response surfaces. More generally, the adaptivity
featured by our technique allows for a reduction of the costs associated with
probabilistic analysis of the quantity under consideration, as measured by the
number of times the quantity needs to be evaluated in order to achieve a certain
accuracy level. The magnitude of the reduction depends on the problem, and it
can be substantial when the problem is well disposed to adaptation.

\section{Problem Formulation}
\slab{interpolant-problem}

Consider a computer system composed of two major components: a platform and an
application. The platform is a collection of heterogeneous processing elements
as defined in \sref{system-model}, and the application is a collection of
interdependent tasks. The designer is interested in studying a quantity \g that
characterizes the system under consideration from a certain perspective.
Examples of \g include the execution delay of the application or a specific
task, energy consumption of the platform or a specific processing element, and
maximum temperature of the platform or a specific processing element.

The quantity of interest \g depends on a set of parameters \vu that are
uncertain at the design stage. Examples of \vu include the amount of data that
the application needs to process, execution times of the tasks, and properties
of the environment. The parameters \vu are given as a random vector $\vu: \Omega
\to \real^\nu$---which is defined on a suitable probability space $(\Omega, \F,
\probability)$ as in \sref{probability-theory}---with an arbitrary but known
distribution whose \ac{CDF} is denoted by $F$.

The dependency of \g on \vu implies that \g is random to the designer. For a
given outcome of \vu, however, the evaluation of \g is purely deterministic.
This operation is typically undertaken by an adequate simulator of the system at
hand, and it is assumed to be doable but computationally expensive.

Our objective is to develop an efficient framework for estimating the
probability distribution of the quantity of interest \g dependent on the
uncertain parameters \vu. The framework is required to be able to handle
nondifferentiable and even discontinuous dependencies between \g and \vu as they
constitute an important class of problems for computer-system design; see
\sref{interpolant-example}.

\section{Prior Work}
\slab{interpolant-prior}

Sampling methods would be a reasonable solution to probabilistic analysis of
computer systems if computer systems were inexpensive (with respect to the
computation time) to simulate. In order to eliminate or reduce the costs
associated with direct sampling, a number of techniques have been introduced.

Let us first discuss physical sources of uncertainty. Since process variation is
the most prominent example in this case, our work in
\cref{design-uncertainty-analog}---which leverages \ac{PC} expansions
\cite{xiu2010} introduced in \sref{polynomial-chaos}---is very much relevant in
this regard, and the discussion here is a brief recapitulation of the one in
\sref{chaos-prior}. Circuit-level timing and power analysis under process
variation are undertaken in \cite{bhardwaj2008} by means of \ac{PC} expansions.
The work in \cite{juan2012} models static steady-state temperature and accounts
for process variation by leveraging the linearity of Gaussian distributions and
time-invariant systems. A stochastic collocation \cite{xiu2010} approach to
static steady-state temperature analysis is given in \cite{lee2013}, which
relies on global interpolation using Newton polynomials.

Let us now turn to what we refer to as virtual sources of uncertainty. In this
context, timing analysis has drawn the major attention \cite{quinton2012}. A
seminal work on response time analysis of periodic tasks with random execution
times on uniprocessor systems is reported in \cite{diaz2002}. A novel analytical
solution to this problem is given in \cite{tanasa2015}, which makes milder
assumptions and allows for addressing larger, previously unsolvable problems.
The framework proposed in \cite{santinelli2011} facilitates task scheduling by
providing probabilistic bounds on the resource given to a task flow and the
resource needed by that task flow; the approach is based on real-time calculus
and applicable to computer systems.

Studying the literature on probabilistic analysis of computer systems related to
\cref{design-uncertainty-analog} and this chapter, one can note a pronounced
trend: the generality and straightforwardness of sampling methods tend to be
lost. To elaborate, a technique typically \one~requires restrictive assumptions
to be fulfilled such as the absence of correlation, \two~is tailored to one
concrete quantity of interest such as the response time, and \three~requires
substantial effort in order to be deployed. However, one should keep in mind
what is practical. First of all, although additional assumptions might make the
mathematics analytically solvable, they often do not hold in reality and
oversimplify the model. An exact analytical solution might also be extremely
complex, requiring a lot of computational resources upon evaluation.
Furthermore, it is often the case that there has been developed a robust
simulator evaluating the quantity under consideration for the deterministic
scenario. Switching to probabilistic analysis based on analytical approaches
means discarding this battle-tested code and implementing something else from
scratch, which is wasteful.

Some of the techniques mentioned earlier, in fact, preserve the generality and
straightforwardness of sampling methods. An example is our uncertainty analysis
presented in \cref{design-uncertainty-analog}. The reason is that our
construction of \ac{PC} expansions is undertaken by means of nonintrusive
spectral projections \cite{xiu2010}, which, similar to sampling methods, do not
need to look inside the ``black box.'' However, as motivated in
\sref{interpolant-example}, nonsmoothness is a serious problem for global
approximation based on polynomials. The convergence of, for instance, \ac{PC}
expansions deteriorates substantially in such cases, requiring partitioning the
stochastic space in order to alleviate the problem. Therefore, it is not
straightforward to apply such techniques as the one given in
\cref{design-uncertainty-analog} in the context of virtual sources of
uncertainty exhibiting nonsmoothness.

To conclude, the available techniques for probabilistic analysis of computer
systems are restricted in use. A flexible and easy-to-deploy framework capable
of tackling nonsmooth uncertainty-quantification problems is needed.

\section{Our Solution}
\slab{interpolant-solution}

Our work brings the following major contribution: we develop an efficient
framework for probabilistic analysis of electronic systems that is
straightforward to use and is applicable to a wide range of
uncertainty-quantification problems.

The usage of our framework is streamlined because it has the same low entrance
requirements as sampling techniques: one only has to be able to evaluate the
metric given a set of deterministic parameters. Moreover, the framework can be
utilized in scenarios with limited knowledge of the joint probability
distribution of the uncertain parameters, which are common in practice (to be
elaborated on in \sref{parameters}).

The scope of our framework is wide because the framework features a powerful
approximation engine. We make use of the hierarchical interpolation with hybrid
adaptivity developed in \cite{jakeman2012, klimke2006, ma2009}, which enables us
to tackle diverse design problems while keeping the associated computation costs
low.

To the best of our knowledge, our framework is the first one to address
systematically and efficiently nonsmooth problems in the context of
probabilistic analysis of electronic systems. The importance of the
characteristic is motivated in \sref{introduction}.

In addition to the aforementioned contribution, we open-source our
implementation \cite{sources}. The code base also includes the whole
experimental setup described in \sref{experiments}.

As noted earlier, making use of a sampling method is a compelling approach to
uncertainty quantification. We would readily apply such a method to study our
metric \g if only evaluating \g had a small cost, which it does not.

Our solution to the above predicament is to construct a light representation of
the heavy \g and study this representation instead of \g. The surrogate that we
build is based on adaptive interpolation: \g is evaluated at a number of
strategically chosen collocation nodes, and any other values of \g are
reconstructed on demand (without involving \g) using a set of basis functions
mediating between the collected values of \g. The benefit of this approach is in
the number of invocations of the metric \g: only a few evaluations of \g are
needed, and the rest of our probabilistic analysis is powered by the constructed
interpolant, which, in contrast to \g, has a negligible cost.

Let us delineate the steps involved in the solution process. Recall that \g is
parameterized by the uncertain parameters \vu, and these variables are the only
source of randomness. 1)~The metric \g is reparameterized in terms of an
auxiliary random vector \vz extracted from \vu; the necessity of this stage will
become clear later on. 2)~An interpolant of \g is constructed by considering \g
as a deterministic function of \vz and evaluating \g at a small set of carefully
chosen points. 3)~The probability distribution of \g is then estimated by
applying an arbitrary sampling method to the constructed interpolant of \g.

The first two of the above steps should be undertaken with a great care as
interpolation of multivariate functions is a challenging task. This aspect will
be discussed in detail in \sref{modeling} and \sref{interpolation}. However,
before we proceed to those sections, we would like to give an illustrative
example.

\section{Illustrative Example}

In this section, we apply our framework to a small problem in order to get a
better understanding of the workflow of the framework. A detailed description of
our experimental setup is given in \sref{configuration}; here we give only the
bare minimum.

The addressed problem is depicted in \fref{example}. We consider a platform with
two processing elements, PE1 and PE2, and an application with four tasks,
T1--T4. The data dependencies between T1--T4 and their mapping onto PE1 and PE2
can be seen in \fref{example}. The metric \g is the end-to-end delay of the
application. The uncertain parameters \vu are the execution times of T2 and T4
denoted by $\u_1$ and $\u_2$, respectively.

The leftmost box in \fref{example} represents a simulator of the system at hand,
and it could involve such tools as Sniper \cite{carlson2011}. It takes an
assignment of the execution times of T2 and T3, $\u_1$ and $\u_2$, and outputs
the calculated end-to-end delay \g. The second box corresponds to the
reparameterization mentioned in \sref{solution} (to be discussed in
\sref{parameters}). It converts the auxiliary variables $\z_1$ and $\z_2$ into
$\u_1$ and $\u_2$ in accordance with $\u_1$ and $\u_2$'s joint distribution. The
third box is our interpolation engine (to be discussed in \sref{interpolation}).
Using a number of strategic invocations of the simulator, the interpolation
engine yields a light surrogate for the simulator; the surrogate corresponds to
the slim box with rounded corners. Having obtained such a surrogate, one
proceeds to sampling extensively the surrogate via a sampling method of choice
(the rightmost box). The surrogate takes $\z_1$ and $\z_2$ and returns an
approximation of \g at that point. Recall that the computation cost of this
extensive sampling is negligible as \g is not involved. The samples are then
used to compute an estimate of the distribution of \g.

In the graph on the right-hand side of \fref{example}, the blue line shows the
probability density function of \g computed by applying kernel density
estimation to the samples obtained from our surrogate. The yellow line (barely
visible behind the blue line) shows the true density of \g; its calculation is
explained in \sref{experiments}. It can be seen that our solution closely
matches the exact one. In addition, the orange line shows the estimation that
one would get if one sampled \g directly 156 times and used only those samples
in order to calculate the density of \g. We see that, for the same budget of
simulations, the solution delivered by our framework is substantially closer to
the true one than the one delivered by na\"{i}ve sampling.

At this point, we are ready to present to the proposed framework. We begin by
elaborating on the modeling of uncertain parameters and metrics of interest. We
shall then proceed to the interpolation engine (\sref{interpolation}).

\section{Modeling}

The agenda for this section is as follows. In \sref{parameters}, the uncertain
parameters \vu are transformed into a form suitable for the subsequent
calculations. This stage is an essential part of our framework, and it is
denoted by $\transform$ in \fref{example}. The rest of the subsections,
\sref{time}--\ref{sec:temperature}, serve a strictly illustrative purpose. They
exemplify the leftmost box in \fref{example} in order to give the reader a
better intuition about the utility of the framework. The subsections introduce a
number of models and a number of metrics \g; however, it should be well
understood that the essence of \g is problem specific. In practice, \g stands
for an adequate simulator of the system under consideration. The modeling
capabilities of this simulator are naturally inherited by the proposed
framework.

\subsection{Uncertainty Parameters}

The foremost step of our framework is to change the parameterization of the
problem from the random vector $\vu = (\u_i)_{i = 1}^\nu \sim F_{\vu}$ to an
auxiliary random vector $\vz = (\z_i)_{i = 1}^\nz \sim F_{\vz}$ such that: 1)
the support of $F_{\vz}$ is the unit hypercube $[0, 1]^\nz$, and 2) $\nz \leq
\nu$ has the smallest value needed to retain the desired level of accuracy. The
first is standardization, which is done primarily for convenience. The second is
model-order reduction, which identifies and eliminates excessive complexity and,
hence, speeds up the solution process. The reduction is possible whenever there
are dependencies between $(\u_i)_{i = 1}^\nu$, in which case one can find such
$(\z_i)_{i = 1}^\nz$, $\nz < \nu$, that each $\u_i$ can be recovered from
$(\z_i)_{i = 1}^\nz$. We shall denote the overall transformation by $\vu =
\transform{\vz}$ where
\[
  \transform: \real^\nu \to [0, 1]^\nz.
\]
For any point $\vz \in [0, 1]^\nz$, we are now able to compute the corresponding
\vu and, consequently, the metric \g as $(\g \circ \transform)(\vz) =
\g(\transform(\vz)) = \g(\vu)$; recall \sref{problem} and see \fref{example}.

Let us consider an example of $\transform$ in order to understand the concept
better. To this end, we begin by assuming that the distribution of $\vu =
(\u_i)_{i = 1}^\nu$, $F_{\vu}$, is given as a set of marginal distribution
functions $\{ F_{\u_i} \}_{i = 1}^\nu$ and a copula \cite{nelsen2006} (see also
\sref{preliminaries}). Furthermore, the copula is assumed to be a Gaussian
copula whose correlation matrix is $\correlation{\vu} \in \real^{\nu \times
\nu}$.

\begin{remark}
A set of marginals and a copula entirely characterize the joint distribution of
\vu, that is, $F_{\vu}$. However, we consider this distribution as an
approximation rather than as the true one. The knowledge of the true joint would
be an impractical assumption to make. A more realistic assumption is the
availability of the marginals and correlation matrix of \vu. In general, these
two pieces are not sufficient to recover the joint of \vu; however, the joint
can be approximated well by accompanying the available marginals by a Gaussian
copula constructed based on the available correlation matrix; see \cite{liu1986}
and also \cite{ukhov2014}. Hence, a set of marginals and a Gaussian copula are
practical inputs to probabilistic analysis.
\end{remark}

The number of variables, which is so far \nu, has a significant impact on the
complexity of the problem at hand. Therefore, an important component of our
framework is model-order reduction, which we shall base on the discrete \acf{KL}
decomposition, also known as the principal component analysis. We proceed as
follows. Since any correlation matrix is real and symmetric, $\correlation{\vu}$
admits the eigendecomposition: $\correlation{\vu} = \m{V} \m{\Lambda} \m{V}^T$
where $\m{V} \in \real^{\nu \times \nu}$ is an orthogonal matrix whose columns
are the eigenvectors of $\correlation{\vu}$, and $\m{\Lambda} =
\diagonal{\lambda_i}_{i = 1}^\nu$ is a diagonal matrix whose diagonal elements
are the eigenvalues of $\correlation{\vu}$. The eigenvalues $(\lambda_i)_{i =
1}^\nu$ correspond to the variances of the corresponding components revealed by
the decomposition. The model-order reduction boils down to selecting those major
components whose cumulative contributions to the total variance is above a
certain threshold. Formally, assuming that $(\lambda_i)_{i = 1}^\nu$ are sorted
in the descending order and given a threshold $\eta \in (0, 1]$ specifying the
fraction of the total variance to be preserved, we identify the smallest \nz
such that
\[
  \frac{\sum_{i = 1}^\nz \lambda_i}{\sum_{i = 1}^\nu \lambda_i} \geq \eta.
\]
Denote by $\tilde{\m{V}} \in \real^{\nu \times \nz}$ and $\tilde{\m{\Lambda}}
\in\real^{\nz \times \nz}$ the matrices obtained by truncating $\m{V}$ and
$\m{\Lambda}$, respectively, to preserve only the first \nz components where \nz
is as shown above.

Now, the transformation $\transform$ in \eref{transformation} is
\[
  \vu = F_{\vu}^{-1} \left( \Phi\left( \tilde{\m{V}} \tilde{\m{\Lambda}}^\frac{1}{2} \, \Phi^{-1}(\vz) \right) \right)
\]
where the random variables $\vz = (\z_i)_{i = 1}^\nz$ are independent and
uniformly distributed on $[0, 1]^\nz$; $\Phi$ and $\Phi^{-1}$ are the
distribution function of the standard Gaussian distribution and its inverse,
respectively, which are applied elementwise; and $F_{\vu}^{-1} = F_{\u_1}^{-1}
\times \cdots \times F_{\u_\nz}^{-1}$ is the Cartesian product of the inverse
marginal distributions of \vu, which are applied to the corresponding element of
the vector yielded by $\Phi$. In the absence of correlations,
\eref{transformation-concrete} is simply $\vu = F_{\vu}^{-1}(\vz)$, and no
model-order reduction is possible ($\nu = \nz$).

To summarize, we have found such a transformation $\transform$ and the
corresponding random vector $\vz \sim F_{\vz}$ that: 1) $F_{\vz}$ is supported
by $[0, 1]^\nz$, and 2) \vz has the smallest number of dimensions \nz needed to
preserve $\eta$ portion of the variance. Let us emphasize that this $\transform$
is an example; the framework works with any $\transform$ that yields $\vz \sim
F_{\vz}$ for some \nz.

\subsection{Application Timing}

Suppose the application is given as a directed acyclic graph. The vertices
represent tasks, and the edges data dependency between these tasks. Suppose
further that a static cyclic scheduling policy is utilized. Note, however, these
assumptions are orthogonal to our framework: the framework can be applied to any
application model and any scheduling policy.

Each task has a start and a finish time. For task $i$, denote these two time
moments by $b_i$ and $d_i$, respectively, and let $\v{b} = (b_i)_{i = 1}^\nt$
and $\v{d} = (d_i)_{i = 1}^\nt$. Other timing characteristics of the application
can be derived from $(\v{b}, \v{d})$. An example is the end-to-end delay, which
is the difference between the finish time of the latest task and the start time
of the earliest task:
\[
  \text{End-to-end delay} = \max_{i = 1}^\nt \, d_i - \min_{i = 1}^\nt \, b_i.
\]

Suppose the execution times of the tasks depend on \vu (see \sref{problem}).
Then the tuple $(\v{b}, \v{d})$ depends on \vu. Then the end-to-end delay given
in \eref{end-to-end-delay} depends on \vu and is a potential metric \g; it is
used in \fref{example}. Note that this \g is nondifferentiable as the $\max$ and
$\min$ functions are such. Hence, \g is nonsmooth, which renders \up{PC}
expansions and similar techniques inadequate for this problem, as illustrated in
\sref{introduction}.

\begin{remark}
In general, the behavior of \g with respect to continuity, differentiability,
and smoothness cannot be inferred from the behavior of \vu. Even when the
parameters are perfectly behaved, \g can still and likely will exhibit
nondifferentiability or even discontinuity, which depends on how \g works
internally. For example, as shown in \cite{tanasa2015}, even if execution times
of tasks are continuous, due to the actual scheduling policy, end-to-end delays
are very often discontinuous.
\end{remark}

\subsection{Power Consumption}

Denote the number of processing elements present on the platform by \np. Let the
dynamic power consumed by task $j$ when running on processing element $i$ be
fixed during the execution of the task and denote this dynamic power by
$\p^\dynamic_{ij}$. The fact that $\p^\dynamic_{ij}$ is constant might seem
restrictive. However, one should keep in mind that it is an example. Our
framework does not have such a restriction. Even in this simple model, the
modeling accuracy can be substantially improved by representing large tasks as
sequences of smaller tasks.

Let the vector $\vp(t) = (\p_i(t))_{i = 1}^\np$ capture the total power
consumption of the system at time $t$. This vector is related to the dynamic
power introduced above as follows:
\[
  \p_i(t) = \sum_{j = 1}^\nt \p^\dynamic_{ij} \: \delta_{ij} (t) + \p^\static_i(t), \text{ for $i = 1, \dots, \np$},
\]
where $\delta_{ij}(t)$ is an indicator function (outputs either zero or one) of
the event that processing element $i$ executes task $j$ at time $t$, and
$\p^\static_i(t)$ is the static power consumed by processing element $j$ at time
$t$. The last component depends on time because the leakage power and
temperature are interdependent \cite{liu2007}, and temperature changes over time
(see the next subsection).

Given a set of \ns points on the timeline $\{ t_i \}_{i = 1}^\ns$, \eref{power}
can be used to construct a power profile of the system as follows:
\[
  \mp = (\p_i(t_j))_{i = 1, j = 1}^{\np, \ns} \in \real^{\np \times \ns}.
\]
The above is a matrix where row $i$ captures the power consumed by processing
element $i$ at the \ns time moments.

The total energy consumed by the system during an application run can be
computed by integrating \eref{power} over the time span of the
application---which is demarcated by the minuend and subtrahend in
\eref{end-to-end-delay}---and the corresponding integral can be estimated using
the power profile as follows:
\[
  \text{Total energy} = \sum_{i = 1}^\np \int \p_i(t) \, \d t \approx \sum_{i = 1}^\np \sum_{j = 1}^\ns \p_i(t_j) \, \Delta t_j
\]
where $\Delta t_j$ is either $t_j - t_{j - 1}$ or $t_{j + 1} - t_j$, depending
on how power values are encoded in \mp. The assumption that \eref{total-energy}
is based on is that each $\Delta t_i$ is sufficiently small so that the power
consumed within the interval does not change significantly.

Since the tuple $(\v{b}, \v{d})$ depends on \vu, the power consumption of the
system depends on \vu too. Consequently, the total energy given in
\eref{total-energy} depends on \vu and is a candidate for \g. Note that
\rref{smoothness} applies in this context to the full extent.

\subsection{Heat Dissipation}

Based on the specification of the platform including its thermal package, an
equivalent thermal \up{RC} circuit is constructed \cite{skadron2004}. The
circuit comprises \nn thermal nodes, and its structure depends on the intended
level of granularity, which impacts the resulting accuracy. For clarity, we
assume that each processing element is mapped onto one corresponding node, and
the thermal package is represented as a set of additional nodes.

The thermal dynamics of the system are modeled using the following system of
differential-algebraic equations \cite{ukhov2014, ukhov2012}:
\begin{subnumcases}{}
  \m{C} \frac{\d\vs(t)}{\d t} + \m{G} \vs(t) = \m{M} \vp(t) \\
  \vq(t) = \m{M}^T \vs(t) + \vq_\ambient
\end{subnumcases}
The coefficients $\m{C} \in \real^{\nn \times \nn}$ and $\m{G} \in \real^{\nn
\times \nn}$ are a diagonal matrix of thermal capacitance and a symmetric,
positive-definite matrix of thermal conductance, respectively. The vectors
$\vp(t) \in \real^\np$,  $\vq(t) \in \real^\np$, and $\vs(t) \in \real^\nn$
correspond the system's power, temperature, and internal state at time $t$,
respectively. The vector $\vq_\ambient \in \real^\np$ contains the ambient
temperature. The matrix $\m{M} \in \real^{\nn \times \np}$ is a mapping that
distributes the power consumption of the processing elements across the thermal
nodes; without loss of generality, $\m{M}$ is a rectangular diagonal matrix
whose diagonal elements are equal to one.

Given a set of \ns points on the timeline $\{ t_i \}_{i = 1}^\ns$,
\eref{thermal-system} can be used to compute a temperature profile of the system
as follows:
\[
  \mq = (\q_i(t_j))_{i = 1, j = 1}^{\np, \ns} \in \real^{\np \times \ns}.
\]
Then the maximum temperature of the system can be estimated using the
temperature profile as follows:
\[
  \text{Max temperature} = \max_{i = 1}^\np \, \sup_{t} \, \q_i(t) \approx \max_{i = 1}^\np \max_{j = 1}^\ns \, \q_i(t_j).
\]

Since the power consumption of the system is affected by \vu (see \sref{power}),
the system's temperature is affected by \vu as well. Therefore, the temperature
in \eref{maximum-temperature} can be considered as a metric \g. Note that, due
to the maximization involved, the metric is nondifferentiable and, hence, cannot
be adequately addressed using polynomial approximations, specially taking into
account the concern in \rref{smoothness}.

To sum up, we have discussed the transformation that needs to be applied to \vu
prior to the interpolation of \g. We have also covered three aspects of
electronic systems, namely, timing, power, and temperature, and introduced a
number of metrics associated with them; we shall come back to these metrics in
the section on experimental results, \sref{experiments}.

\section{Interpolation}

In this section, we present the algorithm that constitutes the core of the
proposed framework for probabilistic analysis of electronic systems. It
corresponds to the third box from the left in \fref{example}. The algorithm
features a sparse-grid structure, hierarchical construction, and hybrid
adaptivity. The benefits of these features are interconnected and can be
summarized as follows: the ability to efficiently tackle multidimensional
problems, the ability to perform gradual refinement of the approximation with a
natural error control, and the ability to make the refinement fine-grained and,
therefore, gain further efficiency. The mathematics presented in
\sref{tensor}--\ref{sec:adaptivity} is a distilled version of the one developed
in \cite{jakeman2012, klimke2006, ma2009}.

Let \f be a function that we would like to approximate; the connection between
\f and \g will be explained in \sref{analysis}. The function is assumed to be in
$\continuous([0, 1]^n)$, the space of continuous functions in the unit hypercube
$[0, 1]^n$. The assumption is a formality and does not impose any restrictions
in practice.

\subsection{Tensor Product}

In one dimension ($n = 1$), \f is approximated by virtue of the following
interpolation formula:
\[
  \tensor{i}(\f) = \sum_{j \in \tensorindex{1}{i}} \f(\x_{ij}) \, e_{ij}
\]
where $i \geq 0$ signifies the level of interpolation; $\X_i = \{ \x_{ij} \}_{j
\in \tensorindex{1}{i}} \subset [0, 1]$ are the collocation nodes; $\E_i = \{
e_{ij} \}_{j \in \tensorindex{1}{i}} \subset \continuous([0, 1])$ are the basis
functions; and $\tensorindex{1}{i} = \{ j - 1 \}_{j = 1}^{n_i}$ is an index set
enumerating (starting from zero) the nodes and functions of level $i$. The
subscript $j \in \tensorindex{1}{i}$ is referred to as the order of a node or
function. The choice of $\X_i$ and $\E_i$ is important and will be discussed
thoroughly later on.

In multiple dimensions ($n > 1$), \f is approximated by the tensor product of
$n$ one-dimensional interpolants:
\[
  \tensor{\vi}(\f)
  = \left( \bigotimes_{k = 1}^n \tensor{1}{i_k} \right)(\f)
  = \sum_{\vj \in \tensorindex{n}{\vi}} \f(\vx_{\vi \vj}) \, e_{\vi \vj}
\]
where $\vi = (i_k)_{k = 1}^n$ and $\vj = (j_k)_{k = 1}^n$ are (multi-)indices
specifying levels and orders, respectively, for each of the dimensions, and
$\tensorindex{n}{\vi} = \tensorindex{n}{i_1} \times \cdots \times
\tensorindex{n}{i_n}$ is an index set obtained by computing the Cartesian
product of one-dimensional index sets. In the above formula,
\begin{align*}
  \X_{\vi} &= \X_{i_1} \times \cdots \times \X_{i_n} \\
           &= \left\{ \vx_{\vi \vj} = (x_{i_k j_k})_{k = 1}^n \right\}_{\vj \in \tensorindex{n}{\vi}} \subset [0, 1]^n
\end{align*}
and
\[
  \E_{\vi}
  = \bigotimes_{k = 1}^n \E_{i_k}
  = \left\{ e_{\vi \vj} = \bigotimes_{k = 1}^n e_{i_k j_k} \right\}_{\vj \in \tensorindex{n}{\vi}} \subset \continuous([0, 1]^n)
\]
are the collocation nodes and basis functions, respectively, corresponding to
index \vi. In \eref{basis-functions}, for any $\vx \in [0, 1]^n$,
\[
  e_{\vi \vj}(\vx)
  = \left( \bigotimes_{k = 1}^n e_{i_k j_k} \right)(\vx)
  = \prod_{k = 1}^n e_{i_k j_k}(x_k).
\]
Finally, the cardinality of $\tensorindex{n}{\vi}$ is as follows:
\[
  n_{\vi}
  = \cardinality{\tensorindex{n}{\vi}}
  = \prod_{k = 1}^n \cardinality{\tensorindex{n}{i_k}}
  = \prod_{k = 1}^n n_{i_k}.
\]
Equation \eref{tensor-cardinality} elucidates the prohibitive expense of the
tensor-product construction shown in \eref{tensor-nd} for multidimensional
problems: the number of nodes grows exponentially as $n$ increases. However,
\eref{tensor-nd} serves well as a building block for more efficient algorithms,
which we discuss next.

\subsection{Smolyak Algorithm}

One of the central algorithms in the field of multidimensional integration and
interpolation is the Smolyak algorithm \cite{smolyak1963}. Intuitively speaking,
the algorithm takes a number of small tensor-product structures and composes
them in such a way that the resulting grid has a drastically reduced number of
nodes while preserving the approximating power of the full tensor-product
construction for the classes of functions that one is typically interested in
integrating or interpolating \cite{klimke2006}.

The Smolyak interpolant for \f is as follows:
\[
  \sparse{n}{l}(\f) = \sum_{\l - n + 1 \leq |\vi| \leq l} (-1)^{l - |\vi|} \, {n - 1 \choose l - |\vi|} \, \tensor{n}{\vi}(\f)
\]
where $l \geq 0$ is the index of the interpolation step, which we shall refer to
as the Smolyak level, and $|\vi| = i_1 + \dots + i_n$. It can be seen that the
algorithm is indeed just a peculiar composition of cherry-picked tensor
products. However, the formula has an implication of paramount importance. The
function \f needs to be evaluated only at the nodes of the grid underpinning
\eref{smolyak-original}:
\[
  \Y_l = \bigcup_{l - n + 1 \leq |\vi| \leq l} \X_{\vi}.
\]
The cardinality of the above set does not have a general closed-form formula;
however, it can be several orders of magnitude smaller than the one of the full
tensor product given in \eref{tensor-cardinality}, which depends on the
dimensionality of the problem at hand and the one-dimensional rules utilized
(\sref{tensor}).

A better intuition about the properties of the Smolyak construction can be
obtained by rewriting \eref{smolyak-original} in an incremental form. To this
end, let $\Delta\tensor{n}{-1}(\f) = 0$,
\begin{align*}
  & \Delta\tensor{n}{i}(\f) = (\tensor{n}{i} - \tensor{n}{i - 1})(\f), \text{ and} \\
  & \Delta\tensor{n}{\vi}(\f) = \left( \bigotimes_{k = 1}^n \Delta\tensor{n}{i_k} \right)(\f).
\end{align*}
Then, \eref{smolyak-original} is identical to
\[
  \sparse{n}{l}(\f)
  = \sum_{\vi \in \sparseindex{n}{l}} \Delta\tensor{n}{\vi}(\f)
  = \sparse{n}{l - 1}(\f) + \sum_{\vi \in \Delta\sparseindex{n}{l}} \Delta\tensor{n}{\vi}(\f)
\]
where $\sparse{n}{-1}(\f) = 0$, and we let $\sparseindex{n}{l} = \{ \vi: |\vi|
\leq l \}$ and $\Delta\sparseindex{n}{l} = \{ \vi: |\vi| = l \}$. It can be seen
that a Smolyak interpolant can be efficiently refined: the work done in order to
attain one Smolyak level $l$ is entirely recycled to go to the next.

The sparsity and incremental refinement of the Smolyak approach, which are shown
in \eref{smolyak-grid} and \eref{smolyak-incremental}, respectively, are
remarkable properties by themselves, but they can be taken even further. To this
end, let $\Delta\X_{-1} = \emptyset$,
\[
  \begin{split}
    & \Delta\X_i = \X_i \setminus \X_{i - 1}, \text{ and } \\
    & \Delta\X_{\vi} = \Delta\X_{i_1} \times \cdots \times \Delta\X_{i_n}.
  \end{split}
\]
Then, \eref{smolyak-grid} can be rewritten as
\[
  \Y_l = \bigcup_{\vi \in \sparseindex{n}{l}} \Delta\X_{\vi} = \Y_{l - 1} \cup \bigcup_{\vi \in \Delta\sparseindex{n}{l}} \Delta\X_{\vi},
\]
which is analogous to \eref{smolyak-incremental}. It can be seen now that it is
beneficial to the refinement to have $\X_{i - 1}$ be entirely included in $\X_i$
since, in that case, the cardinality of $\Y_l \setminus \Y_{l - 1} =
\bigcup_{\vi \in \Delta\sparseindex{n}{l}} \Delta\X_{\vi}$ derived from
\eref{smolyak-grid-incremental} decreases. In words, the values of \f obtained
on lower levels (lower $l$) can be reused to attain higher levels (higher $l$)
if the grid grows without abandoning its previous structure. With this in mind,
the rule used for generating successive sets of points $\{ \X_i \}_i$ should be
chosen to be nested, that is, in such a way that $\X_i$ contains all nodes of
$\X_{i - 1}$.

The final step in this subsection is to rewrite \eref{smolyak-incremental} in a
hierarchical form. To this end, we require the interpolants of higher levels to
represent exactly the interpolants of lower levels. In one dimension, it means
that
\[
  \tensor{1}{i - 1}(\f) = \tensor{n}{i}(\tensor{1}{i - 1}(\f)).
\]
The condition in \eref{tensor-exactness} can be satisfied by an appropriate
choice of collocation nodes and basis functions, which will be discussed later.
If \eref{tensor-exactness} holds, using \eref{tensor-1d} and
\eref{tensor-delta-1d},
\[
  \Delta\tensor{1}{i}(\f) = \sum_{j \in \Delta\tensorindex{1}{i}} \left( \f(\x_{ij}) - \tensor{1}{i - 1}(\f)(\x_{ij}) \right) \, e_{ij}
\]
where $\Delta\tensorindex{1}{i} = \{ j \in \tensorindex{1}{i}: \x_{ij} \in
\Delta\X_i \}$. The above sum is over $\Delta\X_i$ due to the fact that the
difference in the parentheses is zero whenever $\x_{ij} \in \X_{i - 1}$ since
$\X_{i - 1} \subset \X_i$.

In multiple dimensions, using the Smolyak formula,
\[
  \Delta\tensor{n}{\vi}(\f) = \sum_{\vj \in \Delta\tensorindex{n}{\vi}} \left( \f(\vx_{\vi\vj}) - \sparse{|\vi| - 1}(\f)(\vx_{\vi \vj}) \right) \, e_{\vi \vj}
\]
where $\Delta\tensorindex{n}{\vi} = \{ \vj \in \tensorindex{n}{\vi}:
\vx_{\vi \vj} \in \Delta\X_{\vi} \}$. The delta
\[
  \Delta\f(\vx_{\vi \vj}) = \f(\vx_{\vi \vj}) - \sparse{n}{|\vi| - 1}(\f)(\vx_{\vi \vj})
\]
is called a hierarchical surplus. When increasing the interpolation level, this
surplus is nothing but the difference between the actual value of \f at a new
node and the approximation of this value computed by the interpolant constructed
so far.

The final formula for nonadaptive hierarchical interpolation is obtained by
substituting \eref{tensor-delta} into \eref{smolyak-incremental}:
\begin{align*}
  \sparse{l}(\f) &= \sum_{\vi \in \sparseindex{n}{l}} \sum_{\vj \in \Delta\tensorindex{n}{\vi}} \Delta\f(\vx_{\vi \vj}) e_{\vi \vj} \\
                &= \sparse{n}{l - 1}(\f) + \sum_{\vi \in \Delta\sparseindex{n}{l}} \sum_{\vj \in \Delta\tensorindex{n}{\vi}} \Delta\f(\vx_{\vi \vj}) e_{\vi \vj}
\end{align*}
where $\Delta\f(\vx_{\vi \vj})$ is computed according to \eref{surplus}.

\subsection{Collocation Nodes}

A sparse grid that is fully nested and, moreover, well disposed to adaptivity,
as we shall see, can be constructed using the (one-dimensional) Newton--Cotes
rule \cite{ma2009}. For each level, the rule is merely a set of equidistant
nodes on $[0, 1]$.

There are two types of the rule: closed and open. The only difference between
the two is that the former includes the endpoints, that is, 0 and 1, while the
latter does not. Now, in \sref{smolyak}, we postulated that the assumption in
\eref{tensor-exactness} was needed in order to proceed. The closed rule
satisfies this assumption, and it is the one used in the original version of
local adaptivity presented in \cite{ma2009}. The open Newton--Cotes rule, on the
other hand, violates the assumption close to the boundaries of the interval.
However, we found that the open rule is a viable option since it performs well
in practice, which was also noted in \cite{klimke2006}. In fact, we were able to
obtain better results with the open rule and decided to present it here.

The open Newton--Cotes rule of level $i \geq 0$ is
\[
  \X_i = \left\{ \x_{ij} = \frac{j + 1}{n_i + 1} \right\}_{j \in \tensorindex{1}{i}}
\]
where $\tensorindex{1}{i} = \left\{ i - 1 \right\}_{i = 1}^{n_i}$ and $n_i =
2^{i + 1} - 1$. The first three levels of the rule are depicted in \fref{grid}.
It can be seen that the number of nodes (in one dimension) grows as 1, 3, 7, 15,
31, and so on, and that the rule is fully nested. In multiple dimensions, the
nodes are formed as shown in \eref{collocation-nodes}.

\subsection{Basis Functions}

The basis functions that correspond the open Newton--Cotes rule described in
\sref{grid} are the following piecewise linear functions. For $i = 0$ and $j =
0$, we have that $e_{00}(\x) = 1$. For $i > 0$ and $j = 0$ (close to the left
endpoint),
\[
  e_{i0}(\x) =
  \begin{cases}
    2 - \left( n_i + 1 \right) \x, & \text{if } \x < \frac{2}{n_i + 1}, \\
    0, & \text{otherwise}.
  \end{cases}
\]
For $i > 0$ and $j = n_i - 1$ (close to the right endpoint),
\[
  e_{i(n_i - 1)}(\x) =
  \begin{cases}
    \left( n_i + 1 \right) \x - n_i + 1, & \text{if } \x > \frac{n_i - 1}{n_i + 1}, \\
    0, & \text{otherwise}.
  \end{cases}
\]
In other cases,
\[
  e_{ij}(\x) =
  \begin{cases}
    1 - \left( n_i + 1 \right)|\x - \x_{ij}|, & \text{if } |\x - \x_{ij}| < \frac{1}{n_i + 1}, \\
    0, & \text{otherwise}.
  \end{cases}
\]
The basis functions corresponding to the first three levels of one-dimensional
interpolation are depicted in \fref{basis}. In multiple dimensions, the basis
functions are formed as shown in \eref{basis-functions}.

Lastly, let us mention the volumes (integrals over the whole domain) of the
basis functions denoted by $w_{ij}$; they will be needed in continuation.
Namely, $w_{00} = 1$ and, for $i > 0$,
\[
  w_{ij} = \int_0^1 e_{ij}(\x) \, \d \x =
  \begin{cases}
    \frac{2}{n_i + 1}, & \text{if } j \in \{ 0, n_i - 1 \}, \\
    \frac{1}{n_i + 1}, & \text{otherwise}.
  \end{cases}
\]
In multiple dimensions, the volumes are products of the one-dimensional
components, analogous to \eref{basis-function}.

Imagine now a function that is nearly flat on the first half of $[0, 1]$ and
rather irregular on the other. Under these circumstances, it is natural to
expect that, in order to attain the same accuracy, the first half would require
much fewer collocation nodes than the other one; recall \fref{motivation}.
However, if we followed the construction procedure described so far, we would
not be able to benefit from this peculiar behavior: we would treat both sides
equally and would add all the nodes of each level. The solution to the above
problem is to make the interpolation algorithm adaptive, which we shall discuss
next.

\subsection{Adaptivity}

In order to make the algorithm adaptive, we first need to find a way to measure
how good our approximation is at any point in the domain of \f. Then, when
refining the interpolant, instead of evaluating the function at all possible
nodes, we shall only choose those that are located in the regions with poor
accuracy as indicated by the yet-to-be-found criterion.

We already have a good foundation for building such a criterion. Recall
\eref{surplus}. Hierarchical surpluses are natural indicators of the
interpolation error: they are the difference between the values of the true
function and those of an approximation at the nodes of the underlying sparse
grid. Hence, they can be recycled in order to effectively identify
``problematic'' regions. Specifically, we first assign a score to each node
$\vx_{\vi \vj}$ or, equivalently, to each pair of level and order indices $(\vi,
\vj)$:
\[
  s_{\vi \vj} = \left| \Delta\f(\vx_{\vi \vj}) w_{\vi \vj} \right|
\]
where $\Delta\f(\vx_{\vi \vj})$ and $w_{\vi\vj}$ are given by \eref{surplus} and
\eref{volume}, respectively, and this score is then used in order to guide the
algorithm as we shall explain in the rest of this subsection.

The Smolyak construction in \eref{smolyak-hierarchical} is rewritten as follows:
\[
  \interpolant{n}{l}(\f) = \interpolant{n}{l - 1}(\f) + \sum_{\vi \in \Delta\sparseindex{n}{l}} \sum_{\vj \in \Delta\tensorindex{n}{\vi}} \Delta\f(\vx_{\vi \vj}) e_{\vi \vj}.
\]
The different with respect to \eref{smolyak-hierarchical} is that $l \geq 0$ is
no longer the Smolyak level (see \eref{smolyak-original}) but a more abstract
interpolation step, and $\interpolant{n}{l}$ is the interpolant at that step. As
always, $\interpolant{n}{-1} = 0$, and the definition of $\Delta\f$ given in
\eref{surplus} is adjusted accordingly. From now on, all index sets will be
generally subsets of their full-fledged counterparts defined in \sref{smolyak}.

Each $\interpolant{n}{l}$ is characterized by a set of level indices
$\sparseindex{n}{l}$, and each $\vi \in \sparseindex{n}{l}$ by a set of order
indices $\Delta\tensorindex{n}{\vi}$. At each interpolation step $l \geq 0$, a
single index $\vi_l$ is chosen from $\sparseindex{n}{l - 1}$ with
$\sparseindex{n}{-1} = \{ \v{0} \}$. The chosen index then gives birth to
$\Delta\sparseindex{n}{l}$ and $\{ \Delta\tensorindex{n}{\vi} \}_{\vi \in
\Delta\sparseindex{n}{l}}$, which shape the increment in the right-hand side of
\eref{approximation}.

The set $\Delta\sparseindex{n}{l}$ contains the so-called admissible forward
neighbors of $\vi_l$. Let us now parse the previous sentence. First, the forward
neighbors of an index \vi are given by
\[
  \left\{ \vi + \v{1}_k: k = 1, \dots, n \right\}
\]
where $\v{1}_k$ is a vector whose elements are zero except for element $k$ equal
to unity. Next, an index \vi is admissible if its inclusion into the index set
$\sparseindex{n}{l}$ in question keeps the set admissible. Finally,
$\sparseindex{n}{l}$ is admissible if it satisfies the following condition
\cite{klimke2006}:
\[
  \vi - \v{1}_k \in \sparseindex{n}{l}, \text{ for $\vi \in \sparseindex{n}{l}$ and $k = 1, \dots, n$,}
\]
where, naturally, the cases with $i_k = 0$ need no check.

Now, how is $\vi_l$ chosen from $\sparseindex{n}{l - 1}$ at each iteration of
\eref{approximation}? First of all, each index can be obviously picked at most
once. The rest is resolved by prioritizing the candidates. It is reasonable to
assign a priority to a level index $\vi$ based on the scores of the order
indices associated with it, that is, on the scores of $\tensorindex{n}{\vi}$. We
compute the priority as the average score:
\[
  s_{\vi} = \frac{1}{\cardinality{\Delta\tensorindex{n}{\vi}}} \sum_{\vj \in \Delta\tensorindex{n}{\vi}} s_{\vi \vj}
\]
Consequently, the answer to the above question is that, at each step $l$, the
index \vi with the highest $s_{\vi}$ gets promoted to $\vi_l$.

Let us now turn to the content of $\Delta\tensorindex{n}{\vi}$ where $\vi =
\vi_l + \v{1}_k$ for a fixed $k$. It also contains admissible forward neighbors,
but they are order indices, and their construction is drastically different from
the one in \eref{forward-level-neighbors}. Concretely, these indices are
identified by inspecting the backward neighborhood of \vi (analogous to
\eref{forward-level-neighbors}). For each backward neighbor $\hat{\vi} = \vi -
\v{1}_{\hat{k}}$ and each $\vj \in \Delta\tensorindex{n}{\hat{\vi}}$, we begin
by checking the following condition: $s_{\hat{\vi} \vj} \geq \epsilon_s$ where
$\epsilon_s$ is a user-defined constant, which we shall refer to as the score
error. If the condition holds, the forward neighbors of \vj in dimension $k$ are
added to $\Delta\tensorindex{n}{\vi}$. This procedure is illustrated in
\fref{grid} for the open Newton--Cotes rule (see \sref{grid}). The arrows
emerging from a node connect the node with its forward neighbors. It can be seen
that each node has two forward neighbors (for each dimension); their order
indices are
\[
  (j_1, \dots, 2 j_k, \dots, j_n) \text{ and } (j_1, \dots, 2 j_k + 2, \dots, j_n).
\]
The above refinement procedure is repeated for each index $\vi \in
\Delta\sparseindex{n}{l}$ with respect to each dimension $k = 1, \dots, n$.

The final question is the stopping condition of the approximation process in
\eref{approximation}. Apart from the natural constraints on the maximum number
of function evaluations and the maximum allowed Smolyak level $l$ in
\eref{smolyak-original}, we rely on the following criterion. Assume that we are
given two additional constants: $\epsilon_a$ and $\epsilon_r$ referred to as the
absolute and relative error, respective. Then, the process is terminated as soon
as
\[
  \max_{(\vi, \vj)} \, |\Delta\f(\vx_{\vi \vj})| \leq \max \left\{ \epsilon_a, \epsilon_r (f_\text{max} - f_\text{min}) \right\}
\]
where $f_\text{min}$ and $f_\text{max}$ are the minimum and maximum observed
value of \f, respectively, and the left-hand side is the maximum surplus whose
level index has not been refined yet (considered as $\vi_l$ at some step $l$ in
\eref{approximation}). The above criterion is a sound way to curtail the process
as it is based on the actual progress.

The adaptivity presented in this subsection is referred to as hybrid as it
combines features of global and local adaptivity; the combination was proposed
in \cite{jakeman2012}. Local adaptivity, which has already been sufficiently
motivated, is due to \cite{ma2009}, and it operates on the level of individual
nodes. Global adaptivity is due to \cite{klimke2006}, and it operates on the
level of individual dimensions. The intuition behind global adaptivity is that,
in general, the input variables manifest themselves (impact \f) differently, and
the interpolation algorithm is likely to benefit by prioritizing those variables
that are the most influential.

To summarize, we have obtained an efficient algorithm for adaptive hierarchical
interpolation in multiple dimensions. The main equation is \eref{approximation}
where $\Delta\f$, $\vx_{\vi \vj}$, and $e_{\vi \vj}$ are the ones given in
\sref{smolyak}, \sref{grid}, and \sref{basis}, respectively, and the
interpolation procedure is undertaken according to the rules given in
\sref{adaptivity}. Now we discuss the implementation.

\subsection{Implementation}

The life cycle of interpolation has roughly two stages: construction and usage.
The construction stage invokes \f at a set of collocation nodes and produces
certain artifacts. The usage stage estimates the values of \f at a set of
arbitrary points by manipulating the artifacts. In this subsection, we shall
look at the pseudocodes of the two stages. The purpose is to give the big
picture. All the details can be found online \cite{sources}.

Let us first make a general note. We found it beneficial to the clarity and ease
of implementation to collapse the two sums in \eref{approximation} into one.
This requires storing a level index $\vi = (i_k)_{k = 1}^n$ and an order index
$\vj = (j_k)_{k = 1}^n$ for each interpolation element. It is also advantageous
to encode each pair $(i_k, j_k)$ as a single unsigned integer, which, in
particular, eliminates excessive memory usage. In multiple dimensions, this
results in a single vector $\v{\iota} = (\iota_k)_{k = 1}^n$, which we simply
call an index. The encoding that we utilize is as follows: $\iota_k = i_k \lor
(j_k \ll n_\text{bits})$ where $\lor$ and $\ll$ are the bitwise \up{OR} and
logical left shift, respectively, and $n_\text{bits}$ is the number of bits
reserved for storing Smolyak levels (see \eref{smolyak-original}), which can be
adjusted according to the maximum permitted deepness of interpolation.

The pseudocode of the construction stage is given in \aref{construct} called
\texttt{Construct}. The \texttt{target} input is a function $\f$ to be
approximated. The \texttt{surrogate} output is a structure containing the
artifacts of interpolation, which are a set of tuples $\{ (\v{\iota}_k,
\Delta\f(\vx_{\v{\iota}_k}) \}_k$, giving a comprehensive description of an
interpolant. The routine works as follows.

\begin{itemize}

\item[2:] Each iteration is an interpolation step in \eref{approximation}. It
has a state captured by a structure denoted by \texttt{s}. The \texttt{strategy}
object represents an adaptation strategy utilized and works as described in
\sref{adaptivity}. The \texttt{First} method of \texttt{strategy} returns the
initial state of the first step so that the \texttt{indices} field of \texttt{s}
is initialized with the indices of that step. The body of the loop populates the
rest of the fields of \texttt{s} so that \texttt{strategy.Next} can adequately
produce the initial state of the next iteration. The process terminates when a
stopping condition is satisfied, in which case \texttt{Next} returns a null
state.

\item[3:] The \texttt{grid} object represents the interpolation grid utilized
(see \sref{grid}), and its \texttt{Compute} method converts the step's indices
into the coordinates of the corresponding collocation nodes, that is, $\{
\v{\iota}_k \}_k$ into $\{ \vx_{\v{\iota}_k} \}_k$.

\item[4:] \texttt{Invoke} evaluates \texttt{target} at the collocation nodes.
This is by far the most time consuming function of the algorithm as
\texttt{target} is generally expensive to evaluate. This function is also a
prominent candidate for parallelization since the algorithm does not impose any
evaluation order.

\item[5:] \texttt{Evaluate} exercises the interpolant constructed so far at the
collocation nodes, approximating the values obtained on line~4. This function
will be discussed separately.

\item[6:] \texttt{Subtract} computes the difference between the true and
approximated values of \texttt{target}, which yields the step's hierarchical
surpluses $\{ \Delta\f(\vx_{\v{\iota}_k}) \}_k$, similar to \eref{surplus}.

\item[7:] \texttt{strategy.Score} calculates the scores of the new collocation
nodes based on their surpluses; see \eref{score}.

\item[8:] \texttt{Append} improves the interpolant by extending it with the
indices and surpluses of the current iteration.

\end{itemize}

We now turn to the usage stage of an interpolant. The pseudocode is given in
\aref{evaluate} called \texttt{Evaluate}. This algorithm is also involved in
\aref{construct}; see line~5. Let us make a couple of observations regarding
\texttt{Evaluate}.

\begin{itemize}

\item[4:] The inner loop is an unfolded version of \eref{approximation} (there
is no separation between individual interpolation steps taken).

\item[5:] The \texttt{basis} object represents the interpolation basis utilized
(see \sref{basis}), and its \texttt{Compute} method evaluates a single
(multidimensional) basis function at a single point.

\end{itemize}

It is worth noting that the \texttt{basis}, \texttt{grid}, and \texttt{strategy}
objects conform to certain interfaces and can be easily swapped out. This makes
the two algorithms very general and reusable with different configurations. In
particular, the adaptation strategy can be fine-tuned for each particular
problem.

To recapitulate, we have presented the key component of our framework for
probabilistic analysis of electronic systems: an efficient approach to
multidimensional interpolation. The overall technique has been consolidated in
\aref{construct} and \ref{alg:evaluate}.

\section{Analysis}

In \sref{modeling}, we formalized the uncertainty affecting electronic systems
and discussed several aspects of such systems along with metrics \g, which the
designer is interested in evaluating. In \sref{interpolation}, we obtained an
efficient interpolation algorithm for approximating hypothetical
multidimensional functions \f. We shall now amalgamate the ideas developed in
the aforementioned two sections.

Given an electronic system dependent on a number of uncertain parameters $\vu:
\Omega \to \real^\nu$, the goal is to analyze a metric \g representing a certain
aspect of the system. For instance, \vu can correspond to the execution times of
the tasks, and \g can correspond the total energy consumed by the processing
elements, as we exemplify in \sref{time} and \sref{power}. The goal is attained
as follows; recall \fref{example}.

1) The parameterization of \g is changed from \vu to random variables $\vz:
\Omega \to [0, 1]^\nz$ via a suitable transformation $\transform$; this stage is
described in \sref{parameters}. 2) An interpolant of the resulting composition
$\g \circ \transform$ is constructed by treating the composition as a
deterministic function \f of \vz; this stage is detailed in
\sref{interpolation}. 3) An estimation of the probability distribution of \g is
undertaken in the usual sampling-based manner but relying solely on the
constructed interpolant; \g is no longer involved. This last stage boils down to
drawing independent samples from $F_{\vz}$ and evaluating the interpolant
$\interpolant{n}{l}(\f) \equiv \interpolant{n}{l}(\g \circ \transform)$ at those
points. Having collected samples of \g, other statistics about \g, such as
probabilities of particular events, can be straightforwardly estimated. We do
not discuss this estimation stage any further as it is standard.

There are two aspects concerning the usage of the proposed framework that we
would like to cover in what follows.

\subsection{Expectation and Variance}

Since the expected value and variance, which are defined in \eref{expectation}
and \eref{variance}, respectively, usually draw particular attention, we would
like to elaborate on them separately.

As shown in \sref{parameters}, \g can be reparameterized in terms of independent
variables that are uniformly distributed on $[0, 1]^\nz$. This means that the
probability density function of \vz simply equals to one. Therefore, using
\eref{expectation} and \eref{approximation}, we have
\[
  \expectation{\g} \approx \expectation{\interpolant{n}{l}(\f)} = \int_{[0, 1]^\nz} \interpolant{n}{l}(\f)(\vz) \, \d\vz = \sum_{\vi \in \sparseindex{n}{l}} \sum_{\vj \in \Delta\tensorindex{n}{\vi}} \Delta\f(\vx_{\vi \vj}) w_{\vi \vj}
\]
where
\[
  w_{\vi \vj} = \int_{[0, 1]^\nz} e_{\vi \vj}(\vz) \d\vz = \prod_{k = 1}^\nz \int_0^1 e_{i_k j_k}(\z_k) \d\z_k = \prod_{k = 1}^\nz w_{i_k j_k}.
\]
In the above equation, $w_{ij}$ is as shown in \eref{volume}. Consequently, we
have obtained an analytical formula for the expected value of \g, which does not
require any additional sampling.

Regarding the variance of \g, it can be seen in \eref{variance} that the
variance can be assembled from two components: the expected value of \g, which
we already have, and the expected value of $\g^2$, which we are missing. The
solution is to let $\h = (\g, \g^2)$ be the metric instead of \g. Then the
expected values of both \g and $\g^2$ will be available in analytical forms, and
the variance of \g can be computed using \eref{variance}. This approach can be
generalized to probabilistic moments of higher orders.

\subsection{Multiple Outputs}

The careful reader has noted a problem with the calculation of variance in the
previous subsection: \h is vector valued. More generally, the metric \g in
\sref{modeling} and the function \f in \sref{interpolation} have been depicted
as having one-dimensional codomains. This, however, has been done only for the
sake of clarity. All the mathematics and pseudocodes stay the same for
vector-valued functions. The only except is that, since a surplus
$\Delta\f(\vx_{\vi \vj})$ naturally inherits the output dimensionality of \f,
the operations that involve $\Delta\f(\vx_{\vi \vj})$ should be adequately
adjusted. If the outputs are on different scales and/or have different accuracy
requirements, one might want to have different $\epsilon_a$ and $\epsilon_r$ in
\eref{stopping-condition} for different outputs. In that case, one also needs to
device a more sensible strategy for scoring collocation nodes in \eref{score}
such as rescaling individual outputs and then calculating the uniform norm
$\norm[\infty]{\cdot}$ or $\L{2}$ norm $\norm[2]{\cdot}$. Our code
\cite{sources} has been written with multiple outputs in mind.

To summarize, once an interpolant of \g has been constructed, the distribution
of \g is estimated using versatile sampling methods applied to the interpolant.
The framework extends naturally to metrics with multiple outputs, and it
provides analytical formulae for expectations and variances.

Let us remind that the evaluation of \g is an extensive operation. Our technique
is designed to keep this expense as low as possible by choosing the evaluation
points adaptively, which is unlike traditional sampling methods. Moreover, in
contrast to \up{PC} expansions and similar techniques, the proposed framework is
well suited for the nonsmooth response surfaces.

\section{Experimental Results}
\slab{interpolant-results}

In this section, we evaluate the performance of our framework. Our
implementation is open source and can be found at \cite{sources}, which also
includes the experimental setup along with configuration files and input data.
The experiments discussed below are conducted on a \up{GNU}/Linux machine
equipped with 16 processors Intel Xeon E5520 2.27~\up{GH}z and 24~\up{GB} of
\up{RAM}.

We shall address $3 \times 2 \times 3 = 18$ uncertainty-quantification problems.
Specifically, we shall consider three platform sizes \np: 2, 4, and 8 processing
elements; two application sizes \nt: 10 and 20 tasks; and three metrics \g: the
end-to-end delay, total energy consumption, and maximum temperature defined in
\eref{end-to-end-delay}, \eref{total-energy}, and \eref{maximum-temperature},
respectively. At this point, it might be helpful to recall the example in
\fref{example}.

\subsection{Configuration}

A platform with \np processing elements and an application with \nt tasks are
generated randomly by the \up{TGFF} tool \cite{dick1998}. The tool generates \np
tables and a directed acyclic graph with \nt nodes. Each table corresponds to a
processing element, and it describes certain properties of the tasks when they
are mapped to that particular processing element. Namely, each table assigns two
numbers to each task: a reference execution time, chosen uniformly between 10
and 50~ms, and a power consumption, chosen uniformly between 5 and 25~W. The
graph captures data dependencies between the tasks. The application is scheduled
using a list scheduler \cite{adam1974}. The mapping of the application is fixed
and obtained by scheduling the tasks based on their reference execution times
and assigning them to the earliest available processing elements (a shared ready
list).

The construction of thermal \up{RC} circuits needed for temperature analysis is
delegated to the HotSpot tool \cite{skadron2003}. The floorplan of each platform
is a regular grid wherein each processing element occupies $2 \times
2~\text{mm}^2$ on the die. The output of the tool is a pair of a thermal
capacitance matrix $\m{C}$ and a thermal conductance $\m{G}$ matrix used in
\eref{thermal-system}. The leakage modeling is based on a linear fit to a data
set of \up{SPICE} simulations of a series of \up{CMOS} invertors
\cite{ukhov2012, liu2007}; see also \cite{ukhov2014}. The time step of power and
temperature profiles is constant and equal to one microsecond; see \sref{power}
and \sref{temperature}.

The uncertain parameters \vu introduced in \sref{problem} are the execution
times of the tasks; see \sref{time}. All other parameters are deterministic.
Targeting the practical scenario described in \sref{parameters}, the marginal
distributions and correlation matrix of \vu are assumed to be available. Without
loss of generality, the marginal of $\u_i$ is a four-parametric beta
distribution $\text{Beta}(\alpha_i, \beta_i, a_i, b_i)$ where $\alpha_i$ and
$\beta_i$ are the shape parameters, and $a_i$ and $b_i$ are the endpoints of the
support. The left $a_i$ and right $b_i$ endpoint are set to 80\% and 120\%,
respectively, of the reference execution time generated by the \up{TGFF} tool as
described earlier. The parameter $\alpha_i$ and $\beta_i$ are set to two and
five, respectively, for all tasks, which skews the distribution toward the left
endpoint. The execution times of the tasks are correlated based on the structure
of the graph produced by the \up{TGFF} tool: the closer task $i$ and task $j$
are in the graph as measured by the number of edges between vertex $i$ and
vertex $j$, the stronger $\u_i$ and $\u_j$ are correlated. The model-order
reduction parameter $\eta$ in \eref{reduction} (\sref{parameters}) is set to
0.9, which results in $\nz = 2$ and 3 preserved variables for applications with
$\nt = 10$ and 20 tasks, respectively.

The configuration of the interpolation algorithm (the collocation nodes, basis
functions, and adaptation strategy with stopping conditions) is as described in
\sref{interpolation}. The parameters $\epsilon_a$, $\epsilon_r$, and
$\epsilon_s$ are around $10^3$, $10^2$, and $10^4$, respectively, depending on
the problem; the exact values can be found at \cite{sources}, which, again,
contains all other details too.

The performance of our framework with respect to each problem is assessed as
follows. First, we obtain the ``true'' probability distribution of the metric in
question \g by sampling \g directly and extensively. Direct sampling means that
samples are drawn from \g itself (not from a surrogate), and that there is no
any intermediate model-order reduction (see \sref{parameters}). Second, we
construct an interpolant for \g and estimate \g's distribution by sampling the
interpolant. In both cases, we draw $10^5$ samples; let us remind, however, that
the cost of sampling the interpolant is practically negligible. Third, we
perform another round of direct sampling of \g, but this time we draw as many
samples as many times the metric was evaluated during the interpolation process.
In each of the three cases, the sampling is undertaken in accordance with a
Sobol sequence, which is a quasi-random low-discrepancy sequence featuring much
better convergence properties than those of the classical Monte-Carlo (\up{MC})
sampling \cite{joe2008}.

As a result, we obtain three estimates of \g's distribution: reference (the one
considered true), proposed (the one interpolation powered), and direct (the one
equal in terms of the number of \g's evaluations to the proposed solution). The
last two are compared with the first one. For comparing the proximity between
two distributions, we use the well-known Kolmogorov--Smirnov (\up{KS}) statistic
\cite{rao2002}, which is the supremum over the distance (pointwise) between two
empirical distribution functions and, hence, is a rather unforgiving error
indicator.

\subsection{Discussion}

The results of all 18 uncertainty-quantification problems are given in
\fref{results} as a 6-by-3 grid of plots, one plot per problem. The three
columns correspond to the three metrics at hand: the end-to-end delay (left),
total energy (middle), and maximum temperature (right). The three pairs of rows
correspond to the three platform sizes: 2 (top), 4 (middle), and 8 (bottom)
processing elements. The rows alternate between the two application sizes: 10
(odd) and 20 (even) tasks.

The horizontal axis of each plot shows the number of points, that is,
evaluations of the metric \g, and the vertical one shows the \up{KS} statistic
on a logarithmic scale. Each plot has two lines. The solid line represents our
technique. The circles on this line correspond to the steps of the interpolation
process given in \eref{approximation}. They show how the \up{KS} statistic
computed with respect to the reference solution changes as the interpolation
process takes steps (and increases the number of collocation nodes) until the
stopping condition is satisfied (\sref{adaptivity}). Note that only a subset of
the actual steps is displayed in order to make the figure legible. Synchronously
with the solid line (that is, for the same numbers of $\g$'s evaluations), the
dashed line shows the error of direct sampling, which, as before, is computed
with respect to the reference solution.

Let us first describe one particular problem shown in \fref{results}. Consider,
for instance, the one labeled with $\bigstar$. It can be seen that, at the very
beginning, our solution and the solution of direct sampling are poor. The
\up{KS} statistic tells us that there are substantial mismatches between the
estimates and the reference solution. However, as the interpolant is being
adaptively refined, our solution approaches rapidly the reference one and, by
the end of the interpolation process, leaves the solution of na\"{i}ve sampling
approximately an order of magnitude behind.

Studying \fref{results}, one can make a number of observations. First and
foremost, our interpolation-powered approach (solid lines) to probabilistic
analysis outperforms direct sampling (dashed lines) in all the cases. This means
that, given a fixed budget of the computation time---the probability
distributions delivered by our framework are much closer to the true ones than
those delivered by sampling $\g$ directly, despite the fact that the latter
relies on Sobol sequences, which are a sophisticated sampling strategy. Since
direct sampling methods try to cover the probability space impartially,
\fref{results} is a salient illustration of the difference between being
adaptive and nonadaptive.

It can also be seen in \fref{results} that, as the number of evaluations
increases, the solutions computed by our technique approach the exact ones. The
error of our framework decreases generally steeper than the one of direct
sampling. The decrease, however, tends to plateau toward the end of the
interpolation process (when the stopping condition is satisfied). This behavior
can be explained by the following two reasons. First, the algorithm has been
instructed to satiate certain accuracy requirements ($\epsilon_a$, $\epsilon_r$,
and $\epsilon_s$), and it reasonably does not do more than what has been
requested. Second, since the model-order reduction mechanism is enabled in the
case of interpolation, the metric being interpolated is not \g, strictly
speaking; it is a lower-dimensional representation of \g, which already implies
an information loss. Therefore, there is a limit on the accuracy that can be
achieved, which depends on the amount of reduction.

The message of the above observations is that the designer of an electronic
system can benefit substantially in terms of accuracy per computation time by
switching from direct sampling to the proposed technique. If the designer's
current workhorse is the classical \up{MC} sampling, the switch might lead to
even more dramatic savings than those shown in \fref{results}. Needless to
mention that the gain is especially prominent in situations where the analysis
needs to be performed many times such as when it resides in a design-space
exploration loop.

\begin{remark}
The wall-clock time taken by the experiments is not reported in this paper
because this time is irrelevant: since the evaluation of \g is time consuming
(see \sref{problem}), the number of \g's evaluations is the most apposite
expense indicator. For the curious reader, however, let us give an example by
considering the problem labeled with $\clubsuit$ in \fref{results}. Obtaining a
reference solution with $10^5$ simulations in parallel on 16 processors took us
around two hours. Constructing an interpolant with 383 collocation nodes took
around 30 seconds (this is also the time of direct sampling with 383 simulations
of \g). Evaluating the interpolant $10^5$ times took less than a second. The
relative computation cost of sampling an interpolant readily diminishes as the
complexity of \g increases; contrast it with direct sampling, whose cost grows
proportional to \g's evaluation time.
\end{remark}

\subsection{Real-life Example}

Last but not least, we investigate the viability of deploying the proposed
framework in a real environment. It means that we need to couple the framework
with a battle-proven simulator, which is used in both academia and industry, and
let it simulate a real application running on a real platform. Before we
proceed, we would like to remind that all the implementation and configuration
details including the infrastructure developed for this example can be found at
\cite{sources}.

The scenario that we consider is the same as the one depicted in \fref{example}
except for the fact that an industrial-standard simulator is put in place of the
``black box'' on the left side, and that the metric of interest \g is now the
total energy. Unlike the previous examples, there is no true solution to compare
with due to the prohibitive expense of the simulator, which is exactly why our
framework is needed in such cases.

The simulator of choice is the well-known and widely used combination of Sniper
\cite{carlson2011} and McPAT \cite{li2009}. The architecture that we simulate is
Intel's Nehalem-based Gainestown series. Sniper is distributed with a
configuration file for this architecture, and we use it without any changes. The
platform is configured to have three \up{CPU}s sharing one \up{L3} cache.

The application that has been chosen for simulation is \up{VIPS}, which is an
image-processing piece of software taken from the \up{PARSEC} benchmark suite
\cite{bienia2011}. In this scenario, \up{VIPS} applies a fixed set of operations
to a given image. The width and height of the image to process are considered as
the uncertain parameters \vu (see \sref{problem}), which are assumed to be
distributed uniformly within certain ranges.

The real-life deployment has fulfilled our expectations. The interpolation
process successfully finished and delivered a surrogate after 78 invocations of
the simulator. Each invocation took 40 minutes on average. The probability
distribution of the total energy was then estimated by sampling the constructed
surrogate $10^5$ times. These many samples would take around 6 months to obtain
on our machine if we sampled the simulator directly in parallel on 16
processors; using the proposed technique, the whole procedure took approximately
9 hours.

\section{Conclusion}
\slab{interpolant-conclusion}

In this paper, we have presented a framework for probabilistic analysis of
electronic systems. Given a description of the probability distribution of the
uncertain parameters present in the system under consideration and a simulator
of a metric of interest dependent on the parameters, the framework prescribes
the steps that need to be taken in order to computationally efficiently obtain
the probability distribution of the metric.

The proposed approach is powered by hierarchical interpolation following a
hybrid adaptation strategy. The adaptivity makes the framework particularly
suited for problems with idiosyncratic behaviors and steep response surfaces,
which arise in electronic systems due to their digital nature.

The performance of our framework has been assessed by comparing it with the
performance of an advanced sampling technique. The experimental results have
shown that, for a fixed budget of evaluations of the metric, our approach
achieves higher accuracy compared to direct simulations.

Finally, we would like to emphasize that, even though the framework has been
exemplified by considering a specific source of uncertainty and specific
metrics, it is general and can be successfully applied in many other settings.

We develop a framework for system-level analysis of electronic systems whose
runtime behaviors depend on uncertain parameters. The proposed approach thrives
on hierarchical interpolation guided by an advanced adaptation strategy, which
makes the framework general and suitable for studying various metrics that are
of interest to the designer. Examples of such metrics include the end-to-end
delay, total energy consumption, and maximum temperature of the system under
consideration. The framework delivers a light generative representation that
allows for a straightforward, computationally efficient calculation of the
probability distribution and accompanying statistics of the metric at hand. Our
technique is illustrated by considering a number of uncertainty-quantification
problems and comparing the corresponding results with exhaustive simulations.

Computer systems are omnipresent and omniscient. They penetrate deep into
everyday life and are unsettlingly indispensable and increasingly intelligent at
tasks entrusted to them. It is readily understandable that analysis and design
of computer systems are acutely difficult and vastly far-reaching endeavors.

One major concern of the designer of computer systems is the presence of
uncertainty, which, in many cases, is inherent and inevitable. Uncertainty can
be due to different phenomena; it can originate from different sources. From one
of many perspectives, uncertainty in computer systems can be broadly classified
into analog and digital. Analog uncertainty originates from the physical world
while digital uncertainty originates from the virtual, computer world.

A prominent example of analog uncertainty is the one stemming from process
variation \cite{srivastava2010}. In this case, the source of uncertainty is the
imperfection of the contemporary fabrication process: the process parameters of
fabricated nanoscale devices deviate from their nominal values since the
fabrication process cannot be controlled precisely down to the level of
individual atoms. This transistor-level variability then propagates to such
crucial system-level characteristics of a computer system as power consumption
and heat dissipation.

Another example of analog uncertainty is the one originating from performance
variation. In this case, uncertainty arises from natural or accelerated wear or
fatigue \cite{jedec2016}, which degrades the performance of electrical circuits
over time. This physical degradation can cause a permanent fault and, hence,
abruptly end the life of the system at hand. Since the degradation is a
nonuniform and intricate process, the system's lifetime is uncertain to the
designer.

A salient example of digital uncertainty is the one due to workload variation.
In this case, the source of uncertainty is the actual work that computer systems
are instructed to perform. To elaborate, from one activation to another, the
same piece of deterministic software can exhibit drastically different behaviors
depending on the environment and input data, and neither the environment nor
input data that the system under consideration will be exposed to at runtime is
exhaustively known at early development stages.

Process variation has been a topic of many lines of research; see, for instance,
\cite{bhardwaj2006, bhardwaj2008, chandra2010, juan2012, lee2013}. Similarly,
performance uncertainty has been extensively studied in the literature; see, for
instance, \cite{coskun2006, huang2009b, das2014c}. Workload uncertainty has not
been deprived of due attention either, especially in the real-time community;
see, for instance, \cite{diaz2002, santinelli2011, quinton2012, tanasa2015}.
However, as we discuss in detail in the relevant parts of the thesis, certain
problems have not been addressed yet, and the solutions to those that have are
strictly restricted in use, which is frequently due the unrealistic assumptions
that these solutions make.

Nevertheless, such phenomena as the ones mentioned above render the behavior of
computer systems nondeterministic, uncertain to the designer of these systems.
The presence of uncertainty can lead to degradation of the quality of service in
the best case and to severe faults or burnt silicon in the worst-case scenario.
Hence, it is crucial to analyze and quantify uncertainty in computer systems and
to mitigate its damaging consequences by designing these systems in such a way
that they are well aware of uncertainty and well equipped with mechanisms to
effectively and efficiently take it into consideration.

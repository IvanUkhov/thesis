In this thesis, we take the designer's standpoint. Specifically, a certain
aspect of the electronic system under consideration is said to be uncertain if
it is unknown to the designer of this system. The typical scenario is that the
designer is interested in studying a certain quantity---referred to as the
quantity of interest---whose complete knowledge would be highly beneficial but
cannot be attained since the quantity \emph{per se} is uncertain, or it depends
on parameters that are uncertain to the designer. We stay agnostic on the
underlying reason behind this state of uncertainty: it can be aleatory as well
as epistemic.

Uncertainty in electronic systems can originate from different phenomena, and,
in many cases, it is inherent and inevitable (from the designer's standpoint).
Let us consider three examples that are of relevance to the thesis.

\subsection{Process Variation}
\slab{process-variation}

A prominent example of uncertainty is the one stemming from process variation
\cite{chandrakasan2000, srivastava2010}. In this case, the source of uncertainty
is the fabrication process. Specifically, the process parameters of fabricated
nanoscale devices deviate from their nominal values since the contemporary
fabrication process cannot be controlled precisely down to the level of
individual atoms.

The aforementioned transistor-level variability propagates to such crucial
system-level characteristics of an electronic system as power consumption and
heat dissipation and, thereby, makes them uncertain to the designer of the
system. The propagation is due to process variation affecting the key parameters
of a technological process such as the effective channel length and gate oxide
thickness. As a result, the same workload applied to two seemingly identical
dies can lead to two drastically different power profiles and consequently to
two drastically different temperature profiles since power consumption and heat
dissipation depend on the aforementioned quantities, which is to be discussed
further in \sref{power-model}. This concern is especially exigent due to the
power-temperature interplay---which is also covered in \sref{power-model} as
well as in \sref{power-temperature}---whose magnitude depends on process
parameters.

\subsection{Workload Variation}
\slab{workload-variation}

Another salient example of uncertainty is the one emerging from workload
variation. In this case, the source of uncertainty is the actual work that
electronic systems are instructed to perform. To elaborate, from one activation
to another, the same piece of deterministic software can exhibit drastically
different behaviors depending on the environment and input data. Neither the
environment nor input data that the system under consideration will be exposed
to at runtime is exhaustively known at early development stages.

\subsection{Aging Variation}
\slab{aging-variation}

Yet another example of uncertainty is the one originating from aging variation.
In this case, the source of uncertainty is natural or accelerated wear or
fatigue \cite{jedec2016}, which leads to the performance of electrical circuits
degrading over time. This physical degradation can cause terminal faults and,
therefore, abruptly end the life of the system at hand. Since the degradation is
a nonuniform and intricate process, the lifetime of the system is uncertain to
the designer.

There is one factor that it is worth highlighting in this context: temperature.
Temperature has a profound impact on the lifetime of electronic systems, which
is well known and well studied \cite{jedec2016}. The failure mechanisms that are
commonly considered---such as electromigration, time-dependent dielectric
breakdown, and thermal cycling---are directly driven by temperature. One can
also note a connection between process variation and aging variation. More
specifically, the former intensifies the latter via temperature.

\conclusioncut
Having introduced and exemplified uncertainty in electronic systems, we are
ready to consolidate our motivation and solidify our objective.

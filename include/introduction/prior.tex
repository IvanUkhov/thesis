Since the appearance of the first digital computers in 1940s, \ac{MC} sampling
remains one of the most well-known and widely used methods for analyzing
stochastic systems. In this case, the system at hand is treated as an opaque
object, and one only has to evaluate this object a number of times in order to
start to draw conclusions about the system's behavior. This straightforwardness
of application is at the heart of the technique's popularity, but there are two
other prominent reasons: the independence of the number of stochastic dimensions
and the law of large numbers \cite{durrett2010} stating that the quantities
estimated using \ac{MC} sampling asymptotically approach the true values.

The major problem with sampling-based methods, however, is in sampling: one
should be able to obtain sufficient many realizations of the quantity of
interest in order to accurately estimate the needed statistics about this
quantity, and the number of need samples can be considerably large
\cite{diaz-emparanza2002}. In the case of \ac{MC} sampling, for instance, the
error of estimation decreases as $\no^{-1/2}$ where \no is the number of drawn
samples. This means that an additional decimal point of accuracy requires
hundred times more samples. When the subject under analysis is computationally
expensive---which is arguably the case with all nontrivial quantities as they
typically involve a complete realization of the whole system---sampling methods
are rendered slow and often unfeasible.

There are sampling techniques that have better convergence rates than the
aforementioned one of the classical \ac{MC} sampling. Examples of such
techniques include quasi-\ac{MC} sampling and Latin hypercube sampling
\cite{asmussen2007}. However, these rates are still relatively low, and the
corresponding techniques often bring additional restrictions, which limit their
applicability in practice \cite{xiu2010}.

In order to circumvent the high computational costs of sampling methods and to
cover other scenarios, a number of techniques for analysis and design of
electronic systems under uncertainty have been developed. Process variation has
been a topic of many lines of research; see, for instance, \cite{bhardwaj2006,
bhardwaj2008, chandra2010, juan2012, lee2013}. Similarly, performance
uncertainty has extensively been studied in the literature; see, for instance,
\cite{coskun2006, huang2009b, das2014c}. Workload uncertainty has not been
deprived of due attention either, especially in the real-time community; see,
for instance, \cite{diaz2002, santinelli2011, quinton2012, tanasa2015}. However,
as we discuss in detail in the relevant parts of the thesis, certain problems
have not been addressed yet, and the solutions to those that have been
considered are strictly restricted in use, which is frequently due the
unrealistic assumptions that these solutions make.

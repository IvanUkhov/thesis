A bird's eye view of the remaining chapters of the thesis is as follows.

In \cref{background}, we introduce a number of commonly used models that
constitute a starting point for the developments presented in this thesis.
Specifically, these models are a general system model, a temperature-aware power
model, a temperature model, and a temperature-aware reliability model.

In \cref{certainty-development}, we consider techniques for deterministic
system-level analysis of electronic systems. These techniques do not take
uncertainty into account; however, they serve as a solid foundation for those
that do. Our attention revolves primarily around power and temperature since
they are of central importance for attaining robustness and energy efficiency.
We develop a novel approach to dynamic steady-state temperature analysis of
electronic systems and apply it in the context of temperature-aware reliability
optimization. Reliability optimization addresses aging uncertainty by
definition; however, the above-mentioned optimization falls in the scope of this
chapter since the accompanying reliability analysis treats temperature as a
deterministic quantity, which is suboptimal, as we discuss and tackle in the
subsequent chapters.

In \cref{uncertainty-analog-fabrication}, we present our first technique
targeted at uncertainty. We develop a statistical approach to analyzing
process-variation-induced fluctuations of process parameters across silicon
wafers by means of indirect measurements, which can also be incomplete and
noisy. Examples of process parameters include the effective channel length and
gate oxide thickness, and examples of indirect measurements include readings
from thermal sensors. To this end, we make use of a suite of tools borrowed from
Bayesian statistics.

In \cref{uncertainty-analog-development}, we continue working with process
uncertainty and present a technique that is applicable to studying diverse
quantities with respect to process variation. To this end, we leverage the
theory of polynomial chaos expansions. In particular, the proposed approach
allows for analyzing transient and dynamic steady-state power and temperature
profiles of electronic systems as well as such critical metrics of electronic
systems as the maximum temperature and energy consumption. All the
aforementioned quantities can be analyzed from the probabilistic standpoint, and
the utility of this virtue is illustrated by addressing a problem of
design-space exploration with a set of probabilistic constraints related to
reliability. Unlike the reliability model utilized in
\cref{certainty-development}, the one presented in
\cref{uncertainty-analog-development} is well aware of process uncertainty and
consequently allows for a more adequate treatment of aging uncertainty.

In \cref{uncertainty-digital-development}, we develop another system-level
technique targeted at uncertainty. The tool makes use of adaptive hierarchical
interpolation on sparse grids and specializes in tackling workload variation and
similar phenomena whose manifestation is less regular than the smooth,
well-behaved one of process variation. The proposed technique is exemplified by
quantifying the effect that workload units with unknown processing times have on
the timing-, power-, and temperature-related characteristics of electronic
systems.

In \cref{uncertainty-digital-operation}, we elaborate on runtime management of
electronic systems under workload variation. In this context, we perform an
early investigation of the applicability of advanced prediction techniques from
machine learning to fine-grained long-range forecasting of resource usage in
computer systems. More concretely, we study the applicability of recurrent
neural networks.

In \cref{conclusion}, we conclude the thesis by recapitulating our contribution.

The thesis also contains an appendix where we give an overview of a number of
concepts from such interconnected disciplines as linear algebra, probability
theory, statistics, numerical integration, and interpolation. The theory given
in the appendix is utilized extensively throughout the thesis.

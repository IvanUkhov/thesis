Our goal in this section is to build a flexible and computationally efficient
technique for reliability analysis of electronic systems affected by process
variation. The development is based on the general reliability model $R(\cdot |
\vg)$ described in \sref{reliability-model}, which is not aware of process
variation yet. Note also that, in this section, we address not only process
uncertainty but also aging uncertainty, which is introduced in
\sref{aging-variation}, since the latter can be more adequately mitigated when
the former is taken into consideration in reliability analysis.

\subsection{\problemtitle}

Our work in this context is motivated by the following two observations.

First, as emphasized throughout the thesis, temperature is the driving force of
many failure mechanisms. The most prominent examples include electromigration,
time-dependent dielectric breakdown, stress migration, and thermal cycling
\cite{xiang2010}; see \cite{jedec2016} for an overview. All of the
aforementioned mechanisms have strong dependencies on the operating temperature.
At the same time, temperature is tightly related to process parameters---such as
the effective channel length and gate oxide thickness---and can vary
dramatically when these parameters deviate from their nominal values. Despite
the above concerns, the state-of-the-art techniques for reliability analysis of
electronic systems lack a systematic treatment of process variation and, in
particular, the effect of this variation on temperature, which is also the case
in \sref{reliability-model}.

Second, having established a reliability model $R(\cdot | \vg)$ of the system
under consideration, the major portion of the associated computation time is
ascribed to the evaluation of the parameterization \vg rather than to the model
\emph{per se}, that is, for a given \vg. For instance, \vg often contains
estimates of the \ac{MTTF} of each processing element for a range of stress
levels. Thus, \vg typically involves computationally intensive simulations,
including power analysis paired with temperature analysis; see
\sref{reliability-model}.

Guided by the aforementioned observations, we employ the \ac{PC} decomposition
in order to construct a light surrogate for \vg. It is worth emphasizing that
$R(\cdot | \vg)$ stays intact, which means that our approach does not impose any
restrictions on $R(\cdot | \vg)$. Hence, the designer can take advantage of an
arbitrary reliability model in a straightforward manner. Naturally, this also
implies that the modeling errors associated with the chosen $R(\cdot | \vg)$ can
affect the quality of the results delivered by our technique. Therefore,
choosing an adequate reliability model for the problem at hand is the designer's
responsibility.

\begin{remark} \rlab{chaos-nested-uncertainty}
It is important to realize that there are two levels of probabilistic modeling
here. First, $R(\cdot | \vg)$ \emph{per se} is a probabilistic model describing
the lifetime \life of the system. Second, the parameterization \vg is another
probabilistic model characterizing the impact of the uncertainty due to process
variation on the reliability model. Thus, the overall model can be thought of as
a probability distribution over probability distributions. Given an outcome of
the fabrication process and thus \vg, the system's lifetime remains random.
\end{remark}

To conclude, the quantify of interest \g, which is an output of Stage~1 in
\fref{chaos-overview}, is the parameters \vg of the reliability model under
consideration.

\subsection{Surrogate Construction}

Similar to \sref{chaos-dynamic-steady-analysis}, Stage~3 and Stage~4 of the
framework require no particular attention in this section except for noting that
\rref{chaos-multidimensional-output} should be taken into consideration if the
parameterization \vg has multiple entries.

\conclusioncut
In conclusion, the proposed approach to reliability analysis is founded on the
basis of the state-of-the-art reliability models, and it enriches their modeling
capabilities by seamlessly incorporating the deteriorating impact of process
variation. In particular, the technique allows for a straightforward propagation
of uncertainty from process parameters through temperature to the lifetime of
the system, which is an important application since temperature is the driving
force of many failure mechanisms. In contrast to the straightforward use of
\ac{MC} sampling, the light surrogates that we construct make the subsequent
analysis highly efficient from the computational perspective.

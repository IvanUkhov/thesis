Assume the system model given in \sref{system-model}. Suppose that the system
depends on a number of process parameters that are uncertain at the design
stage. Once the fabrication process yields a particular outcome, the considered
process parameters take certain values, which can and are likely to vary across
each fabricated chip, and stay unchanged thereafter. This variability leads to
deviations of power from the nominal values and, therefore, to deviations of
temperature from the one corresponding to the nominal power consumption.

Each process parameter is a characteristic of a single transistor---consider,
for instance, the effective channel length---and, therefore, each device in the
electrical circuit at hand can potentially have a different value of this
parameter since, in general, the variability due to process variation is not
uniform. Consequently, each process parameter can be viewed as a random process
$\x: \Omega \times \real^2 \to \real$ defined on a probability space $(\Omega,
\F, \probability)$---which, in turn, is defined in
\sref{probability-theory}---and a two-dimensional plane. Since our work is
system-level oriented, the random processes are discretized, and each processing
element is modeled with a finite number of random variables. Then the uncertain
parameters of the problem under consideration are defined as
\[
  \vu = (\u_i)_{i = 1}^\nu: \Omega \to \real^\nu
\]
so that all variables of all processing elements are collected into one random
vector with \nu entries. Given this setting, our goal in this chapter is
twofold.

First, we are to develop a system-level framework for transient temperature
(and, hence, power) analysis as well as dynamic steady-state analysis of
electronic systems where power consumption and heat dissipation are stochastic
due to their dependency on the parameters \vu. The designer is required to
specify \one~the probability distribution of \vu and \two~the dependency of the
consumed power on \vu, which can be given as a ``black box.'' The framework is
to provide the designer with the tools for analyzing the system under a given
workload---without imposing constraints on the nature of this workload---and
calculating the corresponding stochastic power \mp and temperature \mq profiles
with a desired level of accuracy and at low computational costs.

Second, taking into consideration the effect of process variation on power and
temperature, we are to find the reliability function of the system and to
develop a computationally efficient design-space exploration scheme exploiting
the proposed techniques for power, temperature, and reliability analysis.

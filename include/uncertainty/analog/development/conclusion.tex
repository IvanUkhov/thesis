Electronic-system designs that ignore process variation are unreliable and
inefficient. Acknowledging this aspect, we have developed a flexible framework
for the analysis of electronic systems subjected to process variation. The
framework is capable of modeling diverse probability laws of the underlying
uncertain parameters and arbitrary dependencies of the system on these
parameters. Our technique delivers a lightweight representation of the quantity
that the designer is interested in studying, and this representation allows for
a computationally efficient estimation of the corresponding probability
distribution as well as of other accompanying statistics about this quantity of
interest.

We have presented a technique targeted at the quantification of transient power
and temperature variations of electronic systems under process variation. Our
technique has been applied in a context of particular importance where the
variability of the effective channel length has been addressed. Note, however,
that the framework can be readily utilized to analyze any other quantities
affected by process variation and to study their combinations. Using this
application, we have drawn a comparison with \ac{MC} sampling, which has
confirmed the efficiency of our approach in terms of both accuracy and speed.
The reduced execution times, by up to five orders of magnitude, implied by our
framework allow for the analysis to be efficiently performed inside
design-space-exploration loops aimed at, for instance, energy and reliability
optimization with temperature-related constraints under process variation.

We have also presented a process-variation-aware approach to dynamic
steady-state temperature analysis as well as such an approach to reliability
analysis that seamlessly takes into account the variability of process
parameters and, in particular, the effect of process variation on temperature.
We have drawn a comparison with \ac{MC} sampling, which has confirmed the
efficiency of our solution with respect to accuracy and speed. The low
computational demand of the proposed technique implies that it is readily
applicable for design-space exploration, which has also been demonstrated by
considering an energy-driven probabilistic optimization procedure under
reliability-related constraints. We have shown that temperature has to be
treated as a stochastic quantity in order to pursue robustness of
electronic-system designs.

Finally, we note that, even though the framework has been illustrated by
considering the effective channel length, gate oxide thickness, and a number of
specific quantities of interest, it can be readily applied to other problems
requiring the uncertainty due to process variation to be considered.

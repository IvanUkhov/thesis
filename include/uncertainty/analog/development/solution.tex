The uncertainty imposed by process variation is addressed as follows. Due to the
inherent complexity, uncertainty-quantification problems are often viewed as
approximation problems: one constructs a computationally efficient surrogate for
the system under consideration and then studies this light representation
instead of the original system. In this chapter, we reside to the \ac{PC}
decomposition---which is introduced in \sref{polynomial-chaos}---for the
construction of such a lightweight surrogate for the quantity of interest. The
technique decomposes stochastic quantities into infinite series of mutually
orthogonal polynomials operating on random variables. These series are an
attractive alternative to \ac{MC} sampling since they possess much faster
convergence properties and provide succinct and intuitive representations of
system responses to stochastic inputs. Having obtained an adequate polynomial
surrogate, we utilize it for calculating the desired statistics about the
quantity of interest such as its \ac{CDF}, \ac{PDF}, probabilistic moments, and
probabilities of various events.

To elaborate, we develop a framework for the analysis of computer systems
subject to the uncertainty due to process variation. The framework is flexible
in modeling diverse probability distributions, specified by the designer, of the
uncertain parameters---such as the effective channel length and gate oxide
thickness. Moreover, there are no assumptions on the probability distribution of
the quantity of interest---such as power and temperature profiles---as this
distribution is unlikely to be known \emph{a priori}. The proposed technique is
capable of capturing arbitrary joint effects of the uncertain parameters on the
system since the impact of these parameters is introduced into the framework as
a ``black box,'' which is also defined by the designer. In particular, it allows
for the power-temperature interplay to be taken into account with no effort.

Leveraging the proposed framework, we extend the deterministic transient and
dynamic steady-state analysis presented in \sref{transient-analysis} and
\sref{dynamic-steady-state-analysis}, respectively, by accounting for the
uncertainty due to process variation. Moreover, the framework allows us to
enrich the reliability analysis presented in \sref{reliability-model}, which is
based on the state-of-the-art reliability models, by taking into consideration
the effect of process variation on temperature.

We illustrate the proposed framework by considering two important process
parameters that are affected by process variation, namely, the effective channel
length and gate oxide thickness; note, however, that our approach can be applied
to other parameters as well. Furthermore, we utilize the framework in order to
construct a computationally efficient design-space exploration procedure
targeted at the minimization of energy consumption under probabilistic
constraints on the temperature and lifetime of the system.

In this section, we consider a particular application of the proposed approach
to probabilistic transient analysis given in \sref{chaos-transient-analysis} in
order to illustrate how it is used in practice. We begin by describing the
considered scenario.

\subsection{\problemtitle}

At Stage~1 in \fref{chaos-overview}, the quantity of interest \g is the
transient power and temperature profiles corresponding to a given workload. Let
us now specify the parameters \vu that make \g uncertain to the designer of the
system.

As discussed in \sref{power-model}, the total dissipation of power is composed
of two major parts: dynamic and static. The influence of process variation on
the dynamic power is known to be negligibly small \cite{srivastava2010}; on the
other hand, the variability of the static power is substantial, in which the
subthreshold leakage current contributes the most \cite{juan2011, juan2012}.
Hence, we focus on the subthreshold leakage and, more specifically, on the
effective channel length denoted by \u since it has the strongest influence on
this leakage---it affects the threshold voltage as well---and is severely
deteriorated by process variation \cite{chandrakasan2000}.

\inputfigure{chaos-beta-gaussian}
It is well known that the dispersion of the effective channel length around its
nominal value resembles the bell shape of Gaussian distributions. Therefore,
such variations are often conveniently modeled using Gaussian random variables
\cite{bhardwaj2006, ghanta2006, huang2009a, shen2009, chandra2010,
srivastava2010, juan2011, juan2012, lee2013}. Due to both the underlying physics
and demonstration purposes, we make a step further and embed into the model the
fact that the effective channel length---occupying the space between the drain
and source of a transistor---cannot be arbitrarily large or take negative
values, which Gaussian distributions allow it to do. In other words, we require
the model of \u to have a bounded support. With this in mind, we propose to
model the effective channel length and other physically bounded parameters using
the four-parametric family of beta distributions as
\begin{equation} \elab{beta-distribution}
  \u \sim \mathrm{Beta}(a, b, c, d)
\end{equation}
where $a$ and $b$ are the shape parameters, and $c$ and $d$ are the left and
right bounds of the support, respectively. The parameters $a$ and $b$ can be
chosen so that the frequently observed bell shape is preserved. An illustration
is given in \fref{chaos-beta-gaussian} where a beta distribution is fitted to
the standard Gaussian distribution; alternatively, one can match probabilistic
moments. It can be seen that the curves are nearly indistinguishable, but the
beta one has a bounded support $[-4, 4]$, which can potentially lead to more
realistic models.

The variability of $u$ is split into global $u_\overall$ and local $u_\local$
parts \cite{shen2009, chandra2010}. Without loss of generality, $u_\overall$ can
be treated as a composition of independent inter-lot, inter-wafer, and inter-die
variations; likewise, $u_\local$ can be treated as a composition of independent
and dependent local variations. The variability $u_\overall$ is assumed to be
shared by all the \np processing elements whereas each processing element is
assumed to have its own local parameter $u_{\local, i}$. Thus, the effective
channel length of processing element $i$ is modeled as follows:
\[
  u_i = u_\nominal + u_\overall + u_{\local, i}
\]
where $u_\nominal$ is the nominal value of the effective channel length.
Consequently, the uncertain parameters of the problem are
\[
  \vu = (u_{\local, 1}, \dotsc, u_{\local, \np}, u_\overall): \Omega \to \real^{\np + 1}.
\]

Global variations are typically assumed to be uncorrelated with respect to the
local ones. The latter, however, are known to have high spatial correlations.
Similar to the treatment in \cref{uncertainty-analog-fabrication}, we model
these correlations using the composite correlation function given in
\eref{bayes-correlation}, which is inspired by the correlation patterns induced
by the fabrication process \cite{friedberg2005, chandrakasan2000, cheng2011}.
Specifically, the correlation function imposes similarities between those
locations on the die that are close to each other as well as between those
locations that are at the same distance from the center of the die; see also
\cite{ghanem1991, ghanta2006, bhardwaj2008, huang2009a, lee2013}.

Although \eref{bayes-correlation} captures certain features inherent to the
fabrication process, it is still an idealization. In practice, it can be
difficult to make a justifiable choice and tune such a formula, which is a
prerequisite for the techniques discussed in \sref{chaos-prior} that are based
on the continuous \ac{KL} decomposition. A correlation matrix, on the other
hand, can be readily estimated from measurements and, thus, is a more probable
input to probabilistic analysis. Therefore, we use \eref{bayes-correlation} with
the only purpose of constructing a correlation matrix of $\{ u_{\local, i} \}_{i
= 1}^\np$. For convenience, the resulting matrix is extended by one dimension in
order to accommodate $u_\overall$ along with $\{ u_{\local, i} \}_{i = 1}^\np$.
In this case, the correlation matrix obtains one additional nonzero diagonal
element equal to unity; the result is the correlation matrix of \vu denoted by
$\correlation{\vu}$.

Let us now be more specific about the power model in \eref{chaos-power-model}.
In the ongoing scenario, \f can be rewritten as the following summation:
\[
  \f(i, \vq, \vu) = \f_\dynamic(i) + \f_\static(\vq, \vu)
\]
where $\f_\dynamic: \natural[+] \to \real^\np$ and $\f_\static: \real^\np \times
\real^\nu \to \real^\np$. Without loss of generality, the dynamic part
$\f_\dynamic$ is assumed to be given as a dynamic power profile denoted by
$\mp_\dynamic$. Similar to \sref{bayes-results}, the modeling of the static part
$\f_\static$ is based on \up{SPICE} simulations of a reference electrical
circuit composed of \up{BSIM4} devices \cite{bsim} configured according to the
45-nm \up{PTM} \up{HP} model \cite{ptm}; specifically, we use a series of
\up{CMOS} invertors. The simulations are performed with respect to a
sufficiently wide fine-grained two-dimensional grid---the effective channel
length against temperature---and the results are tabulated. An interpolation
technique is subsequently utilized when the static power is required to be
calculated at a particular point within the range of the grid.

Regarding temperature analysis, an adequate thermal \up{RC} circuit should be
constructed. Given the specification of the platform---including the floorplan
of the die and the configuration of the thermal package---the circuit is
constructed by HotSpot \cite{skadron2003} adhering to the structure described in
\sref{temperature-model}.

To conclude, in this section, we address the variability of the effective
channel length. The input to our analysis is composed of the marginal
distributions of the uncertain parameters \vu, which are beta distributions, and
the corresponding correlation matrix $\correlation{\vu}$. Let us now go over the
other stages of our methodology discussed in \sref{chaos-uncertainty-analysis}
and depicted in \fref{chaos-overview}.

\subsection{Probability Transformation}

At Stage~2 in \fref{chaos-overview}, \vu should be preprocessed in order to
extract a vector of mutually independent random variables \vz via a suitable
transformation $\transform$; see \eref{chaos-transformation}. Following the
guidance given in \sref{chaos-transformation}, the most apposite $\transform$
for the ongoing scenario is the Nataf transformation. The whole procedure is
described in detail in \xref{probability-transformation} and can be seen in
\eref{probability-transformation}.

Using this specific $\transform$, one can prescribe arbitrary marginal
distributions to \vz. There are no restrictions in this regard as long as one
can subsequently construct a suitable polynomial basis as described in
\sref{chaos-construction}. We let \vz have beta distributions, staying in the
same family of distributions with \vu.

Since the number of stochastic dimensions, which is $\nu = \np + 1$ in the case
of \vu, directly impacts the computational cost of \ac{PC} expansions, which is
elaborated on in \sref{chaos-construction}, one should consider a possibility of
model order reduction before constructing \ac{PC} expansions. Therefore, the
reduction procedure described in \xref{probability-transformation} in connection
with $\transform$ is assumed to be engaged in the transformation. The reduce
dimensionality is denoted by \nz.

\subsection{Surrogate Construction}

At Stage~3 in \fref{chaos-overview}, the uncertain parameters, power model, and
temperature model developed in the previous subsections are to be fused together
under the desired workload $\mp_\dynamic$ in order to produce the corresponding
stochastic power and temperature profiles denoted by \mp and \mq, respectively.

In the current scenario, the construction of \ac{PC} expansions is based on the
Jacobi polynomial basis as it is preferable in situations involving
beta-distributed parameters \cite{xiu2010}. To give a concrete example, for a
dual-core platform ($\np = 2$) with two stochastic dimensions ($\nz = 2$), the
second-level \ac{PC} expansion ($\lc = 2$) of temperature at step $i$ is as
follows:
\begin{equation} \elab{chaos-expansion-example}
  \begin{split}
    \chaos{2}{2}{\vq_i}
    =    {} & \hat{\vq}_{i, (0, 0)} \, \psi_{(0, 0)} +
              \hat{\vq}_{i, (1, 0)} \, \psi_{(1, 0)} +
              \hat{\vq}_{i, (0, 1)} \, \psi_{(0, 1)} \\
    {} + {} & \hat{\vq}_{i, (1, 1)} \, \psi_{(1, 1)} +
              \hat{\vq}_{i, (2, 0)} \, \psi_{(2, 0)} +
              \hat{\vq}_{i, (0, 2)} \, \psi_{(0, 2)}
  \end{split}
\end{equation}
where the coefficients $\{ \hat{\vq}_{i \vj} \}$ are vectors with two elements
corresponding to the two processing elements. Regarding the basis,
\begin{align*}
  & \psi_{(0, 0)}(\vz) = 1, \\
  & \psi_{(1, 0)}(\vz) = 2 z_1, \\
  & \psi_{(0, 1)}(\vz) = 2 z_2, \\
  & \psi_{(1, 1)}(\vz) = 4 z_1 z_2 \\
  & \psi_{(2, 0)}(\vz) = \frac{15}{4} z_1^2 - \frac{3}{4}, \text{ and} \\
  & \psi_{(0, 2)}(\vz) = \frac{15}{4} z_2^2 - \frac{3}{4}.
\end{align*}
The Jacobi polynomials have two parameters \cite{xiu2010}, and the ones shown
above correspond to the case where both parameters are equal to two. Such a
series might be shorter or longer depending on the accuracy requirements given
by \lc. The expansion of power has the same structure but different
coefficients.

The next step is to compute the coefficients of power $\{ \hat{\vp}_{i \vj} \}$
in \eref{chaos-recurrence}, which subsequently yield the coefficients of
temperature $\{ \hat{\vq}_{i \vj} \}$. As discussed in
\sref{chaos-construction}, these computations involve multidimensional
integration with respect to the distribution of \vz, and they should be
performed numerically using an adequate quadrature $\quadrature{\nz}{\lq}$. When
beta distributions are concerned, the natural choice is Gauss--Jacobi
quadratures, which belong to the class of Gaussian quadratures introduced in
\xref{sparse-integration}. Given $\quadrature{\nz}{\lq}$, the coefficient are
computed as shown in \eref{chaos-coefficient}. It is important to note that \lq
should be chosen in such a way that the quadrature is exact for polynomials of
total order up to at least $2 \lc$, that is, twice the level of \ac{PC}
expansions, which is motivated in \sref{chaos-construction}. Consequently, $\lq
\geq \lc$ since the quadrature is Gaussian.

To summarize, we have completed four out of five stages of the proposed
framework depicted in \fref{chaos-overview}. The result is a light surrogate for
the entire system. At each time step, the surrogate is composed of two
\np-valued polynomials---one is for power, and the other one is for
temperature---which are defined in terms of \nz mutually independent random
variables.

\subsection{Post-Processing}

At Stage~4 in \fref{chaos-overview}, the constructed expansions are utilized in
order to assist the designer in analyzing the impact of process variation on
power- and temperature-related characteristics of the system under development.
Consider, for example, \eref{chaos-expansion-example}. It can be seen in that
the surrogate model has a negligibly small computational cost: for any outcome
of \vz, one can straightforwardly compute the corresponding temperature by
plugging this outcome into \eref{chaos-expansion-example}; the same applies to
power. Hence, the representation can be trivially analyzed in order to retrieve
various statistics about the system. Let us illustrate a few of them using this
example in \eref{chaos-expansion-example}.

\inputfigure{chaos-application-power}
\inputfigure{chaos-application-temperature}
Assume that the dynamic power profile $\mp_\dynamic$ is the one shown in
\fref{chaos-application-power}. Having constructed a surrogate with respect to
this profile, we can calculate, for instance, the expectation and variance of
the temperature that the system has at a certain time moment, which is a trivial
operation given the formulae in \eref{chaos-moments}. For the whole time span of
$\mp_\dynamic$, these quantities are plotted in
\fref{chaos-application-temperature} where the dashed lines correspond to one
standard deviation above the corresponding expectations. The displayed curves
closely match those obtained via \ac{MC} sampling with $\no = 10^4$ samples;
however, our method takes less than a second while \ac{MC} sampling takes more
than a day as we shall see in \sref{chaos-transient-results}. In addition, the
\ac{PDF} of the temperature at that time moment can also be estimated. This
operation is undertaken by sampling the surrogate, in which case we might obtain
curves similar to those shown \fref{chaos-application-density}, which is a part
of a different example given in \sref{chaos-transient-results}.

In this section, we develop a technique for probabilistic transient power and
temperature analysis of electronic systems using the uncertainty-unaware
approach presented in \sref{transient-solution} combined with the machinery
described in \sref{chaos-uncertainty-analysis}. Our goal is to obtain such a
technique that preserves the gradual solution process which is at the heart of
transient analysis; see \sref{transient-analysis}. In this way, the designer
retains fine control over transient calculations.

\subsection{\problemtitle}

As stated in \sref{chaos-uncertainty-analysis}, the designer is supposed to
decide on the system model producing the quantity of interest. Assuming the
general system and temperature models given in \sref{system-model} and
\sref{temperature-model}, respectively, the only component that is left to
define is the power model since it is the one introducing the actual workload
into the system. Denote the power model by
\begin{equation} \elab{chaos-power-model}
  \vp = \f(i, \vq, \vu)
\end{equation}
where $\f: \natural[+] \times \real^\np \times \real^\nu \to \real^\np$ is a
function that evaluates the power consumption $\vp \in \real^\np$ of the
processing elements at time step $i$ given their heat dissipation $\vq \in
\real^\np$ and an assignment of the uncertain parameters $\vu \in \real^\nu$.

\begin{remark}
It should be understood that \vp, \vq, and \vu are random vectors in general,
and that \f consumes $\vq(\omega)$ and $\vu(\omega)$ and yields $\vp(\omega)$
for some particular outcome $\omega \in \Omega$. The function \f \perse is
purely deterministic.
\end{remark}

The designer can choose any \f. For instance, it can be a closed-form formula or
a piece of code. The only assumption we make is that \f is smooth in \vz and,
when viewed as a random variable, belongs to $\L{2}(\Omega, \F, \probability)$
(see \xref{probability-theory}), which is generally applicable to most physical
systems \cite{xiu2010}. The definition of \f is flexible enough to take account
of such phenomena as the interdependence between power and temperature discussed
in \sref{power-model}.

Our solution to transient analysis under process variation is based on the one
presented in \sref{transient-solution}. The major difference is that
\eref{temperature-model} implicitly operates on stochastic quantities in the
context of this chapter. Consequently, the recurrent solution in
\eref{transient-recurrence}, that is,
\begin{equation} \elab{chaos-recurrence-original}
  \vs_i = \m{E} \vs_{i - 1} + \m{F} \vp_i
\end{equation}
for $i = \range{1}{\ns}$, is stochastic as well. In the deterministic case, it
can be readily employed in order to perform transient power and temperature
analysis. In the probabilistic case, however, the situation is substantially
different since $\vp_i$ and hence $\vs_i$ and $\vq_i$ are stochastic quantities.
Moreover, at each step, $\vp_i$ is an arbitrary transformation of the uncertain
parameters \vu and stochastic temperature $\vq_i$, which results in a random
vector with a generally unknown probability distribution. Furthermore, $\vp_i$,
$\vq_i$, $\vs_i$, and \vu are dependent random vectors since the first three are
functions of the last one. Therefore, the operations in
\eref{chaos-recurrence-original} are to be performed on dependent random vectors
with arbitrary distributions, which, in general, have no closed-form solutions.

Let us now summarize Stage~1 in \fref{chaos-overview}. In this section, the
quantity of interest \g is the transient power and temperature
profiles---denoted by \mp and \mq, respectively---that correspond to the
workload specified by \f as shown in \eref{chaos-power-model}. Regarding the
uncertain parameters \vu, they are left unspecified since the construction below
is general with respect to \vu; however, a concrete scenario is to be considered
in the next section, \sref{chaos-transient-application}. We now proceed directly
to Stage~3 as Stage~2 requires no additional attention in this case.

\subsection{Surrogate Construction}

The goal now is to transform the recurrence in \eref{chaos-recurrence-original}
in such a way that the distributions of power and temperature can be efficiently
estimated. Based on the methodology presented in
\sref{chaos-uncertainty-analysis}, we construct a \ac{PC} expansion of \f so
that it can then be propagated through the recurrence in
\eref{chaos-recurrence-original} in order to obtain a \ac{PC} expansion of power
and a \ac{PC} expansion of temperature.

The expansion of $\vp_i$, which is needed in \eref{chaos-recurrence-original}
and computed via \f shown in \eref{chaos-power-model}, is as follows:
\[
  \chaos{\nz}{\lc}{\vp_i} = \sum_{\vj \in \sparseindex{\nz}{\lc}} \hat{\vp}_{i \vj} \psi_{\vj}
\]
where $\{ \psi_{\vj}: \real^\nz \to \real \}$ are the basis polynomials, $\{
\hat{\vp}_{i \vj} \} \subset \real^\np$ are the corresponding coefficients, and
the index set $\sparseindex{\nz}{\lc}$ is the one given in
\eref{index-total-order-anisotropic}. In addition,
\rref{chaos-multidimensional-output} is worth recalling. It can be seen in
\eref{chaos-recurrence-original} that, due to the linearity of the operations
involved in the recurrence, $\vs_i$ attains such a \ac{PC} expansion that has
the same structure as the expansion of $\vp_i$. Then the recurrence in
\eref{chaos-recurrence-original} can be rewritten as follows:
\[
  \chaos{\nz}{\lc}{\vs_i} = \m{E} \, \chaos{\nz}{\lc}{\vs_{i - 1}} + \m{F} \, \chaos{\nz}{\lc}{\vp_i}
\]
for $i = \range{1}{\ns}$. Consequently, there are two interwoven \ac{PC}
expansions: one is for power, and the other for temperature. The two expansions
have the same polynomial basis but different coefficients. In order to
understand the structure of the above formula better, let us spell it out as
\[
  \sum_{\vj \in \sparseindex{\nz}{\lc}} \hat{\vs}_{i \vj} \psi_{\vj} =
  \sum_{\vj \in \sparseindex{\nz}{\lc}} \left(\m{E} \, \hat{\vs}_{i - 1, \vj} + \m{F} \, \hat{\vp}_{i \vj}\right) \psi_{\vj}.
\]
Making use of the orthogonality property, which can be seen in
\eref{chaos-orthogonality}, we obtain the following recurrence:
\begin{equation} \elab{chaos-recurrence}
  \hat{\vs}_{i \vj} = \m{E} \hat{\vs}_{i - 1, \vj} + \m{F} \hat{\vp}_{i \vj}
\end{equation}
for $i = \range{1}{\ns}$ and $\vj \in \sparseindex{\nz}{\lc}$. The coefficients
$\{ \hat{\vp}_{i \vj} \}$ needed in this recurrence are found using a suitable
quadrature $\quadrature{\nz}{\lq}$ (see \xref{sparse-integration}) as follows:
\[
  \hat{\vp}_{i \vj} = \quadrature{\nz}{\lq}{\vp_i \psi_{\vj}}
\]
for $i = \range{1}{\ns}$ and $\vj \in \sparseindex{\nz}{\lc}$ where the
operation is elementwise, and it can be efficiently undertaken by virtue of the
projection matrix in \eref{chaos-projection-matrix}.

The final step of the construction process is to combine \eref{chaos-recurrence}
with \eref{temperature-algebraic} in order to obtain the coefficients of the
\ac{PC} expansion of the temperature vector $\vq_i$. Note that, since power
depends on temperature, which is discussed in \sref{power-model}, at each step
of the recurrence in \eref{chaos-recurrence}, the computation of $\hat{\vp}_{i
\vj}$ via \f should be done with respect to the expansion of $\vq_{i - 1}$.

The construction process of the stochastic power and temperature profiles is
estimated to have the following time complexity per time step:
\[
  \bigo{n_n^2 \nc + \nn \np \nq \nc + \nq \f(\np)}
\]
where \nn, \np, \nc, and \nq are the number of thermal nodes, processing
elements, polynomial terms, and quadrature points, respectively, and $\f(\np)$
denotes the contribution of the power model shown in \eref{chaos-power-model}.
The expression can be detailed further by expanding \nc and \nq. Assuming the
isotropic total-order index set in \eref{index-total-order-isotropic}, \nc can
be calculated as shown in \eref{index-total-order-isotropic-length}. This
formula behaves as $\nz^\lc / \lc!$ in the limit with respect to \nz. Regarding
\nq, for quadratures based on the full tensor product given in
\eref{quadrature-tensor}, $\log(\nq) \propto \nz$, which means that the
dependency of \nq on \nz is exponential.

It can be seen that the theory of \ac{PC} expansions suffers from the curse of
dimensionality \cite{eldred2008, xiu2010}: when \nz increases, the number of
polynomial terms as well as the complexity of the corresponding coefficients
exhibit a growth, which is exponential without special treatments. The problem
does not have a general solution and is one of the central topics of many
ongoing studies.

We mitigate the curse of dimensionality by \one~keeping the number of stochastic
dimensions low via model order reduction, which is a part of $\transform$ shown
in \eref{chaos-transformation} and is based on the \ac{KL} decomposition
described in \xref{probability-transformation}, and \two~utilizing efficient
construction techniques, which is discussed in \sref{chaos-construction} and
\xref{sparse-integration}. For instance, in the case of isotropic integration
grids based on the Smolyak algorithm and Gaussian quadratures, $\log(\nq)
\propto \log(\nz)$, which means that the dependency of \nq on \nz is only
polynomial \cite{heiss2008}. Anisotropic constructions allow for further
reduction.

Let us summarize Stage~3 in \fref{chaos-overview}. Recall the stochastic
recurrence in \eref{chaos-recurrence-original} where, in the presence of
dependencies, an arbitrary function $\vp_i$ (since it is based on \f as shown
\eref{chaos-power-model}) of the stochastic temperature $\vq_i$ and uncertain
parameters \vu needs to be evaluated and combined with the random vector
$\vs_i$. This recurrence has been replaced with a purely deterministic one shown
in \eref{chaos-recurrence}. More generally, the whole system, including the
temperature model in \eref{temperature-model-original}, has been substituted
with a light surrogate defined by a set of polynomials $\{ \psi_{\vj} \}$, a set
of coefficients $\{ \hat{\vp}_{i \vj} \}$ for power, and a set of coefficient
$\{ \hat{\vq}_{i \vj} \}$ for temperature. These quantities constitute the
desired stochastic power and temperature profiles, and these profiles are ready
to be analyzed at each step of the process as discussed in
\sref{chaos-processing}.

Before we proceed, let us draw attention to the ease and generality of taking
process variation into consideration using the proposed approach. The
description given above is delivered from any explicit formula of any particular
process parameter. In contrast, the solutions from the literature related to
process variation are typically based on ad~hoc expressions and should be
tailored by the designer for each new parameter individually; see, for instance,
\cite{ghanta2006, bhardwaj2008, huang2009a}. The proposed framework provides
great flexibility in this regard.

In this chapter, we shift our attention from analyzing process variation to
designing with process variation. In other words, instead of studying process
variation \emph{per se}, we now study its deteriorating impact on higher-level
characteristics of a design so that this impact can be adequately taken into
account.

\section{Introduction}
\inputsection{introduction}

\section{Motivational Example}
\slab{chaos-example}
\inputsection{example}

\section{Problem Formulation}
\slab{chaos-problem}
\inputsection{problem}

\section{Prior Work}
\slab{chaos-prior}
\inputsection{prior}

\section{Our Solution}
\slab{chaos-solution}
\inputsection{solution}

\section{Uncertainty Analysis}
\slab{chaos-uncertainty-analysis}
\inputsection{uncertainty-analysis}

\section{Transient Analysis}
\slab{chaos-transient-analysis}
\inputsection{transient-analysis}

\section{Transient Analysis Application}
\slab{chaos-transient-application}
\inputsection{transient-application}

\section{Transient Analysis Results}
\slab{chaos-transient-results}
\inputsection{transient-results}

\section{Dynamic Steady-State Analysis}
\slab{chaos-dynamic-steady-state-analysis}
\inputsection{dynamic-steady-state-analysis}

\section{Reliability Analysis}
\slab{chaos-reliability-analysis}
\inputsection{reliability-analysis}

\section{Reliability Optimization}
\slab{chaos-reliability-optimization}
\inputsection{reliability-optimization}

\section{Reliability Optimization Application}
\slab{chaos-reliability-application}
\inputsection{reliability-application}

\section{Reliability Optimization Results}
\slab{chaos-reliability-results}
\inputsection{reliability-results}

\section{Conclusion}
\slab{chaos-conclusion}
\inputsection{conclusion}

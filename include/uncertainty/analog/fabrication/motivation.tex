Consider the distribution of the effective channel length, which we denote by
\g, across a silicon wafer. The effective channel length has one of the
strongest effects on the leakage current and, consequently, on power and
temperature \cite{juan2012}; see also \sref{power-model}. At the same time, \g
is well known to be severely deteriorated by process variation
\cite{chandrakasan2000, srivastava2010}. Therefore, the distribution of \g is
not uniform across the wafer. For concreteness, let this distribution be the one
depicted on the left-hand side of \fref{inference-example}. The gradient from
blue to yellow represents the transition of \g from low to high values, and the
scale is given in terms of the number of standard deviations away from the mean
value; the exact experimental setup is to be described in detail in
\sref{inference-results}. Hence, the blue regions have a high level of power
consumption and heat dissipation.

Assume that the technological process imposes a lower bound $\g_\minimum$ on \g.
This bound separates defective dies ($\g < \g_\minimum$) from those that are
acceptable ($\g \geq \g_\minimum$). In order to reduce costs, the manufacturer
is interested in detecting the faulty dies and taking them out of the production
process at early stages. Then the possible actions with respect to a single die
on the wafer are \one~to keep the die if it conforms to the specification and
\two~to recycle it otherwise.

\inputfigure{inference-example}
In order to analyze the variability of the effective channel length \g across
the wafer, one can remove the top layer of (and, consequently, destroy) the dies
and measure \g directly. Alternatively, despite the fact that the knowledge of
\g is more preferable, one can step back and decide to quantify process
variation using an auxiliary parameter \h that can be measured without damaging
the dies; for instance, \h can be the leakage current. It should be noted that,
in this second case, \h is the final product of the analysis, and \g is left
unknown.

In either case, adequate test structures have to be present on the dies in order
to take measurements at sufficiently many locations with a sufficient level of
granularity. Such a sophisticated test structure might not always be readily
available, and its deployment might significantly increase production expenses.
Moreover, as noted earlier, the first approach implies that the measured dies
have to be recycled afterwards, and the second approach implies that the further
design decisions will be based on a surrogate quantity \h instead of the primary
target of process variation \g, which can compromise these decisions. The latter
concern is particularly prominent in situations where the production process is
not yet completely stable, and, consequently, design decisions based on the
primary subjects of process variation are preferable.

Our technique operates differently. In this example, in order to characterize
the effective channel length \g, we begin by measuring an auxiliary quantity \h
as well. The quantity is required to depend on \g, and it can be chosen to be
straightforward from the measurement perspective. The distribution of \g across
the whole wafer is then obtained by inferring it from the collected measurements
of \h. Our technique permits these measurements to be taken only at a small
number of locations on the wafer and to be corrupted by noise, which can be due
to the imperfection of the utilized measurement equipment.

Let us consider one particular \h that can be used to study the effective
channel length \g. Specifically, let \h be temperature, which is to be discussed
further in \sref{inference-results}. We can then apply a fixed workload---for
instance, we can run the same application under the same conditions---to a few
dies on the wafer and measure the corresponding temperature profiles. Since
temperature does not require extra equipment to be deployed on the wafer and can
be tracked using infrared cameras \cite{mesa-martinez2007} or built-in
facilities of the dies, our approach can reduce the costs associated with the
analysis of process variation. The results of our framework applied to a set of
noisy temperature profiles measured at only 7\% of the dies are shown on the
right-hand side of \fref{inference-example}, and the locations of the measured
dies are depicted on the left-hand side of \fref{inference-application}. It can
be seen that the two maps in \fref{inference-example} closely match each other,
implying that the distribution of \g is reconstructed with a high level of
accuracy.

\inputfigure{inference-application}
Another characteristic of the proposed framework is that probabilities of
various events, for instance, $\probability{\g \geq \g_\minimum}$, can readily
be estimated. This is important since, in reality, the true values are
unknown---otherwise, we would not need to infer them---and, therefore, we can
rely on our decisions only up to a certain probability. We can then reformulate
the decision rule given earlier as follows: \one~to keep the die if
$\probability{\g \geq \g_\minimum}$ is larger than a certain threshold and
\two~to recycle it otherwise. An illustration of following this rule is given on
the right-hand side of \fref{inference-application} where $\g_\minimum$ is set
to two standard deviations below the mean value of \g; the probability threshold
of the first action is set to 0.9; the crosses mark both the true and inferred
defective dies (they coincide); and the gradient from white to orange
corresponds to the inferred probability of defect. It can be seen that the
inference accurately detects the faulty dies.

In addition, we can introduce a trade-off action between action \one and action
\two as follows: \three~to expose the die to a thorough inspection (for
instance, via a test structure) if $\probability{\g \geq \g_\minimum}$ is
smaller than the threshold of action \one but larger than another threshold. For
instance, action \three can be taken if $0.1 < \probability{\g \geq \g_\minimum}
< 0.9$. In this case, we can reduce costs by examining only those dies for which
there is no sufficiently strong evidence of their satisfactory and
unsatisfactory condition. Furthermore, one can take into consideration a
so-called utility function, which, for each combination of an outcome of \g and
a taken action, returns the gain that the decision maker obtains. For example,
such a function can favor a rare omission of malfunctioning dies to a frequent
inspection of correct dies as the latter might involve much higher costs. The
optimal decision is given by the action that maximizes the expected utility with
respect to both the observed data and prior knowledge on \g. Thus, all possible
\g's weighted by their probabilities are taken into account in the final
decision, incorporating also the preferences of the designer via the utility
function.

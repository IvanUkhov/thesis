As discussed in \sref{process-variation}, process variation is an exigent
concern of electronic-system designs since it can lead to deterioration in
efficiency and to faults of various magnitudes. Therefore, process variation
should be adequately analyzed as the foremost step toward an efficient and
robust product.

An important problem in this regard is the characterization of the on-wafer
distribution of a quantity of interest that is deteriorated by process variation
based on measurements. The problem belongs to the class of inverse problems
since the measured data can be seen as an output of the system at hand, and the
desired quantity as an input. Such an inverse problem is addressed here.

Our goal is to estimate on-wafer distributions of arbitrary process parameters
with high accuracy and at low costs. The goal is accomplished by measuring
auxiliary quantities that are more convenient and less expensive to work with
and then employing statistics in order to infer the desired parameters from the
measurements. Specifically, we propose a novel approach to the quantification of
process variation based on indirect, incomplete, and noisy measurements.
Moreover, we develop a solid framework around the proposed idea and perform a
thorough study of various aspects of our technique.

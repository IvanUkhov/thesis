In this section, we assess our workflow for making fine-grained long-range
predictions of resource usage in computer systems, which is described in
\sref{brain-solution}. All the experiments are conducted on a \up{GNU}/Linux
machine equipped with 8 Intel Core i7-3770 3.4~\up{GH}z processors, 24~\up{GB}
of \up{RAM}, and an \up{HDD} of 500~\up{GB}. All the source code, configuration
files, and input data used in the experiments are available online at
\cite{eslab2017b}; see also \cite{eslab2017c}.

\hiddensubsection{Data Pipeline}

Recall that we study the Google cluster-usage traces \cite{reiss2011}, which are
introduced in \sref{brain-data-pipeline}. Without loss of generality, we focus
on the consumption of one particular resource, namely \up{CPU}; thus, $\ng = 1$
in \eref{brain-profile}. In this regard, the data set provides two apposite
pieces of information: the average and maximum \up{CPU} usage over five-minute
intervals; we extract the former.

The grouping and indexing steps, which are described in
\sref{brain-data-pipeline} and depicted in \fref{brain-overview}, take around 57
hours (no parallelism). Since they have to be performed only once, their
computational cost can be considered negligible. Regarding the selection step,
we filter those resource-usage profiles that contain 5--50 data points; hence,
$\nsi{i} \in [5, 50]$ in \eref{brain-profile}. Such profiles constitute about
74\% of the total number of profiles (about 18 out of 24 million). We experiment
with a random subset of two million profiles, which is roughly 11\% of the 5--50
resource-usage profiles; thus, $\no = 2 \times 10^6$ in \eref{brain-dataset}.
The data sets $G_\training$, $G_\validation$, and $G_\testing$ constitute 70\%,
10\%, and 20\% of $G$, respectively. The process of fetching this number of
profiles and storing them on disk takes around four hours. Recall that this
operation has to be repeated only when the selection criteria change, which
happens rarely.

\hiddensubsection{Learning Pipeline}

The training step (see the leftmost blue box in \fref{brain-overview}) is
configured as follows. Ten buckets are used according to the following
partition:
\[
  5 < 6 < 7 < 8 < 9 < 10 < 15 < 20 < 30 < 40 \leq 50.
\]
For instance, the first and last buckets collect profiles with $\nsi{i} = 5$ and
$\nsi{i} \in [40, 50]$, respectively. The batch size \nb is set to 64. The
optimization algorithm employed for minimizing the loss function is Adam
\cite{kingma2014}, which is an adaptive technique. The algorithm is applied with
its default settings.

Let us now discuss the validation step, which corresponds to the middle blue box
in \fref{brain-overview}. The hyperparameters being considered are the number of
recurrent layers \nl (the blue boxes in \fref{brain-predictive-model}), number
of units per layer \nu (the double circles in \fref{brain-predictive-model}),
and probability of dropout $\rho_\drop$; these hyperparameters are introduced in
\sref{brain-predictive-model}. Specifically, we let \nl be in $\set{1, 2, 3, 4,
5}$, \nu in $\set{100, 200, 400, 800, 1600}$, and $\rho_\drop$ in $\set{0, 0.25,
0.5}$, which yields 75 different combinations in total. The candidate solutions
are explored by means of the Hyperband algorithm with its default settings; this
algorithm is introduced in \sref{brain-learning-pipeline}. The maximum budget
granted to one configuration is set to four training epochs, which correspond to
$4 \times 0.7 \times 2 \times 10^6 = 5.6 \times 10^6$ resource-usage profiles or
$5.6 \times 10^6 \div 64 = 87{,}500$ training steps.

The aforementioned exploration, which encompasses both training and validation,
takes roughly 15 days to finish. During this process, we run up to four learning
sessions in parallel, which typically keeps all eight cores busy. It should be
noted that, since the training, validation, and testing data sets have been
cached on disk as a result of the data pipeline described in
\sref{brain-data-pipeline}, individual learning sessions do not have any
overhead in this regard. It is also worth noting that the machine utilized in
these experiments has no modern \up{GPU}s; therefore, there is a great deal of
room for performance improvement.

\inputtable{brain-validation}
The results of the validation step are reported in \tref{brain-validation}. The
table shows the \acp{MSE} of the top 10 and bottom 10 configurations of the
hyperparameters as measured using $G_\validation$ ($0.1 \times 2 \times 10^6 = 2
\times 10^5$ profiles). The total number of distinct cases sampled by Hyperband
is 62. It can be seen that the error changes rapidly at the bottom of the table
and slows down toward the top. Relative to the least accurate trained model
(rank~62), the error drops by around 22\% in the bottom 10 and by around 3\% in
the top 10 configurations. The overall improvement in accuracy with respect to
the bottommost configuration is about 36\%, which indicates that the validation
step is highly beneficial.

The best trained predictive model (rank~1) is found to have the following
hyperparameters: $\nl = 3$, $\nu = 1600$, and $\rho_\drop = 0$. In general,
larger architectures tend to outperform smaller ones, which is expected.
However, there are complex configurations at the bottom of
\tref{brain-validation} as well, which could be due to the dropout mechanism
engaged in those cases. To elaborate, in these experiments, the impact of
dropouts is found to be mostly neutral or negative; compare, for instance, the
candidates of ranks~2 and 60. This could be due to the amount of training data
being sufficient for regularizing the model.

\tref{brain-validation} also shows an estimate of the memory required by each
configuration, which includes the trainable parameters and internal state of the
model. It can be seen that, if memory usage is a concern, one could trade a
small decrease in accuracy for a considerable saving in memory. For example, the
fourth best solution requires around 85\% less memory than the first one.

After the validation step, the best trained predictive model is taken to the
testing step; see \fref{brain-overview}. The solution is assessed extensively by
predicting resource usage for individual tasks multiple steps ahead at each time
step of the profiles in $G_\testing$ ($0.2 \times 2 \times 10^6 = 4 \times 10^5$
profiles). In these experiments, we predict four time steps into the future;
therefore, $h = 4$ in \sref{brain-problem}. This elaborate and mostly sequential
procedure takes approximately 18 hours.

In order to attain a better assessment of the accuracy of the chosen trained
model, we employ an alternative solution referred to as the reference solution.
This alternative solution is based on random walk. It postulates that the best
prediction of what will happen tomorrow is what happens today plus an optional
random offset, which we set to zero. In other words, the next value for the
resource usage of each analyzed task is estimated to be the current one, which
subsequently results in four identical predictions at each time step.

\inputfigure{brain-testing}
The results of the testing step can be seen in \fref{brain-testing}, which shows
the \acp{MSE} of our solution (the blue line) and the reference one (the orange
line) with respect to $G_\testing$. The magnitude of the errors of our
predictive model suggests that the amount of regularity present in the data is
not sufficient for making highly accurate resource-usage predictions.
Nevertheless, one can observe in \fref{brain-testing} that, relative to the
reference solution, the trained model provides an error reduction of
approximately 47\% at each of the four time steps. This observation indicates
that some structure does exist, and that this structure can be identified and
utilized in order to make educated predictions.

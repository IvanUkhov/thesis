Consider a system of computers that is serving a stream of tasks that are
distributed across the individual machines of the system by a resource manager.
The tasks consume certain resources, such as \up{CPU} and memory, during their
execution. Let task~$i$ be characterized by a resource-usage profile defined as
a sequence of \ng-dimensional measurements taken with a certain sampling
interval and captured by the following $\ng \times \nsi{i}$ matrix:
\begin{equation} \elab{brain-profile}
  \mg_i = (\vg_{ij})_{j = 1}^\nsi{i}
\end{equation}
where $\vg_{ij} \in \real^\ng$ is the measurement taken at time step~$j$, and
\nsi{i} denotes the length of the sequence. Such information is called fine
grained, as it contains multiple measurements over the execution of the task as
opposed to having only one aggregate measurement, such as the average or maximum
value.

Suppose now that the current time step with respect to task~$i$ is $j$; an
illustration for $j \in \set{3, 4, 5}$ is given in \fref{brain-application}.
This means that $\range{\vg_{i1}}{\vg_{ij}}$ are known. Given these previous
values, the objective is to estimate the next $h$ values of the sequence, which
we denote by $\range{\hat{\vg}_{i, j + 1}}{\hat{\vg}_{i, j + h}}$; in
\fref{brain-application}, $h = 4$. Such prediction is called long range, since
it provides multiple future values as opposed to providing only one. The
operation should be performed with respect to each active task of interest at
each moment of interest.

In order to tackle this problem, one is supposed to learn from historical data,
that is, from a data set of past profiles. Denote this data set by
\begin{equation} \elab{brain-dataset}
  G = \set{\mg_i}{i = \range{1}{\no}}
\end{equation}
where \no is the total number of profiles, and $\mg_i$ is as in
\eref{brain-profile}. Such data can be collected straightforwardly provided that
the system at hand has an adequate monitoring facility deployed, which is
commonplace in practice.

It should be understood that, in order for learning to be possible, the
available resource-usage profiles have to have a certain structure that could be
extracted and utilized for meaningful prediction. Therefore, an important
question in this regard is whether real-life profiles of this kind exhibit such
a structure at all. Investigating this question is part of our objective in this
chapter.

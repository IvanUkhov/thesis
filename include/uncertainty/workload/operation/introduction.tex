Resource management is of great importance, since it is the activity that, if
done well, allows one to exploit the full potential of the computer system under
consideration. However, in many cases, there is very little control over the
operating environment. In particular, the actual runtime workload of a
general-purpose system is rarely, if ever, known in advance. In such cases,
resource management inevitably has to contend with workload uncertainty.

Workload uncertainty can be mitigated at runtime by predicting the future and
acting accordingly. This is the general principle that proactive resource
managers thrive on. However, accurate and useful prediction is not
straightforward: modern computer systems are reasonably complex, and their
resource management reasonably requires elaborate forecasting mechanisms.

Prediction traditionally falls in the scope of machine learning
\cite{hastie2013}. The field has recently received a great deal of attention due
to the renaissance in neural networks \cite{goodfellow2016}. This family of
techniques constitutes a highly promising assistant for resource management;
however, although resource management has already seen a number of applications
of neural networks (to be discussed in \sref{brain-past}), the research in this
area is limited. In particular, only primitive architectures of these networks
have been considered, and they have been applied to relatively simple problems.
This state of affairs is unfortunate given that neural networks have been all
but revolutionary in other disciplines. Therefore, we feel strongly that more
research should be conducted in order to investigate the potential aid that the
recent advancements in machine learning could provide to the design of resource
managers for computer systems.

\inputfigure{brain-application}
In this chapter, we conduct one such body of research. More specifically, we
study the usage of resources in a large system of computers and aim to predict
this usage multiple steps ahead at the level of individual tasks that are
executed by the machines. To this end, we use recurrent neural networks
\cite{goodfellow2016}.

In order to give a better sense of the scenario being considered,
\fref{brain-application} illustrates how the aforementioned prediction is
supposed to work in practice. In this example, the \up{CPU} usage of a single
task running on a single machine is depicted three times; see the solid lines.
The three cases correspond to three different moments in time as viewed by the
resource manager of the system; see the black circles. The solid orange lines
represent the history of the usage, which is known to the manager, while the
solid blue lines represent the future usage, which is unknown to the manager.
Our objective is then to predict this future usage for each task several steps
ahead. In \fref{brain-application}, our potential predictions up to four steps
ahead are depicted as a set of dashed blue lines.

Information about future resource usage with the level of detail (individual
tasks) and foresight (multiple steps ahead) that \fref{brain-application}
depicts could be of great help to the resource manager of the system under
consideration. In particular, the manager can more intelligently decide which
machine the next incoming task should be delegated to provided that it is able
to foresee how the currently active tasks will utilize the system's resources in
the future.

In this chapter, we have developed a framework for system-level analysis of
electronic systems whose runtime behaviors depend on uncertain parameters. The
proposed approach thrives on hierarchical interpolation guided by an advanced
adaptation strategy, which makes our framework general and suitable for studying
various quantities that are of interest to the designer. Examples include the
end-to-end delay, total energy consumption, and maximum temperature of the
system. The hybrid adaptivity featured by the framework makes it particularly
suited for problems with idiosyncratic behaviors and steep response surfaces,
which often arise in electronic systems due to their nature.

Provided a means of evaluating the quantity of interest for a given outcome of
the uncertain parameters and a description of the probability distribution of
the parameters, the proposed framework prescribes the steps that need to be
taken in order to computationally efficiently characterize the quantity from the
probabilistic perspective. Concretely, the framework delivers a light
representation that allows for a straightforward calculation of the probability
distribution of the quantity of interest and other statistics about this
quantity.

The performance of our technique has been evaluated by addressing a number of
problems that often appear in electronic-system design. The results delivered by
our approach have been compared with the ones produced by an advanced sampling
technique with a large number of samples. The comparison has shown that, for a
fixed budget of evaluations of the quantity of interest, the framework achieves
higher accuracy compared to direct sampling. Our technique has also been applied
to a real-life problem, which has confirmed that the deployment of the framework
in a real-life context is straightforward.

Finally, note that, even though the proposed framework has been exemplified by
considering random execution times and three specific quantities of interest, it
is general and can be applied in many other settings. Additionally, the approach
to reliability analysis presented in \sref{chaos-reliability-analysis} can
benefit from the development given in this chapter by constructing surrogate
representations via adaptive hierarchical interpolation instead of the \ac{PC}
decomposition and thereby making it possible to work with nonsmooth response
surfaces.

At Stage~2 of the framework, we change the parameterization of the problem from
the random vector $\vu: \Omega \to \real^\nu$ to an auxiliary random vector
$\vz: \Omega \to \real^\nz$ such that the support of the \ac{PDF} of \vz is the
unit hypercube $[0, 1]^\nz$, and that \nz has the smallest value required to
retain the desired level of accuracy. The first task is standardization, which
is done primarily for convenience. The second one is model order reduction,
which identifies and eliminates excessive complexity and hence speeds up the
subsequent solution process. The overall transformation is denoted by
\begin{equation} \elab{frame-transformation}
  \vu = \transform{\vz}
\end{equation}
where $\transform: [0, 1]^\nz \to \real^\nu$. The quantity of interest \g can
now be computed as
\[
  \g(\vu) = (\g \circ \transform)(\vz) = \g(\transform(\vz)).
\]

The attentive reader might already have a suitable candidate for $\transform$:
it is the one described in \xref{probability-transformation}, consolidated in
\eref{probability-transformation}, and utilized throughout
\cref{uncertainty-process-development} starting from
\sref{chaos-transformation}. The way this transformation is applied in this
chapter is to be discussed further in \sref{frame-application}.

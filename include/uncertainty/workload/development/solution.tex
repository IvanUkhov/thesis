The general solution strategy here is similar to the one outlined in
\sref{chaos-solution}. Recall first that making use of a sampling method is a
compelling approach to uncertainty quantification. We would readily apply such a
method to study the quantity of interest \g if evaluating \g had only a small
cost, which, most of the time, it does not. Our solution to this quandary is to
construct a lightweight representation of the heavy \g and study this
representation instead of \g.

The surrogates that we build in this chapter are based on hierarchical
interpolation with hybrid adaptivity, which is developed in \cite{klimke2006,
ma2009, jakeman2012}. In this case, \g is evaluated at a number of strategically
chosen collocation nodes, and any other values of \g are reconstructed on
demand, without involving \g, using a set of basis functions that mediate
between the collected values of \g. The benefit of this approach is in the
number of invocations of the quantity \g: only a few evaluations of \g are
needed, and the rest of the analysis is powered by the constructed interpolant,
which, in contrast to \g, has a negligible cost.

The proposed framework is efficient at characterizing the impact of workload
uncertainty and is straightforward to use in practice. The effectiveness of our
approach is due to the aforementioned powerful approximation engine, which
enables tackling diverse design problems while keeping the associated costs low.
The usage of our approach is streamlined, since it has the same low entrance
requirements as sampling techniques, which is also the case with our framework
in \cref{uncertainty-process-development}: one only has to be able to evaluate
the quantity of interest given a particular outcome of the uncertain parameters.
Moreover, it can be utilized in scenarios with limited knowledge of the joint
probability distribution of the uncertain parameters, which are common in
practice.

The solution process has four stages, which reflect the ones depicted in
\fref{chaos-overview}. At Stage~1, the quantity of interest \g and the uncertain
parameters \vu are decided upon by the designer. At Stage~2, \g is
reparameterized in terms of an auxiliary random vector \vz extracted from \vu,
which is described in \sref{frame-transformation}. At Stage~3, an interpolant of
\g is constructed by considering \g as a deterministic function of \vz and
evaluating \g at a small set of carefully chosen points, which is detailed in
\sref{frame-construction}. At Stage~4, the constructed interpolant of \g is
post-processed in order to calculate the desired statistics about \g; in
particular, the probability distribution of \g is estimated by applying an
arbitrary sampling method to the interpolant, which is discussed in
\sref{frame-processing}.

The first stage of the framework is problem specific, and it will be exemplified
in \sref{frame-application}. In the following, we proceed directly to the second
stage, which together with the third one should be approached with great care,
since interpolation of multivariate functions is a challenging undertaking.

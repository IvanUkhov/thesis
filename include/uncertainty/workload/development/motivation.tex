Due to its nature, the variability originating from process variation is
typically smooth and well behaved. In such cases, uncertainty quantification
based on \ac{PC} expansions \cite{xiu2010} and other approximation techniques
relying on global polynomials generally work well, which is demonstrated in
\cref{uncertainty-process-development}. On the other hand, the variability
coming from such sources as workload often has steep gradients and favors
nondifferentiability and even discontinuity. In such cases, \ac{PC} expansions
and similar techniques fail: they require extremely many evaluations of the
quantity of interest in order to deliver an acceptable level of accuracy and are
consequently not worth it.

\inputfigure{frame-motivation}
In order to illustrate this concern, let us consider an example. Suppose that
our system has only one processing element, and it is running an application
with only one task. Suppose further that the task has two branches and takes
either one depending on the input data. Assume that one branch takes 0.1~s to
execute and has probability 0.6, and the other branch takes 1~s and has
probability 0.4. Our goal is to find the distribution of the end-to-end delay of
the application. In this example, the quantity of interest is the end-to-end
delay, and it coincides with the execution time of the task; hence, we already
know the answer. Let us pretend we do not and try to obtain it by other means.

Suppose the above scenario is modeled by a uniformly distributed random variable
$\u: \Omega \to [0, 1]$. The execution time of the task (the end-to-end delay of
the application) is 0.1~s if $\u \in [0, 0.6]$, and it is 1~s if $\u \in (0.6,
1]$. The response in this case is a step function, which is illustrated by the
orange line in \fref{frame-motivation}.

First, we try to quantify the end-to-end delay by constructing and subsequently
sampling a \ac{PC} expansion founded on the Legendre polynomial basis
\cite{xiu2010}; see \xref{polynomial-chaos}. The green line in
\fref{frame-motivation} shows a ninth-level \ac{PC} expansion ($\lc = \lq = 9$),
which uses 10 points ($\nq = 10$). It can be seen that the approximation is
poor---not to mention negative execution times---which means that the follow-up
sampling will also yield a poor approximation of the true distribution. The
observed oscillating behavior is the well-known Gibbs phenomenon stemming from
the discontinuity of the response. Regardless of the number of points spent, the
oscillations will never go away completely.

Let us now see how the framework developed in this chapter solves the same
problem. For the purpose of this experiment, our technique is constrained to
make use of the same number of points as the \ac{PC} expansion does. The result
is the blue curve in \fref{frame-motivation}, and the adaptively chosen points
are plotted on the horizontal axis. It can be seen that the approximation is
good, and, in fact, it would be indistinguishable from the true response with a
few additional points. One can note that the adaptive procedure started to
concentrate collocation nodes at the jump and left the insipid regions on both
sides of the jump with no particular attention. Having constructed such a
representation, one can proceed to the calculation of the probability
distribution of the quantity of interest, which, in general, is done via
sampling followed by such techniques as kernel density estimation
\cite{hastie2013}. The crucial point to note is that, analogously to the \ac{PC}
expansions leveraged in \cref{uncertainty-process-development}, this follow-up
sampling does not involve the original system in any way, which implies that the
operation costs practically nothing in terms of computation time.

The example described above illustrates the fact that the proposed framework is
well suited for nonsmooth response surfaces. More generally, the adaptivity
featured by our technique allows for a reduction in the costs associated with
probabilistic analysis of the quantity under consideration as measured by the
number of times the quantity needs to be evaluated in order to achieve a certain
accuracy level. The magnitude of reduction depends on the problem, and it can be
substantial when the problem is well disposed to adaptation.

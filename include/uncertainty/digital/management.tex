In the previous chapters, we have been primarily concerned with making
uncertainty-aware decisions at design- and fabrication-time, that is, during
various development and manufacturing stages of a computer system. In this
chapter, we elaborate on making such decisions at runtime and, more
specifically, on resource management in the present of workload uncertainty.

\section{\introductiontitle}
\slab{network-introduction}
\inputsection{introduction}

\section{\problemtitle}
\slab{network-problem}
\inputsection{problem}

\section{\priortitle}
\slab{network-prior}
\inputsection{prior}

\section{\solutiontitle}
\slab{network-solution}

In order to attain the objective established in \sref{network-problem}, an
adequate model of the available data $G$ is constructed offline by means of
machine learning, and this model is then used in order to make predictions at
runtime, assisting the resource manager in accounting for workload uncertainty.

\inputfigure{network-overview}
Let us now adumbrate our workflow, which is illustrated in
\fref{network-overview}. Given raw resource-usage profiles, we first process
them in order to make the data suitable for the subsequent computations. This
part can be seen at the top of \fref{network-overview} and is explained in the
next section, \sref{network-processing}. The processed profiles are then used in
order to obtain an adequately configured predictive model. The modeling part is
covered in \sref{network-model} while the learning one in
\sref{network-learning}; the latter can also be seen at the bottom of
\fref{network-overview}. The above operations are to be undertaken offline while
the obtained model is supposed to be used by the resource manager of the
computer system at hand at runtime.

Before we proceed the model construction, we develop a data-processing
pipeline for working with large data sets that makes the data readily accessible
for machine learning. The pipeline is applied to a data set of real
resource-usage traces collected in a large computer cluster. Second, we
construct a model for making fine-grained long-range prediction of the system's
resource usage that relies on the state-of-the-art recurrent neural networks
\cite{goodfellow2016}.

\subsection{Data Processing}
\slab{network-processing}

Before we describe our data-processing pipeline, let us first give a brief
introduction to the considered data set: the Google cluster-usage traces
\cite{reiss2011}. The traces were collected over 29 days in May 2011 and
encompass more than 12 thousand machines serving requests from more than 900
users.

In the data set's parlance, a user request is a job; a job comprises one or
several tasks; and a task is a Linux program to be run on a single machine.
Each job is ascribed a unique \up{ID}, and each task is given an \up{ID} that is
unique in the scope of the corresponding job. Apart from other tables, the data
set contains the so-called resource-usage table. The table records the resource
usage of the executed tasks with the granularity of five minutes. Each record
corresponds to a specific task and a specific five-minute interval, and it
provides such measurements as the average and maximum values of the \up{CPU},
memory, and disk usage. There are more than 1.2 billion records, which
correspond to more than 24 million tasks or, equivalently, resource-usage traces
and to more than 670 thousand jobs.

The resource-usage table is provided in the form of 500 archives. Each archive
contains a single file with measurements over a certain time window. Such a
format is inconvenient and inefficient to work with, which is what we address in
this section. Consider now the three leftmost boxes in \fref{workflow}.

At the first stage, the data from the 500 archives are distributed into separate
databases so that each such database contains all the data points that belong to
a particular job, resulting in as many databases as there are jobs. In order to
reduce the space requirements, only the used columns of the table are preserved.
In our experiments, these columns are the start time stamp, job \up{ID}, task
\up{ID}, and average \up{CPU} usage.

At the second stage, an index of the traces is created in order to be able to
efficiently navigate the catalog of the databases created at the previous stage.
Each record in the index contains metadata about a single task, the most
important of which are the task ID and the path to the corresponding database.
We also include the length of the trace in question into the index in order to
be able to efficiently filter the traces by their lengths.

At the last stage of our pipeline, a subset of the resource-usage traces is
selected using the index according to the needs of a particular training session
(to be discussed in \sref{result}) and then stored on disk. Concretely, the
traces are fetched from the databases and stored in the native format of the
machine-learning library utilized; we shall refer to a file in such a format as
a binary file. Instead of writing all the selected traces into one binary file,
we distribute them across several files. Such a catalog of binary files is
created for each of the three commonly considered parts \cite{hastie2013} of the
data at hand: one is for training, one for validation, and one for testing; see
also \fref{workflow}. We shall refer to these parts as $X_1$, $X_2$, and $X_3$,
respectively.

Lastly, it is common practice to standardize data before feeding them into a
machine-learning algorithm \cite{hastie2013}. In our case, it is done along with
creating the aforementioned three catalogs, which requires a separate pass over
the training set.

In conclusion, the benefit of the above pipeline is in the streamlined access to
the data during one or multiple training sessions. The binary files can be read
efficiently as many times as needed, and they can be straightforwardly
regenerated whenever the selection criteria change; note that the artifacts of
the procedures in \sref{grouping} and \sref{indexing} stay the same. The
presence of multiple binary files allows also for shuffling the training data at
the beginning of each training epoch.

Now we are ready to make use of the processed data.

\subsection{Modeling}

As mentioned in \sref{introduction} and \sref{problem}, a part of our goal is to
assess the applicability of the latest advancements in neural networks
\cite{goodfellow2016} to modeling and prediction of fine-grained resource-usage
data. The architectures of neural networks are very diverse: one network can be
nothing like another. Since the data that we study are inherently sequential, it
is natural to found our model on the basis of recurrent neural networks
\cite{goodfellow2016}, which are specifically designed for such cases.

A schematic representation of our model can be seen in \fref{model}; many of the
actual connections between the model's parts are simplified or not shown at all
in order to make the figure legible. In the following subsections, we shall
describe each part in detail. A number of important operational aspects of the
model will be covered in the next section, \sref{learning}.

The input to the model is a single $d$-dimensional data point, which can be seen
on the left-hand side of \fref{model}. Similarly, the output is a single
$d$-dimensional data point, which is depicted on the right-hand side of
\fref{model}. The input $x_{ik}$ is the value of the resource usage of an
individual task at step $k$, and the output $\hat{x}_{i,k + 1}$ is a
one-step-ahead prediction of the usage.

The core of the model is formed by a number of recurrent layers, which are also
called cells. They are shown as a group of blue boxes in \fref{model}. The
network can be made as many layers deep as needed. Each cell is composed of a
number of units, which are depicted by double circles in \fref{model}. We let
$c$ and $u$ be the number of cells and units per cell, respectively.

A unit is the smallest processing elements. The key characteristics of a unit
are that it has internal memory, and that it has access to its previous output,
which makes it recurrent. There are different types of units; each one defines
how a unit lets data flow through it and updates its memory. One notable type is
called \up{LSTM} \cite{hochreiter1997}, which stands for \emph{long short-term
memory}. It has been designed to overcome the problems of traditional recurrent
networks---such as vanishing gradient when training---and it is now one of the
most widely used types. The recurrent layers of our model are \up{LSTM} cells.

In addition, during training, each cell is enhanced by a dropout mechanism
\cite{zaremba2014}, which gives control over the regularization of the model and
is to prevent potential overfitting \cite{hastie2013}. We let $p$ be the
probability of dropping an output of a cell.

The output of the last cell is typically a large tensor, which is proportional
in size to the number of units in the cell. Each entry of such a tensor can be
considered as a feature that the network has extracted and activated in
accordance with the trace that is currently being fed into the model. The task
now is to combine these features in order to produce a single prediction. To
this end, we mount a regression layer on top of the last cell, which is depicted
by a red box in \fref{model}. Unlike the recurrent layers in \sref{recurrent},
which feature highly nonlinear transformations, this layer performs an affine
transformation.

To summarize, we have described a predictive model that is composed of a number
of \up{LSTM} cells and a linear regression layer. Due to its internal memory,
the model is capable of efficiently taking into account the entire past of the
resource-usage trace under consideration when making predictions. Let us now
discuss how the model is supposed to be used.

\subsection{Learning}

As mentioned in \sref{introduction} and \sref{problem}, a part of our goal is to
assess the applicability of the latest advancements in neural networks
\cite{goodfellow2016} to modeling and prediction of fine-grained resource-usage
data. The architectures of neural networks are very diverse: one network can be
nothing like another. Since the data that we study are inherently sequential, it
is natural to found our model on the basis of recurrent neural networks
\cite{goodfellow2016}, which are specifically designed for such cases.

A schematic representation of our model can be seen in \fref{model}; many of the
actual connections between the model's parts are simplified or not shown at all
in order to make the figure legible. In the following subsections, we shall
describe each part in detail. A number of important operational aspects of the
model will be covered in the next section, \sref{learning}.

The input to the model is a single $d$-dimensional data point, which can be seen
on the left-hand side of \fref{model}. Similarly, the output is a single
$d$-dimensional data point, which is depicted on the right-hand side of
\fref{model}. The input $x_{ik}$ is the value of the resource usage of an
individual task at step $k$, and the output $\hat{x}_{i,k + 1}$ is a
one-step-ahead prediction of the usage.

The core of the model is formed by a number of recurrent layers, which are also
called cells. They are shown as a group of blue boxes in \fref{model}. The
network can be made as many layers deep as needed. Each cell is composed of a
number of units, which are depicted by double circles in \fref{model}. We let
$c$ and $u$ be the number of cells and units per cell, respectively.

A unit is the smallest processing elements. The key characteristics of a unit
are that it has internal memory, and that it has access to its previous output,
which makes it recurrent. There are different types of units; each one defines
how a unit lets data flow through it and updates its memory. One notable type is
called \up{LSTM} \cite{hochreiter1997}, which stands for \emph{long short-term
memory}. It has been designed to overcome the problems of traditional recurrent
networks---such as vanishing gradient when training---and it is now one of the
most widely used types. The recurrent layers of our model are \up{LSTM} cells.

In addition, during training, each cell is enhanced by a dropout mechanism
\cite{zaremba2014}, which gives control over the regularization of the model and
is to prevent potential overfitting \cite{hastie2013}. We let $p$ be the
probability of dropping an output of a cell.

The output of the last cell is typically a large tensor, which is proportional
in size to the number of units in the cell. Each entry of such a tensor can be
considered as a feature that the network has extracted and activated in
accordance with the trace that is currently being fed into the model. The task
now is to combine these features in order to produce a single prediction. To
this end, we mount a regression layer on top of the last cell, which is depicted
by a red box in \fref{model}. Unlike the recurrent layers in \sref{recurrent},
which feature highly nonlinear transformations, this layer performs an affine
transformation.

To summarize, we have described a predictive model that is composed of a number
of \up{LSTM} cells and a linear regression layer. Due to its internal memory,
the model is capable of efficiently taking into account the entire past of the
resource-usage trace under consideration when making predictions. Let us now
discuss how the model is supposed to be used.

\section{\resultstitle}
\slab{network-results}

The infrastructure developed for the experiments presented below is open source
and available online at \cite{eslab2017b}. The implementation is based on
TensorFlow \cite{abadi2015}. The experiments are conducted on a \up{GNU}/Linux
machine equipped with 8 \up{CPU}s Intel Core i7-3770 3.4~\up{GH}z, 16~\up{GB} of
\up{RAM}, and an \up{HDD} of 500~\up{GB}. The machine has no modern \up{GPU}s;
therefore, the reported results have an immense room for improvement.

\hiddensubsection{Data Processing}

Recall that the considered data set is the Google cluster-usage traces
\cite{reiss2011} described in \sref{data}. In the experiments, we focus on one
particular resource ($d = 1$), which is the \up{CPU} usage of the tasks executed
in the cluster. The resource-usage table contains two apposite columns: the
average and maximal \up{CPU} usage over five-minute intervals; we extract the
former.

The grouping and indexing steps of the data-processing pipeline described in
\sref{grouping} and \sref{indexing}, respectively, and depicted in
\fref{workflow} take approximately 57 hours to finish (no parallelism). Since
they have to be done only once, their computational cost can be safely
considered negligible.

Regarding the selection stage delineated in \sref{selection}, we filter those
resource-usage traces that contain 5--50 data points; consequently, $l_i \in [5,
50]$ in \eref{trace}. Such traces constitute around 74\% of the total number of
traces (around 18 out of 24 million). We experiment with a random subset of two
million traces, which is around 11\% of the 5--50 resource-usage traces; hence,
$n = 2 \times 10^6$ in \eref{traces}. The data sets $X_1$, $X_2$, and $X_3$
constitute 70\%, 10\%, and 20\% of $X$, respectively. Fetching and storing on
disk these many traces take approximately four hours. Recall that this operation
has to be repeated only when the selection criteria change, which happens rarely
in practice.

\hiddensubsection{Learning}

The training stage (see \sref{training} and recall \fref{workflow}) is
configured as follows. Ten buckets or, equivalently, queues are used according
to the following rule: $l < 6 < 7 < 8 < 9 < 10 < 15 < 20 < 30 < 40 \leq 50$. The
batch size $b$ is set to 64. The optimization algorithm that is employed for
minimizing the loss function is Adam \cite{kingma2014}, which is an adaptive
technique. The algorithm is applied with its default settings.

Regarding the validation stage, the considered hyperparameters are the number of
cells $c$ (the blue boxes in \fref{model}), number of units per cell $u$ (the
double circles in \fref{model}), and probability of dropout $p$ as discussed in
\sref{validation}. More concretely, we let $c \in \{1, 2, 3, 4, 5\}$, $u \in
\{100, 200, 400, 800, 1600\}$, and $p \in \{0, 0.25, 0.5\}$, which yields 75
different combinations in total. The candidates are explored by means of the
Hyperband algorithm introduced in \sref{validation} with its default settings.
The maximum budget for one configuration is set to four training epochs, which
correspond to $4 \times 0.7 \times 2 \times 10^6 = 5.6 \times 10^6$
resource-usage traces or $5.6 \times 10^6 \div 64 = 87\,500$ training steps.

The above exploration, which encompasses both the training and validation
stages, takes approximately 15 days to finish. During this process, we run up to
four training sessions in parallel, which typically keeps all the eight
\up{CPU}s busy. It should be noted that, since the training, validation, and
testing data sets have been cached on disk as a result of our data-processing
pipeline described in \sref{data}, individual training sessions do not have any
overhead in this regard.

The results of the validation stage are given in \tref{validation}, which shows
the mean squared error (\up{MSE}) of the top 10 configurations of the
hyperparameters as measured using $X_2$ ($0.1 \times 2 \times 10^6 = 2 \times
10^5$ traces). The best trained model is found to have the following
hyperparameters: $c = 3$, $u = 1600$, and $p = 0$. In general, deeper and wider
architectures tend to outperform shallower and narrower ones (the depth and
breadth are measured by $c$ and $u$, respectively), which is expected. In these
experiments, the dropout mechanism does not have much impact on the resulting
accuracy, which is likely due to the amount of data being enough for
regularizing the model.

\tref{validation} also shows an estimate of the memory required by each
configuration, including the model's trainable parameters and internal state. It
can be seen that, if the memory usage is a concern, one could trade a small
decrease in accuracy for a considerable memory saving. For example, the fourth
best configuration requires 75\% less memory than the first one.

After the exploration stage, the best trained model is taken to the testing
stage (see \sref{testing}), which is undertaken using $X_3$ ($0.2 \times 2
\times 10^6 = 4 \times 10^5$ traces). At this stage, the model is extensively
assessed by predicting the resource usage of individual tasks multiple time
steps ahead at each step of the testing traces in $X_3$. In these experiments,
we predict four steps into the future ($h = 4$ in \sref{problem}). This
elaborate sequential testing procedure takes around 18 hours from start to
finish.

In order to assess better the accuracy of our model, we employ also an
alternative one, which we shall refer to as the reference model. The reference
model is based on random walk. It postulates that the best prediction of what
will happen tomorrow is what happens today plus an optional random offset, which
we set to zero. In other words, the next value of a resource-usage trace is
estimated to be the current one, which results in four identical predictions at
each time step.

The results of the testing stage can be seen in \fref{testing}, which shows the
\up{MSE} of our model (the blue line) as well as the one of the reference model
(the red line) with respect to $X_3$. The magnitude of our model's errors
suggests that the amount of regularity present in the data is not sufficient to
make the resource-usage predictions highly accurate. Nevertheless, it can be
seen in \fref{testing} that, relative to the reference model, our model provides
an error reduction of approximately 47\% at each of the four future time
moments. This observation indicates that a certain structure does exist, and
that it can be identified and utilized in order to make educated predictions.

\section{\conclusiontitle}
\slab{network-conclusion}

We have presented our experience of working with the state-of-the-art recurrent
neural networks in the context of fine-grained long-range prediction of the
resource usage in a computer cluster. Our workflow---which starts from making
the data readily available for learning and finishes by predicting the resource
usage of individual tasks multiple time steps into the future---has been
described in detail and applied to a large data set of real resource-usage
traces of a computer cluster.

The experimental results suggest that the considered fine-grained traces possess
a certain structure, and that this structure can be extracted by advanced
machine-learning techniques and subsequently utilized for making educated
predictions. This information can be of use to such a crucial component as the
resource manager of the computer cluster in question, allowing the manager to
more intelligently orchestrate the cluster.

In order to facilitate the development of intelligent resource managers of
computer clusters, we investigate the utility of the state-of-the-art neural
networks for the purpose of fine-grained long-range prediction of the resource
usage in one such cluster. We consider a large data set of real-life traces and
describe in detail our workflow, starting from making the data accessible for
learning and finishing by predicting the resource usage of individual tasks
multiple steps ahead. The experimental results indicate that such fine-grained
traces as the ones considered possess a certain structure, and that this
structure can be extracted by advanced machine-learning techniques and
subsequently utilized for making informed predictions.

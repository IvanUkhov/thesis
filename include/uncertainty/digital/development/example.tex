Depending on their origins, the uncertainties present in computer systems can be
classified as analog or digital---in order to build intuition, these two
categories can also be called physical and virtual, respectively. An example of
an analog source of uncertainty is process variation since it originates from
the imperfection of the fabrication process, which the topic of
\cref{uncertainty-analog-fabrication}, \cref{design-uncertainty-analog}, and
many other lines of research. An example of a digital source of uncertainty is
workload. To elaborate, the characteristics of the codes running on modern
computer systems change from one activation to another depending on the
environment and input data. This source of uncertainty has not been deprived of
attention either, especially in the real-time community \cite{quinton2012,
diaz2002, santinelli2011, tanasa2015}. Regardless of the origin, such phenomena
as the ones mentioned above render the behavior of the system at hand
nondeterministic to the designer.

Due to its nature, the variability coming from the analog world is typically
smooth, well behaved. In such cases, uncertainty quantification based on
\acf{PC} expansions \cite{xiu2010} and other approximation techniques making use
of global polynomials generally work well, which is demonstrated in
\cref{design-uncertainty-analog}. On the other hand, the variability coming from
the digital world has often steep gradients and favors nondifferentiability and
even discontinuity. In such cases, \ac{PC} expansions and similar techniques
fail: they require extremely many evaluations of the studied quantity in order
to deliver an acceptable level of accuracy and, consequently, are not worth it.

\inputfigure{interpolant-example}
In order to illustrate this concern, let us consider an example. Suppose that
our system has only one processing element, and it is running an application
with only one task. Suppose further that the task has two branches and takes
either one depending on the input data. Assume that one branch takes 0.1~s to
execute and has probability 0.6, and the other branch takes 1~s and has
probability 0.4. Our goal is to find the distribution of the end-to-end delay of
the application. In this example, the quantity of interest is the end-to-end
delay, and it coincides with the execution time of the task; hence, we already
know the answer. Let us pretend we do not and try to obtain it by other means.

Suppose the above scenario is modeled by a uniformly distributed random variable
$\u: \Omega \to [0, 1]$: the execution time of the task (the end-to-end delay of
the application) is 0.1~s if $\u \in [0, 0.6]$, and it is 1~s if $\u \in (0.6,
1]$. The response in this case is a step function, which is illustrated by the
yellow line in \fref{interpolant-example}.

First, we try to quantify the end-to-end delay by constructing and subsequently
sampling a \ac{PC} expansion founded on the Legendre polynomial basis
\cite{xiu2010}; see \sref{polynomial-chaos}. The orange line in
\fref{interpolant-example} shows a \ac{PC} expansion of level nine ($\lc = \lq =
9$), which uses 10 points ($\nq = 10$). It can be seen that the approximation is
poor---not to mention negative execution times---which means that the follow-up
sampling will also yield a poor approximation of the true distribution. The
observed oscillating behavior is the well-known Gibbs phenomenon stemming from
the discontinuity of the response. Regardless of the number of points spent, the
oscillations will never go away completely.

Let us now investigate how the framework developed in this chapter solves the
same problem. For the purpose of the experiment, our technique is constrained to
make use of the same number of points as the \ac{PC} expansion does. The result
is the blue curve in \fref{interpolant-example}, and the adaptively chosen
points are plotted on the horizontal axis. It can be seen that the approximation
is good, and, in fact, it would be indistinguishable from the true response with
a few additional points. One can note that the adaptive procedure started to
concentrate interpolation points at the jump and left the insipid regions on
both sides of the jump with no particular attention. Having constructed such a
representation, one can proceed to the calculation of the probability
distribution of the quantity of interest, which, in general, is done via
sampling followed by such techniques as kernel density estimation
\cite{hastie2013}. The crucial point to note is that this follow-up sampling
does not involve the original system in any way, which implies that it costs
practically nothing in terms of the computation time.

The example discussed above illustrates the fact that the proposed framework is
well suited for nonsmooth response surfaces. More generally, the adaptivity
featured by our technique allows for a reduction of the costs associated with
probabilistic analysis of the quantity under consideration, as measured by the
number of times the quantity needs to be evaluated in order to achieve a certain
accuracy level. The magnitude of the reduction depends on the problem, and it
can be substantial when the problem is well disposed to adaptation.

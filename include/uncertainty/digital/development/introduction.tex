Similar to \cref{uncertainty-analog-development}, we propose a design-time
system-level framework for the analysis of electronic systems that are dependent
on uncertain parameters. At it is the case with sampling methods introduced in
\sref{prior}, our technique treats the system at hand as a ``black box'' and,
therefore, is straightforward to apply since no handcrafting is required, and
existing codes need no change. Hence, the quantities that the framework is able
to tackle are diverse. Examples include those concerned with timing-, power-,
and temperature-related characteristics of applications running on heterogeneous
platforms.

In contrast to \cref{uncertainty-analog-development} and sampling methods, the
framework presented in this chapter explores and exploits the nature of the
problem---that is, the way the quantity of interest depends on the uncertain
parameters---by exercising the aforementioned ``black box'' at a set of points
chosen adaptively. The adaptivity that we leverage is hybrid \cite{jakeman2012}:
it tries to pick up both global (that is, on the level of individual dimensions
\cite{klimke2006}) and, more importantly, local (that is, on the level on
individual points \cite{ma2009}) variations. This means that the framework is
able to benefit from any particularities that might be present in the stochastic
space, that is, in the space of the uncertain parameters.

The remainder of the chapter is organized as follows. The adaptivity is the
capital feature of our technique, and we motivate and illustrate it in the next
section, \sref{frame-motivation}. The problem formulation is given in
\sref{frame-problem}. In \sref{frame-prior}, the prior work is discussed. Our
solution is outlined in \sref{frame-solution} and then detailed in
\sref{frame-transformation}, \sref{frame-construction}, and
\sref{frame-processing}. In \sref{frame-application}, an illustrative
application is given. The experimental results are reported in
\sref{frame-results}. Finally, \sref{frame-conclusion} concludes the chapter.

The agenda for this section is as follows. First, we exemplify Stage~1 of the
proposed framework by introducing a number of quantities of interest \g, which
illustrate the broad applicability of the framework, thereby giving the reader a
better sense of its utility. Second, we turn to Stage~2 and highlight a
transformation $\transform$ that is the one to use in the majority of cases.
Third, we proceed directly to Stage~4 and illustrate a potential output of the
proposed framework. For concreteness, throughout this section, the framework is
applied to a particular problem involving one of the introduced quantities.

\subsection{\problemtitle}
\slab{frame-application-problem}

Assume the system, power, and temperature models given in \sref{system-model},
\sref{power-model}, \sref{temperature-model}, respectively. Assume also the
application model utilized in \sref{dream-optimization-problem} except for the
requirement about being periodic.

Let us first touch upon the timing aspects of the system in question. Each task
of the application has a start time and a finish time, which are denoted by
$b_i$ and $d_i$, respectively. Let also $\v{b} = (b_i)_{i = 1}^\nt$ and $\v{d} =
(d_i)_{i = 1}^\nt$. Then other timing characteristics can be derived from
$(\v{b}, \v{d})$. For example, the end-to-end delay of the application---which
is the difference between the finish time of the latest task and the start time
of the earliest task---is as follows:
\begin{equation} \elab{frame-delay}
  \text{End-to-end delay}
  = \max_{i = 1}^\nt d_i - \min_{i = 1}^\nt b_i.
\end{equation}
Suppose now that the execution times of the tasks depend on the uncertain
parameters \vu; see \sref{frame-problem}. Then the tuple $(\v{b}, \v{d})$
depends on \vu. Hence, the end-to-end delay given in \eref{frame-delay} does so
too, and it constitutes an example of a quantity \g that the designer might be
interested in studying. Note that this \g is nondifferentiable since the $\max$
and $\min$ functions are such. Therefore, \g is nonsmooth, which renders \up{PC}
expansions and similar techniques inadequate for this problem, which is
elaborated on in \sref{frame-motivation}.

\begin{remark} \rlab{frame-nonsmoothness}
In general, the behavior of \g with respect to continuity, differentiability,
and smoothness cannot be inferred from the behavior of \vu. Even when the
parameters behave perfectly, \g might still exhibit nondifferentiability or even
discontinuity, which depends on how \g functions internally. For example, as
shown in \cite{tanasa2015}, even if execution times of tasks are continuous, due
to the actual scheduling policy, end-to-end delays are very often discontinuous.
\end{remark}

Let us move on to power. The total energy consumed by the system during an
application run can be estimated using a power profile \mp as follows:
\begin{equation} \elab{frame-energy}
  \text{Total energy}
  = \sum_{i = 1}^\np \int \p_i(t) \d t
  \approx \dt \norm[1]{\mp}
\end{equation}
where $\p_i$ stands for the power consumption of processing element $i$, and \dt
is the sampling interval, which is assumed to be sufficiently small. Since the
tuple $(\v{b}, \v{d})$ depends on \vu, the power consumption of the system is
dependent on \vu as well. Consequently, the total energy given in
\eref{frame-energy} depends on \vu and is a candidate for \g.
\rref{frame-nonsmoothness} applies in this context to the full extent.

Let us now turn to temperature. The maximum temperature of the system can be
estimated using a temperature profile \mq as follows:
\begin{equation} \elab{frame-temperature}
  \text{Maximum temperature}
  = \max_{i = 1}^\np \sup_{t} \q_i(t)
  \approx \norm[\infty]{\mq}
\end{equation}
where $\q_i$ stands for the heat dissipation of processing element $i$. Since
the power consumption of the system is affected by \vu, its heat dissipation is
affected by \vu as well. Thus, the maximum temperature in
\eref{frame-temperature} is also a potential quantity of interest \g. Note that,
due to the maximization involved, the quantity is nondifferentiable and hence
cannot be adequately addressed using polynomial approximations; recall also the
concern in \rref{frame-nonsmoothness}.

To summarize, we have covered three aspects of electronic systems, namely
timing, power, and temperature, and introduced a number of quantities that the
designer is typically interested in analyzing. These quantities are to be
discussed further in the section on experimental results, \sref{frame-results}.

\inputfigure{frame-application}
Let us also employ one of the introduced quantities in order to have a concrete
example to work with in this section. The setup that we are about to describe is
to be detailed, extended, and scaled up in the experiments reported in
\sref{frame-results}. The problem addressed in this section is depicted on the
left-hand side of \fref{frame-application}. We consider a heterogeneous platform
with two processing elements, PE1 and PE2, and an application with four tasks,
T1--T4. The data dependencies between the tasks and their mapping onto the
processing elements can be seen in \fref{frame-application}. The quantity of
interest \g is the end-to-end delay of the application defined in
\eref{frame-delay}. The uncertain parameters \vu are the execution times of T2
and T4 denoted by $\u_1$ and $\u_2$, respectively.

The leftmost component in \fref{frame-application} is a ``black box'' capable of
evaluating \g given \vu. It takes an assignment of the execution times $\u_1$
and $\u_2$ and outputs the calculated end-to-end delay \g. In practice, this
evaluation often involves an appropriate system simulator, such as Sniper
\cite{carlson2011}, in which case the modeling capabilities of this simulator
are naturally inherited by our technique.

Targeting the practical scenario described in \sref{chaos-formulation}, the
marginal distributions and correlation matrix of \vu are assumed to be
available. Without loss of generality, each marginal is a four-parameter beta
distribution shown in \eref{beta-distribution}. The execution times are assumed
to be correlated based on the dependencies between them defined by the structure
of the task graph: the closer task $i$ and task $j$ are in the graph as measured
by the number of edges between vertex $i$ and vertex $j$, the stronger $\u_i$
and $\u_j$ are correlated.

\subsection{Probability Transformation}
\slab{frame-application-transformation}

At Stage~2 of our workflow outlined in \sref{frame-solution}, a suitable
$\transform$ needed in \eref{frame-transformation} is to be decided upon. In
this example, we base $\transform$ on the one shown in
\eref{probability-transformation} and reduce $\vu: \Omega \to \real^\nu$ to
$\vz: \Omega \to [0, 1]^\nz$ so that the latter is uniformly distributed. In
this case, the rightmost $\transform_1$ in \eref{probability-transformation} is
simpler than what it is according to its original definition since the marginal
distributions of \vz are already uniform. It should be understood that the
model-order-reduction functionality of \eref{probability-transformation} is
engaged; it eliminates redundant stochastic dimensions and, therefore, assists
the subsequent interpolation.

Note that the chosen $\transform$ makes \vz conform to the requirements listed
in \sref{frame-transformation}. Namely, the codomain of \vz is $[0, 1]^\nz$, and
it has the smallest number of dimensions that are needed to preserve the desired
level of accuracy.

The transformation is also depicted in \fref{frame-application}; see the small
box to the right of the ``black box'' described earlier. In this particular
example, the dimensionality is found to be the same before and after
$\transform$. The depicted component takes an assignment of the auxiliary
variables $\z_1$ and $\z_2$ and outputs the execution times $\u_1$ and $\u_2$ in
accordance with their joint distribution.

In what follows, we proceed directly to discussing Stage~4 since Stage~3, which
is given in the middle of \fref{frame-application}, is standard: using a number
of strategic invocations of the simulator, it yields a light surrogate for the
simulator; in \fref{frame-application}, this surrogate corresponds to the small
rounded rectangle.

\subsection{Post-Processing}

Having constructed an interpolant of \g, the designer proceeds to working
exclusively with this interpolant, which is Stage~4 of the presented framework.
For concreteness, assume that the designer is interested in the distribution of
\g, in which case the interpolant is to be sampled extensively; it is denoted by
the rightmost box in \fref{frame-application}. Then the interpolant displayed in
the figure corresponds roughly to \aref{frame-evaluation}: it takes $\z_1$ and
$\z_2$ and returns an approximation of \g at that point. Recall that the
computational cost of this extensive sampling is negligible as \g is not
involved. The collected samples, which we denote by $G$, are then used to
compute an estimate of the distribution of \g.

\inputfigure{frame-density}
\fref{frame-density} depicts the result. To elaborate, the blue line shows the
\ac{PDF} of \g computed by applying kernel density estimation to $G$ obtained
from the interpolant. The yellow line---which is barely visible behind the blue
line---shows the true density of \g; its calculation is explained in
\sref{frame-results}. It can be seen that our solution closely matches the exact
one. In addition, the orange line shows the estimate that one would get if one
sampled \g directly consuming the same number of \g's evaluations as the one
consumed by our framework. We see that, given the same budget, the solution
delivered by our technique is substantially closer to the true one than the one
delivered by naïve sampling.

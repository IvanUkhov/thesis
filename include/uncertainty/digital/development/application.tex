The agenda for this section is as follows. First, we exemplify Stage~1 of the
proposed framework by introducing a number of quantities of interest \g, which
illustrate the broad applicability of the framework and thereby give the reader
a better sense of its utility. Second, we turn to Stage~2 and highlight a
transformation $\transform$ that is the one to use in the majority of cases.
Third, we proceed directly to Stage~4 and illustrate a potential output of our
framework.

\subsection{\problemtitle}
\slab{frame-application-problem}

Assume the system, power, and temperature models described in
\sref{system-model}, \sref{power-model}, \sref{temperature-model}, respectively.
Assume also the application model utilized in \sref{dream-optimization-problem}
except for the requirement about being periodic.

Let us first touch upon the timing aspects of the system in question. Each of
the \nt tasks of the application has a start time and a finish time, which are
denoted by $b_i$ and $d_i$, respectively. Let also $\v{b} = (b_i)_{i = 1}^\nt$
and $\v{d} = (d_i)_{i = 1}^\nt$. Then other timing characteristics can be
derived from $\v{b}$ and $\v{d}$. For example, the end-to-end delay of the
application, which is the difference between the finish time of the latest task
and the start time of the earliest task, is as follows:
\begin{equation} \elab{frame-delay}
  \text{End-to-end delay}
  = \max_{i = 1}^\nt d_i - \min_{i = 1}^\nt b_i.
\end{equation}

Suppose now that the execution times of the tasks depend on the uncertain
parameters \vu introduced in \sref{frame-problem}. Then $\v{b}$ and $\v{d}$
depend on \vu. Hence, the end-to-end delay given in \eref{frame-delay} does so
too, and it constitutes a quantity \g that the designer might be interested in
analyzing. Note that this \g is nondifferentiable since the $\max$ and $\min$
functions are such. Therefore, \g is nonsmooth, which renders the \up{PC}
decomposition and similar techniques inappropriate in this case in general,
which is elaborated on in \sref{frame-motivation}.

\begin{remark} \rlab{frame-nonsmoothness}
The behavior of \g with respect to continuity, differentiability, and smoothness
cannot generally be inferred from the behavior of \vu. Even when the parameters
behave perfectly, \g might still exhibit nondifferentiability or even
discontinuity, which depends on how \g functions internally. For example, as
shown in \cite{tanasa2015}, even if execution times of tasks are continuous, due
to the actual scheduling policy, end-to-end delays are very often discontinuous.
\end{remark}

Let us move on to power. The total energy consumed by the \np processing
elements during an application run can be estimated using a power profile \mp,
which is defined in \eref{power-profile}, as follows:
\begin{equation} \elab{frame-energy}
  \text{Total energy}
  = \sum_{i = 1}^\np \int \p_i(t) \d t
  \approx \dt \norm[1]{\mp}
\end{equation}
where $\p_i$ denotes the power consumption of processing element~$i$, \dt is the
sampling interval, and $\norm[1]{\cdot}$ stands for the Manhattan norm. Since
$\v{b}$ and $\v{d}$ depend on \vu, the power consumption of the system is also
dependent on \vu. Consequently, the total energy given in \eref{frame-energy}
depends on \vu and is a candidate for \g. \rref{frame-nonsmoothness} applies in
this context to the full extent.

Let us now turn to temperature. The maximum temperature that the platform
attains during an application run can be estimated using a temperature profile
\mq, which is defined in \eref{temperature-profile}, as follows:
\begin{equation} \elab{frame-temperature}
  \text{Maximum temperature}
  = \max_{i = 1}^\np \sup_{t} \q_i(t)
  \approx \norm[\infty]{\mq}
\end{equation}
where $\q_i$ denotes the heat dissipation of processing element~$i$, and
$\norm[\infty]{\cdot}$ stands for the uniform norm. Since the power consumption
of the platform is affected by \vu, the corresponding heat dissipation is
influenced by \vu as well. Therefore, the maximum temperature in
\eref{frame-temperature} is also a potential quantity of interest \g. Note that,
due to the maximization involved in the calculations, the quantity is
nondifferentiable and hence cannot generally be adequately addressed using
polynomial approximations; recall also the concern in
\rref{frame-nonsmoothness}.

To summarize, we have covered three aspects of electronic systems, namely
timing, power, and temperature, and introduced a number of quantities that the
designer is typically interested in analyzing. These quantities are to be
discussed further in the section on experimental results, \sref{frame-results}.

\inputfigure{frame-application}
Let us employ one of the introduced quantities in order to have a concrete
example to work with in this section. The addressed problem is depicted on the
left-hand side of \fref{frame-application}. We consider a heterogeneous platform
with two processing elements denoted by PE1 and PE2 and an application with four
tasks denoted by T1--T4; the setup is to be described in detail in
\sref{frame-results}. The data dependencies between the tasks and their mapping
onto the processing elements can be seen in \fref{frame-application}. The
quantity of interest \g is the application's end-to-end delay defined in
\eref{frame-delay}. The uncertain parameters \vu are the execution times of T2
and T4 denoted by $\u_1$ and $\u_2$, respectively.

The large rectangle on the left-hand side of \fref{frame-application} is a
``black box'' that evaluates \g given \vu. It takes an assignment of the
execution times $\u_1$ and $\u_2$ and outputs the calculated end-to-end delay
\g. In practice, this evaluation often involves a system simulator, such as
Sniper \cite{carlson2011}, in which case the modeling capabilities of this
simulator are naturally inherited by our technique.

Targeting the practical scenario described in \sref{chaos-formulation}, the
marginal distributions and correlation matrix of \vu are assumed to be
available. Without loss of generality, each marginal distribution is a
four-parameter beta distribution shown in \eref{beta-distribution}. Furthermore,
the execution times are assumed to be correlated based on the dependencies
between the corresponding tasks as defined by the structure of the task graph of
the application. Specifically, the closer task~$i$ and task~$j$ are in the graph
as measured by the number of edges between vertex~$i$ and vertex~$j$, the
stronger $\u_i$ and $\u_j$ are correlated.

\subsection{Probability Transformation}
\slab{frame-application-transformation}

At Stage~2 of our workflow outlined in \sref{frame-solution}, a suitable
transformation $\transform$, which is required in \eref{frame-transformation},
is decided upon. We utilize the one shown in \eref{probability-transformation}
and reduce $\vu: \Omega \to \real^\nu$ to $\vz: \Omega \to [0, 1]^\nz$ so that
the latter is uniformly distributed. In this case, the rightmost $\transform_1$
in \eref{probability-transformation} is simplified since the marginals of \vz
are already uniform. Note that the model-order-reduction functionality of the
chosen $\transform$ is engaged; it eliminates redundant stochastic dimensions
and, therefore, assists the subsequent interpolation.

The result is that the obtained vector \vz conforms to the requirements listed
in \sref{frame-transformation}: the codomain of \vz is $[0, 1]^\nz$, and it has
the smallest number of dimensions that are necessary in order to preserve a
certain level of accuracy.

In \fref{frame-application}, $\transform$ is depicted as a small square. In this
particular example, the stochastic dimensionality is found to be the same before
and after $\transform$, which is indicated by the two incoming and two outgoing
arrows. The depicted component takes an assignment of the auxiliary variables
$\z_1$ and $\z_2$ and outputs the execution times $\u_1$ and $\u_2$ in
accordance with their joint distribution.

In what follows, we proceed directly to Stage~4 since Stage~3, which is given in
the middle of \fref{frame-application}, is standard: using a number of strategic
invocations of the simulator of \g (the ``black box''), it delivers a light
representation $\interpolant{\nz}{\ls}(\g)$, which is illustrated by a small
rounded rectangle in \fref{frame-application}.

\subsection{Post-Processing}

Having constructed the interpolant $\interpolant{\nz}{\ls}(\g)$, the designer
starts to work solely with this interpolant, which corresponds to Stage~4 of our
framework.

Suppose the designer is interested in the probability distribution of \g. In
this scenario, $\interpolant{\nz}{\ls}(\g)$ should be sampled, which is
represented by the rightmost box in \fref{frame-application}. The operation
corresponds roughly to \aref{frame-evaluation}: the interpolant receives $\z_1$
and $\z_2$ and returns an approximation of the value of \g at that point. Recall
that the computational cost of this sampling is negligible since \g is not
involved. The collected samples, which are denoted by $G$ in
\fref{frame-application}, are then utilized in order to estimate the
distribution of \g.

\inputfigure{frame-density}
\fref{frame-density} depicts the result. The blue line shows the \ac{PDF} of \g
computed by applying kernel density estimation \cite{hastie2013} to the data set
$G$. The yellow line, which is barely visible behind the blue line, shows the
actual density of \g; its calculation is explained in \sref{frame-results}. It
can be seen that our solution closely matches the exact one. In addition, the
orange line illustrates the estimate that the designer would obtain if \g was
sampled directly using the same number of evaluations as the one consumed by the
proposed framework. It can be seen that, given the same budget, the solution
delivered by our technique is substantially closer to the exact one than the one
delivered by direct sampling.

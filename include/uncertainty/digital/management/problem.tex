Consider a large system of computers that is serving a stream of tasks which are
distributed across the individual machines of the system by a resource manager.
Each task consumes certain resources---such as \up{CPU} and memory---during its
execution. Each task $i$ is characterized by a resource-usage profile, which is
a sequence of \ng-dimensional measurements taken with a certain sampling
interval and captured by the following matrix of size $\ng \times \nsi{i}$:
\begin{equation} \elab{network-profile}
  \mg_i = (\vg_{ij})_{j = 1}^\nsi{i}
\end{equation}
where $\vg_{ij} \in \real^\ng$ is the measurement taken at time step $j$, and
\nsi{i} denotes the length of the sequence. Such information is called
fine grained as it contains multiple measurements over the execution of the task
as opposed to having only one aggregate measurement such as the average or
maximum value.

Consider now task $i$ and suppose that the current time step is $j$; an
illustration for $j \in \{3, 4, 5\}$ is given in \fref{network-example}. This
means that $\range{\vg_{i1}}{\vg_{ij}}$ are known. Given these previous values,
the goal is to estimate the next $h$ values of the sequence, which we denote by
$\range{\hat{\vg}_{i,j + 1}}{\hat{\vg}_{i,j + h}}$; in \fref{network-example},
$h = 4$. Such an estimation is called a long-range prediction as it provides
multiple future values as opposed to providing only one. The described operation
is to be performed for each active task of interest at each time moment of
interest.

In order to tackle the problem, one is to learn from historical data.
Specifically, there is a data set of past profiles at one's disposal, which is
denoted by
\begin{equation} \elab{network-dataset}
  G = \left\{ \mg_i: i = \range{1}{\no} \right\}
\end{equation}
where \no is the total number of profiles, and $\mg_i$ is as in
\eref{network-profile}. Such a data set can be straightforwardly collected
provided that the system at hand has an adequate monitoring facility deployed,
which is commonplace in practice.

It should be understood that, in order for learning to be possible, the
available resource-usage profiles have to have a certain structure that could be
extracted and utilized for intelligent prediction. An important question is
whether real-life profiles of this kind exhibit such a structure at all.
Investigating this question is part of our objective in this chapter.

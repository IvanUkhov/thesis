Let us discuss a number of studies that leverage machine-learning techniques in
order to facilitate resource management in computer systems.

In \cite{coskun2008}, the subject of forecasting is temperature, and the
objective is attained by means of an autoregressive--moving-average model
\cite{hastie2013}, resulting in an efficient thermal management strategy for
multiprocessor systems. The work in \cite{kumar2010} enhances runtime thermal
management by providing an on-chip temperature predictor based on feedforward
neural networks \cite{hastie2013}. The analysis and mitigation of the impact of
process variation undertaken in \cite{juan2014} are facilitated by a linear
regression model \cite{hastie2013} constructed based on measurements of the
static power with the goal of predicting peak temperatures.

Closer to the topic of this chapter, the work in \cite{dabbagh2015} is concerned
with cloud data centers. The authors propose a framework for predicting the
number of virtual-machine requests together with the required amount of \up{CPU}
and memory. The framework makes use of k-means clustering \cite{hastie2013} for
identifying different types of requests, and then it utilizes Wiener filters in
order to estimate the aggregate workload with respect to each identified type.
Similarly to \cite{dabbagh2015}, the work in \cite{ismaeel2015} is focused on
forecasting virtual-machine requests in cloud data centers and relies on k-means
clustering as the first step. Unlike \cite{dabbagh2015}, the main workhorse in
the case of the technique in \cite{ismaeel2015} is extreme learning machines,
which are feedforward neural networks mentioned earlier. An ensemble model
\cite{hastie2013} is presented in \cite{cao2014} targeted at predicting the
\up{CPU} usage in cloud environments. It relies on multiple traditional models,
and the final prediction is obtained by combing them by means of a scoring
algorithm.

It can be seen that, in general, machine learning has been utilized extensively
for aiding the design of resource managers of computer systems. However, as
noted in \sref{brain-introduction}, the most recent advancements in machine
learning have not been sufficiently explored yet in this context. In particular,
the utility of neural networks have been studied only marginally: feedforward
neural networks---utilized, for instance, in \cite{kumar2010, ismaeel2015}---are
arguably the simplest and least powerful members of the family. However, the
family is rich and potent, which is widely demonstrated. For instance, recurrent
neural networks accompanied by adequate training and regularization techniques
\cite{goodfellow2016} are a salient candidate for resource management in
computer systems.

In addition, it should be noted that the predictions delivered by the techniques
proposed in \cite{cao2014, dabbagh2015, ismaeel2015} are coarse, aggregate. They
treat virtual-machine requests or computational resources as a fluid and predict
the level that this fluid will attain at the next time moment. This means that
they are not capable of characterizing individual tasks executed by the
machines. More generally, resource-usage prediction at the level of individual
tasks and potentially multiple steps ahead has been deprived of attention.
However, such prediction could provide the resource manager with more detailed
and foresighted information, thereby allowing for a more intelligent
orchestration of the system.

To summarize, only primitive architectures of neural networks have been
investigated in the literature on resource management of computer systems, and
only aggregate prediction has been considered so far. Therefore, there is a
palpable need for further exploration and development in this direction.

Resource management is of great importance. It is the activity that, if
adequately executed, enables one to exploit optimally the full potential of the
computer system at hand. However, there is typically very little control over
the production environment. In particular, the actual workload that the system
will have to process at runtime is rarely, if ever, known in advance. Therefore,
resource management has to cope with inherent workload uncertainty.

Workload uncertainty can be mitigated at runtime by predicting the future and
acting accordingly, which is what proactive resource managers thrive on.
However, accurate and useful prediction is not an easy task. Computer systems
are elaborate, and their resource managers require elaborate forecasting
mechanisms, which traditionally fall in the scope of machine learning
\cite{hastie2013}. Machine learning has recently received a great amount of due
attention due the renaissance in neural networks \cite{goodfellow2016}, which
seemingly effortlessly superseded the then state-of-the-art techniques for
modeling and prediction.

Modern neural networks constitute a highly promising assistant for resource
management. Despite the fact that resource management has already seen a number
of applications of neural networks---which we touch upon in
\sref{network-prior}---the research in this direction has been limited. In
particular, only primitive architectures of neural networks have been
considered, and they have been applied to relatively simple problems. This state
of affairs is unfortunate provided that neural networks have been nearly
revolutionary in other disciplines. Therefore, we feel strongly that more
research should be conducted in order to investigate the aid that the recent
advancements in machine learning can give to the design of resource managers of
computer systems.

\inputfigure{network-example}
In this chapter, we conduct one such body of research. More specifically, we
study the resource usage in a large system of computers and aim to predict this
usage multiple steps ahead at the level of individual tasks executed by the
machines. To this end, we intend to use recurrent neural networks
\cite{goodfellow2016}.

In order to give a better intuition about the considered scenario,
\fref{network-example} illustrates how the aforementioned prediction is supposed
to work. In this example, the \up{CPU} usage (the solid lines) of a single task
running on a single computer is depicted three times. The three cases correspond
to three different time moments (the black circles) as viewed by the resource
manager of the system. The solid orange lines represent the history of the
usage, which is known to the manager, while the solid blue lines represent the
future usage, which is unknown to the manager. The latter is what we are to
predict for each task of interest and several steps ahead. In
\fref{network-example}, our potential predictions up to four steps ahead are
depicted as a set of dashed blue lines.

Such detailed (since for individual tasks) and foresighted (since multiple steps
ahead) information about the future resource usage as the one depicted in
\fref{network-example} can be of great help to the resource manager of the
system. For instance, having at its disposal the information about the future
resource usage of the tasks that are currently being executed, the manager can
more intelligently decide which machine the next incoming task should be
delegated to.

We proceed as follows. In \sref{network-problem}, the problem that we address is
formulated. \sref{network-prior} provides an overview of the prior work in this
context. In \sref{network-solution}, we present our solution. The experimental
results are reported and discussed in \sref{network-results}. Lastly,
\sref{network-conclusion} concludes the chapter.

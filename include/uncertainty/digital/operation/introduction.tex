Resource management is of great importance. It is the activity that, if
adequately executed, enables one to optimally exploit the full potential of the
computer system at hand. However, there is typically very little control over
the operating environment. In particular, the actual workload that the system
will have to process at runtime is rarely, if ever, known in advance. Therefore,
resource management has to cope with inherent workload uncertainty.

Workload uncertainty can be mitigated at runtime by predicting the future and
acting accordingly. This is the general principle that proactive resource
managers thrive on. However, accurate and useful prediction is not
straightforward: modern computer systems are reasonably complex, and their
resource management reasonably requires elaborate forecasting mechanisms.

Prediction traditionally falls in the scope of machine learning
\cite{hastie2013}. The field has recently received a great amount of due
attention due to the renaissance in neural networks \cite{goodfellow2016}, which
constitute a highly promising assistant for resource management. Despite the
fact that resource management has already seen a number of applications of
neural networks---which we touch upon in \sref{brain-prior}---the research in
this direction has been limited. In particular, only primitive architectures of
neural networks have been considered, and they have been applied to relatively
simple problems. This state of affairs is unfortunate provided that neural
networks have been nearly revolutionary in other disciplines. Therefore, we feel
strongly that more research should be conducted in order to investigate the aid
that the recent advancements in machine learning can give to the design of
resource managers of computer systems.

\inputfigure{brain-application}
In this chapter, we conduct one such body of research. More specifically, we
study the resource usage in a large system of computers and aim to predict this
usage multiple steps ahead at the level of individual tasks executed by the
machines. To this end, we intend to use recurrent neural networks
\cite{goodfellow2016}.

In order to give a better intuition about the considered scenario,
\fref{brain-application} illustrates how the aforementioned prediction is
supposed to work in practice. In this example, the \up{CPU} usage of a single
task running on a single computer is depicted three times; see the solid lines.
The three cases correspond to three different time moments as viewed by the
resource manager of the system; see the black circles. The solid orange lines
represent the history of the usage, which is known to the manager, while the
solid blue lines represent the future usage, which is unknown to the manager.
The latter is what we are to predict for each task of interest and several steps
ahead. In \fref{brain-application}, our potential predictions up to four steps
ahead are depicted as a set of dashed blue lines.

Such detailed (since for individual tasks) and foresighted (since multiple steps
ahead) information about the future resource usage as the one depicted in
\fref{brain-application} can be of great help to the resource manager of the
system. For instance, having at its disposal the information about the future
resource usage of the tasks that are currently being executed, the manager can
more intelligently decide which machine the next incoming task should be
delegated to.

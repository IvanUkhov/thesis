In this section, we assess our framework for characterizing process variation
presented in \sref{bayes-solution}. All the experiments are conducted on a
\up{GNU}/Linux machine equipped with an Intel Core i7 2.66~\up{GH}z and
8~\up{GB} of \up{RAM}. All the configuration files used in the experiments are
available online at \cite{eslab2014a}.

Our goal is to infer the effective channel length \g from temperature \h. Such a
high-level parameter as temperature constitutes a challenging task for the
inference of such a low-level parameter as the effective channel length, which
implies a rigorous assessment of the proposed technique. The performance of our
approach is only expected to increase when the auxiliary parameter \h is closer
to the target parameter \g with respect to the data model $\h = \f(\g)$
described in \sref{bayes-data-model}. For instance, such a closer quantity \h
could be the leakage current, but this might not always be the most preferable
parameter to measure. Lastly, let us note that the chosen effective channel
length is an important target, as it is strongly affected by process variation
and considerably impacts power consumption and heat dissipation
\cite{chandrakasan2000, srivastava2010, juan2011, juan2012}. It also affects
other process-related parameters, such as the threshold voltage.

We first describe the default configuration, which will be adjusted later on
according to the purpose of each particular experiment. We consider a 45-nm
technological process. The diameter of the wafer is 20 dies, and the total
number of dies \nd is 316. The number of measured dies \hnd is 20, and these
dies are chosen by an algorithm that strives for even coverage of the wafer. The
fabricated platform has four processing elements, and they are the points at
which measurements are taken, that is, $\np = 4$. The floorplan of the platform
is constructed in such a way that the processing elements form a regular grid.
The dynamic power profiles involved in the experiments are based on simulations
of applications randomly generated by \up{TGFF} \cite{dick1998}. The model of
the static power parameterized by temperature and the effective channel length
is constructed by fitting to \up{SPICE} simulations of reference electrical
circuits composed of \up{BSIM4} devices \cite{bsim} configured according to the
45-nm \up{PTM} \up{HP} model \cite{ptm}. The construction of thermal \up{RC}
circuits is delegated to HotSpot \cite{skadron2003}, and temperature analysis is
undertaken via the approach described in \sref{transient-solution}. The sampling
interval of power and temperature profiles is 1~ms.

The input data set $H$ is obtained as follows: \one~draw a sample of \g from a
Gaussian distribution with a mean of 17.5~nm (in accordance with the
technological process under consideration \cite{ptm}) and a covariance function
equal to \eref{bayes-covariance} with a standard deviation of 2.25~nm;
\two~perform one fine-grained temperature simulation for each of the \hnd dies
selected for measurement; \three~thin the obtained temperature profiles so that
each has only \ns, which is 20 by default, evenly spaced moments in time; and
\four~perturb the resulting data using white Gaussian noise with a standard
deviation of \celsius{1}.

Let us turn to the statistical model in \sref{bayes-statistical-model} and
summarize the intuition behind the model's parameters and the process by which
they are assigned. In the covariance function given in \eref{bayes-covariance},
the weight parameter $w$ and the two length-scale parameters $\ell_\SE$ and
$\ell_\OU$ should be set according to the variation patterns that are typical
for the fabrication process at hand \cite{chandrakasan2000, cheng2011}; we set
$w$ to 0.7 and $\ell_\SE$ and $\ell_\OU$ to half the radius of the wafer. The
threshold parameter $\eta$ of the model-order-reduction procedure shown in
\eref{karhunen-loeve} and utilized in \eref{bayes-reduction} should be set high
enough in order to preserve a sufficiently large portion of the variance of the
data, thereby keeping the model sufficiently accurate; we set it to 0.99. The
resulting dimensionality \nz of \vz in \eref{bayes-reduction} is found to be 27
or 28. The parameters $\mu_0$ and $\tau_\g$ of the prior given in
\eref{bayes-prior} are specific to the technological process under
consideration; we set $\mu_0$ to 17.5~nm and $\tau_\g$ to 2.25~nm. The
parameters $\sigma_0$ and $\nu_\g$ used in \eref{bayes-prior} determine the
precision of the information about $\mu_0$ and $\tau_\g$ and are set according
to the beliefs of the designer; we set $\sigma_0$ to 0.45~nm and $\nu_\g$ to 10.
The latter can be thought of as the number of imaginary observations that the
choice of $\tau_\g$ is based on. The parameter $\tau_\epsilon$ in
\eref{bayes-prior} represents the precision of the equipment utilized for
collecting $H$ and can be found in the technical specification of that
equipment; we set $\tau_\epsilon$ to \celsius{1}. The parameter $\nu_\epsilon$
in \eref{bayes-prior} has the same interpretation as $\nu_\g$; we set it to 10
as well. In \eref{bayes-proposal}, $\nu$ and $\alpha$ are tuning parameters,
which are configured based on experiments; we set $\nu$ to 8 and $\alpha$ to
0.5. The number of sample draws is another tuning parameter, which we set to
\power{10}{4}. The first half of these samples is ascribed to the burn-in period
mentioned in \xref{bayesian-statistics}, and the second one constitutes $G$; in
this case, $\no = 5 \times 10^3$. In the optimization described in
\sref{bayes-optimization}, we use the quasi-Newton algorithm \cite{press2007}.
For parallel computations, we utilize four computational cores.

In order to ensure that the experimental setup is adequate, we first perform a
detailed inspection of the results obtained for one particular example with the
default configuration. The true and inferred distributions of the quantity of
interest are shown in \fref{bayes-motivation-distribution} where the \ac{NRMSE}
is below 2.8\%, and the absolute error is bounded by 1.4~nm, which suggests that
our framework produces a close match to the true value of the quantity. We also
investigate the behavior of the constructed Markov chains and the quality of the
proposal distribution. All the observations indicate that the optimization and
sampling procedures are properly configured.

In the following subsections, we consider the experimental setup described above
and alter a single parameter at a time in order to investigate its impact.
Specifically, we change \one~the number of measurement sites \hnd, which is 20
by default; \two~the number of measurement points per site \np, which is 4 by
default; \three~the number of data instances per point \ns, which is 20 by
default; and \four~the standard deviation of measurement noise
$\sigma_\epsilon$, which is \celsius{1} by default.

\hiddensubsection{Number of Measurement Sites}

\inputtable{bayes-sites}
Let us vary the number of measured sites or dies \hnd. The scenarios being
considered are 1, 10, 20, 40, 80, and 160 dies. The obtained results are shown
in \tref{bayes-sites}. In this and the following tables, we report the
optimization time and sampling time separately, which correspond to Stage~2 and
Stage~3 in \fref{bayes-overview}, respectively. In addition, the sampling time
is given for two cases: sequential and parallel computing, which is followed by
the total time and resulting error. The computation time of the post-processing
stage, Stage~4, is not given as it is negligibly small. The sequential sampling
time is the most representative indicator of the computational complexity of the
proposed framework, since the number of samples is always fixed, and there is no
parallelization. Therefore, we refer to this value in most of the discussions
given below.

It can be seen in \tref{bayes-sites} that the more data the proposed framework
needs to process, the longer the execution time becomes, which is reasonable.
The trend, however, is modest: when \hnd is doubled, the computation time
increases by less than a factor of two. Regarding accuracy, the error decreases
definitively and drops below 4\% when around 20 sites are measured, which is
only 6\%--7\% of the total number of dies on the wafer under consideration.

\hiddensubsection{Number of Measurement Points}

\inputtable{bayes-points}
In this subsection, we consider five platforms with different numbers of
processing elements or, equivalently, measurement points \np. The scenarios
being considered are 2, 4, 8, 16, and 32 processing elements. The results are
summarized in \tref{bayes-points}. The computation time grows along with \np.
This is expected, since the granularity of the temperature model is bound to the
number of processing elements: each processing element contributes four thermal
nodes to the thermal \up{RC} circuit that temperature analysis is based on;
recall \sref{temperature-model}. Hence, temperature analysis becomes more
expensive. Nevertheless, even for large examples, taking into account the
complexity of the inference procedure and the yielded accuracy, the time
requirement is readily acceptable. An interesting observation can be made with
respect to the \ac{NRMSE}: the error tends to decrease as \np grows. The reason
is that, with each measurement point, $H$ delivers more information for the
inference to work with.

\hiddensubsection{Number of Data Instances}

\inputtable{bayes-instances}
Now we change the number of data instances \ns, which, in this case, is the
number of moments in time captured by temperature profiles. The scenarios being
considered are 1, 10, 20, 40, 80, and 160 moments. The results are aggregated in
\tref{bayes-instances}. It can be seen that the growth in computation time is
relatively small. One might expect this growth due to \ns to be the same as the
one due to \np, since, technically, the influence of \np and \ns on the
dimensionality of $H$ is identical; recall that $\hvh \in \real^{\hnd \np \ns}$.
However, the meanings of \np and \ns are completely different, and the ways they
manifest themselves in the inference algorithm are also different, which
explains the discordant figures shown in \tref{bayes-points} and
\tref{bayes-instances}. The \ac{NRMSE} in \tref{bayes-instances} has a
decreasing trend; however, this trend is less steady than the ones noted before.
The finding can be explained as follows. The temporal distribution of the
moments in time that are present in $H$ changes, since these moments are kept
evenly spaced across the time spans of the corresponding applications. Some
moments can be more informative than others. Consequently, more or less
representative samples can be accumulated in $H$, helping or misleading the
inference. Additionally, we can conclude that a larger number of spatial
measurements is more advantageous than a larger number of temporal measurements.

\hiddensubsection{Deviation of Measurement Noise}

\inputtable{bayes-noise}
In this subsection, we vary the standard deviation of measurement noise, which
corrupts $H$. The cases being considered are 0, 0.5, 1, and \celsius{2}
\cite{mesa-martinez2007}. Note that the corresponding prior in
\eref{bayes-prior} is kept unchanged. The results are given in
\tref{bayes-noise}. It can be seen that the sampling time is approximately
constant. However, we observe an increase in the optimization time when the
level of noise decreases, which can be ascribed to greater opportunities for
perfection for the optimization procedure. A more important observation revealed
by this experiment is that, in spite of the fact that the inference operates on
indirect and incomplete data, a thoroughly calibrated piece of equipment can
considerably improve the quality of prediction. However, even with a noise of
\celsius{2}---meaning that measurements are dispersed over a wide band of
\celsius{8} with a probability of more than 0.95---the \ac{NRMSE} is still only
4\%.

\hiddensubsection{Sequential and Parallel Sampling}

Lastly, we elaborate on the sequential and parallel sampling strategies. In the
sequential Metropolis--Hastings algorithm, the optimization time is typically
smaller than the time needed for drawing posterior samples. The situation
changes when parallel computing is utilized. When four cores are working in
parallel, the sampling time decreases by a factor of 3.81 on average, which
indicates good parallelization properties of the chosen sampling strategy. The
overall speedup ranges from 1.49 to 2.75 with an average value of 1.77, which
can be pushed even further by employing more computational cores.

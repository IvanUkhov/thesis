There are a number of related studies that we would like to highlight. Bayesian
inference, which is briefly introduced in \xref{bayesian-statistics}, is
utilized in \cite{zhang2010} for identifying the optimal set of locations on the
wafer where the parameter under consideration should be measured in order to
characterize it with the maximal accuracy. The expectation-maximization
algorithm is considered in \cite{reda2009} in order to estimate missing test
measurements. In \cite{paek2012}, the authors consider an inverse problem
focused on the inference of power consumption based on transient temperature
maps by means of Markov random fields. Another temperature-based
characterization of power is developed in \cite{mesa-martinez2007} where a
genetic algorithm is employed for the reconstruction of the power model.

It should be noted that the procedures proposed in \cite{reda2009, zhang2010}
operate on direct measurements, meaning that the output is the same quantity as
the one being measured. In particular, these procedures rely heavily on the
availability of adequate test structures on the dies and are practical only for
secondary quantities affected by process variation, such as delays and currents,
but not for the primary ones, such as various physical dimensions. Consequently,
they often lead to excessive costs and have a limited range of applicability. On
the other hand, the approaches given in \cite{mesa-martinez2007, paek2012},
which concentrate on the power consumption of a single die, are not concerned
with process variation.

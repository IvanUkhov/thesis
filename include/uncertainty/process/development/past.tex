As we elaborate in \sref{motivation} and \sref{past}, uncertainty-unaware
techniques are inadequate, and sampling methods---including \ac{MC}
sampling---as a means of uncertainty quantification are computationally
expensive. In order to overcome the limitations of deterministic approaches and,
at the same time, to eliminate or at least mitigate the computational costs
associated with direct sampling, a number of probabilistic techniques have been
introduced, which are discussed below. We are particularly interested in power
and temperature and, therefore, pay special attention to these metrics in the
exposition. Since the static component of power consumption is influenced by
process variation the most, due mainly to the leakage current, the techniques
discussed below focus primarily on the variability in this component.

A solely power-targeted but temperature-aware solution is proposed in
\cite{chandra2010} where the working force of the analysis is \ac{MC} sampling
with partially precomputed data. A learning-based approach is presented in
\cite{juan2011} in order to estimate the maximum temperature under the static
steady-state condition; recall \sref{static-steady-analysis}.
Temperature-related issues originating from process variation are also
considered in \cite{juan2012} where a statistical model of the static
steady-state temperature is derived based on the linearity of Gaussian
distributions and time-invariant systems. A stochastic temperature simulator
targeted at the static steady state is developed in \cite{huang2009a} using the
\ac{PC} decomposition and the continuous \ac{KL} decomposition; see
\xref{probability-transformation} and \xref{polynomial-chaos}. A stochastic
collocation \cite{maitre2010, xiu2010} approach to static steady-state analysis
is presented in \cite{lee2013}, which relies on the \ac{KL} decomposition and on
Newton polynomials for interpolation. In \cite{shen2009}, \ac{PC} expansions are
employed in order to estimate the static power of entire chips. The \ac{KL}
decomposition is utilized in \cite{bhardwaj2006} for calculating static power.
Static power is also quantified in \cite{bhardwaj2008} via the \ac{PC} and
\ac{KL} techniques. The same combination of tools is employed in
\cite{vrudhula2006} and \cite{ghanta2006} in order to analyze the response of
interconnect networks and power grids, respectively, under process variation.

The last five of the above techniques, that is, \cite{bhardwaj2006,
vrudhula2006, ghanta2006, bhardwaj2008, shen2009}, perform only probabilistic
power analysis and ignore the interdependence between power and temperature
described in \sref{power-model}. The other ones are temperature-related
approaches, but none of them attempts to tackle probabilistic transient
analysis, that is, to compute the probability distributions of power and
temperature that evolve over time. However, such transient curves are of
practical importance. First, certain procedures cannot be undertaken without
knowledge of time-dependent variations; one example of this is reliability
optimization concerned with thermal-cycling fatigue, which is discussed in
\sref{dream-optimization}. Second, the static steady-state assumption that is
considered, for instance, in \cite{huang2009a, juan2011, juan2012, lee2013} can
rarely be justified, since power is not invariant in reality.

In addition, one frequently encounters the assumption that power and temperature
follow \apriori known probability distributions; Gaussian and log-normal
distributions are popular choices; see, for instance, \cite{bhardwaj2006,
srivastava2010, juan2012}. However, this assumption often fails in
practice---which is also noted in \cite{juan2012} regarding the normality of the
leakage current---due to \one~the nonlinear dependence of power on process
parameters and \two~the nonlinear interdependence between power and temperature.
In order to illustrate this, let us return to the example given in
\sref{chaos-motivation} and assume the widespread Gaussian model of the
effective channel length. We can then simulate the example \power{10}{4} times
and apply the Jarque--Bera test of normality to the collected temperature
samples as well as to the samples obtained by passing the temperature samples
through the log transformation. We observe that the null hypothesis, which avers
that the data are from an unspecified Gaussian distribution, is firmly rejected
in both cases at a significance level of 5\% \cite{rao2002}. This means that, if
the null hypothesis were true, the probability of observing these data would be
less than 0.05. Consequently, the distribution is very unlikely to be Gaussian
or log-normal, which can also be seen in \fref{chaos-application-density} shown
in \sref{chaos-transient-results}.

One can observe in the above discussion that the overwhelming majority of the
literature related to temperature in the presence of process variation relies on
static steady-state temperature analysis, which is inadequate in practice, and
the other two types of temperature analysis---transient analysis and dynamic
steady-state analysis---are not given enough attention. However, as discussed
earlier, their availability is of practical importance to the designer.

Let us now discuss reliability analysis. Reliability analysis is probabilistic
by nature. However, certain components of a reliability model can be treated as
either stochastic or deterministic, depending on the phenomena that the model is
designed for. Temperature is an example of such a component: it can be
considered deterministic if the effect of process variation on temperature is
neglected, and it can be considered stochastic if this effect is accounted for.
The former scenario is the one that is typically addressed in the literature
related to reliability. For instance, the reliability model proposed in
\cite{xiang2010} has a treatment of process variation; however, temperature is
included in the model as a deterministic quantity. In \cite{das2014a}, a design
methodology that minimizes energy consumption of and temperature-induced wear on
multiprocessor systems is introduced; yet neither energy nor temperature is
modeled with an awareness of uncertainty due to process variation. A similar
observation can be made with respect to the work reported in \cite{das2014c}
where a reinforcement-learning algorithm is used in order to improve the
lifetime of multiprocessor systems. An extensive survey of reliability-aware
system-level design techniques given in \cite{das2014b} confirms the trend
emphasized above: the widespread device-level models of failure mechanisms
generally ignore the impact of process variation on temperature. However, it is
unwise to assume that temperature is deterministic, since it could lead to a
substantial yield loss.

An example of a different kind is the work presented in \cite{lee2013}. It
introduces a statistical simulator for reliability analysis under process
variation and does consider temperature as a stochastic parameter. However, as
discussed previously, this study is limited to static steady-state temperatures.
Moreover, the reliability analysis that it presents is an analysis of maximum
temperatures without any direct connection to the common failure mechanisms
\cite{jedec2016}. The work in \cite{kiamehr2016} is worth mentioning as well.
Although it is not directly concerned with reliability analysis, it considers
aging variation together with process variation and presents a framework for
timing analysis based on \ac{MC} sampling.

To summarize, the prior techniques for probabilistic power and temperature
analysis are restricted in use due to one or several of the following traits:
\one~based on \ac{MC} simulations (potentially slow) \cite{chandra2010},
\two~limited to power analysis \cite{bhardwaj2006, ghanta2006, vrudhula2006,
bhardwaj2008, shen2009, chandra2010}, \three~ignoring the power-temperature
interplay \cite{bhardwaj2006, ghanta2006, vrudhula2006, bhardwaj2008,
huang2009a, shen2009}, \four~limited to the static steady-state temperature
\cite{huang2009a, juan2011, juan2012, lee2013}, \five~exclusive focus on the
maximum temperature \cite{juan2011}, and \six~\apriori chosen distributions of
power and temperature \cite{bhardwaj2006, srivastava2010, juan2012}. The
designer's toolbox does not yet include tools for transient analysis and dynamic
steady-state analysis under process variation, which are of great importance for
certain applications. Furthermore, reliability models lack a flexible approach
to taking the effect of process variation on power and temperature into
consideration.

\inputfigure{chaos-motivation}
Consider a quad-core architecture whose leakage current is uncertain to the
designer due to its dependence on a number of process parameters affected by
process variation. Assume first that these parameters have nominal values. We
can then simulate the system under a certain workload and observe the
corresponding temperature profile. The result is depicted in
\fref{chaos-motivation} by a blue line, which corresponds to the temperature of
one of the processing elements; the experimental setup is to be detailed in
\sref{chaos-transient-application} and \sref{chaos-transient-results}. It can be
seen that the temperature is always below \celsius{90}. Let us now assume a mild
deviation of the parameters from their nominal values in the direction where the
leakage current is higher, and let us perform temperature analysis once again.
The result is the orange line in \fref{chaos-motivation}; the maximum
temperature is approaching \celsius{100}. Finally, we repeat the experiment
considering a severe deviation of the parameters in the same direction and
observe the yellow line in \fref{chaos-motivation}; in this case, the maximum
temperature is almost \celsius{110}.

Suppose now that the designer is tuning a solution constrained by a maximum
temperature of \celsius{90}, and that the designer is guided exclusively by the
nominal values of the process parameters. In this scenario, even with mild
deviations, the electrical circuits might be burnt. Another path that the
designer might take is to design the system for severe conditions. In this
scenario, however, the system might easily become too conservative,
overdesigned.

The conclusion drawn from the example given above is that the presence of
uncertainty has to be adequately addressed in order to pursue efficiency and
robustness. Nevertheless, the majority of the literature related to power,
temperature, and reliability analysis of electronic systems ignores this
important aspect; see, for instance, \cite{rao2009, rai2011, thiele2011}. This
negligence is also present in the analysis and optimization described in
\cref{certainty-development}, and the goal of this chapter is to eliminate this
concern in the case of process variation.

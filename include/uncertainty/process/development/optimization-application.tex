In order to give a better sense of the approach to reliability analysis and
optimization presented in \sref{chaos-reliability-analysis} and
\sref{chaos-optimization}, we consider a concrete application, meaning that we
specify the uncertain parameters and discuss the accompanying computations. This
application is also utilized for the quantitative evaluation of our technique
given in the next section, \sref{chaos-optimization-results}.

\subsection{\problemtitle}

Assume that the structure of the reliability model $R(\cdot | \vg)$ of the
system at hand is the one given in \eref{reliability-model} where each
individual reliability function $R_i(\cdot | \vg_i)$ is the one shown in
\eref{weibull-reliability} with its own parameters $\scale_i$ and $\shape_i$.

During each iteration, the temperature of processing element~$i$ exhibits \nk{i}
cycles. Each cycle generally has different characteristics and, therefore,
causes different damage to the processing element. This aspect is taken into
account by adjusting $\scale_i$ as shown in \eref{thermal-cycling-scale}. The
shape parameter $\shape_i$ is known to be indifferent to temperature
\cite{chang2006}. For simplicity, assume that $\shape_i$ does not depend on
process parameters either, and that $\shape_i = \shape$ for $i =
\range{1}{\np}$.

Under the above assumptions, \rref{weibull-homogeneity} applies, and the
lifetime $\life: \Omega \to \real$ of the system has a Weibull distribution as
follows:
\[
  \life | (\scale, \shape) \sim \mathrm{Weibull}(\scale, \shape)
\]
where $\scale$ is the one given in \rref{weibull-homogeneity} combined with
\eref{thermal-cycling-scale}. Even though the reliability model has two
parameters, only one of them is uncertain to the designer, namely $\scale$.
Therefore, we treat the model as if it was parameterized only by $\scale$. The
shape parameter $\shape$ is assumed to be implicitly given.

In the case of reliability analysis under process variation without any
accompanying exploration of the design space, one can proceed to constructing a
\ac{PC} expansion of $\scale$. Having obtained this lightweight surrogate, the
reliability of the system can be studied from various perspectives. In the
current scenario, however, the quantity of interest \g is the one given in
\eref{chaos-optimization-quantity} since it allows for evaluating the objective
function and constraints defined in \eref{chaos-optimization-objective} and
\eref{chaos-optimization-constraints}, respectively. In
\eref{chaos-optimization-quantity}, the component denoted by \life stands for
the parameterization of the reliability model; consequently, it is $\scale$ in
the illustrative application developed in this section.

Let us now turn our attention to the uncertain parameters \vu of the addressed
problem. We focus on two crucial process parameters: the effective channel
length and gate oxide thickness. Then each processing element is ascribed two
random variables corresponding to the two process parameters, which means that
$\nu = 2 \np$ in the current example; see also \sref{chaos-problem}.

\begin{remark}
The variability of a process parameter at a spatial location can be modeled as a
composition of several parts---such as inter-lot, inter-wafer, inter-die, and
intra-die variations---which is demonstrated in
\sref{chaos-transient-application}. In this section, we illustrate a different
approach. From the mathematical standpoint, it is sufficient to consider only
one random variable per location with an adequate distribution and correlations
with respect to the other locations.
\end{remark}

Based on \sref{chaos-formulation}, the parameters \vu are assumed to be given as
a set of marginal distributions and a correlation matrix denoted by
$\set{F_i}_{i = 1}^\nu$ and $\correlation{\vu}$, respectively. Note that the
number of distinct marginals is only two since \np components of \vu correspond
to the same process parameter.

Both considered process parameters, the effective channel length and gate oxide
thickness, correspond to Euclidean distances; they take values on bounded
intervals of the positive half of the real line. Thus, similarly to
\sref{chaos-transient-application}, we model the two process parameters using
the four-parameter family of beta distributions shown in
\eref{beta-distribution}. Without loss of generality, the parameters are assumed
to be independent of each other, and the correlations between those elements of
\vu that correspond to the same process parameter are assumed to be given by the
correlation function shown in \eref{bayes-correlation}.

The process parameters manifest themselves in the calculations associated with
the power model shown in \eref{chaos-power-model-bulk} through the static power.
Analogously to \sref{chaos-transient-application}, the modeling here is based on
\up{SPICE} simulations of a series of \up{CMOS} invertors. The invertors are
taken from the 45-nm open cell library by NanGate \cite{nangate} and configured
according to the 45-nm \up{PTM} \up{HP} model \cite{ptm}. The simulations are
performed on a fine-grained and sufficiently broad three-dimensional grid
comprising the effective channel length, gate oxide thickness, and temperature;
the results are tabulated. An interpolation algorithm is subsequently employed
whenever the static power is to be evaluated at a particular point within the
range of the grid. The output of this model is scaled up to account for about
40\% of the total power consumption \cite{liu2007}. Regarding temperature, the
thermal \up{RC} circuit utilized for dynamic steady-state analysis is
constructed by virtue of HotSpot \cite{skadron2003} as described in
\sref{temperature-model}.

At this point, the two outputs of Stage~1 are now specified.

\subsection{Probability Transformation}

At Stage~2 in \fref{chaos-overview}, the uncertain parameters \vu are
transformed into a vector of independent random variables \vz via a suitable
transformation $\transform$. Specifically, we use the one given in
\eref{probability-transformation}, which also includes model order reduction.
Unlike \sref{chaos-transient-application}, in this section, we let \vz obey the
standard Gaussian distribution and, therefore, tailor $\transform$ accordingly;
see \xref{probability-transformation}.

\subsection{Surrogate Construction}

Since the auxiliary variables $\vz = (\z_i)_{i = 1}^\nz$ are Gaussian, the
polynomial basis considered at Stage~3 is to be composed of Hermite polynomials,
which is the exact scenario described in \xref{polynomial-chaos}. The variables
also tell us how to approach numerical integration needed for the evaluation of
the coefficients of \ac{PC} expansions: since we are interested in integrals
with respect to the standard Gaussian measure, Gauss--Hermite quadratures
\cite{maitre2010} are worth considering. These quadratures are especially
efficient since they belong to the class of Gaussian quadratures and thus
inherit their properties; see \xref{numerical-integration}.

Lastly, let us illustrate the Hermite basis. In the case of working with only
one standard Gaussian variable ($\nz = 1$), a second-level \ac{PC} expansion
($\lc = 2$) of a three-dimensional quantity of interest \vg is as follows:
\[
  \chaos{1}{2}{\vg}
  = \hat{\vg}_{(0)} \psi_{(0)}
  + \hat{\vg}_{(1)} \psi_{(1)}
  + \hat{\vg}_{(2)} \psi_{(2)}
\]
where $\set{\hat{\vg}_{\vi}} \subset \real^3$,
\begin{align*}
  & \psi_{(0)}(\vz) = 1, \\
  & \psi_{(1)}(\vz) = \z_1, \text{ and} \\
  & \psi_{(2)}(\vz) = \z_1^2 - 1.
\end{align*}
At Stage~4, the expansion is post-processed as described in
\sref{chaos-optimization}.

In this section, we discuss dynamic steady-state power and temperature analysis
under process variation. Unlike transient analysis, this analysis cannot be done
one time step at a time, since the repetitive workload needs to be taken into
account at once in order to calculate the corresponding dynamic steady state.
Thus, similarly to the contrast between \sref{transient-analysis} and
\sref{dynamic-steady-analysis}, the solutions in \sref{chaos-transient-analysis}
and this section differ. However, they still rely on the same methodology
outlined in \sref{chaos-uncertainty-analysis} and shown in
\fref{chaos-overview}.

\subsection{\problemtitle}

The system model is the same as the one in \sref{chaos-transient-analysis}. The
only difference is that it is more convenient to define the power model as
follows:
\begin{equation} \elab{chaos-power-model-bulk}
  \mp = \f(\vu, \mq).
\end{equation}
The function $\f: \real^\nu \times \real^{\np \times \ns} \to \real^{\np \times
\ns}$ is supposed to return the periodic power profile that corresponds to the
periodic workload under analysis.

\inputalgorithm{chaos-dynamic-steady-solution-iterative}
The solution to probabilistic dynamic steady-state analysis is based on the
deterministic one presented in \sref{dynamic-steady-solution}. The power,
temperature, and other vectors that appear in \sref{dynamic-steady-solution}
become stochastic in the present context, which also concerns the boundary
condition given in \eref{dynamic-steady-condition}. The pseudocode of a
procedure that delivers \mq for a fixed \vu is listed in
\aref{dream-dynamic-steady-solution}. The algorithm does not take account of the
interdependence between power and temperature; however, this interdependence can
be addressed via one of the techniques presented in \sref{power-temperature}. In
this section, we use the iterative approach illustrated in
\aref{dream-dynamic-steady-solution-iterative}. For clarity, this algorithm is
rewritten here as shown in \aref{chaos-dynamic-steady-solution-iterative}; the
main difference is that the calculation of power on line~3 is now based on
\eref{chaos-power-model-bulk}.

\begin{remark}
The application of the linear approximation described in
\sref{power-temperature} is problematic in this case. This technique is suitable
when the only varying parameter is temperature, and all other parameters have
nominal values. In that case, it is relatively easy to decide on a
representative temperature range and apply a curve-fitting procedure. In this
case, however, the power model has multiple parameters that range far from their
nominal values.
\end{remark}

To recapitulate, the quantity of interest \g in \fref{chaos-overview} is the
dynamic steady-state power and temperature profiles, which are denoted by \mp
and \mq, respectively. The calculation of this quantity is shown in
\aref{chaos-dynamic-steady-solution-iterative}.

\subsection{Surrogate Construction}

At Stage~3 in \fref{chaos-overview}, the procedure delineated in
\sref{chaos-construction} is applied to
\aref{chaos-dynamic-steady-solution-iterative} from Stage~1 with respect to the
output of Stage~2. Concretely, \aref{chaos-dynamic-steady-solution-iterative} is
utilized inside \aref{chaos-construction} via Algorithm~G. In this case,
Algorithm~G calls \aref{chaos-dynamic-steady-solution-iterative} and returns \mp
or \mq or both, depending on what is actually needed for the subsequent
calculations, in a suitable format.

Suppose, for instance, that the designer is interested in analyzing solely the
dynamic steady-state temperature profile \mq. Following
\rref{chaos-multidimensional-output} concerning vector-valued quantities, \g is
treated as an $\np \ns$-element row vector, in which case each coefficient
$\hat{\g}_{\vi}$ in \eref{chaos-expansion} is also such a vector. The projection
matrix, which is defined in \eref{chaos-projection-matrix}, and
\aref{chaos-construction} should be reinterpreted accordingly: \vg is an $\nq
\times \np \ns$ matrix whose row~$j$ is \mq computed at point~$j$ of the
quadrature in \eref{chaos-coefficient} and reshaped into a row vector.
Similarly, $\hat{\vg}$ should be understood as an $\nc \times \np \ns$ matrix
whose row~$i$ is coefficient~$i$ of the expansion in \eref{chaos-expansion}.
Recall that a certain ordering is assumed to be imposed on quadrature points and
polynomial terms.

The constructed \ac{PC} expansion can now be post-processed as required, which
is already the topic of Stage~4 in \fref{chaos-overview}; see
\sref{chaos-processing}.

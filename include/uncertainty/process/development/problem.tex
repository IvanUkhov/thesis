Assume the system model given in \sref{system-model}. Suppose that the system
depends on a number of process parameters that are uncertain at the design
stage. Once the fabrication process yields a particular outcome, the process
parameters take certain values and stay unchanged thereafter. However, these
values are different for different fabricated chips, and they vary within each
fabricated chip, since, in general, the variability due to process variation is
not uniform. As emphasized earlier, this variability leads to such phenomena as
deviations in power from the nominal values and, therefore, to deviations in
temperature from the one corresponding to the nominal power consumption.

Each process parameter is a characteristic of a single transistor; consider, for
instance, the effective channel length. Therefore, each device in the electrical
circuit at hand can potentially have a different value for this parameter. The
process parameters can then be modeled as a stochastic process
\[
  \u: \Omega \times \real^2 \to \real^\nu
\]
that is defined on a suitable probability space $(\Omega, \F, \probability)$
(see \xref{probability-theory}) and a two-dimensional plane and takes values in
$\real^\nu$ where \nu is the number of the process parameters. For practical
computations, the stochastic process is discretized, and each processing element
is modeled via a finite set of random variables. The uncertain parameters of the
problem are then defined as
\[
  \vu = (\u_i)_{i = 1}^\nu: \Omega \to \real^\nu
\]
where the union of all random variables of all processing elements is arranged
into a single random vector, and \nu is redefined to be the number of elements
that this vector has. Given the above setting, our goal is twofold as follows.

First, we are to develop a system-level framework for transient temperature (and
hence power) analysis as well as dynamic steady-state analysis of electronic
systems where power consumption and heat dissipation are stochastic due to their
dependency on the parameters \vu. The designer is required to specify \one~the
probability distribution of \vu and \two~the dependency of the system's power
consumption on \vu, which can be given as a ``black box.'' The framework is to
provide the designer with tools for analyzing the system under a given
workload---without imposing constraints on the nature of this workload---and
calculating the corresponding stochastic power \mp and stochastic temperature
\mq profiles with a desired level of accuracy and at low computational costs.

Second, taking into consideration the effect of process variation on power and
temperature, we are to find the reliability function of the system and to
develop a computationally efficient design-space-exploration scheme exploiting
the proposed techniques for power, temperature, and reliability analysis.

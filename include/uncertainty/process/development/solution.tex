Due to the inherent complexity, uncertainty-quantification problems are often
viewed as approximation problems: one constructs a computationally efficient
surrogate for the system under consideration and then studies this light
representation instead of the original system. In this chapter, we resort to the
\ac{PC} decomposition, which is thoroughly introduced in
\xref{polynomial-chaos}, for the construction of such a lightweight surrogate
for the quantity of interest \g. The technique decomposes stochastic quantities
into infinite series of mutually orthogonal polynomials operating on random
variables. These series are an attractive alternative to \ac{MC} sampling, since
they possess much faster convergence properties and provide succinct and
intuitive representations of the system's responses to stochastic inputs. Having
obtained an adequate polynomial surrogate for \g, we utilize it for calculating
the desired statistics about \g, such as its \ac{CDF}, \ac{PDF}, probabilistic
moments, and probabilities of various events.

The solution is consolidated in a framework targeted at analyzing electronic
systems subject to the uncertainty due to process variation. The framework is
flexible in modeling diverse probability distributions of the uncertain
parameters specified by the designer. Examples of such parameters include the
effective channel length and gate oxide thickness. Moreover, there are no
assumptions on the probability distribution of the quantity of interest, as this
distribution is unlikely to be known \apriori. Examples of such quantities
include as power and temperature profiles. The proposed technique is capable of
capturing arbitrary joint effects of the uncertain parameters on the system,
since the impact of these parameters is introduced into the framework as a
``black box,'' which is also defined by the designer. In particular, it allows
the power-temperature interplay to be taken into account with no effort.

Leveraging the proposed framework, we extend the deterministic transient
analysis and the deterministic dynamic steady-state analysis presented in
\sref{transient-analysis} and \sref{dynamic-steady-analysis}, respectively, by
taking account of process variation. Moreover, the framework allows us to enrich
the reliability analysis presented in \sref{reliability-model}, which is based
on the state-of-the-art reliability models, by taking into consideration the
effect of process variation on temperature.

We illustrate the proposed framework by considering two important process
parameters that are affected by process variation, namely the effective channel
length and gate oxide thickness; note, however, that our approach can be applied
to other parameters as well. Furthermore, we utilize the framework in order to
construct a computationally efficient design-space-exploration procedure
targeted at minimizing energy consumption under a set of probabilistic
constraints on the temperature and lifetime of the system at hand.

In what follow sections, we present our general approach to uncertainty analysis
and then specialize it for the aforementioned scenarios.

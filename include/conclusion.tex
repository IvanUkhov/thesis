Modern computer systems are complex, and they operate in complex environments
and perform complex tasks. Design as well as analysis of such systems are
already highly challenging endeavors; however, their difficulty is considerably
escalated by the inevitable uncertainty associated with computer systems.

The objective of this thesis has been to assist the designer by providing the
tools that would allow the designer to effectively and efficiently characterize
uncertainty and take it into account. To this end, we have developed a number of
system-level techniques for analyzing and designing with three classes of
uncertainty, namely, process, performance, and workload uncertainty.

In \cref{certainty-development}, we have elaborated on the Utopian deterministic
scenario, which has served as an adequate starting point for the developments in
the subsequent chapters. We have refined transient power and temperature
analysis and proposed a novel approach to dynamic steady-state analysis. The
latter has been applied it in the context of reliability optimization,
constituting our first attempt at mitigating the consequences of performance
variation.

In \cref{uncertainty-analog-fabrication}, we have considered the variability of
process parameters across silicon wafers induced by process variation and
presented a versatile statistical framework for inferring this variability by
means of indirect, incomplete, and noisy measurements such as readings from
thermal sensors.

In \cref{uncertainty-analog-development}, we have presented a probabilistic
technique for studying diverse system-level quantities of interest deteriorated
by process variation. Examples include transient and dynamic steady-state power
and temperature profiles of the system under consideration as well as its
maximal temperature and total energy consumption. The approach has been applied
to reliability optimization operating on stochastic quantities. In this context,
reliability analysis has been enhanced in order to account for the impact of
process uncertainty, making the treatment of performance uncertainty more
comprehensive.

In \cref{uncertainty-digital-development}, we have developed another technique
for probabilistic analysis of system-level quantities that are of interest to
the designer. In this case, we have striven to give a computationally efficient
and adequate characterization of the variability that originates from digital
sources of uncertainty such as the runtime workload, which tends to be less
regular than the one originating from analog sources of uncertainty such as the
fabrication process.

In \cref{uncertainty-digital-management}, we have elaborated on the mitigation
of workload uncertainty at runtime in the context of resource management.
More specifically, we have performed an early investigation of the applicability
of advanced prediction techniques from the field of machine learning to the
problem of fine-grained long-range forecasting of the resource usage in large
computer systems.

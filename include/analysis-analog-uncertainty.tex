Starting from this chapter and relying on the exposition given in
\cref{design-certainty}, we turn our attention to techniques that address the
uncertainty that is present in computer systems. In this particular chapter, we
develop a framework for the analysis of process variation across semiconductor
wafers.

\section{Introduction}

Process variation is an acute concern of electronic-system designs
\cite{chandrakasan2000, srivastava2010}. A crucial implication of process
variation is that it renders the key parameters of a technological
process---such as the effective channel length and gate oxide thickness---as
uncertain quantities. Therefore, the same workload applied to two seemingly
identical dies can lead to two different power profiles and, consequently, to
two different temperature profiles since the power consumption and heat
dissipation depend on the aforementioned quantities, which is primarily due to
the static component of power discussed in \sref{interdependence}. As it is the
case with any other type of uncertainty in computer systems, the uncertainty due
to process variation leads to performance degradation and faults of various
magnitudes, and, therefore, process variation should be adequately analyzed as
the foremost step toward efficient and robust products.

An important problem in this regard is the characterization of the on-wafer
distribution of a quantity of interest that is deteriorated by process
variation, based on measurements. The problem belongs to the class of inverse
problems since the measured data can be seen as an output of the system at hand,
and the desired quantity as an input. Such an inverse problem is addressed here.

Our goal is to characterize arbitrary process parameters with high accuracy and
at low costs. The goal is accomplished by measuring auxiliary quantities that
are more convenient and less expensive to work with and employing statistics in
order to infer the desired parameters from the measurements. More specifically,
we propose a novel approach to the quantification of process variation based on
indirect, incomplete, and noisy measurements. Moreover, we develop and implement
a solid framework around the proposed idea and perform a thorough study of
various aspects of our technique.

There are a number of related studies that we would like to highlight. Bayesian
inference is utilized in \cite{zhang2010} for identifying the optimal set of
locations on the wafer where the parameter under consideration should be
measured in order to characterize it with the maximal accuracy. The
expectation-maximization algorithm is considered in \cite{reda2009} in order to
estimate missing test measurements. In \cite{paek2012}, the authors consider an
inverse problem focused on the inference of the power dissipation based on
transient temperature maps using Markov random fields. Another temperature-based
characterization of power is developed in \cite{mesa-martinez2007} where a
genetic algorithm is employed for the reconstruction of the power model. It
should be noted that the procedures in \cite{zhang2010, reda2009} operate on
direct measurements, meaning that the output is the same quantity as the one
being measured. In particular, these procedures rely heavily on the availability
of adequate test structures on the dies and are practical only for secondary
quantities affected by process variation such as delays and currents, but not
for the primary ones such as various geometrical properties. Hence, they often
lead to excessive costs and have a limited range of applications. The approaches
in \cite{paek2012, mesa-martinez2007}, on the other hand, concentrating on the
power dissipation of a single die, are not concerned with process variation.

The remainder of the chapter is structured as follows. A motivational example is
given in \sref{inference-example}. In \sref{inference-problem}, we formulate the
problem and the requirements to a potential solution. The solution that we
propose is presented in \sref{inference-solution}, and the corresponding
experimental results are reported and discussed in \sref{inference-results}.
\sref{inference-conclusion} concludes this chapter.

\section{Motivational Example}
\slab{inference-example}

Consider the distribution of the effective channel length across a silicon
wafer. The effective channel length, which we shall denote by \g, has one of the
strongest effects on the leakage current and, consequently, on power and
temperature \cite{juan2012}; see also \sref{interdependence}. At the same time,
\g is well known to be severely deteriorated by process variation
\cite{chandrakasan2000, srivastava2010}. Therefore, the distribution of \g is
not uniform across the wafer. For concreteness, let this distribution be the one
depicted on the left-hand side of \fref{inference-example}. The gradient from
blue to yellow represents the transition of \g from low to high values, and the
scale is given in terms of the number of standard deviations away from the mean
value (the exact experimental setup will be described in detail in
\sref{inference-results}). Hence, the blue regions have a high level of the
power consumption and heat dissipation. Assume that the technological process
imposes a lower bound $\g_\minimum$ on \g. This bound separates defective dies
($\g < \g_\minimum$) from those that are acceptable ($\g \geq \g_\minimum$). In
order to reduce costs, the manufacturer is interested in detecting the faulty
dies and taking them out of the production process at early stages. The possible
actions with respect to a single die on the wafer are \one~to keep the die if it
conforms to the specification and \two~to recycle it otherwise.

\inputfigure{inference-example}
In order to analyze the variability of the effective channel length \g across
the wafer, one can remove the top layer of (and, thus, destroy) the dies and
measure \g directly. Alternatively, despite the fact that the knowledge of \g is
more preferable, one can step back and decide to quantify process variation
using some other parameter \h that can be measured without the need for damaging
the dies; an example is the leakage current. It should be noted that, in this
second case, the chosen surrogate is the final product of the analysis, and \g
is left unknown. In either way, adequate test structures have to be present on
the dies in order to take measurements at sufficiently many locations with the
desired level of granularity. Such a sophisticated test structure might not
always be readily available, and its deployment might significantly increase
production costs. Moreover, the first approach implies that the measured dies
have to be recycled afterwards, and the second implies that the further design
decisions will be based on a surrogate quantity \h instead of the primary source
of uncertainty \g, which can compromise these decisions. The latter concern is
particularly prominent in the situations where the production process is not yet
completely stable, and, consequently, design decisions based on the primary
subjects of process variation are preferable.

Our technique operates differently. In this example, in order to characterize
the effective channel length \g, we begin by measuring an auxiliary quantity \h
as well. The quantity is required to depend on \g, and it can be chosen to be
straightforward from the measurement perspective. The distribution of \g across
the whole wafer is then obtained by inferring it from the collected measurements
of \h. Our technique permits these measurements to be taken only at a small
number of locations on the wafer and to be corrupted by noise, which can be due
to the imperfection of the measurement equipment utilized.

\inputfigure{inference-measured-defective}
Let us consider one particular \h that can be used to study the effective
channel length \g; specifically, let \h be temperature (we elaborate further on
this choice in \sref{inference-results}). We can then apply a fixed workload
---for instance, we can run the same application under the same conditions---to
a few dies on the wafer and measure the corresponding temperature profiles.
Since temperature does not require extra equipment to be deployed on the wafer
and can be tracked using infrared cameras \cite{mesa-martinez2007} or built-in
facilities of the dies, our approach can reduce the costs associated with the
analysis of process variation. The results of our framework applied to a set of
noisy temperature profiles measured for only 7\% of the dies are shown on the
right-hand side of \fref{inference-example}, and the locations of the measured
dies are depicted in \fref{inference-measured}. It can be seen that the two maps
in \fref{inference-example} closely match each other, implying that the
distribution of \g is reconstructed with a high level of accuracy.

Another characteristic of the proposed framework is that probabilities of
various events, for instance, $\probability{\g \geq \g_\minimum}$, can readily
be estimated. This is important since, in reality, the true values are
unknown---otherwise, we would not need to infer them---and, therefore, we can
rely on our decisions only up to a certain probability. We can then reformulate
the decision rule given earlier as follows: \one~to keep the die if
$\probability{\g \geq \g_\minimum}$ is larger than a certain threshold and
\two~to recycle it otherwise. An illustration of following this rule is given in
\fref{inference-defective} where $\g_\minimum$ is set to two standard deviations
below the mean value of \g; the probability threshold of the first action is set
to 0.9; the crosses mark both the true and inferred defective dies (they
coincide); and the gradient from white to orange corresponds to the inferred
probability of a die to be defective. It can be seen that the inference
accurately detects faulty dies.

In addition, we can introduce a trade-off action between action \one and action
\two as follows: \three~to expose the die to a thorough inspection (for
instance, via a test structure) if $\probability{\g \geq \g_\minimum}$ is
smaller than the threshold of action \one but larger than another threshold. For
instance, action \three can be taken if $0.1 < \probability{\g \geq \g_\minimum}
< 0.9$. In this case, we can reduce costs by examining only those dies for which
there is no sufficiently strong evidence of their satisfactory and
unsatisfactory conditions. Furthermore, one can take into consideration a
so-called utility function, which, for each combination of an outcome of \g and
a taken action, returns the gain that the decision maker obtains. For example,
such a function can favor a rare omission of malfunctioning dies to a frequent
inspection of correct dies as the latter might involve much higher costs. The
optimal decision is given by the action that maximizes the expected utility with
respect to both the observed data and prior knowledge on \g. Thus, all possible
\g weighted by their probabilities will be taken into account in the final
decision, incorporating also the preferences of the user via the utility
function.

\section{Problem Formulation}
\slab{inference-problem}

Consider a generic electronic system, which is fabricated on a silicon wafer
that accommodates \nd dies. Let \spec refer collectively to the specification of
the manufacturing process and of the system itself including the layout of the
wafer and the floorplan of the dies. The system depends on a process parameter
\g, which we are interested in studying and shall refer to as the \ac{QOI}. Due
to the presence of process variation, the value of \g deviates from the nominal
one, and this deviation can be different at different locations on the wafer.
The \ac{QOI} is assumed to be impractical to be measured directly.

Given \spec, the goal is to develop a framework targeted at the identification
of the on-wafer distribution of \g with the following properties: \one~low
measurement costs, \two~high computational speed, \three~robustness to the
measurement noise, \four~ability to accommodate prior knowledge on \g, and
\five~ability to assess the trustworthiness of the collected data and
corresponding predictions.

\section{Our Solution}
\slab{inference-solution}

In order to achieve the established goal, we make use of indirect measurements.
Specifically, instead of \g, we measure an auxiliary parameter \h, which we
shall refer to as the \ac{QOM}. The observations of \h are then processed via
Bayesian inference, which is introduced in \sref{bayesian-statistics}, in order
to derive the distribution of \g. The \ac{QOM} is chosen such that \one~\h is
convenient and cheap to be measured; \two~\h depends on \g, which is signified
by $\h = f(\g)$; and \three~there is a way to compute \h for a given \g. The
last means that $f$ should be known; however, it does not have to be explicitly
given: our framework treats $f$ as a ``black box.'' For example, $f$ can be a
piece of code.

As the first step, the user of the proposed framework is supposed to harvest a
set of observations of \h at several locations on the wafer; recall
\sref{inference-example}. Without loss of generality, we adhere to the following
convention. One die corresponds to one potential measurement site, and $\ndm <
\nd$ denotes the number of those sites that have been selected for measurement.
Each site comprises \np inner points, and each point contains \ns data
instances. For instance, in the example given in \sref{inference-example}, each
observation is a temperature profile $\mq \in \real^{\np \times \ns}$, which is
an matrix capturing temperature of \np processors at \ns time moments as defined
in \eref{temperature-profile}. Denote by $H = \{ \hh_i \}_{i = 1}^{\ndm}$ the
collected data set where $\hh_i \in \real^{\np \times \ns}$ stands for one
observation (one site) of \h. Lastly, it is assumed that the corresponding
locations are recorded along with $H$.

It is worth noting that, if $f$ is the identity function, that is, $\h = f(\g) =
\g$, the proposed technique will primarily focus on the reconstruction of any
missing observations in $H$, that is, on the unobserved sites on the wafer. From
this standpoint, our approach is a generalization of those developed in
\cite{zhang2010, reda2009}.

\inputfigure{inference-overview}
In the rest of this section, we present our framework for the characterization
of process variation. The technique revolves around \eref{bayes-theorem} and is
divided into four major stages depicted in \fref{inference-overview}. Stage~1 is
the data-harvesting stage where the user collects a set of observations of the
\ac{QOM}, forming the data set $H$. At Stage~2, we undertake an optimization
procedure, which is to assist the sampling procedure at Stage~3. The latter
produces a collection of samples of the \ac{QOI}---such as the effective channel
length---denoted by $G$, which is then processed at Stage~4 in order to estimate
the desired characteristics of this \ac{QOI}---such as the probability of the
effective channel length to be smaller than a certain threshold as motivated in
\sref{inference-example}. As it can be seen in \fref{inference-overview},
Stage~2 and Stage~3 actively communicate with the two models on the right-hand
side called the data and statistical models, which we discuss now.

\subsection{Data Model}
\slab{inference-data-model}

The data model relates the \ac{QOI} \g with the \ac{QOM} \h as follows:
\[
  \h = f(\g).
\]
The function $f$ depends on the choice of \h and is specified by the user. The
model is utilized in order to predict the values of \h at the same sites, at the
same inner points, and with the same amount as the ones in $H$ from Stage~1. The
resulting data are stacked into a single vector denoted by $\vh \in \real^{\ndm
\np \ns}$. Also, let $\hvh \in \real^{\ndm \np \ns}$ be a stacked version of the
data in $H$ such that the respective elements of \vh and \hvh correspond to the
same locations on the wafer.

In order to acquire a better understanding of the data model, let us return to
the example given in \sref{inference-example}. In this case, \g stands for the
effective channel length, and \h stands for the temperature profile
corresponding to a fixed workload. The data model can be roughly divided into
two transitions: \one~from the effective channel length \g to the static power
$\p_\static$ and \two~from the static power $\p_\static$ to the corresponding
temperature profile \h. At this point, it is worth recalling the power model
presented in \sref{power-model}. The first transition is due to the dependence
of the leakage current on the effective channel length, which is implicitly
present in \eref{static-power}. Then the transition can be realized using the
model in \eref{static-power} or any of its variations; see, for instance,
\cite{chandrakasan2000, srivastava2010, juan2012}. In particular, an adequate
model of the static power can be constructed via a fitting procedure applied to
a data set of \up{SPICE} simulations of reference electrical circuits. The only
requirement to such a model is that it should be parameterized by \g. In
addition, it can also be parameterized by temperature in order to account for
the interdependence between power and temperature described in
\sref{interdependence}. The second transition is undertaken by combining the
static power $\p_\static$ with the dynamic power $\p_\dynamic$ that corresponds
to the considered workload. The obtained total power along with the
temperature-related information contained in $\spec$---such as the floorplan and
thermal parameters of the platform at hand---are fed to a thermal simulator in
order to acquire the corresponding temperature \h, which is the topic of
\cref{design-certainty}.

\subsection{Statistical Model}

\inputfigure{inference-statistical-model}
Once the wafer has been fabricated, the values of \g across the wafer are fixed;
however, they remain unknown to us. In order to infer them, we employ the
procedure developed in the current subsection, which can also be seen in
\fref{inference-statistical-model}. The model consists of the five steps
described below.

Step~1 is to assign an adequate model to the unknown \g. We model \g as a
Gaussian process \cite{rasmussen2006} since \one~it is flexible in capturing the
correlation patterns induced by the manufacturing process; \two~it is
computationally efficient; and \three~Gaussian distributions are often natural
and accurate models for uncertainties due to process variation
\cite{srivastava2010, reda2009, juan2012}. We denote this as
\begin{equation} \elab{inference-prior}
  \g | \t_\g \sim \text{Gaussian Process}(\mu, k)
\end{equation}
where $\mu(\cdot)$ and $k(\cdot, \cdot)$ are the mean and covariance functions
of \g, respectively, which take locations on the wafer as arguments. The
notation also indicates that \g depends on a set of parameters $\t_\g$, which we
shall identify later on. Prior to taking any measurements, \g is assumed to be
spatially unbiased; therefore, we let $\mu$ be a single location-independent
parameter $\mu_\g$, that is, $\mu(r) = \mu_\g$ for any $r \in \real^2$. The
covariance function $k$ is chosen to be as follows:
\begin{equation} \elab{inference-covariance}
  k(r_1, r_2) = \sigma_\g^2 \left(\eta k_\text{SE}(r_1, r_2) + (1 - \eta) k_\text{OU}(r_1, r_2)\right)
\end{equation}
for $r_1 \in \real^2$ and $r_2 \in \real^2$ where
\begin{align*}
  k_\text{SE}(r_1, r_2) & = \exp\left(-\frac{\norm[2]{r_1 - r_2}^2}{\ell_\text{SE}^2}\right) \text{ and} \\
  k_\text{OU}(r_1, r_2) & = \exp\left(-\frac{\absolute{\,\norm[2]{r_1} - \norm[2]{r_2}\,}}{\ell_\text{OU}}\right)
\end{align*}
are the squared exponential and Ornstein--Uhlenbeck correlation functions
\cite{rasmussen2006}, respectively; $\sigma_\g^2$ is the variance of \g; $\eta
\in [0, 1]$ is a weighting coefficient; $\ell_\text{SE} > 0$ and $\ell_\text{OU}
> 0$ are so-called length-scale parameters; and $\norm[2]{\cdot}$ stands for the
Euclidean distance. The choice of $k$ is based on the observations of the
correlation structures induced by the manufacturing process
\cite{chandrakasan2000, cheng2011}. More specifically, $k_\text{SE}$ imposes
similarities between locations on the wafer that are close to each other, and
$k_\text{OU}$ imposes similarities between locations that are at the same
distance from the center of the wafer. The parameters $\ell_\text{SE}$ and
$\ell_\text{OU}$ control the extend of these similarities, that is, the range
where the correlation between two locations is significant. Although all the
above parameters of \g can be inferred from data, for simplicity, we shall focus
on $\mu_\g$ and $\sigma_\g^2$. The rest of the parameters---namely, $\eta$,
$\ell_\text{SE}$, and $\ell_\text{OU}$---are assumed to be determined prior to
our analysis based on the knowledge of the correlation patterns typical for the
production process utilized; see \cite{marzouk2009} and the references therein.

Step~2 is to make the above model of \g computationally tractable. The model is
an infinite-dimensional object as it characterizes a continuum of locations. For
practical computations, it should be reduced to a finite-dimensional one. First,
\g is discretized with respect to the union of two sets of locations. The first
one is composed of the $\ndm \np$ points where the observations in $H$ are made
($\ndm$ selected sites with $\np$ inner points each), and the other of the
locations where the user wishes to characterize \g. For simplicity, we assume
that the user is interested in all the sites, which is $\nd \np$ locations in
total. Let $\vg \in \real^{\nd \np}$ store the values of \g at these locations.
Second, the dimensionality is reduced even further by performing the
transformation in \eref{model-order-reduction} with respect to the correlation
matrix of \vg computed via \eref{inference-covariance}. The result is
\begin{equation} \elab{inference-reduction}
  \vg = \mu_\g \v{1} + \sigma_\g \m{L} \vz
\end{equation}
where $\v{1} = (\range{1}{1}) \in \real^{\nd \np}$, and $\vz = (\z_i) \in
\real^\nz$ independent random variables that obey the standard Gaussian
distribution. The number \nz is the final dimensionality of the model of \g.
Typically, $\nz \ll \nd \np$. The model is now ready for practical computations.
In addition, the parameters $\t_\g$ in \eref{inference-prior} are now known, and
they are $\t_\g = \{ \vz, \mu_\g, \sigma^2_\g \}$; see
\fref{inference-statistical-model}.

Step~3 is to define a likelihood function, which is where the observed
information is taken into account; see \sref{bayesian-statistics}. In our case,
the observed information is the measurements $H$ stacked into \hvh as described
in \sref{inference-data-model}. Since the measurement process is not perfect, we
also take into consideration the measurement noise. To this end, the observed
\hvh is assumed to deviate from the data model prediction \vh as follows:
\[
  \hvh = \vh + \v{\epsilon}
\]
where $\v{\epsilon}$ is an $\ndm \np \ns$-dimensional vector of noise, which is
typically assumed to be a white Gaussian noise \cite{rasmussen2006,
marzouk2009}. Without loss of generality, the noise is assumed to be independent
of \g and to have the same magnitude for all measurements. Hence, the model of
the noise is
\[
  \v{\epsilon} | \sigma^2_\epsilon \sim \text{Gaussian}(\v{0}, \sigma^2_\epsilon \m{I})
\]
where $\sigma^2_\epsilon$ is the variance of the noise. At this point, all the
parameters of the inference are identified, and they are $\t = \t_\g \cup \{
\sigma_\epsilon^2 \} = \{ \vz, \mu_\g, \sigma_\g^2, \sigma_\epsilon^2 \}$; see
\fref{inference-statistical-model}. Taking the above into account, we obtain
\begin{equation} \elab{inference-likelihood}
  \hvh | \t \sim \text{Gaussian}(\vh, \sigma_\epsilon^2 \m{I}).
\end{equation}
The density function of this distribution is the likelihood $p(H | \t)$ of our
statistical model, which is the first element of the posterior in
\eref{bayes-theorem}.

Step~4 is to decide on the second element of the posterior in
\eref{bayes-theorem}, that is, on the prior $p(\t)$. We put the following priors
on \t:
\begin{align} \elab{inference-prior}
  \begin{split}
    \vz               & \sim \text{Gaussian}(\v{0}, \m{I}), \\
    \mu_\g            & \sim \text{Gaussian}(\mu_0, \sigma^2_0), \\
    \sigma^2_\g       & \sim \text{Scale-inv-$\chi^2$}(\nu_\g, \tau^2_\g), \text{ and} \\
    \sigma^2_\epsilon & \sim \text{Scale-inv-$\chi^2$}(\nu_\epsilon, \tau^2_\epsilon).
  \end{split}
\end{align}
The prior of \vz is due to the decomposition in \eref{inference-reduction}. The
other three priors, that is, a Gaussian and two scaled inverse chi-squared
distributions, are a common choice for a Gaussian model whose mean and variance
are unknown. The parameters $\mu_0$, $\tau^2_\g$, and $\tau^2_\epsilon$
represent the presumable values of $\mu_u$, $\sigma^2_\g$, and
$\sigma^2_\epsilon$, respectively, and are set by the user based on the prior
knowledge of the technological process and measurement equipment employed. The
parameters $\sigma_0$, $\nu_\g$, and $\nu_\epsilon$ reflect the precision of
this prior information. When the prior knowledge is weak, less specific priors
can be considered \cite{gelman2004}. Taking the product of the densities of the
priors in \eref{inference-prior}, we obtain $p(\t)$ in \eref{bayes-theorem}.

Step~5 is to compile the posterior in \eref{bayes-theorem}. To this end, the
likelihood function $p(H | \t)$, which is the density of the distribution in
\eref{inference-likelihood}, and the prior $p(\t)$, which is the product of the
densities of the distributions in \eref{inference-prior}, are put together. The
density of the resulting posterior distribution is as follows:
\begin{equation} \elab{inference-posterior}
  p(\t | H) \propto p(\hvh | \vz, \mu_\g, \sigma^2_\g, \sigma^2_\epsilon) p(\vz) p(\mu_\g) p(\sigma^2_\g) p(\sigma^2_\epsilon).
\end{equation}
Provided that there is a way to draw samples from \eref{inference-posterior},
the \ac{QOI} can be readily analyzed as we shall see in
\sref{inference-post-processing}. The problem, however, is that direct sampling
of the posterior is difficult due to the data model involved in the likelihood
function via \vh; see \eref{inference-likelihood}. In order to circumvent this
problem, we utilize the Metropolis--Hastings algorithm \cite{gelman2004}
outlined in \sref{bayesian-statistics}, which operates on an auxiliary
distribution called the proposal distribution. The construction of an adequate
proposal is discussed next.

\subsection{Optimization Procedure}
\slab{inference-optimization}

In this subsection, we describe the objective of Stage~2 in
\fref{inference-overview}. Although the requirements to the proposal
distribution are mild, it is often difficult to pick an efficient proposal, that
is, such a proposal that would yield a good approximation with as few
evaluations of the posterior and, thus, of the data model in
\sref{inference-data-model} as possible. This choice is especially difficult for
high-dimensional problems, and our problem---which involves around 30 parameters
as we shall see in \sref{inference-results}---is one them. Therefore, a careful
construction of the proposal is an essential component of our framework.

A common technique to construct a high-quality proposal is to optimize the
posterior given in \eref{inference-posterior}. More specifically, we seek such a
value $\t^*$ of \t that maximizes \eref{inference-posterior} and, hence, has the
maximal posterior probability. In addition, we calculate the negative of the
Hessian matrix at $\t^*$, which is called the observed information matrix and
denoted by $\m{J}$; see the output of Stage~2 in \fref{inference-overview}.
Using $\t^*$ and $\m{J}$, we can construct a proposal that would allow the
Metropolis--Hastings algorithm \one~to start producing samples directly from the
desired regions of high probability and \two~to explore those regions more
rapidly. In the next subsection, the exact usage of $\t^*$ and $\m{J}$ is
explained.

\subsection{Sampling Procedure}

Let us turn to Stage~3 in \fref{inference-overview}. We have at our disposal
$\t^*$ and $\m{J}$ from Stage~2 in order to construct an adequate proposal and
utilize it for sampling. A commonly used proposal is a multivariate Gaussian
distribution where the mean is the current location of the chain of samples
started at $\t^*$, and the covariance matrix is the inverse of $\m{J}$
\cite{gelman2004}. In order to speed up the sampling process, we would like to
make use of parallelization. The aforementioned proposal, however, is purely
sequential as the mean for the next sample draw is dependent on the previous
sample. Therefore, we appeal to a variation of the Metropolis--Hastings
algorithm known as the independence sampler \cite{gelman2004}. In this case, a
typical choice of the proposal is a multivariate t-distribution independent of
the current position of the chain as follows:
\begin{equation} \elab{inference-proposal}
  \t \sim t_\nu(\t^*, \alpha^2 \m{J}^{-1})
\end{equation}
where $\t^*$ and $\m{J}$ are as in \sref{inference-optimization}, $\nu$ is the
number of degrees of freedom, and $\alpha$ is a tuning constant controlling the
standard deviation of the proposal. Now sampling the proposal in
\eref{inference-proposal} and evaluating the posterior in
\eref{inference-posterior} can be done in parallel. The obtained samples can
then be accepted or rejected subsequently as in the usual Metropolis--Hastings
algorithm.

Having completed the sampling procedure, we obtain a collection of samples of
\t. The first portion of the drawn samples is typically discarded as being
unrepresentative; this portion is known as the burn-in period. The preserved
samples of $\t = \{ \vz, \mu_\g, \sigma^2_\g \}$ are then passed through
\eref{inference-reduction} in order to compute samples of \g, which we denote by
$G = \{ \vg_i \}$ where $\vg_i \in \real^{\nd \np}$.

\subsection{Post-Processing}
\slab{inference-post-processing}

At Stage~4 in \fref{inference-overview}, using the set of samples $G$, the user
computes the desired statistics about the \ac{QOI} such as the most probable
value of the effective channel length at some location of interest and the
probability of a certain area on the wafer to be defective. These computations
reduce to the estimation of expected values with respect to the posterior
distribution of \t in \eref{inference-posterior}. Specifically, in order to
compute an arbitrary quantity dependent on \g, one evaluates this quantity for
each $\vg_i$ in $G$ and then takes the average value.

The strength of the Bayesian approach to inference starts to shine when one is
also interested in assessing the trustworthiness of the measured data and,
therefore, the reliability of estimates and decisions based on these data. Such
an assessment can readily be undertaken using our framework since the delivered
posterior distribution contains all the needed information about the \ac{QOI}.
This is especially helpful in decision-making as exemplified in
\sref{inference-example}.

\section{Experimental Results}
\slab{inference-results}

In this section, we assess our framework by inferring the effective channel
length \g from temperature \h. Such a high-level parameter as temperature
constitutes a challenging task for the inference of such a low-level parameter
as the effective channel length, which implies a strong assessment of the
proposed technique. The effective channel length is an important target as it is
strongly affected by process variation and considerably impacts the power
consumption and heat dissipation \cite{chandrakasan2000, srivastava2010,
juan2012}. It also influences other process-related parameters such as the
threshold voltage. The performance of our approach is expected to only increase
when the auxiliary parameter \h resides closer to the target parameter \g with
respect to the data model $\h = f(\g)$. For instance, such a closer quantity \h
can be the leakage current, which, however, might not always be the most
preferable parameter to measure.

We describe now the default configuration of our setup, which will be later
adjusted according to the purpose of each particular experiment. We consider a
45-nanometer technological process. The diameter of the wafer is 20 dies, and
the total number of dies \nd is 316. The number of measured dies \ndm is 20, and
these dies are chosen by an algorithm that strives for an even coverage of the
wafer. The fabricated platform has four processors, and they are the points of
taking measurements, that is, $\np = 4$. The floorplan of the platform is
constructed in such a way that the processors form a regular grid. The dynamic
power profiles involved in the experiments are based on simulations of
applications generated randomly by \up{TGFF} \cite{dick1998}. The model of the
static power parameterized by temperature and the effective channel length is
constructed by fitting to \up{SPICE} simulations of reference electrical
circuits composed of \up{BSIM4} devices \cite{bsim} configured according to the
45-nm \up{PTM} \up{HP} model \cite{ptm}. The temperature calculations are
undertaken using the approach described in \sref{transient-state-solution}; the
construction of thermal \up{RC} circuits is delegated to HotSpot
\cite{skadron2003}. The sampling interval of power and temperature profiles is
1~ms. All the configuration files of the experiments are available online at
\cite{eslab2013}.

The input data set $H$ is obtained as follows: (a) draw a sample of \g from a
Gaussian distribution with the mean value equal to 17.5~nm, according to the
considered technological process \cite{ptm}, and the covariance function given
by \eref{inference-covariance} wherein the standard deviation is 2.25~nm; (b)
perform one fine-grained temperature simulation per each of the $\ndm$ selected
dies under the corresponding dynamic power profile; (c) shrink the temperature
profiles to keep only \ns, which is equal to 20 by default, evenly spaced
moments of time; and (d) perturb the obtained data set using a white Gaussian
noise with the standard deviation of 1$\,$K (Kelvin).

Let us turn to the statistical model in \sref{statistical-model} and summarize
the intuition and our assignment for each parameter of this model. In the
covariance function given by \eref{inference-covariance}, the weight parameter
$\eta$ and the two length-scale parameters $\ell_\text{SE}$ and $\ell_\text{OU}$
should be set according to the correlation patterns typical for the production
process at hand \cite{chandrakasan2000, cheng2011}; we set $\eta$ to 0.7 and
$\ell_\text{SE}$ and $\ell_\text{OU}$ to half the radius of the wafer. The
threshold parameter of the model order reduction procedure described in
\sref{model-order-reduction} and utilized in \eref{inference-reduction} should
be set high enough to preserve a sufficiently large portion of the variance of
the data and, thus, to keep the corresponding results accurate; we set it to
0.99 preserving 99\% of this variance. The resulting dimensionality \nz of \vz
in \eref{inference-reduction} was found to be 27--28. The parameters $\mu_0$ and
$\tau_\g$ of the priors in \eref{mu-u-prior} and \eref{sigma2-u-prior},
respectively, are specific to the considered technological process; we set
$\mu_0$ to 17.5~nm and $\tau_\g$ to 2.25~nm. The parameters $\sigma_0$ and
$\nu_\g$ in \eref{mu-u-prior} and \eref{sigma2-u-prior}, respectively, determine
the precision of the information on $\mu_0$ and $\tau_\g$ and are set according
to the beliefs of the user; we set $\sigma_0$ to 0.45~nm and $\nu_\g$ to 10. The
latter can be thought of as the number of imaginary observations that the choice
of $\tau_\g$ is based on. The parameter $\tau_\epsilon$ in
\eref{sigma2-noise-prior} represents the precision (deviation) of the equipments
utilized to collect the data set $H$ and can be found in the technical
specification of these equipments; we set $\tau_\epsilon$ to 1~K. The parameter
$\nu_\epsilon$ in \eref{sigma2-noise-prior} has the same interpretation as
$\nu_\g$ in \eref{sigma2-u-prior}; we set it to 10 as well. In \eref{proposal},
$\nu$ and $\alpha$ are tuning parameters, which can be configured based on
experiments; we set $\nu$ to eight and $\alpha$ to 0.5. The number of sample
draws is another tuning parameter, which we set to $10^4$; the first half of
these samples is ascribed to the burn-in period leaving $5 \cdot 10^3$ effective
samples $n$. For the optimization in \sref{optimization}, we use the
Quasi-Newton algorithm \cite{press2007}. For parallel computations, we utilize
four processors. All the experiments are conducted on a GNU/Linux machine with
Intel Core i7 2.66~GHz and 8~GB of RAM.

To ensure that the experimental setup is adequate, we first perform a detailed
inspection of the results obtained for one particular example with the default
configuration. The true and inferred distributions of the \ac{QOI} are shown in
\fref{wafer-qoi} where the normalized root-mean-square error (NRMSE) is below
2.8\%, and the absolute error is bounded by 1.4$\,$nm, which suggests that the
framework produces a close match to the true value of the \ac{QOI} We have also
looked at the behavior of the constructed Markov chains and the quality of the
proposal distribution; however, due to the shortage of space, these results are
not presented here. All the observations suggest that the optimization and
sampling procedures are properly configured.

Next we use the assessed configuration and alter only one parameter at a time:
the number of measured sites/dies $\ndm$; the number of processing
elements/measured points $\np$ on a site; the amount of data per measurement
point \ns; and the noise deviation $\sigma_\epsilon$.

\subsection{Number of Measured Sites}

Let us change the number of dies $\ndm$ that have been measured. The considered
scenarios are 1, 10, 20, 40, 80, and 160 measured dies, respectively. The
results are reported in \tref{spatial-measurements}. In this and the following
tables, we report the optimization (Stage~2 in \fref{algorithm}) and sampling
(Stage~3 in \fref{algorithm}) times separately (given in minutes). In addition,
the sampling time is given for two cases: sequential and parallel computing,
which is followed by the total time and error (NRMSE). The computational time of
the post-processing phase (Stage~4 in \fref{algorithm}) is not given as it is
negligibly small. The sequential sampling time is the most representative
indicator of the computational complexity scaling as the number of samples is
always fixed, and there is no parallelization; thus, we shall watch this value
in most of the discussions below (highlighted in bold).

We see in \tref{spatial-measurements} that the more data the proposed framework
needs to process, the longer the execution times, which is reasonable. The
trend, however, is rather modest: with the doubling of $\ndm$, all the
computational times increase less than two times. The error firmly decreases and
drops below 4\% with around 20 sites measured, which is only 6.3\% of the total
number of dies on the wafer.

\subsection{Number of Measured Points Per Site}

Here we consider five platforms with the number of processing
elements/measurement points $\np$ on each die equal to 2, 4, 8, 16, and 32,
respectively. The results are summarized in \tref{processing-elements}. All the
computational times grow with $\np$. This behavior is expected as the
granularity of the utilized thermal model (see \sref{data-model} and
\cite{ukhov2012}) is bound to the number of processing elements; therefore, the
temperature simulations become more intensive. Nevertheless, even for large
examples, the timing is readily acceptable, taking into account the complexity
of the inference procedure behind and the yielded accuracy. An interesting
observation can be made from the NRMSE: the error tends to decrease as $\np$
grows. The explanation is that, with each processing element, $H$ delivers
more information to the inference to work with since the temperature profiles
are collected for all the processing elements simultaneously.

\subsection{Amount of Data Per Measured Point}

In this subsection, we sweep the number of moments of time \ns captured by the
measured temperature profiles. The scenarios are 1, 10, 20, 40, 80, and 160 time
moments, respectively. The results are aggregated in
\tref{temporal-measurements}. As we see, the growth of the computational time is
relatively small. One might have expected this growth due to \ns to be the same
as the one due to $\np$ since, formally, the influence of $\np$ and \ns on the
dimensionality of $H$ is identical (recall $\vh^\measured \in \real^{\ndm \np
\ns}$). However, the meaning of the two numbers, \np and \ns, is completely
different, and, therefore, the way they manifest themselves in the algorithm is
also different. Therefore, the corresponding amounts of extra data are being
treated differently leading to the discordant timing shown in
\tref{processing-elements} and \tref{temporal-measurements}. The NRMSE in
\tref{temporal-measurements} has a decreasing trend; however, this trend is less
steady than the ones discovered before. The finding can be explained as follows.
The distribution of the time moments in $H$ changes since these moments are kept
evenly spaced across the corresponding time spans of the input power profiles.
Some moments of time can be more informative than the other. Hence, more or less
representative samples can end up in $H$ helping or misleading the inference. We
can also conclude that a larger number of spatial measurements is more
advantageous than a larger number of temporal measurements.

\subsection{Deviation of the Measurement Noise}

Next we vary the standard deviation of the noise (in Kelvins), affecting the
data $H$, within the set $\{ 0, 0.5, 1, 2 \}$ coherent with the literature
\cite{mesa-martinez2007}. Note that the corresponding prior distribution in
\eref{sigma2-noise-prior} is kept unchanged. The results are given in
\tref{noise-deviation}. The sampling time is approximately constant. However, we
observe an increase of the optimization time with the decrease of the noise
level, which can be ascribed to wider possibilities of perfection for the
optimization procedure. A more important observation, revealed by this
experiment, is that, in spite of the fact that the inference operates on
indirect and drastically incomplete data, a thoroughly calibrated equipment can
considerably improve the quality of predictions. However, even with a high level
of noise of two degrees---meaning that measurements are dispersed over a wide
band of 8~K with a large probability of more than 0.95---the NRMSE is still only
4\%.

\subsection{Sequential vs. Parallel Sampling}

Let us summarize the results of the sequential and parallel sampling strategies.
In the sequential MH algorithm, the optimization time is typically smaller than
the time needed for drawing posterior samples. The situation changes when
parallel computing is utilized. With four parallel processors, the sampling time
decreases 3.81 times on average, which indicates good parallelization properties
of the chosen sampling strategy. The overall speedup ranges from 1.49 to 2.75
with the average value of 1.77 times, which can be pushed even further employing
more parallel processors.

\section{Conclusion}
\slab{inference-conclusion}

We proposed a framework for the analysis of process variation across
semiconductor wafers based on cost-efficient, indirect measurements. The
technique was exposed to an extensive study of various aspects concerning its
implementation. The obtained results support the computational efficiency and
accuracy of our approach.

We would like to note that, although the framework was demonstrated on the
effective channel length and temperature, it can be readily utilized to analyze
any other \acp{QOI} based on any other \acp{QOM}.

The framework is capable of quantifying the primary parameters affected by
process variation such as the effective channel length, which is in contrast
with the former techniques wherein only secondary parameters were considered
such as the leakage current. Instead of taking direct measurements of the
quantity of interest, we employ Bayesian inference to draw conclusions based on
indirect observations such as on temperature. The proposed approach has low
costs since no deployment of expensive test structures might be needed or only a
small subset of the test equipments already deployed for other purposes might
need to be activated. The experimental results present an assessment of our
framework for a wide range of configurations.

Finally, we would like to emphasize that temperature is just one option. In
certain situations, it might be preferable to perform the above inference based
on measurements of some other auxiliary quantity \h provided that it depends on
the one that we wish to characterize, that is, on \g. For example, \h can be the
leakage current, which can be readily measured if adequate test structures have
already been deployed on the wafer for other purposes.

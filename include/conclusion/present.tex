Modern electronic systems are complex, and they operate in complex environments
and perform complex tasks. Design as well as analysis of such systems are
already highly challenging endeavors; however, their difficulty is considerably
escalated by the inevitable uncertainty associated with electronic systems.

The objective of this thesis has been mainly to assist the designer of
electronic systems by providing the tools that would allow the designer to
effectively and efficiently characterize uncertainty and subsequently take it
into account. To this end, we have developed a number of system-level techniques
for analyzing and designing under process, workload, and aging uncertainty.

In \cref{certainty-development}, we have elaborated on the Utopian deterministic
scenario, which has served as an adequate starting point for the developments in
the subsequent chapters. We have refined transient power and temperature
analysis and proposed a novel approach to performing dynamic steady-state
analysis. The latter has been illustrated in the context of reliability
optimization.

In \cref{uncertainty-process-fabrication}, we have considered the variability of
process parameters across silicon wafers engendered by process variation and
presented a versatile statistical framework for inferring this variability by
means of indirect measurements, which can potentially be incomplete and
corrupted by noise.

In \cref{uncertainty-process-development}, we have presented a technique for
studying diverse system-level quantities that are of interest to the designer
but are also uncertain due the variability stemming from process variation.
Examples include transient and dynamic steady-state power and temperature
profiles of the system under consideration as well as the system's maximum
temperature and energy consumption. The proposed approach has been applied to
energy optimization with reliability-related constraints. In this context,
reliability analysis has been enhanced in order to take account of the impact of
process uncertainty.

In \cref{uncertainty-workload-development}, we have developed another technique
for probabilistic analysis of system-level quantities that are of interest to
the designer. In this case, we have striven to provide a computationally
efficient and adequate characterization of the variability that originates from
workload uncertainty, which tends to be less regular than the one originating
from process uncertainty.

In \cref{uncertainty-workload-operation}, we have elaborated on the mitigation
of workload uncertainty at runtime in the context of resource management. More
specifically, we have performed an early investigation of the applicability of
advanced prediction techniques from the field of machine learning to the problem
of fine-grained long-range forecasting of resource usage in large computer
systems.

We have developed a number of system-level techniques for analyzing and
designing electronic systems under uncertainty stemming from the fabrication
process, workload, and aging. The features that are common to our solutions are
efficiency, generality, and straightforwardness of application.

In \cref{certainty-development}, we have elaborated on the deterministic
scenario, which has served as an adequate starting point for the developments in
the subsequent chapters. We have presented an auxiliary transformation of the
temperature model that simplifies certain operations associated with temperature
calculations. Furthermore, we have proposed a novel approach to dynamic
steady-state power and temperature analysis. The high accuracy and speed of our
technique make it readily applicable inside intensive design-space-exploration
loops, which has been illustrated by undertaking reliability optimization.

In \cref{uncertainty-process-fabrication}, we have considered the variability of
process parameters that is induced by process variation across silicon wafers.
In this context, we have presented a versatile statistical framework for
inferring the aforementioned variability by means of indirect measurements,
which can potentially be incomplete and corrupted by noise. The ability to work
with such measurements implies low costs, since there is no need for deploying
any specialized test structures, or only a small subset of test equipment might
need to be engaged.

In \cref{uncertainty-process-development}, we have presented a methodology for
studying diverse system-level quantities that are of interest to the designer,
but that are also uncertain due to process variation. Examples of such
quantities include transient and dynamic steady-state power and temperature
profiles of the system at hand as well as the system's maximum temperature and
total energy consumption. Our approach treats the quantity under analysis as an
opaque object, which makes it flexible and convenient to use, and delivers a
surrogate for this object that allows the designer to efficiently estimate
various probabilistic characteristics of this quantity. The efficiency makes the
proposed technique suitable for exploring the design space. This virtue has been
exemplified in the context of energy optimization with reliability-related
constraints, in which reliability analysis has been enhanced to consider process
uncertainty.

In \cref{uncertainty-workload-development}, we have described another technique
for probabilistic analysis of system-level quantities that are of interest to
the designer. In this case, we have striven to provide an adequate
characterization of the variability that originates from workload. This
variability tends to be less regular than the one stemming from the fabrication
process and, therefore, necessitates a different treatment. The presented
approach allows the designer to analyze such important quantities subject to
workload variation as end-to-end delays, energy consumption, and peak
temperatures. The experimental results have demonstrated that, given a fixed
budget of evaluations of the quantity under consideration, the designer can
benefit substantially in terms of accuracy per computation time by utilizing the
proposed technique instead of direct sampling.

In \cref{uncertainty-workload-operation}, we have elaborated on mitigation of
workload uncertainty at runtime in the context of resource management.
Specifically, we have performed an early investigation of the applicability of
advanced prediction techniques from the field of machine learning to the problem
of fine-grained long-range forecasting of resource usage in large computer
systems. The obtained results indicate that the real-life data which have been
studied possess certain regularities, and that these regularities can be modeled
by advanced machine-learning techniques and thereby utilized for making sensible
predictions.

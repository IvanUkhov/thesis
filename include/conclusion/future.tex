As many other bodies of research, the one presented in this thesis has a
beginning and an end. The latter rarely implies that the subject under
consideration has been exhaustively studied and requires no further
investigation. Instead, it merely means that the research undertaking has come
to a certain conclusion and made a certain contribution to the understanding or
treatment of the subject. Naturally, this advancement leaves a number of
peripheral questions open and poses new research questions, which is also the
case with our work.

Before we discuss particularities, let us first make one general remark. It
would be highly beneficial to investigate how our uncertainty-aware techniques
perform in practice. We have conducted such an investigation in the case of the
interpolation-based technique presented in
\cref{uncertainty-workload-development}; however, this investigation is
relatively small and has been done in an academic setting. Industrial settings
are different, and they might help to reveal blind spots.

The early investigation reported in \cref{uncertainty-workload-operation} is
arguably the most prominent direction for further development in the scope of
this thesis. In situations where workload can vary dramatically and without any
apparent regularity, mitigation of uncertainty at runtime is considered
advantageous, since one has access to large amounts of relevant and previously
inaccessible-by-definition information and hence can make more deliberate
decisions. As emphasized throughout \cref{uncertainty-workload-operation},
modern architectures of neural networks accompanied by modern approaches to
training have great potential, and resource management is the foremost
beneficiary in this regard. This potential, however, is yet to be materialized.
Extensive research in this direction is highly needed.

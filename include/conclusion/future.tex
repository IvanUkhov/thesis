As many other bodies of research, the one presented in this thesis has a
beginning and an end. The latter rarely implies that the subject under
consideration has been exhaustively studied and requires no further attention.
Instead, it merely means that the research undertaking has come to a certain
conclusion and made a certain contribution to the understanding or treatment of
this subject. Naturally, this can leave a number of research questions open and
is likely to pose new questions, which is also the case with our work.

Before we discuss particularities, let us first make one general remark about
our research. It would be valuable to investigate how our uncertainty-aware
techniques perform in practice. Although we have conducted such an investigation
in the case of the interpolation-based technique presented in
\cref{uncertainty-workload-development}, this study is relatively small and has
been done in an academic setting. Industrial settings are different, and they
might help to reveal our blind spots.

In the experiments reported in \cref{uncertainty-process-fabrication},
\cref{uncertainty-process-development}, and
\cref{uncertainty-workload-development}, we assume certain probability
distributions. Even though our assumptions are motivated, it would be helpful to
assess our solutions using probability distributions that are estimated based on
real-life measurements. In particular, as noted in the chapters, the assumed
correlations impact not only accuracy but also feasibility of applying our
techniques. The concern about feasibility is particularly acute in situations
where the curse of dimensionality is an issue, and this curse could be an issue
in \cref{uncertainty-process-development} and
\cref{uncertainty-workload-development}. Recall that the former leverages the
\ac{PC} decomposition while the latter makes use of adaptive hierarchical
interpolation. Having identified a number of problematic real-life use cases,
one could investigate how the proposed techniques should be configured or even
extended in order to make them applicable in those cases. For instance, the
attentive reader might have noted that the control over anisotropy which is
provided by our framework based on \ac{PC} expansions has not been sufficiently
explored in the experimental results. However, if used properly, it could
effectively mitigate the curse of dimensionality.

It would also be meaningful to perform an elaborate comparison and explore the
boundaries between the techniques proposed in
\cref{uncertainty-process-development} and
\cref{uncertainty-workload-development}. Our default policy is to use the former
for process uncertainty and the latter for workload uncertainty, which is
motivated by the response surfaces that are characteristic of the two types of
uncertainty. However, it does not mean that one of the techniques cannot be
successfully used in the primary application area of the other. For instance,
workload uncertainty might lead to smooth variations, in which case a polynomial
surrogate might be beneficial.

The early investigation reported in \cref{uncertainty-workload-operation} is
arguably the most prominent direction for further development in the scope of
this thesis. Mitigation of uncertainty at runtime is considered advantageous,
since one has access to large amounts of relevant and previously
inaccessible-by-definition information and, therefore, can make more educated
decisions. As emphasized throughout \cref{uncertainty-workload-operation},
modern architectures of neural networks accompanied by modern approaches to
training have great potential, and resource management is the foremost
beneficiary in this regard. This potential, however, is yet to be materialized.
Extensive research in this direction is highly needed.

In this section, we illustrate the importance of temperature analysis for
design-space exploration. Specifically, we consider reliability optimization and
demonstrate the utility of our solution to dynamic steady-state analysis
presented in \sref{dynamic-steady-state} for mitigating the fatigue due to
thermal cycling \cite{jedec2016}, which is one of the most common failure
mechanism. To this end, we develop a thermal-cycling-aware technique for
scheduling of periodic applications.

The remainder of the section is organized as follows. First, we consider a
motivational example in \sref{thermal-cycling-example}. The reliability model is
presented in \sref{reliability-model}. In \sref{thermal-cycling-problem}, we
formulate our optimization objective and, in \sref{thermal-cycling-solution},
describe the optimization technique that we utilize in order to attain it.
Lastly, the experimental results are reported in \sref{thermal-cycling-result}.

\subsection{Motivational Example}
\slab{thermal-cycling-example}

\inputfigure{thermal-cycling-task-graph}
\inputfigure{thermal-cycling-example}
We begin by giving a motivational example. Consider a periodic application with
six tasks denoted by T1--T6 and a heterogeneous platform with two processing
elements denoted by P1 and P2. The task graph of the application is given in
\fref{thermal-cycling-task-graph} along with the execution times of the tasks
for both processing elements. The period of the application is 60~ms. The first
alternative schedule and the resulting dynamic steady-state temperature profile
are shown on the left-hand side of \fref{thermal-cycling-example} where the
height of a task represents its power consumption. It can be seen that P1 is
experiencing three thermal cycles, that is, there are three intervals wherein
temperature starts from a certain value, reaches an extremum, and then comes
back. If we move T6 to P2, the number of cycles decreases to two, which can be
seen in the middle of \fref{thermal-cycling-example}. If we also swap T2 and T4,
the number of cycles of P1 drops to one, which is depicted on the right-hand
side. According to the reliability model that we shall describe shortly, these
two changes improve the lifetime of the system by around 45\% and 55\%,
respectively, relative to the initial schedule.

The above example shows that, when exploring the design space, it is important
to take into consideration the number of temperature fluctuations as well as
their characteristics. In order to acquire this information, dynamic
steady-state temperature analysis has to be undertaken.

\subsection{Reliability Model}
\slab{reliability-model}
\newcommand{\mean}{\mu}
\newcommand{\scale}{\eta}
\newcommand{\shape}{\beta}

The reliability model that we utilize is the one presented in \cite{huang2009b,
xiang2010}. The model relies heavily on Weibull distributions, which we overview
in brief now.

A Weibull distribution has two parameters: $\scale$ and $\shape$, which are
called the scale and shape parameters, respectively. Let $T$ be a random
variable that is distributed according to such a distribution, which is denoted
by $T \sim \mathrm{Weibull}(\scale, \shape)$. Then the distribution function
\cite{durrett2010} of $T$ is
\begin{equation} \elab{weibull-distribution}
  F(t) = 1 - \exp\left(-\left(\frac{t}{\scale}\right)^\shape\right);
\end{equation}
the complementary distribution function of $T$ is
\begin{equation} \elab{weibull-reliability}
  R(t) = 1 - F(t) = \exp\left(-\left(\frac{t}{\scale}\right)^\shape\right);
\end{equation}
and the expectation of $T$ is
\begin{equation} \elab{weibull-expectation}
  \mean = \expectation{T} = \scale \: \Gamma\left(1 + \frac{1}{\shape}\right)
\end{equation}
where $\Gamma$ is the gamma function. In this context of reliability analysis,
$T$ represents the time to failure of the component under consideration; $F$
gives the probability of failure before a certain time moment; $R$ gives the
probability of survival up to a certain time moment, and it is called the
reliability function; and $\mean$ corresponds to the \ac{MTTF}.

It is natural to expect that the distribution of $T$ is different for different
usage conditions, which is not prominent in \eref{weibull-distribution}. In
order to take this into account, the application period \period is split into
\ns time intervals $\{ \dt_i: i = \range{1}{\ns} \}$ so that the conditions that
are relevant to the model stay unchanged within each interval. Let $T_i \sim
\mathrm{Weibull}(\scale_i, \shape_i)$ be the time to failure that the component
would have if interval $i$ was the only interval present and denote by $\mean_i$
the corresponding \ac{MTTF} according to \eref{weibull-expectation}. In the case
of temperature-induced failures, we have that $\shape_i = \shape$ for $i =
\range{1}{\ns}$, that is, all the shape parameters are equal. The reason is
that, unlike the scale parameters, the shape parameters are independent of the
operating temperature \cite{chang2006}.

As shown in \cite{xiang2010}, in the above scenario, the reliability function of
the component can still be approximated by means of \eref{weibull-reliability}
by letting
\[
  \scale = \frac{\sum_{i = 1}^\ns \dt_i}{\sum_{i = 1}^\ns \frac{\Delta t_i}{\scale_i}}.
\]
Applying \eref{weibull-expectation} at the level of individual cycles, the
parameter is rewritten as
\[
  \scale = \frac{\sum_{i = 1}^\ns \dt_i}{\Gamma\left(1 + \frac{1}{\shape}\right) \sum_{i = 1}^\ns \frac{\Delta t_i}{\mean_i}}.
\]
Note that the reliability model in \eref{weibull-reliability} becomes fully
specified as soon as $\dt_i$ and $\mean_i$ are identified for $i =
\range{1}{\ns}$. This part of the model depends on the particular failure
mechanism considered. In the rest of this subsection, we shall tailor the model
to the thermal-cycling fatigue, which is of interest to us due to its prominent
dependence on temperature oscillations.

In the case of thermal cycling, the time intervals with constant relevant
conditions correspond to thermal cycles. In order to detect them in a given
temperature curve, we utilize the rainflow counting method \cite{xiang2010}. As
a result, a set of \ns thermal cycles is obtained. Each detected cycle is
characterized by a number of properties including its duration, which is the
desired $\dt_i$. Regarding the corresponding $\mean_i$, it can be expressed as
follows:
\[
  \mean_i = \nc_i \dt_i
\]
where $\nc_i$ stands for the mean number of such cycles to failure. This number
is estimated using a modified version of the Coffin--Manson equation with the
Arrhenius term as shown in the following equation \cite{xiang2010, jedec2016}:
\begin{equation} \elab{reliability-mean-cycles}
  \nc = a (\Delta \q - \Delta \q_0)^{-b} \exp\left(\frac{c}{k \q_\maximum}\right)
\end{equation}
where $a$, $b$ (called the Coffin--Manson exponent), and $c$ (called the
activation energy) are empirically determined constants; $k$ is the Boltzmann
constant; $\Delta \q$ is the excursion of the cycle in question; $\Delta \q_0$
is the portion of the temperature excursion that resides in the elastic region,
which does not cause damage; and $\q_\maximum$ is the maximum temperature during
the cycle.

The reliability model of the abstract component under consideration is now fully
specified. The reliability function is the one in \eref{weibull-reliability}
with
\begin{equation} \elab{reliability-scale}
  \scale = \frac{\sum_{i = 1}^\ns \dt_i}{\Gamma\left(1 + \frac{1}{\shape}\right) \sum_{i = 1}^\ns \frac{1}{\nc_i}}
\end{equation}
where $\nc_i$ is as in \eref{reliability-mean-cycles}. Using
\eref{weibull-expectation}, the \ac{MTTF} of the
component is then
\begin{equation} \elab{reliability-mean-time}
  \mean = \frac{\sum_{i = 1}^\ns \dt_i}{\sum_{i = 1}^\ns \frac{1}{\nc_i}}.
\end{equation}

In conclusion, it should be noted that the reliability model requires detailed
information about the thermal cycles that the component is exposed to, which can
be obtained by performing dynamic steady-state temperature analysis.

\subsection{Problem Formulation}
\slab{thermal-cycling-problem}

A heterogeneous platform with \np processing elements is to execute a periodic
application with \nt tasks. The application is given as a directed acyclic graph
whose vertices and edges correspond to the tasks and to data dependencies
between the tasks, respectively; see \fref{thermal-cycling-task-graph} for an
example. The period of the application is denoted by \period and assumed to be
equal to the application's deadline. Each pair of a task and a processing
element is characterized by an execution time and a power consumption. Lastly,
the application is assumed to be scheduled using a list scheduler
\cite{adam1974} whose inputs are a set of priorities of the tasks and a mapping
of the tasks onto the processing elements.

The time to failure of each processing element is modeled using the reliability
model presented in \sref{reliability-model}. With a slight abuse of notation,
let $\mean_i$ be the \ac{MTTF} of processing element $i$ computed according to
\eref{reliability-mean-time}. Our optimization objective is to find a schedule
that maximizes
\begin{equation} \elab{thermal-cycling-objective}
  \min_{i = 1}^\np \mean_i
\end{equation}
such that
\begin{align} \elab{thermal-cycling-constraints}
  \begin{split}
    & \max_{i = 1}^\nt t_{\ns_i} \leq \period \text{ and} \\
    & \norm[\infty]{\mq} \leq \q_\maximum
  \end{split}
\end{align}
where $t_{\ns_i}$ is the completion time of task $i$, \mq is the dynamic
steady-state temperature profile of the system, and $\norm[\infty]{\cdot}$
denotes the uniform norm. The first constraint enforces a deadline on the
schedule, and the second constraint enforces a maximum temperature $\q_\maximum$
on the resulting \mq.

\subsection{Our Solution}
\slab{thermal-cycling-solution}

The optimization is undertaken using a genetic algorithm \cite{schmitz2004}. In
this paradigm inspired by biology, a population of chromosomes, which represent
candidate solutions, is carried through a number of generations in order to
produce a chromosome with the best possible fitness, that is, a solution that
maximizes the objective function. The fitness function in this case is
\eref{thermal-cycling-objective}.

Each chromosome contains $2 \nt$ genes. The first \nt genes encode priorities of
the tasks, and the second \nt genes encode a mapping of the tasks onto the
processing elements; the usage of this information is to be discussed shortly.
The population contains $4 \nt$ chromosomes. In each generation, a number of
chromosomes are chosen for breeding by the tournament selection with the number
of competitors proportional to the population size. The chosen chromosomes
undergo the two-point crossover with probability 0.8 and the uniform mutation
with probability 0.01. The evolution mechanism follows the elitism model where
the best chromosome always survives. The stopping condition is the absence of
improvement within 200 successive generations.

The fitness of a chromosome is evaluated in a number of steps. First, the
chromosome is decoded, and the priorities of the tasks and their mapping onto
the processing elements are fed to the list scheduler. The scheduler produces a
schedule. If the schedule violates the deadline constraint given in
\eref{thermal-cycling-constraints}, the chromosome is penalized proportionally
to the amount of violation and is not further processed. Otherwise, based on the
parameters of the processing elements and tasks, a power profile \mp is
constructed, and the corresponding temperature profile \mq is computed using our
technique presented in \sref{dynamic-steady-state-solution}. If the profile
violates the temperature constraint in \eref{thermal-cycling-constraints}, the
chromosome is penalized proportionally to the amount of violation and is not
further processed. Otherwise, the \ac{MTTF} of each processing element is
estimated according to \eref{reliability-mean-time}, and the fitness function in
\eref{thermal-cycling-objective} is computed.

\subsection{Experimental Results}
\slab{thermal-cycling-result}

In this subsection, we proceed to the optimization itself. We shall first
consider a set of synthetic applications and then study a real-life one. The
general experimental setup is the same as the one described in
\sref{dynamic-steady-state-result}. All the configuration files used in the
experiments are available online at \cite{eslab2012}.

Heterogeneous platforms and periodic applications are generated randomly using
the \up{TGFF} tool \cite{dick1998} in such a way that the execution times of
tasks are uniformly distributed between 1 and 10~ms, and that the static power
accounts for 30--60\% of the total power consumption. The area of a processing
element is set to 4~mm\textsuperscript{2}. The modeling of the leakage power is
based on the linear approximation discussed in
\sref{power-temperature-interplay}. The maximum-temperature constraint
$\q_\maximum$ in \eref{thermal-cycling-constraints} is set to \celsius{100}. In
\eref{reliability-mean-cycles}, the Coffin--Manson exponent $b$ is set to 6, the
activation energy $c$ to 0.5, and the elastic region $\Delta \q_0$ to 0
\cite{jedec2016}; the coefficient of proportionality $a$ in
\eref{reliability-mean-cycles} is irrelevant since we are concerned with
relative improvements, which will be explained shortly.

The initial population of chromosomes is initialized partially randomly and
partially based on a temperature-aware heuristic proposed in \cite{xie2006},
which we shall refer to as the initial solution. The heuristic relies on spatial
temperature variations and tries to minimize the maximum temperature while
satisfying real-time constraints. This solution is also used to decide on the
deadline constraint \period in \eref{thermal-cycling-constraints}: it is set to
the duration of the initial schedule extended by 5\%. Furthermore, the initial
solution serves as a baseline for evaluating the performance of the solutions
delivered by our optimization procedure.

\inputtable{thermal-cycling-elements}
In the first set of experiments, we change the number of processing elements \np
while keeping the number of tasks \nt per processing element constant and equal
to 20. For each problem, we generate 20 random task graphs and compute the
average change in the \ac{MTTF} relative to the initial solution. In addition,
we calculate the average change in the energy consumption. The obtained results
are reported in \tref{thermal-cycling-elements}, which also shows the average
time taken by the optimization procedure. It can be seen that the
reliability-aware optimization increases the \ac{MTTF} by a factor of 13--40.
Even for large problems---such as those with 16 processing elements executing
320 tasks---a feasible schedule that significantly improves the lifetime of the
system can be found in an affordable time. Moreover, as shown in
\tref{thermal-cycling-elements}, the performed optimization does not impact much
the energy efficiency of the system.

\inputtable{thermal-cycling-tasks}
In the second set of experiments, we keep the quad-core platform and vary the
number of tasks \nt of the application. As before, for each problem, we generate
20 random task graphs and monitor the changes in the \ac{MTTF} and energy
consumption relative to the initial solution. The results can be seen in
\tref{thermal-cycling-tasks}. The observations are similar to those made with
respect to \tref{thermal-cycling-elements}.

\inputtable{thermal-cycling-methods}
The experiments show that our optimization is able to effectively increase the
\ac{MTTF} of the system at hand. The efficiency is due to the fast and accurate
approach to dynamic steady-state temperature analysis presented in
\sref{dynamic-steady-state-solution}, which is at the heart of the optimization
procedure. Due to its speed, the technique allows a large portion of the design
space to be explored. In order to illustrate this, let us replace, inside the
optimization framework, our method with \one~the one based on iterative
transient analysis with HotSpot and \two~the one based on static steady-state
analysis; see \sref{dynamic-steady-state-prior}. The goal is to compare the
results in \tref{thermal-cycling-tasks} with the results produced by the two
alternative methods when they are given the same amount of time as the one taken
by our method. For each problem, the optimized \ac{MTTF} produced by either of
the two approximate methods is re-evaluated by our exact method. The results are
summarized in \tref{thermal-cycling-methods} where \ac{MTTF}\textsubscript{i}
and \ac{MTTF}\textsubscript{ii} stand for the two alternative methods,
respectively. It can be seen that, for example, the lifetime of the platform
running 160 tasks can be extended by a factor of 18 using our technique whereas
the best solutions found by the other two techniques within the same time frame
are only around 2--5 times better than the initial solution. The reason for the
poor performance of iterative transient analysis with HotSpot is the excessively
long execution time of the calculation of dynamic steady-state temperature
profiles, which means that this method allows for a very shallow exploration of
the design space. In the case of static steady-state analysis, the reason is
different: the method is fast but also very inaccurate as it is discussed and
illustrated in \sref{dynamic-steady-state-prior}. The inaccuracy drives the
optimization towards solutions that turn out to be of low quality.

\inputfigure{thermal-cycling-pareto}
The experiments show that the impact of the optimization on the energy
consumption is insignificant. This is not surprising: the optimization searches
toward low-temperature solutions, which are also implicitly the low-leakage
ones. In order to explore this further, let us perform a multi-objective
optimization along the dimensions of energy and reliability, and let us use
\up{NSGA-II} \cite{deb2002}. The resulting Pareto front averaged over 20
applications with 80 tasks deployed on a quad-core platform is displayed in
\fref{thermal-cycling-pareto}. It can be observed that the variation of the
energy change is less than 2\%. This means that the solutions optimized for the
\ac{MTTF} have an energy consumption that is almost identical to the one of the
solutions optimized for energy. At the same time, the difference along the
\ac{MTTF} dimension is huge. This means that ignoring the reliability aspect the
designer may end up with a significantly decreased \ac{MTTF} without any
significant gain in the energy consumption.

Finally, we consider a real-life application, namely, an \up{MPEG-2} decoder
\cite{ffmpeg}, which is assumed to be deployed on a dual-core platform. The
decoder is analyzed and split into 34 tasks. The parameters of each task are
obtained through a system-level simulation using \up{MPARM} \cite{benini2005}.
The deadline is set to 40~ms assuming 25 frames per second. The solution found
by the proposed method improves the lifetime of the system by a factor of 23.59
with a 5\% energy saving compared to the initial solution. The solutions found
by undertaking the same optimization via the two alternative methods mentioned
earlier are only 5.37 and 11.50 times better than the initial one, respectively.

To conclude, the experimental experiments have demonstrated the superiority of
the proposed approach to dynamic steady-state temperature analysis in the
context of reliability optimization compared to the state-of-the-art.

\newcommand{\mean}{\mu}
\newcommand{\scale}{\eta}
\newcommand{\shape}{\beta}

The reliability model that we utilize is the one presented in \cite{huang2009b,
xiang2010}. The model relies heavily on Weibull distributions, which we overview
in brief now.

A Weibull distribution has two parameters: $\scale$ and $\shape$, which are
called the scale and shape parameters, respectively. Let $T$ be a random
variable that is distributed according to such a distribution, which is denoted
by $T \sim \mathrm{Weibull}(\scale, \shape)$. Then the distribution function
\cite{durrett2010} of $T$ is
\begin{equation} \elab{weibull-distribution}
  F(t) = 1 - \exp\left(-\left(\frac{t}{\scale}\right)^\shape\right);
\end{equation}
the complementary distribution function of $T$ is
\begin{equation} \elab{weibull-reliability}
  R(t) = 1 - F(t) = \exp\left(-\left(\frac{t}{\scale}\right)^\shape\right);
\end{equation}
and the expectation of $T$ is
\begin{equation} \elab{weibull-expectation}
  \mean = \expectation{T} = \scale \: \Gamma\left(1 + \frac{1}{\shape}\right)
\end{equation}
where $\Gamma$ is the gamma function. In this context of reliability analysis,
$T$ represents the time to failure of the component under consideration; $F$
gives the probability of failure before a certain time moment; $R$ gives the
probability of survival up to a certain time moment, and it is called the
reliability function; and $\mean$ corresponds to the \ac{MTTF}.

It is natural to expect that the distribution of $T$ is different for different
usage conditions, which is not prominent in \eref{weibull-distribution}. In
order to take this into account, the application period \period is split into
\ns time intervals $\{ \dt_i: i = \range{1}{\ns} \}$ so that the conditions that
are relevant to the model stay unchanged within each interval. Let $T_i \sim
\mathrm{Weibull}(\scale_i, \shape_i)$ be the time to failure that the component
would have if interval $i$ was the only interval present and denote by $\mean_i$
the corresponding \ac{MTTF} according to \eref{weibull-expectation}. In the case
of temperature-induced failures, we have that $\shape_i = \shape$ for $i =
\range{1}{\ns}$, that is, all the shape parameters are equal. The reason is
that, unlike the scale parameters, the shape parameters are independent of the
operating temperature \cite{chang2006}.

As shown in \cite{xiang2010}, in the above scenario, the reliability function of
the component can still be approximated by means of \eref{weibull-reliability}
by letting
\[
  \scale = \frac{\sum_{i = 1}^\ns \dt_i}{\sum_{i = 1}^\ns \frac{\Delta t_i}{\scale_i}}.
\]
Applying \eref{weibull-expectation} at the level of individual cycles, the
parameter is rewritten as
\[
  \scale = \frac{\sum_{i = 1}^\ns \dt_i}{\Gamma\left(1 + \frac{1}{\shape}\right) \sum_{i = 1}^\ns \frac{\Delta t_i}{\mean_i}}.
\]
Note that the reliability model in \eref{weibull-reliability} becomes fully
specified as soon as $\dt_i$ and $\mean_i$ are identified for $i =
\range{1}{\ns}$. This part of the model depends on the particular failure
mechanism considered. In the rest of this subsection, we shall tailor the model
to the thermal-cycling fatigue, which is of interest to us due to its prominent
dependence on temperature oscillations.

In the case of thermal cycling, the time intervals with constant relevant
conditions correspond to thermal cycles. In order to detect them in a given
temperature curve, we utilize the rainflow counting method \cite{xiang2010}. As
a result, a set of \ns thermal cycles is obtained. Each detected cycle is
characterized by a number of properties including its duration, which is the
desired $\dt_i$. Regarding the corresponding $\mean_i$, it can be expressed as
follows:
\[
  \mean_i = \nc_i \dt_i
\]
where $\nc_i$ stands for the mean number of such cycles to failure. This number
is estimated using a modified version of the Coffin--Manson equation with the
Arrhenius term as shown in the following equation \cite{xiang2010, jedec2016}:
\begin{equation} \elab{reliability-mean-cycles}
  \nc = a (\Delta \q - \Delta \q_0)^{-b} \exp\left(\frac{c}{k \q_\maximum}\right)
\end{equation}
where $a$, $b$ (called the Coffin--Manson exponent), and $c$ (called the
activation energy) are empirically determined constants; $k$ is the Boltzmann
constant; $\Delta \q$ is the excursion of the cycle in question; $\Delta \q_0$
is the portion of the temperature excursion that resides in the elastic region,
which does not cause damage; and $\q_\maximum$ is the maximum temperature during
the cycle.

The reliability model of the abstract component under consideration is now fully
specified. The reliability function is the one in \eref{weibull-reliability}
with
\begin{equation} \elab{reliability-scale}
  \scale = \frac{\sum_{i = 1}^\ns \dt_i}{\Gamma\left(1 + \frac{1}{\shape}\right) \sum_{i = 1}^\ns \frac{1}{\nc_i}}
\end{equation}
where $\nc_i$ is as in \eref{reliability-mean-cycles}. Using
\eref{weibull-expectation}, the \ac{MTTF} of the
component is then
\begin{equation} \elab{reliability-mean-time}
  \mean = \frac{\sum_{i = 1}^\ns \dt_i}{\sum_{i = 1}^\ns \frac{1}{\nc_i}}.
\end{equation}

In conclusion, it should be noted that the reliability model requires detailed
information about the thermal cycles that the component is exposed to, which can
be obtained by performing dynamic steady-state temperature analysis.

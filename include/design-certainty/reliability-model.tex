\newcommand{\mean}{\mu}
\newcommand{\scale}{\eta}
\newcommand{\shape}{\beta}

Let $(\Omega, \mathcal{F}, \probability)$ be a probability space as defined in
\sref{probability-theory} and let $T: \Omega \to \real$ be a random variable
that represents the lifetime of the system at hand. The lifetime is the time
until the system experiences a fault after which it no longer meets certain
requirements. Let also $F(\cdot | \vg)$ be the distribution function of $T$,
which gives the probability of failure before a certain time moment, where \vg
is a vector of parameters. The expectation $\expectation{T}$ is called the
\ac{MTTF}. Lastly, the complementary distribution function of $T$ is
\[
  R(t | \vg) = 1 - F(t | \vg),
\]
which, in the context of reliability analysis, gives the probability of survival
up to a certain time moment and is called the reliability function of the
system.

The lifetime of the system $T$ is a function of the lifetimes of the \np
processing elements that the system is composed of. Denote these individual
lifetimes by $\{ T_i: \Omega \to \real \}_{i = 1}^\np$. Each $T_i$ is
characterized by a physical model of wear \cite{jedec2016} that describes the
stress that the corresponding processing element is exposed to. Each $T_i$ is
also assigned an individual survival function $R_i(\cdot | \vg)$ modeling the
failures due to that stress. The structure of $R(\cdot | \vg)$ with respect to
$\{ R_i(\cdot | \vg) \}_{i = 1}^\np$ is problem specific, and it can be
especially diverse in the context of fault-tolerant systems. Hence, $R(\cdot |
\vg)$ is to be specified by the designer of the system. To give an example,
suppose that the failure of any of the \np processing elements makes the system
fail, and $\{ T_i \}_{i = 1}^\np$ are conditionally independent given the
parameters gathered in \vg. In this scenario,
\begin{align*}
  & T = \min_{i = 1}^\np T_i \text{ and} \\
  & R(t | \vg) = \prod_{i = 1}^\np R_i(t | \vg).
\end{align*}

\subsection{Thermal Fatigue}

Let us now describe one wide-spread reliability model \cite{huang2009b,
xiang2010}, which we use for characterizing the lifetime of a single processing
element. For convenience, this lifetime is denoted by $T$ instead of $T_i$. The
lifetime is assumed to have a Weibull distribution, which is denoted as follows:
\[
  T | \vg \sim \mathrm{Weibull}(\scale, \shape)
\]
where $\scale$ and $\shape$ are called the scale and shape parameters,
respectively, and $\vg = (\scale, \shape)$. The distribution function of $T$ is
\begin{equation} \elab{weibull-distribution}
  F(t | \vg) = 1 - \exp\left(-\left(\frac{t}{\scale}\right)^\shape\right);
\end{equation}
the survival function of the processing element is
\begin{equation} \elab{weibull-reliability}
  R(t | \vg) = 1 - F(t | \vg) = \exp\left(-\left(\frac{t}{\scale}\right)^\shape\right);
\end{equation}
and the corresponding \ac{MTTF} is
\begin{equation} \elab{weibull-expectation}
  \mean = \expectation{T} = \scale \: \Gamma\left(1 + \frac{1}{\shape}\right)
\end{equation}
where $\Gamma$ is the gamma function.

Let \period be the time horizon of interest. It is natural to expect that the
usage conditions and, hence, wear change over time. Then the distribution of $T$
should reflect this aspect, which is not yet prominent in
\eref{weibull-distribution}. In order to account for this, the time span is
split into \ns time intervals $\{ \dt_i: i = \range{1}{\ns} \}$ so that the
conditions that are relevant to the model stay unchanged within each interval.
Let $T_i | \vg_i \sim \mathrm{Weibull}(\scale_i, \shape_i)$ be the time to
failure that the processing element would have if interval $i$ was the only
interval present and denote by $\mean_i$ the corresponding \ac{MTTF} according
to \eref{weibull-expectation}. In the case of temperature-induced failures, we
have that $\shape_i = \shape$ for $i = \range{1}{\ns}$, that is, all the shape
parameters are equal. The reason is that, unlike the scale parameters, the shape
parameters are independent of the operating temperature \cite{chang2006}.

As shown in \cite{xiang2010}, in the above scenario, the reliability function of
the processing element can still be approximated by means of
\eref{weibull-reliability} by letting
\[
  \scale = \frac{\sum_{i = 1}^\ns \dt_i}{\sum_{i = 1}^\ns \frac{\Delta t_i}{\scale_i}}.
\]
Applying \eref{weibull-expectation} at the level of individual intervals,
$\scale$ is rewritten as
\[
  \scale = \frac{\sum_{i = 1}^\ns \dt_i}{\Gamma\left(1 + \frac{1}{\shape}\right) \sum_{i = 1}^\ns \frac{\Delta t_i}{\mean_i}}.
\]
Note that the reliability model in \eref{weibull-reliability} becomes fully
specified as soon as $\dt_i$ and $\mean_i$ are identified for $i =
\range{1}{\ns}$. This part of the model depends on the particular failure
mechanism considered, which we discuss next.

\subsection{Thermal-Cycling Fatigue}

Let us tailor the Weibull model to the thermal-cycling fatigue, which is of
interest to us due to its prominent dependence on temperature oscillations. In
this case, the time intervals with constant relevant conditions correspond to
thermal cycles. In order to detect them in a given temperature curve, the
rainflow counting method is utilized \cite{xiang2010}. As a result, a set of \ns
thermal cycles is obtained. Each detected cycle is characterized by a number of
properties including its duration, which is the desired $\dt_i$. Regarding the
corresponding $\mean_i$, it can be expressed as follows:
\[
  \mean_i = \nc_i \dt_i
\]
where $\nc_i$ stands for the mean number of such cycles to failure. This number
is estimated using a modified version of the Coffin--Manson equation with the
Arrhenius term as shown in the following equation \cite{xiang2010, jedec2016}:
\begin{equation} \elab{reliability-mean-cycles}
  \nc = a (\Delta \q - \Delta \q_0)^{-b} \exp\left(\frac{c}{k \q_\maximum}\right)
\end{equation}
where $a$, $b$ (called the Coffin--Manson exponent), and $c$ (called the
activation energy) are empirically determined constants; $k$ is the Boltzmann
constant; $\Delta \q$ is the excursion of the cycle in question; $\Delta \q_0$
is the portion of the temperature excursion that resides in the elastic region,
which does not cause damage; and $\q_\maximum$ is the maximum temperature during
the cycle.

The reliability model of a single processing element is now fully specified. The
reliability function is the one in \eref{weibull-reliability} with
\begin{equation} \elab{reliability-scale}
  \scale = \frac{\sum_{i = 1}^\ns \dt_i}{\Gamma\left(1 + \frac{1}{\shape}\right) \sum_{i = 1}^\ns \frac{1}{\nc_i}}
\end{equation}
where $\nc_i$ is as in \eref{reliability-mean-cycles}. Using
\eref{weibull-expectation}, the \ac{MTTF} of the processing element is
\begin{equation} \elab{reliability-mean-time}
  \mean = \frac{\sum_{i = 1}^\ns \dt_i}{\sum_{i = 1}^\ns \frac{1}{\nc_i}}.
\end{equation}

In conclusion, it should be noted that the reliability model requires detailed
information about the thermal cycles that the processing element is exposed to,
which is the topic of the subsequent sections.

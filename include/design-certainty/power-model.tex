The total power consumption of an electrical circuit consists of two main
components: dynamic and static. The dynamic power is due to the actual useful
work done by the circuit. It is modeled as follows:
\[
  \p_\dynamic = C f V^2
\]
where $C$ is the effective switched capacitance, $f$ is the frequency, and $V$
is the supply voltage of the circuit. In contrast, the static power is due to
various parasitic currents that cannot be entirely eliminated. One such current
is the leakage current, which by far dominates, especially in modern \up{CMOS}
transistors. The leakage and, by extension, static power are modeled as follows
\cite{liao2005}:
\begin{equation} \elab{static-power}
  \p_\static = \numberof{g} V I_0 \left(a \q^2 e^{\frac{\alpha V + \beta}{\q}} + b e^{\gamma V + \delta}\right)
\end{equation}
where $\q$ is the current temperature, $V$ is the current supply voltage, $I_0$
is the average leakage current at the reference temperature and reference supply
voltage, and $\numberof{g}$ is the number of gates in the circuit. Quantities
$a$, $b$, $\alpha$, $\beta$, $\gamma$, and $\delta$ are technology-dependent
constants, which can found in \cite{liao2005}.

It can be seen in the above equations that the dynamic power does not depend on
the operating temperature whereas the static power does. The dependence is
important; however, we shall ignore it for the moment and return to it in
\sref{power-temperature-interplay} in order to take it adequately into account.

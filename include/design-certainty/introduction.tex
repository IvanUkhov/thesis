Due to the rapidly increasing power densities, temperature has evolved into an
exigent concern of computer-system designs, and temperature analysis has become
an essential component of design workflows. However, this analysis has
traditionally been a computationally expensive operation. The long computation
times are severely limiting, especially when temperature analysis is to be
performed very many times for such purposes as temperature-aware exploration of
the design space. Therefore, there is a palpable need for developing efficient
techniques for temperature analysis of computer systems. With the above concerns
in mind, we focus on temperature analysis is this chapter.

There are two types of temperature analysis: \one~transient analysis and
\two~steady-state analysis. The first delivers the temperature that the system
at hand transitions over within an arbitrary time interval staring from
arbitrary initial conditions when it is exposed to an arbitrary, both spatially
and temporally, power distribution. The second delivers the temperature that the
system attains and retains once it has reached a thermal equilibrium under a
power distribution that is spatially arbitrary but temporally constant or
repeating.

Steady-state analysis can be further classified into two categories, which have
already been alluded to: \one~static analysis and \two~dynamic analysis. The
first is concerned with temporally constant power distributions, that is, with
those that do not change over time. In this case, temperature distributions do
not change over time either. The second is concerned with temporary arbitrary
power distributions that are periodic, that is, with those that repeat with
certain periods. In this case, temperature distributions change over time with
the same periods as the corresponding repeating power distributions.

The agenda for the rest of this chapter is as follows. In
\sref{temperature-model}, we present the general model that is used in all the
three aforementioned types of temperature analysis. Transient analysis, static
steady-state analysis, and dynamic steady-state analysis are then discussed
separately in \sref{transient-state}, \sref{static-steady-state}, and
\sref{dynamic-steady-state}, respectively. The interdependence between power and
temperature is elaborated on in \sref{power-temperature-interplay}. Lastly, the
importance of temperature analysis for design-space exploration is covered in
\sref{thermal-cycling-exploration}.

Computer systems are omnipresent and omniscient. They penetrate deep into
everyday life and are unsettlingly indispensable and increasingly intelligent at
the tasks entrusted to them. It is then readily understandable that the design
of computer systems is an acutely difficult and vastly far-reaching process.

One major concern of the designer of a computer system is the presence of
uncertainty, which, in many cases, is inherent and inevitable. A prominent
source of uncertainty is process variation: the process parameters of fabricated
nanoscale devices deviate from their nominal values since the contemporary
fabrication process cannot be controlled precisely down to the level of
individual atoms. Another source is performance variation: the performance of
the system under consideration degrades over time due to natural or accelerated
wear. Yet another and arguably the most consequential source of uncertainty is
workload variation: the actual runtime workload that the system will be exposed
to in a production environment is rarely, if ever, known in advance.

The aforementioned uncertainty can lead to degradation of the quality and
efficiency of service in the best case and to severe faults and burnt silicon in
the worst-case scenario. Therefore, it is crucial to acknowledge and analyze it
and to mitigate its deteriorating consequences by designing computer systems
well aware of uncertainty and well equipped with mechanisms for addressing it.

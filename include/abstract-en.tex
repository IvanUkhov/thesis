Computer systems are omnipresent and omniscient. They penetrate deep into
everyday life and are unsettlingly indispensable and increasingly intelligent at
the tasks entrusted to them. It is then readily understandable that the design
of computer systems is an acutely difficult and vastly far-reaching process.

One major concern of the designer of a computer system is the presence of
uncertainty due to such phenomena as process variation and workload variation.
In many cases, uncertainty is inherent and inevitable, and it leads to
degradation of the quality and efficiency of service in the best case and to
severe faults or burnt silicon in the worst-case scenario. Therefore, it is
crucial to acknowledge and analyze uncertainty and to mitigate its deteriorating
consequences by designing computer systems in such a way that they are well
aware of uncertainty and well equipped to effectively and efficiently tackle it.

In this work, we first consider techniques for deterministic analysis of certain
aspects of computer systems. These techniques do not take uncertainty into
account; however, they serve as a foundation for those that do. Our attention
revolves primarily around temperature analysis as temperature is of central
importance for attaining robustness and energy efficiency. We develop a novel
and fast approach to dynamic steady-state temperature analysis of multiprocessor
systems and apply it in the context of reliability optimization.

We then proceed to developing techniques that are particularly concerned with
uncertainty. The first technique of this kind is to infer the variability of
process parameters that is induced by process variation, across silicon wafers
based on nonintrusive measurements such as readings from thermal sensors. The
second technique is to quantify the variability of transient power and
temperature profiles of multiprocessor systems originated from process
variation. This technique is then extended to dynamic steady-state temperature
analysis and utilized for design-space exploration with probabilistic
constraints related to reliability. The next technique is to address less smooth
sources of uncertainty than process variation such as workload variation. The
technique is exemplified by quantifying the effect of random execution times on
the timing-, power-, and temperature-related characteristics of multiprocessor
systems.

In this section, our objective is to build a flexible and computationally
efficient model for reliability analysis of electronic systems affected by
process variation. The development here is based on the reliability model
described in \sref{reliability-model}, which is not aware of process variation
yet.

\begin{remark}
It is important to realize that there are two levels of probabilistic modeling
here. First, the reliability model itself is a probabilistic model describing
the lifetime of the system. Second, the parameterization \vg is another
probabilistic model characterizing the impact of the uncertainty due to process
variation on the reliability model. Thus, the overall model can be thought of as
a probability distribution over probability distributions. Given an outcome of
the manufacturing process and, hence, \vg, the lifetime remains random.
\end{remark}

\subsection{Our Solution}

Our work in this context is motivated by the following two observations.

First, as emphasized throughout the thesis, temperature is the driving force of
many failure mechanisms. The most prominent examples include electromigration,
time-dependent dielectric breakdown, stress migration, and thermal cycling
\cite{xiang2010}; see \cite{jedec2016} for an exhaustive overview. All of the
aforementioned mechanisms have strong dependencies on the operating temperature.
At the same time, temperature is tightly related to process parameters---such as
the effective channel length and gate-oxide thickness---and can vary
dramatically when those parameters deviate from their nominal values. Meanwhile,
the state-of-the-art techniques for reliability analysis of electronic systems
lack a systematic treatment of process variation and, in particular, of the
effect of process variation on temperature, which is also the case in
\sref{reliability-model}.

Second, having established a reliability model $R(\cdot | \vg)$ of the system
under consideration, the major portion of the associated computation time is
ascribed to the evaluation of the parameterization \vg rather than to the model
itself, that is, for a given \vg. For instance, \vg often contains estimates of
the \ac{MTTF} of each processing element for a range of stress levels.
Therefore, \vg typically involves computationally intensive full-system
simulations including power analysis paired with temperature analysis; see
\sref{reliability-model}.

Guided by the aforementioned observations, we propose to use \ac{PC} expansions
in order to construct a light surrogate for \vg. The proposed technique is
founded on the basis of the state-of-the-art reliability models by enriching
their modeling capabilities with respect to process variation and by speeding up
the associated computational process. This approach allows one to seamlessly
incorporate into reliability analysis the effect of process variation on process
parameters. In particular, the framework allows for a straightforward
propagation of the uncertainty from process parameters through power and
temperature to the lifetime of the system. In contrast to the straightforward
use of \ac{MC} sampling, the \ac{PC} expansions that we construct make the
subsequent analysis highly efficient from the computational perspective.

Note that $R(\cdot | \vg)$ is left intact, meaning that our approach does not
impose any restrictions on $R(\cdot | \vg)$. Thus, the designer can take
advantage of various reliability models in a straightforward manner. Naturally,
this also implies that the modeling errors associated with the chosen $R(\cdot |
\vg)$ can affect the quality of the results delivered by our technique.
Therefore, choosing an adequate reliability model for the problem at hand is a
responsibility of the designer.

Let us now illustrate how this approach works in practice. To this end, we
consider the thermal-cycling fatigue \cite{jedec2016}, which is introduced in
\sref{thermal-cycling-fatigue}. Recall that, in this case, the system is exposed
a periodic workload, and that the corresponding temperature profile is a dynamic
steady-state profile, which, in the deterministic case, can be computed as
described in \sref{dynamic-steady-state-solution}.

Assume that the structure of the reliability model is the one shown in
\eref{reliability-model} where each individual reliability function $R_i$ is the
one given in \eref{weibull-reliability} with its own $\eta_i$ and $\beta_i$ for
$i = \range{1}{\np}$. During each iteration, the temperature of processing
element $i$ exhibits \nk{i} cycles. Each cycle generally has different
characteristics and, therefore, causes different damage. This aspect is taken
into account by adjusting $\eta_i$ as shown in \eref{thermal-cycling-scale}. The
shape parameter $\beta_i$ is known to be indifferent to temperature. For
simplicity, let us also assume that $\beta_i$ does not depend on process
parameters, and that $\beta_i = \beta$ for $i = \range{1}{\np}$.

Under the above assumptions, \rref{weibull-homogeneity} applies, and the
lifetime $T: \Omega \to \real$ of the system has a Weibull distribution as
follows:
\[
  T | \vg \sim \mathrm{Weibull}(\eta, \beta)
\]
where $\eta$ is as in \rref{weibull-homogeneity} combined with
\eref{thermal-cycling-scale}. The high-level parameters of the above model are
two parameters: $\eta$ and $\beta$, among which only $\eta$ is uncertain.
Thus, $\vg = (\eta)$ in this particular scenario; $\beta$ is given implicitly.

Now we apply the technique in \sref{uncertainty-analysis} to this quantity. In
this case, Algorithm~G in \aref{surrogate-construction} is an auxiliary function
that makes a call to \aref{temperature-solution}, processes the resulting
temperature profile as it was described earlier in this subsection, and returns
$\eta$ computed according to the formula in \eref{compound-weibull-eta}.
Consequently, we obtain a light polynomial surrogate of the parameterization of
the reliability model, which can be then studied from various perspectives. The
example for a dual-core system given at the end of \sref{temperature-solution}
can be considered in this context as well with the only change that the
dimensionality of the polynomial coefficients would be two here (since $\eta \in
\real^\np$ and $\np = 2$).

In this section, our objective is to build a flexible and computationally
efficient technique for reliability analysis of electronic systems affected by
process variation. The development here is based on the general reliability
model $R(\cdot | \vg)$ described in \sref{reliability-analysis}, which is not
aware of process variation yet.

\subsection{Problem Formulation}

Our work in this context is motivated by two observations.

First, as emphasized throughout the thesis, temperature is the driving force of
many failure mechanisms. The most prominent examples include electromigration,
time-dependent dielectric breakdown, stress migration, and thermal cycling
\cite{xiang2010}; see \cite{jedec2016} for an exhaustive overview. All of the
aforementioned mechanisms have strong dependencies on the operating temperature.
At the same time, temperature is tightly related to process parameters---such as
the effective channel length and gate-oxide thickness---and can vary
dramatically when those parameters deviate from their nominal values. Meanwhile,
the state-of-the-art techniques for reliability analysis of electronic systems
lack a systematic treatment of process variation and, in particular, of the
effect of process variation on temperature, which is also the case in
\sref{reliability-analysis}.

Second, having established a reliability model $R(\cdot | \vg)$ of the system
under consideration, the major portion of the associated computation time is
ascribed to the evaluation of the parameterization \vg rather than to the model
itself, that is, for a given \vg. For instance, \vg often contains estimates of
the \ac{MTTF} of each processing element for a range of stress levels.
Therefore, \vg typically involves computationally intensive full-system
simulations including power analysis paired with temperature analysis; see
\sref{reliability-analysis}.

Guided by the aforementioned observations, we use \ac{PC} expansions in order to
construct a light surrogate for \vg. It is worth emphasizing that $R(\cdot |
\vg)$ is left intact, which means that our approach does not impose any
restrictions on $R(\cdot | \vg)$. Consequently, the designer can take advantage
of various reliability models in a straightforward manner. Naturally, this also
implies that the modeling errors associated with the chosen $R(\cdot | \vg)$ can
affect the quality of the results delivered by our technique. Therefore,
choosing an adequate reliability model for the problem at hand is a
responsibility of the designer.

\begin{remark}
It is important to realize that there are two levels of probabilistic modeling
here. First, $R(\cdot | \vg)$ itself is a probabilistic model describing the
lifetime of the system. Second, the parameterization \vg is another
probabilistic model characterizing the impact of the uncertainty due to process
variation on the reliability model. Thus, the overall model can be thought of as
a probability distribution over probability distributions. Given an outcome of
the manufacturing process and, hence, \vg, the lifetime remains random.
\end{remark}

To conclude, the quantify of interest \g, which is the output of Stage~1 in
\fref{chaos-overview}, is the parameters \vg of the chosen reliability model.

\subsection{Surrogate Construction}

Similar to \sref{chaos-dynamic-steady-state-analysis}, Stage~3 and Stage~4 in
\fref{chaos-overview} require no particular attention in the scope of this
section apart from noting that \vq is a vector-valued random variable, and,
therefore, \rref{chaos-multidimensional-output} should be taken into account.

In conclusion, the proposed approach to reliability analysis is founded on the
basis of the state-of-the-art reliability models, and it enriches their modeling
capabilities by seamlessly incorporating into the analysis the effect of process
variation on process parameters. In particular, the technique allows for a
straightforward propagation of the uncertainty from process parameters through
temperature---which is the driving force of many failure mechanisms---to the
lifetime of the system. In contrast to the straightforward use of \ac{MC}
sampling, the \ac{PC} expansions that we construct make the subsequent analysis
highly efficient from the computational perspective.

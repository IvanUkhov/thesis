Let $(\Omega, \mathcal{F}, \probability)$ be a probability space as defined in
\sref{probability-theory} and assume the same system model as the one described
in \sref{system-model}. The system depends on \nu process parameters that are
uncertain at the design stage. These parameters are modeled as a random vector
$\vu: \Omega \to \real^\nu$. Once the fabrication process yields a particular
outcome, \vu takes potentially different values across each fabricated chip and
stays unchanged thereafter. This variability leads to deviations of power from
the nominal values and, therefore, to deviations of temperature from the one
corresponding to the nominal power consumption. Given the above setting, our
goal in this chapter is twofold.

First, we are to develop a system-level framework for transient temperature
(and, hence, power) analysis as well as dynamic steady-state temperature
analysis of electronic systems where the power consumption and heat dissipation
are stochastic due to their dependency on the parameters \vu. The designer is
required to specify \one~the probability distribution of \vu and \two~the
dependency of the power consumption on \vu, which can be given as a ``black
box.'' The framework is to provide the designer with the tools for analyzing the
system under a given workload, without imposing constraints on the nature of
this workload, and obtaining the corresponding stochastic power \mp and
temperature \mq profiles with a desired level of accuracy and at low costs.

Second, taking into consideration the effect of process variation on power and
temperature, we are to find the reliability function of the system and to
develop a computationally efficient design-space exploration scheme exploiting
the proposed techniques for power, temperature, and reliability analysis.

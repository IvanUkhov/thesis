As emphasized in \sref{inference-introduction}, process variation is a major
concern of the designer of electronic systems. A design that is exclusively
based on deterministic techniques for power and temperature analysis is, in the
context of current and future technologies, both inefficient and unreliable
since the presence of the deteriorating uncertainty due to process variation is
disregarded. Under these circumstances, uncertainty quantification
\cite{maitre2010} has evolved into an indispensable asset of uncertainty-aware
design workflows in order to provide them with guaranties on the efficiency and
robustness of products.

In order to assist the designer, we develop a framework that allows the designer
to propagate the uncertainty due to process variation through the system under
development and, thereby, investigate and account for its impact on the system's
behavior. In particular, it enables power and temperature analysis as well as
reliability analysis and optimization under process variation.

We proceed as follows. A motivational example is given in \sref{chaos-example}.
The objective of our study is formulated in \sref{chaos-problem}.
\sref{chaos-prior} provides an overview of the prior work. The proposed
framework is delineated in \sref{chaos-solution} and
\sref{chaos-uncertainty-analysis}. The case of transient analysis is considered
in \sref{chaos-transient-analysis}, \sref{chaos-transient-application}, and
\sref{chaos-transient-result}. In \sref{chaos-dynamic-steady-state-analysis},
dynamic steady-state analysis is addressed. The approach to reliability analysis
is described in \sref{chaos-reliability-analysis}. In
\sref{chaos-reliability-optimization}, \sref{chaos-reliability-application}, and
\sref{chaos-reliability-result}, reliability optimization is undertaken.
\sref{chaos-conclusion} concludes the chapter.

First, we extend the deterministic transient and dynamic steady-state analysis
presented in \sref{transient-analysis} and \sref{dynamic-steady-state-analysis},
respectively, in order to account for the uncertainty due to process variation.
Second, we develop a framework for the reliability analysis of electronic
systems that enriches the state-of-the-art reliability models by taking into
consideration the effect of process variation on temperature. Third, we
construct a computationally efficient design-space exploration procedure
targeted at the minimization of the energy consumption, which is \emph{a priori}
random, under probabilistic constraints on the thermal behavior and lifetime of
the system.

We develop a framework for the analysis of the transient power and temperature
profiles of electronic systems subject to the uncertainty due to process
variation. The proposed technique is flexible in modeling diverse probability
distributions, specified by the designer, of the uncertain parameters, such as
the effective channel length and gate oxide thickness. Moreover, there are no
assumptions on the distributions of the resulting power and temperature traces
as these distributions are unlikely to be known \emph{a priori}. The proposed
technique is capable of capturing arbitrary joint effects of the uncertain
parameters on the system since the impact of these parameters is introduced into
the framework as a ``black box,'' which is also defined by the designer. In
particular, it allows for the power-temperature interplay to be taken into
account with no effort. Our approach is founded on the basis of \acf{PC}
expansions, which constitute an attractive alternative to \ac{MC} sampling. This
is due to the fact that \ac{PC} expansions possess much faster convergence
properties and provide succinct and intuitive representations of system
responses to stochastic inputs. In addition, we illustrate the framework
considering one of the most important parameters affected by process variation,
namely, the effective channel length. Note, however, that our approach is not
bound to any particular source of variability and, apart from the effective
channel length, can be applied to other process parameters such as the gate
oxide thickness.

The main idea is to construct a surrogate model for the joint power and
temperature model of the system using \ac{PC} expansions. Having constructed
this surrogate, such quantities as the \acf{CDF} and \acf{PDF} can be easily
estimated. Moreover, the representation that we compute provides analytical
formulae for probabilistic moments, which means that the expectation and
variance are readily available.

\inputfigure{chaos-overview}
The major stages of our technique are depicted in \fref{chaos-overview}. At
Stage~1 (\sref{chaos-probability-transformation}), the uncertain parameters \vu
are transformed into independent random variables \vz since independence is a
prerequisite of the \ac{PC} approach. At Stage~2 (\sref{chaos-power-model}), the
designer specifies the power model of the system via a ``black-box'' function
$f$ that computes the total power \vp for a particular temperature \vq and a
particular outcome of the parameters \vu. At Stage~3
(\sref{chaos-temperature-model}), based on \sref{transient-state-solution}, a
mathematical description of the thermal behavior of the system is obtained. At
Stage~4 (\sref{chaos-surrogate-model}), the surrogate model is computed by
traversing the desired time span and gradually constructing \ac{PC} expansions
of the stochastic power and temperature profiles. At Stage~5
(\sref{chaos-post-processing}), the computed \ac{PC} expansions are analyzed in
order to obtain the needed characteristics of the system such as \acp{PDF} and
moments.

Let us now detail each of the five stages of the proposed framework.

We develop a framework for the analysis of the transient power and temperature
profiles of electronic systems subject to the uncertainty due to process
variation. The proposed technique is flexible in modeling diverse probability
distributions, specified by the user, of the uncertain parameters, such as the
effective channel length and gate oxide thickness. Moreover, there are no
assumptions on the distributions of the resulting power and temperature traces
as these distributions are unlikely to be known \emph{a priori}. The proposed
technique is capable of capturing arbitrary joint effects of the uncertain
parameters on the system since the impact of these parameters is introduced into
the framework as a ``black box,'' which is also defined by the user. In
particular, it allows for the power-temperature interplay to be taken into
account with no effort. Our approach is founded on the basis of \acf{PC}
expansions, which constitute an attractive alternative to \ac{MC} sampling. This
is due to the fact that \ac{PC} expansions possess much faster convergence
properties and provide succinct and intuitive representations of system
responses to stochastic inputs. In addition, we illustrate the framework
considering one of the most important parameters affected by process variation,
namely, the effective channel length. Note, however, that our approach is not
bound to any particular source of variability and, apart from the effective
channel length, can be applied to other process parameters such as the gate
oxide thickness.

The main idea is to construct a surrogate model for the joint power and
temperature model of the system using \ac{PC} expansions. Having constructed
this surrogate, such quantities as the \acf{CDF} and \acf{PDF} can be easily
estimated. Moreover, the representation that we compute provides analytical
formulae for probabilistic moments, which means that the expectation and
variance are readily available.

\inputfigure{chaos-overview}
The major stages of our technique are depicted in \fref{chaos-overview}. At
Stage~1 (\sref{chaos-probability-transformation}), the uncertain parameters \vu
are transformed into independent random variables \vz since independence is a
prerequisite of the \ac{PC} approach. At Stage~2 (\sref{chaos-power-model}), the
user specifies the power model of the system via a ``black-box'' function $f$
that computes the total power \vp for a particular temperature \vq and a
particular outcome of the parameters \vu. At Stage~3
(\sref{chaos-temperature-model}), based on \sref{transient-state-solution}, a
mathematical description of the thermal behavior of the system is obtained. At
Stage~4 (\sref{chaos-surrogate-model}), the surrogate model is computed by
traversing the desired time span and gradually constructing \ac{PC} expansions
of the stochastic power and temperature profiles. At Stage~5
(\sref{chaos-post-processing}), the computed \ac{PC} expansions are analyzed in
order to obtain the needed characteristics of the system such as \acp{PDF} and
moments.

Let us now detail each of the five stages of the proposed framework.

\subsection{Probability Transformation}
\slab{chaos-probability-transformation}

Independence is required by \ac{PC} expansions. In general, the \nu individual
variables in \vu are dependent and, therefore, should be preprocessed in order
to fulfill the requirement. To this end, an adequate probability transformation
should be undertaken \cite{eldred2008}. Denote such a transformation by $\vu =
\transform{\vz}$ where $\vz: \Omega \to \real^\nz$ is a random vector with \nz
mutually independent components.

Correlated random variables can be transformed into linearly uncorrelated ones
via the \ac{KL} decomposition given in \eref{karhunen-loeve}. If, in addition,
the correlated variables form a Gaussian vector, the uncorrelated ones are also
mutually independent. In the general case (non-Gaussian), the most prominent
solutions to attain independence are the Rosenblatt \cite{rosenblatt1952} and
Nataf transformations \cite{liu1986}. Rosenblatt's approach is suitable when the
joint distribution function of \vu is known; however, such information is rarely
available. A set of marginal distributions and a correlation matrix are more
likely to be given, which are already sufficient for the Nataf transformation.
See \sref{probability-transformation} for more detail.

Apart from the extraction of the independent variables \vz, an essential
operation at this stage is model order reduction since the number of stochastic
dimensions---that is, the dimensionality of \vz---directly impacts the
complexity of the rest of the computations. This operation is often treated as a
part of the \ac{KL} decomposition, which is also covered in
\sref{probability-transformation}.

\subsection{Power Model}
\slab{chaos-power-model}

As stated in \sref{chaos-problem}, the user is supposed to decide on the power
model for the system, which we denote as follows:
\begin{equation} \elab{chaos-power-model}
  \vp(t) = f(t, \vq, \vu)
\end{equation}
where $f$ is a function that evaluates the power $\vp \in \real^\np$ of the
processing elements given their temperature $\vq \in \real^\np$ and parameters
$\vu \in \real^\nu$. It should be understood that \vp, \vq, and \vu are random
vectors in general, and that $f$ consumes $\vq(t, \omega)$ and $\vu(\omega)$ and
yields $\vp(t, \omega)$ for some particular outcome $\omega \in \Omega$.

The user can choose any $f$. It can be, for instance, a closed-form formula or a
piece of code. The only assumption we make about $f$ is that it is smooth in \vz
and belongs to $\L{2}(\Omega, \mathcal{F}, \probability)$ when it is viewed as a
random variable (see \sref{probability-theory}), which is generally applicable
to most physical systems \cite{xiu2010}. Note also that the operation performed
by this ``black box'' is purely deterministic. The definition of $f$ is flexible
enough to account for such effects as the interdependence between power and
temperature discussed in \sref{power-model}.

\subsection{Temperature Model}
\slab{chaos-temperature-model}

The temperature model is based on the one described in \sref{temperature-model},
and transient temperature analysis is based on the one presented in
\sref{transient-state-solution}. The major difference is that
\eref{temperature-model} implicitly operates on stochastic quantities.
Consequently, the recurrent solution in \eref{transient-state-recurrence}, that
is,
\begin{equation} \elab{chaos-recurrence-original}
  \vs_i = \m{E} \vs_{i - 1} + \m{F} \vp_i
\end{equation}
for $i = \range{1}{\ns}$ where $\vs_0 = \v{0}$, is such as well. In the
deterministic case, it can be readily employed to perform deterministic
transient power and temperature analysis via the techniques in
\sref{power-temperature-interplay}. In the stochastic case, however, the
situation is substantially different since $\vp_i$ and, consequently, $\vs_i$
and $\vq_i$ are probabilistic quantities. Moreover, at each step, $\vp_i$ is an
arbitrary transformation of the uncertain parameters \vu and stochastic
temperature $\vq_i$, which results in a multivariate random variable with a
generally unknown probability distribution. Furthermore, $\vp_i$, $\vq_i$,
$\vs_i$, and \vu are dependent random vectors since the first three are
functions of the last. Consequently, the operations involved in the recurrence
are to be performed on dependent random vectors with arbitrary probability
distributions, which, in general, have no closed-form solutions, which we
address in the following subsection.

\subsection{Surrogate Model}
\slab{chaos-surrogate-model}

The goal now is to transform the recurrence in \eref{chaos-recurrence-original}
in such a way that the distributions of power and temperature can be estimated
efficiently.

Due to the inherent complexity, uncertainty-quantification problems are
typically viewed as approximation problems: one constructs a computationally
efficient surrogate for the system under consideration and then studies this
light representation instead of the original system. Our solution is then to
construct such a surrogate for the power function $f$, which is then propagated
through the recurrence in \eref{chaos-recurrence-original} in order to obtain an
approximation for power and an approximation for temperature. To this end, we
employ \acf{PC} \cite{xiu2010}, which is introduce in \sref{polynomial-chaos}.
The technique decomposes stochastic quantities into infinite series of mutually
orthogonal polynomials of random variables. Such series are especially
attractive from the post-processing perspective as they are nothing more than
the familiar polynomials; therefore, \ac{PC} expansions are easy to interpret
and easy to evaluate.

The first step toward a polynomial expansion is the choice of a suitable
polynomial basis, which is typically made based on the Askey scheme of
orthogonal polynomials \cite{xiu2010}. The step is crucial as the rate of
convergence of \ac{PC} expansions closely depends on it. Although there are no
strict rules that guarantee the optimal choice \cite{knio2006}, there are best
practices saying that one should be guided by the probability distributions of
the random variables that drive the stochastic system at hand. For instance,
when a random variable follows a beta distribution, the Jacobi basis is worth
being tried first; on the other hand, the Hermite basis is preferable for
Gaussian distributions.

Having chosen an appropriate basis, we apply the \ac{PC} formalism to the power
term $\vp_i$ in \eref{chaos-recurrence-original} as shown in
\eref{chaos-expansion}. The result is
\[
  \chaos{\nz}{\lc}{\vp_i} = \sum_{\multiindex[j] \in \multiindices[J]{\lc}} \hat{\vp}_{i, \multiindex[j]} \psi_{\multiindex[j]}
\]
where $\{ \psi_{\multiindex[j]}: \real^\nz \to \real \}$ are the basis
polynomials; $\{ \hat{\vp}_{i, \multiindex[j]} \in \real^\np \}$ are the
corresponding coefficients; and the multi-index set $\multiindices[J]{\lc}$ is
given by \eref{chaos-multi-index-isotropic}.

It can be seen in \eref{chaos-recurrence-original} that, due to the linearity of
the operations involved in the recurrence, $\vs_i$ attains an expansion that has
the same structure as the one of $\vp_i$, and the recurrence can be rewritten as
follows:
\[
  \chaos{\nz}{\lc}{\vs_i} = \m{E} \, \chaos{\nz}{\lc}{\vs_{i - 1}} + \m{F} \, \chaos{\nz}{\lc}{\vp_i}
\]
for $i = \range{1}{\ns}$. Consequently, there are two concurrent \ac{PC}
expansions: one is for power, and one is for temperature; the two expansions
have the same basis but different coefficients. In order to understand the
structure of the above formula better, let us spell it out as
\[
  \sum_{\multiindex[j] \in \multiindices[J]{\lc}} \hat{\vs}_{i \multiindex[j]} \psi_{\multiindex[j]} =
  \sum_{\multiindex[j] \in \multiindices[J]{\lc}} \left(\m{E} \, \hat{\vs}_{i - 1, \multiindex[j]} + \m{F} \, \hat{\vp}_{i \multiindex[j]}\right) \psi_{\multiindex[j]}.
\]
Multiplying the above equation by each polynomial from the basis and making use
of the orthogonality property in \eref{chaos-orthogonality}, we obtain the
following recurrence:
\begin{equation} \elab{chaos-recurrence}
  \hat{\vs}_{i \multiindex[j]} = \m{E} \hat{\vs}_{i - 1, \multiindex[j]} + \m{F} \hat{\vp}_{i \multiindex[j]}
\end{equation}
for $i = \range{1}{\ns}$ and $\multiindex[j] \in \multiindices[J]{\lc}$.
Combining the recurrence with \eref{temperature-algebraic}, we obtain the
coefficients of the \ac{PC} expansion of the temperature vector $\vq_k$.

Let us now find the coefficients $\{ \hat{\vp}_{i \multiindex[j]} \}$ of the
\ac{PC} expansion of power, which will subsequently be propagated to
temperature. The coefficients are computed using spectral projections as shown
in \eref{chaos-projection}. In general, the inner product in
\eref{chaos-projection} should be evaluated numerically as it is described in
\sref{numerical-integration}. To this end, an adequate quadrature
$\quadrature{\nz}{\lq}$ should be used. The result is
\begin{equation} \elab{chaos-coefficient}
  \hat{\vp}_{i \multiindex[j]} = \quadrature{\nz}{\lq}{\vp_i \psi_{\multiindex[j]}}
\end{equation}
for $i = \range{1}{\ns}$ and $\multiindex[j] \in \multiindices[J]{\lc}$ where
the integration should be understood element-wise. It is worth emphasizing that,
since power depends on temperature as shown in \sref{power-model}, at each step
of the recurrence in \eref{chaos-recurrence}, the computation of $\hat{\vp}_{i
\multiindex[j]}$ should be done with respect to the \ac{PC} expansion of $\vq_{i
- 1}$.

The construction process of the stochastic power and temperature profiles is
estimated to have the following time complexity per time step:
\[
  \bigo{n_n^2 \nc + \nn \np \nq \nc + \nq f(\np)}
\]
where \nn, \np, \nc, and \nq are the number of thermal nodes, processing
elements, polynomial terms, and quadrature points, respectively, and $f(\np)$
denotes the contribution of the power model in \eref{chaos-power-model}. The
expression can be detailed further by expanding \nc and \nq. The formula of \nc,
which is the cardinality of $\multiindices[J]{\lc}$, is given in
\eref{chaos-multi-index-isotropic-length}; it behaves as $\nz^\nc / \nc!$ in the
limit with respect to \nz. For quadratures based on the full tensor product,
$\log(\nq) \propto \nz$, which means that the dependency of \nq on \nz is
exponential. It can be seen that the theory of \ac{PC} expansions suffers from
the curse of dimensionality \cite{eldred2008, xiu2010}: when \nz increases, the
number of polynomial terms as well as the complexity of the corresponding
coefficients exhibit a growth, which is exponential without special treatments.
The problem does not have a general solution and is one of the central topics of
many ongoing studies. We mitigate this issue by \one~keeping the number of
stochastic dimensions low using the model order reduction based on the \ac{KL}
decomposition discussed in \sref{probability-transformation} and \two~utilizing
efficient integration techniques discussed in \sref{numerical-integration}. In
particular, for sparse integration grids based on Gaussian quadrature rules,
$\log(\nq) \propto \log(\nz)$, which means that the dependency of \nq on \nz is
only polynomial \cite{heiss2008}.

Let us summarize this subsection. Recall the stochastic recurrence in
\eref{chaos-recurrence-original} where, in the presence of dependencies, an
arbitrary functional $\vp_k$ of the uncertain parameters \vu and stochastic
temperature $\vq_k$---which can be seen in \eref{chaos-power-model}---needs to
be evaluated and combined with another random vector $\vs_k$. This recurrence in
\eref{chaos-recurrence-original} has been replaced with a purely deterministic
recurrence in \eref{chaos-recurrence}. More globally, the whole system including
the heavy temperature model in \eref{temperature-model-original} has been
substituted with a light surrogate defined by a set of polynomials $\{
\psi_{\multiindex[j]} \}$ in \vz, a set of coefficients $\{ \hat{\vp}_{i
\multiindex[j]} \}$ for power, and a set of coefficient $\{ \hat{\vq}_{i
\multiindex[j]} \}$ for temperature. These quantities constitute a stochastic
power profile \mp and a stochastic temperature profile \mq, and these profiles
are ready to be analyzed, which we discuss in the following subsection.

Before we proceed, let us draw attention to the ease and generality of taking
uncertainty into consideration using the proposed approach. The above
description is delivered from any explicit formula of any particular uncertain
parameter. In contrast, the solutions from the literature related to process
variation are typically based on ad~hoc expressions and should be tailored by
the user for each new parameter individually; see, for instance,
\cite{ghanta2006, bhardwaj2008, huang2009a}. Our framework provides a great
flexibility in this regard.

\subsection{Post-Processing}
\slab{chaos-post-processing}

Due to the properties of \ac{PC} expansions---in particular, due to the mutual
orthogonality of the basis functions discussed in \sref{polynomial-chaos}---the
obtained polynomial representations allow for various prospective analyses to be
performed with no effort. For instance, consider the expansion of temperature at
step $i$
\[
  \chaos{\nz}{\lc}{\vq_i} = \sum_{\multiindex[j] \in \multiindices[J]{\lc}} \hat{\vq}_{i \multiindex[j]} \psi_{\multiindex[j]}
\]
where $\{ \hat{\vq}_{i \multiindex[j]} \}$ are computed using
\eref{chaos-recurrence} and \eref{temperature-algebraic}. The above formula
given in terms of operators can be better understood when these operators are
applied to a particular \vz as follows:
\[
  \chaos{\nz}{\lc}{\vq_i}(\vz) = \sum_{\multiindex[j] \in \multiindices[J]{\lc}} \hat{\vq}_{i \multiindex[j]} \psi_{\multiindex[j]}(\vz).
\]
Let us now find, for example, the expectation and variance of the temperature at
this time moment. Since the first polynomial $\psi_{\multiindex[0]}$ in a
polynomial basis is unity by definition \cite{xiu2010},
$\expectation{\psi_{\multiindex[0]}(\vz)} = 1$. Therefore, using the
orthogonality property in \eref{chaos-orthogonality}, we conclude that
$\expectation{\psi_{\multiindex[j]}(\vz)} = 0$ for $\multiindex[j] \in
\multiindices[J]{\lc} \setminus \{ \multiindex[0] \}$. Consequently, the
expected value and variance have the following straightforward expressions
solely based on the coefficients:
\begin{equation} \elab{chaos-moments}
  \begin{split}
    & \expectation{\vq_i} = \hat{\vq}_{i \multiindex[0]} \text{ and} \\
    & \variance{\vq_i} = \sum_{\multiindices[J]{\lc} \setminus \{ \multiindex[0] \}} \hat{\vq}_{i \multiindex[j]}^2,
  \end{split}
\end{equation}
respectively, where the squaring should be understood element-wise. Such
quantities as \acp{CDF}, \acp{PDF}, and probabilities of certain events can be
estimated by means of \ac{MC} sampling applied to the expansion. In this case,
each sample is a trivial evaluation of a polynomial, and, hence, \ac{MC}
sampling is readily adequate. Furthermore, global and local sensitivity analysis
of deterministic and stochastic quantities can be readily conducted on a \ac{PC}
expansion.

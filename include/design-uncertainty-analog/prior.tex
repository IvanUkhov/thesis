Since the appearance of the first digital computers in 1940s, \acf{MC} sampling
remains one of the most well-known and widely used methods for the analysis of
stochastic systems. The reason for this popularity lies in the ease of
implementation, in the independence of the stochastic dimensionality of the
problem at hand, and in the fact that, by the law of large numbers
\cite{durrett2010}, the quantities estimated using \ac{MC} sampling
asymptotically approach the true values. The main problem with \ac{MC} sampling,
however, is the low rate of convergence: the error decreases as $\no^{-1/2}$
where \no is the number of drawn samples. This means that an additional decimal
point of accuracy requires hundred times more samples. Each such sample implies
a complete realization of the whole system, which renders \ac{MC}-based methods
slow and often unfeasible since the needed number of simulations can be
extremely large \cite{diaz-emparanza2002}. There are other sampling techniques
that have better convergence rates than the one of the classical \ac{MC}
sampling such as quasi-\ac{MC} sampling; however, due to additional
restrictions, their applicability is often limited \cite{xiu2010}.

In order to overcome the limitations of deterministic techniques and, at the
same time, to eliminate or, at least, mitigate the computation costs associated
with \ac{MC} sampling, a number of stochastic techniques have been introduced,
which we elaborate on in what follows. We are particularly interested in power
and temperature and, therefore, shape the exposition accordingly. Since the
static component of the total power dissipation is influenced by process
variation the most, which is due to the leakage current, the techniques
discussed below primarily focus on the variability of this component.

A solely power-targeted but temperature-aware solution is proposed in
\cite{chandra2010} where the driving force of the analysis is \ac{MC} sampling
with partially precomputed data. A learning-based approach is presented in
\cite{juan2011} in order to estimate the maximum temperature under the static
steady-state condition; recall \sref{static-steady-state}. Temperature-related
issues originating from process variation are also considered in \cite{juan2012}
where a statistical model of the static steady-state temperature based on
Gaussian distributions is derived. A statistical simulator of the static
steady-state temperature is developed in \cite{huang2009a} using \ac{PC}
expansions and the \acf{KL} decomposition; see
\sref{probability-transformation}. A \ac{KL}-aided stochastic collocation
\cite{xiu2010} approach to static steady-state temperature analysis is presented
in \cite{lee2013}. In \cite{shen2009}, \ac{PC} expansions are employed in order
to estimate the leakage power of the entire chip. The \ac{KL} decomposition is
utilized in \cite{bhardwaj2006} for leakage calculations. In
\cite{bhardwaj2008}, the total leakage is quantified using the \ac{PC} and
\ac{KL} methods. The same combination of tools is employed in
\cite{vrudhula2006} and \cite{ghanta2006} in order to analyze the response of
interconnect networks and power grids, respectively, under process variation.

The last five of the aforementioned techniques, that is, \cite{bhardwaj2006,
vrudhula2006, ghanta2006, bhardwaj2008, shen2009}, perform only stochastic power
analysis and ignore the interdependence between power and temperature described
in \sref{power-model}. The others are temperature-related approaches, but none
of them attempts to tackle stochastic transient temperature analysis and to
compute the probability distribution of temperature that evolves over time.
However, such transient curves are of practical importance. First, certain
procedures cannot be undertaken without the knowledge of time-dependent
temperature variations; an example is reliability optimization based on the
thermal-cycling fatigue, which is discussed in
\sref{thermal-cycling-exploration}. Second, the static steady-state assumption
considered, for instance, in \cite{huang2009a, juan2011, juan2012, lee2013} can
rarely be justified since power profiles are not invariant in reality. In
addition, one frequently encounters the assumption that power and temperature
follow \emph{a priori} known probability distributions; Gaussian and log-normal
distributions are popular choices as in \cite{bhardwaj2006, srivastava2010,
juan2012}. However, this assumption often fails in practice---which is also
noted in \cite{juan2012} regarding the normality of the leakage current---due to
\one~the nonlinear dependence of power on process parameters and \two~the
nonlinear interdependence between power and temperature. In order to illustrate
this, we simulate $10^4$ times the example given in \sref{chaos-example}
assuming the widespread Gaussian model of the effective channel length and apply
the Jarque--Bera test of normality to the collected temperature directly as well
as after processing them with the log transformation. The null hypothesis that
the data are from an unspecified Gaussian distribution is firmly rejected in
both cases at the significance level of 5\%. Therefore, the two distributions
are neither Gaussian nor log-normal, which can also be seen in
\fref{chaos-example-density} described in \sref{chaos-result}.

To conclude, the prior techniques for stochastic power and temperature analysis
are restricted in use due to one or several of the following traits: based on
\ac{MC} simulations (potentially slow) \cite{chandra2010}, limited to power
analysis \cite{bhardwaj2006, ghanta2006, vrudhula2006, bhardwaj2008, shen2009,
chandra2010}, ignoring the power-temperature interplay \cite{bhardwaj2006,
ghanta2006, vrudhula2006, bhardwaj2008, huang2009a, shen2009}, limited to the
static steady-state temperature \cite{huang2009a, juan2011, juan2012, lee2013},
exclusive focus on the maximum temperature \cite{juan2011}, and \emph{a priori}
chosen distributions of power and temperature \cite{bhardwaj2006,
srivastava2010, juan2012}. Consequently, there is a lack of flexible techniques
for stochastic power and temperature analysis.

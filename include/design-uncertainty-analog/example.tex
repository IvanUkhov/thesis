\inputfigure{chaos-example}
Consider a quad-core architecture exposed to the uncertainty due to parameters
that affect the leakage current. Assume first that these parameters have nominal
values. We can then simulate the system under a certain workload and observe the
corresponding temperature profile; the experimental setup is to be detailed in
\sref{chaos-transient-application} and \sref{chaos-transient-results}. The
result is depicted in \fref{chaos-example} by a blue line, which corresponds to
the temperature of one of the processors. It can be seen that the temperature is
always below \celsius{90}. Let us now assume a mild deviation of the parameters
from the nominal values and simulate once again. The result is the orange line
in \fref{chaos-example}; the maximum temperature is approaching \celsius{100}.
Finally, we repeat the experiment considering a severe deviation of the
parameters and observe the yellow line in \fref{chaos-example}; the maximum
temperature is almost \celsius{110}. Suppose that the designer is tuning a
solution constrained by a maximum temperature of \celsius{90} and is guided
exclusively by the nominal parameters. In this case, even with mild deviations,
the circuits might be burnt. Another path that the designer can take is to
design the system for severe conditions. In this scenario, however, the system
might easily become too conservative, overdesigned. Consequently, such
uncertainty has to be addressed in order to pursue efficiency and robustness.
Nevertheless, the majority of the literature related to power, temperature, and
reliability analysis of electronic systems ignores this important aspect; see,
for instance, \cite{rao2009, rai2011, thiele2011}. This negligence is also
present in the analysis and optimization described in \cref{design-certainty},
and the goal of this chapter is to eliminate it in the case of the uncertainty
that is induced by process variation.

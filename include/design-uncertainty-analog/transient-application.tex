In this section, we consider a particular application of the proposed approach
to transient analysis given in \sref{chaos-transient-analysis} in order to
illustrate how it is supposed to be used in practice. We begin by describing the
considered scenario.

\subsection{Problem Formulation}

At Stage~1 in \fref{chaos-overview}, the quantity of interest \g is the
transient (temperature-aware) power and temperature profiles corresponding to a
given workload. Let us now specify the parameters \vu that make \g uncertain to
the designer.

As discussed in \sref{power-model}, the total dissipation of power is composed
of two major parts: dynamic and static. The influence of process variation on
the dynamic power is known to be negligibly small \cite{srivastava2010}; on the
other hand, the variability of the static power is substantial, in which the
subthreshold leakage current contributes the most \cite{juan2011, juan2012}.
Hence, we focus on the subthreshold leakage and, more specifically, on the
effective channel length denoted by \u since it has the strongest influence on
the leakage current and is severely deteriorated by process variation
\cite{chandrakasan2000}. In particular, \u also affects the threshold voltage.

\inputfigure{beta-gaussian}
It is well known that the dispersion of the effective channel length around its
nominal value resembles the bell shape of Gaussian distributions. Therefore,
such variations are often conveniently modeled using Gaussian random variables
\cite{bhardwaj2006, ghanta2006, huang2009a, shen2009, chandra2010,
srivastava2010, juan2011, juan2012, lee2013}. Due to both the underlying physics
and demonstration purposes, we make a step further and bake into the model the
fact that the effective channel length---occupying the space between the drain
and source of a transistor---cannot be arbitrarily large or take a negative
value, which Gaussian distributions allow it to do. In other words, we require
the model of \u to have a bounded support. With this in mind, we propose to
model the effective channel length and other physically-bounded parameters using
the four-parametric family of beta distributions as
\begin{equation} \elab{beta-distribution}
  \u \sim \mathrm{Beta}(a, b, c, d)
\end{equation}
where $a$ and $b$ are the shape parameters, and $c$ and $d$ are the left and
right bounds of the support, respectively. The parameters $a$ and $b$ can be
chosen so that the typically found bell shape of the distribution is preserved.
An illustration is given in \fref{beta-gaussian} where a beta distribution is
fitted to the standard Gaussian distribution; alternatively, one can match
probabilistic moments. It can be seen that the curves are nearly
indistinguishable, but the beta one has a bounded support $[-4, 4]$, which can
potentially lead to more realistic models.

The variability of $u$ is split into global $u_\overall$ and local $u_\local$
parts \cite{shen2009, chandra2010}. Without loss of generality, $u_\overall$ can
be treated as a composition of independent inter-lot, inter-wafer, and inter-die
variations; likewise, $u_\local$ can be treated as a composition of independent
and dependent local variations. The variability $u_\overall$ is assumed to be
shared by all the \np processing elements whereas each processing element has
its own local parameter $u_{\local, i}$. Therefore, the effective channel length
of processing element $i$ is modeled as follows:
\[
  u_i = u_\nominal + u_\overall + u_{\local, i}
\]
where $u_\nominal$ is the nominal value of the effective channel length. Hence,
the uncertain parameters of the problem are
\[
  \vu = (u_{\local, 1}, \dotsc, u_{\local, \np}, u_\overall): \Omega \to \real^{\np + 1}.
\]

Global variations are typically assumed to be uncorrelated with respect to the
local ones. The latter, however, are known to have high spatial correlations.
Similar to \cref{analysis-uncertainty-analog}, these correlations are modeled
here using the correlation function given in \eref{inference-correlation}. This
choice is guided by the observations of the correlation structured induced by
the manufacturing process \cite{friedberg2005, chandrakasan2000, cheng2011}:
$k_\SE$ imposes similarities between the spatial locations that are close to
each other, and $k_\OU$ imposes similarities between the locations that are at
the same distance from the center of the die; see also \cite{ghanem1991,
ghanta2006, bhardwaj2008, huang2009a, lee2013}.

Although \eref{inference-correlation} captures certain features inherent to the
manufacturing process, it is still an idealization. In practice, it can be
difficult to make a justifiable choice and tune such a formula, which is a
prerequisite for the techniques that are discussed in \sref{chaos-prior} and
based on the continuous \ac{KL} decomposition. A correlation matrix, on the
other hand, can readily be estimated from measurements and, thus, is a more
probable input to probabilistic analysis. Therefore, we use
\eref{inference-correlation} with the only purpose of constructing a correlation
matrix of $\{ u_{\local, i} \}_{i = 1}^\np$. For convenience, the resulting
matrix is extended by one dimension in order to accommodate $u_\overall$ along
with $\{ u_{\local, i} \}_{i = 1}^\np$. In this case, the correlation matrix
obtains one additional nonzero diagonal element equal to unity; the result is
the correlation matrix of \vu denoted by $\correlation{\vu}$.

Let us now be more specific about the power model in \eref{chaos-power-model}.
The equation is decomposed into the sum of the dynamic and static components
\[
  f(t, \vq, \vu) = f_\dynamic(t) + f_\static(\vq, \vu)
\]
where $f_\dynamic: \real \to \real^\np$ and $f_\static: \real^\np \times
\real^\nu \to \real^\np$.

\begin{remark}
The function $f_\dynamic$ has no dependency on \vu as the influence of process
variation on the dynamic power is known to be negligibly small
\cite{srivastava2010}. On the other hand, the variability of the static power
given by $f_\static$ is substantial and magnified by the power-temperature
interplay discussed in \sref{power-temperature-interplay}.
\end{remark}

We assume that the desired workload is given as a dynamic power profile denoted
by $\mp_\dynamic$. Similar to \sref{inference-results}, without loss of
generality, the development of the static part is based on \up{SPICE}
simulations of a reference electrical circuit composed of \up{BSIM4} devices
\cite{bsim} configured according to the 45-nm \up{PTM} \up{HP} model \cite{ptm}.
Specifically, we use a series of \up{CMOS} invertors. The simulations are
performed for a sufficiently wide fine-grained two-dimensional grid, the
effective channel length and temperature, and the results are tabulated. An
interpolation technique is then utilized whenever we need to evaluate the static
power for a particular point within the range of the grid.

Regarding temperature analysis, an adequate thermal \up{RC} circuit should be
constructed. Given the specification of the platform---including the floorplan
of the die and the configuration of the thermal package---the circuit is
constructed by HotSpot \cite{skadron2003} adhering to the structure described in
\sref{temperature-model}.

To conclude, the input to our analysis is composed of the marginal distributions
of the uncertain parameters \vu, which are beta distributions, and the
corresponding correlation matrix $\correlation{\vu}$. Let us now go over the
other stages of our methodology discussed in \sref{chaos-uncertainty-analysis}
and depicted in \fref{chaos-overview}.

\subsection{Probability Transformation}

At Stage~2 in \fref{chaos-overview}, \vu should be preprocessed in order to
extract a vector of mutually independent random variables denoted by \vz.
Following the guidance given in \sref{chaos-probability-transformation}, the
most suitable transformation for the ongoing scenario is the one given in
\eref{probability-transformation}, which we denote here by $\transform$; see
\eref{chaos-transformation}.

Since the number of stochastic dimensions, which is $\nu = \np + 1$ for \vu,
directly impacts the computational cost of \ac{PC} expansions, which is
elaborated on in \sref{chaos-surrogate-construction}, one should consider a
possibility for model order reduction before constructing \ac{PC} expansions. To
this end, the model order reduction described in
\sref{probability-transformation} is an essential part of and assumed to be
active in the above transformation. The reduce dimensionality is denoted by \nz.

The marginal distributions of \vz can be arbitrary as long as one can construct
a suitable polynomial basis as described in \sref{chaos-surrogate-construction}.
We let \vz have beta distributions, staying in the same family of distributions
with \vu.

\subsection{Surrogate Construction}

At Stage~3 in \fref{chaos-overview}, the uncertain parameters, power model, and
temperature model developed in the previous subsections are to be fused together
under the desired workload $\mp_\dynamic$ in order to produce the corresponding
stochastic power and temperature profiles denoted by \mp and \mq, respectively.

In the current scenario, the construction of \ac{PC} expansions is based on the
Jacobi polynomial basis as it is preferable in situations involving
beta-distributed parameters \cite{xiu2010}. To give an example, for a dual-core
platform, that is, $\np = 2$, with two stochastic dimensions, that is, $\nz =
2$, the second-level \ac{PC} expansion, that is, $\lc = 2$, of temperature at
step $i$ is as follows:
\begin{equation} \elab{chaos-expansion-example}
  \begin{split}
    \chaos{2}{2}{\vq_i}
    =    {} & \hat{\vq}_{i, (0, 0)} \, \psi_{(0, 0)} +
              \hat{\vq}_{i, (1, 0)} \, \psi_{(1, 0)} +
              \hat{\vq}_{i, (0, 1)} \, \psi_{(0, 1)} \\
    {} + {} & \hat{\vq}_{i, (1, 1)} \, \psi_{(1, 1)} +
              \hat{\vq}_{i, (2, 0)} \, \psi_{(2, 0)} +
              \hat{\vq}_{i, (0, 2)} \, \psi_{(0, 2)}
  \end{split}
\end{equation}
where the coefficients $\{ \hat{\vq}_{i \multiindex[j]} \}$ are vectors with two
elements corresponding to the two processing elements. Regarding the basis,
\begin{align*}
  & \psi_{(0, 0)}(\vz) = 1, \\
  & \psi_{(1, 0)}(\vz) = 2 z_1, \\
  & \psi_{(0, 1)}(\vz) = 2 z_2, \\
  & \psi_{(1, 1)}(\vz) = 4 z_1 z_2 \\
  & \psi_{(2, 0)}(\vz) = \frac{15}{4} z_1^2 - \frac{3}{4}, \text{ and} \\
  & \psi_{(0, 2)}(\vz) = \frac{15}{4} z_2^2 - \frac{3}{4}.
\end{align*}
The Jacobi polynomials have two parameters \cite{xiu2010}, and the ones shown
above correspond to the case where both parameters are equal to two. Such a
series might be shorter or longer depending on the accuracy requirements given
by \lc. The expansion for power has the same structure but different
coefficients.

The next step is to compute the coefficients of power $\{ \hat{\vp}_{i
\multiindex[j]} \}$ in \eref{chaos-recurrence}, which subsequently yield the
coefficients of temperature $\{ \hat{\vq}_{i \multiindex[j]} \}$. As discussed
in \sref{polynomial-chaos}, these computations involve multidimensional
integration with respect to the distribution of \vz, which should be done
numerically using a quadrature rule; recall \sref{chaos-surrogate-construction}.
When beta distributions are concerned, the natural choice is the Gauss--Jacobi
quadrature; see \sref{numerical-integration}. Then, using
\eref{quadrature-summation}, the coefficient are computed as shown in
\eref{chaos-coefficient}. It is important to note that \lq should be chosen in
such a way that the rule is exact for polynomials of a total order up to $2
\lc$, that is, twice the level of \ac{PC} expansions, which can be seen in
\eref{chaos-projection} \cite{eldred2008}. Therefore, $\lq \geq \lc + 1$ as the
quadrature is Gaussian.

To summarize, we have completed four out of five stages of the proposed
framework depicted in \fref{chaos-overview}. The result is a light surrogate for
the entire system. At each time step, the surrogate is composed of two
\np-valued polynomials---one is for power, and one is for temperature---which
are defined in terms of \nz mutually independent random variables.

\subsection{Post-Processing}

At Stage~4 in \fref{chaos-overview}, the constructed expansions are utilized in
order to assist the designer by analyzing the impact of process variation on
power- and temperature-related characteristics of the system. It can be seen in,
for example, \eref{chaos-expansion-example} that the surrogate model has a
negligibly small computational cost: for any outcome of \vz, one can easily
compute the corresponding temperature by plugging this outcome into
\eref{chaos-expansion-example}; the same applies to power. Hence, the
representation can be trivially analyzed in order to retrieve various statistics
about the system. Let us illustrate a few of them using the example in
\eref{chaos-expansion-example}.

\inputfigure{chaos-example-power}
\inputfigure{chaos-example-temperature}
Assume that the dynamic power profile $\mp_\dynamic$ is the one shown in
\fref{chaos-example-power}. Having constructed a surrogate with respect to this
profile, we can calculate, for instance, the expectation and variance of the
temperature that the system has at a certain time moment, which is a trivial
operation given the formulae in \eref{chaos-moments}. For the whole time span of
$\mp_\dynamic$, these quantities are plotted in \fref{chaos-example-temperature}
where the dashed lines correspond to one standard deviation above the
corresponding expectations. The displayed curves closely match those obtained
via \ac{MC} sampling with $\no = 10^4$ samples; however, our method takes less
than a second while \ac{MC} sampling takes more than a day as we shall see in
\sref{chaos-transient-results}. In addition, the \ac{PDF} of the temperature at
that time moment can also be estimated. This operation is undertaken by sampling
the surrogate, in which case we might obtain curves similar to those shown
\fref{chaos-example-density}, which is a part of a different example given in
\sref{chaos-transient-results}.

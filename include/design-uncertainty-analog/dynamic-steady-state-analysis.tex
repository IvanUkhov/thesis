In this section, we elaborate on dynamic steady-state analysis under process
variation. Unlike transient analysis, this analysis cannot be done one time step
at a time since the whole repetitive workload needs to be taken into account at
once in order to calculate the corresponding dynamic steady state. Consequently,
similar to the contrast between \sref{transient-analysis} and
\sref{dynamic-steady-state-analysis}, the solutions in
\sref{chaos-transient-analysis} and this section differ. However, they still
rely on the same strategy outlined in \sref{chaos-uncertainty-analysis} and
shown in \fref{chaos-overview}.

\subsection{Problem Formulation}

The system model is the same as the one in \sref{chaos-transient-analysis}. The
only difference is that the power model is more convenient to be defined as
follows:
\begin{equation} \elab{chaos-power-model-bulk}
  \mp = \f(\vu, \mq).
\end{equation}
The function $\f: \real^\nu \times \real^{\np \times \ns} \to \real^{\np \times
\ns}$ is supposed to return the periodic power profile that corresponds to the
desired periodic workload.

The solution to dynamic steady-state analysis is based on the one in
\sref{dynamic-steady-state-solution}; however, in the present context, the
power, temperature, and state vectors are random, which also concerns the
boundary condition in \eref{dynamic-steady-state-condition}.

\inputalgorithm{chaos-dynamic-steady-state-iterative}
The pseudocode of a procedure that delivers \mq is given in
\aref{dynamic-steady-state}. The algorithm does not account for the
interdependence between power and temperature, and this interdependence can be
addressed by means of one of the techniques presented in
\sref{power-temperature-interplay}. In the context of this chapter, the most
appropriate technique is the iterative approach illustrated in
\aref{dynamic-steady-state-iterative}, which we rewrite here as listed in
\aref{chaos-dynamic-steady-state-iterative}.

\begin{remark}
The application of the linear approximation described in
\sref{power-temperature-interplay} is not straightforward in this case. This
technique is suitable when the only varying parameter is temperature, and all
other parameters have nominal values. In that case, it is relatively easy to
decide on a representative temperature range and apply a curve-fitting
procedure. In this case, however, the power model has multiple parameters
stepping far from their nominal values.
\end{remark}

In this section, the quantity of interest \g is exactly the dynamic steady-state
temperature profile \mq (or the corresponding power profile \mp if preferable).
Therefore, the output of Stage~1 in \fref{chaos-overview} is
\aref{chaos-dynamic-steady-state-iterative}.

\subsection{Surrogate Construction}

At Stage~3 in \fref{chaos-overview}, relying on the output of Stage~2, the
procedure delineated in \sref{chaos-construction} is applied to
\aref{chaos-dynamic-steady-state-iterative} from Stage~1. The algorithm computes
\mp or \mq for a given \vu and returns it in a suitable format.

Following \rref{chaos-multidimensional-output} regarding the treatment of
vector-valued quantities, \g is treated as an $\np \ns$-element row vector, in
which case each coefficient $\hat{\g}_{\vi}$ in \eref{chaos-expansion} is also
such a vector. The projection matrix in \eref{chaos-projection-matrix} and,
consequently, \aref{chaos-construction} should be interpreted as follows: \vg is
an $\nq \times \np \ns$ matrix whose row $i$ is the temperature or power profile
computed at point $i$ of the quadrature in \eref{chaos-coefficient} and reshaped
into a row vector. Similarly, $\hat{\vg}$ should be understood as an $\nc \times
\np \ns$ matrix whose row $i$ is coefficient $i$ of the expansion in
\eref{chaos-expansion}; recall that an ordering is assumed to be imposed on the
index set.

The constructed expansion can now be post-processed as needed, which is already
the topic of Stage~4 in \fref{chaos-overview}; see \sref{chaos-processing}

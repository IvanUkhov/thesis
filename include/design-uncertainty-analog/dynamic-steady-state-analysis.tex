In this section, we detail our temperature analysis, which is suitable for
system-level studies. We shall cover both the transient and dynamic steady-state
scenarios as the former is a prerequisite for the latter. Since temperature is a
direct consequence of power, we begin with the power model utilized in the
proposed framework.

\subsection{Our Solution}

The temperature model is the one described in \sref{temperature-model}.

Let us fix $\omega \in \Omega$, meaning that \vu is assumed to be known, and
consider the system in \eref{thermal-model} as deterministic. In general,
\eref{thermal-model-inner} is a system of ordinary differential equations which
is nonlinear due to the power term, given in \eref{power-model} as an arbitrary
function. Hence, the system in \eref{thermal-model} does not have a general
closed-form solution. A robust and computationally efficient solution to
\eref{thermal-model} for a given \vu is an essential part of our probabilistic
framework. In order to attain such a solution, we utilize a numerical method
from the family of exponential integrators \cite{hochbruck2010}. The procedure
is described in \aref{temperature-solution}, and here we use the final result.

Let us move on to the dynamic steady-state case. The pseudocode of this
algorithm, which delivers the exact solution under the above assumptions, is
given in \aref{dynamic-steady-state}. In the pseudocode, auxiliary variables are
written with hats, and $\mq_\ambient$ is a matrix of the ambient temperature.

Let us now bring the leakage-temperature interdependence into the picture. This
procedure is illustrated in \aref{temperature-solution}.

In \aref{temperature-solution}, $\mp_\static(\vu, \mq)$ should be understood as
a call to a subroutine that returns an $\np \times \ns$ matrix wherein the $(i,
k)$th element is the static component of the power dissipation of the $i$th
processing element at the $k$th moment of time with respect to \vu and the
temperature given by the $(i, k)$th entry of \mq.

\begin{remark}
A widespread approach to account for leakage is to linearize it with respect to
temperature. As shown in \cite{liu2007}, already one linear segment can deliver
sufficiently accurate results. One notable feature of such a linearization is
that no iterating-until-convergence is needed in this case; see
\cite{ukhov2012}. However, this technique assumes that the only varying
parameter of leakage is temperature, and all other parameters have nominal
values. In that case, it is relatively easy to decide on a representative
temperature range and undertake a one-dimensional curve-fitting procedure with
respect to it. In our case, the power model has multiple parameters stepping far
from their nominal values, which makes it difficult to construct a good linear
fit with respect to temperature. Thus, in order to be accurate, we use a
nonlinear model of leakage.
\end{remark}

So far in this subsection, \vu has been assumed to be deterministic. Now we turn
to the stochastic scenario and let \vu be random. Then we apply
\aref{surrogate-construction} to one particular quantity of interest \g.
Specifically, \g is now the temperature profile \mq corresponding to a given
$\mp_\dynamic$. Since \mq is an $\np \times \ns$ matrix, following
\rref{multiple-dimensions}, \g is viewed as an $\np \ns$-element row vector, in
which case each coefficient $\hat{\g}_{\multiindex}$ in
\eref{spectral-decomposition} is also such a vector. The projection in
\eref{coefficient-evaluation} and, consequently, \aref{surrogate-construction}
should be interpreted as follows: $\v{v}$ is an $\nq \times \np \ns$ matrix, and
the $i$th row of this matrix is the temperature profile computed at the $i$th
quadrature point and reshaped into a row vector. Similarly, $\hat{\v{v}}$ is an
$\nc \times \np \ns$ matrix, and the $i$th row of this matrix is the $i$th
coefficient $\hat{\g}_{\multiindex_i}$ of the spectral decomposition in
\eref{spectral-decomposition} (recall that a fixed ordering is assumed to be
imposed on the multi-indices). Keeping the above in mind, a call to
\aref{surrogate-construction} should be made such that Algorithm~X points at an
auxiliary routine which receives \vu, forwards it to \aref{temperature-solution}
along with $\mp_\dynamic$, and returns the resulting temperature profile to
\aref{surrogate-construction}. The constructed expansion can now be
postprocessed as needed; see \sref{postprocessing}.

To give a concrete example, for a dual-core system ($\np = 2$) with one
independent random variable ($\nz = 1$), a second-level expansion ($\lc = 2$) of
the temperature profile with 100 time steps ($\ns = 100$) can be written as
follows (see \eref{spectral-decomposition}):
\[
  \g \approx \hat{\g}_0 + \hat{\g}_1 \z + \hat{\g}_2(\z^2 - 1),
\]
which is a polynomial in \z, and each coefficient is a vector with $\np \ns =
200$ elements. Then, for any outcome $\z \equiv \z(\o)$, the corresponding
temperature profile \mq can be evaluated by plugging in \z into the above
equation and reshaping the result into an $\np \times \ns = 2 \times 100$
matrix. In this case, the three rows of $\hat{\v{v}}$ are $\hat{\g}_0$,
$\hat{\g}_1$, and $\hat{\g}_2$; the first one is also a flattened version of the
expected value of $\mq$ as shown in \eref{probabilistic-moments}.

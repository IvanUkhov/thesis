In this section, we elaborate on dynamic steady-state analysis under process
variation. Unlike transient analysis, this analysis cannot be done one time step
at a time since the whole repetitive workload needs to be taken into account at
once in order to calculate the corresponding dynamic steady state. Consequently,
similar to the contrast between \sref{transient-analysis} and
\sref{dynamic-steady-state-analysis}, the solutions in
\sref{chaos-transient-analysis} and this section differ. However, they still
rely on the same strategy outlined in \sref{chaos-uncertainty-analysis} and
shown in \fref{chaos-overview}.

\subsection{System Model}

The system model is the same as the one in \sref{chaos-transient-analysis}. The
only difference is that the power model is more convenient to be defined as
follows:
\begin{equation} \elab{chaos-power-model-bulk}
  \mp = f(\vu, \mq).
\end{equation}
The function $f: \real^\nu \times \real^{\np \times \ns} \to \real^{\np \times
\ns}$ is supposed to return the periodic power profile that corresponds to the
desired periodic workload.

The solution to dynamic steady-state analysis is based on the one in
\sref{dynamic-steady-state-solution}; however, in the present context, the
power, temperature, and state vectors are random, which also concerns the
boundary condition in \eref{dynamic-steady-state-condition}.

\inputalgorithm{chaos-dynamic-steady-state-iterative}
The pseudocode of a procedure that delivers \mq is given in
\aref{dynamic-steady-state}. The algorithm does not account for the
interdependence between power and temperature, and this interdependence can be
addressed by means of one of the techniques presented in
\sref{power-temperature-interplay}. In the context of this chapter, the most
appropriate technique is the iterative approach illustrated in
\aref{dynamic-steady-state-iterative}, which we rewritten here as shown in
\aref{chaos-dynamic-steady-state-iterative}.

\begin{remark}
The application of the linear approximation described in
\sref{power-temperature-interplay} is not straightforward in this case. This
technique is suitable when the only varying parameter is temperature, and all
other parameters have nominal values. In that case, it is relatively easy to
decide on a representative temperature range and apply a curve-fitting
procedure. In this case, however, the power model has multiple parameters
stepping far from their nominal values.
\end{remark}

In this section, the quantity of interest \g is exactly the dynamic steady-state
temperature profile \mq (or the corresponding power profile \mp if preferable).
Therefore, the output of Stage~1 in \fref{chaos-overview} is
\aref{chaos-dynamic-steady-state-iterative}.

\subsection{Surrogate Construction}

The workflow in \sref{chaos-surrogate-construction} is applied to an algorithm
that computes \mq for a given \vu and returns it in a suitable format. See
\rref{chaos-multidimensional-output} regarding the treatment of vector-valued
quantities.

So far in this subsection, \vu has been assumed to be deterministic. Now we turn
to the stochastic scenario and let \vu be random. Then we apply
\aref{surrogate-construction} to one particular quantity of interest \g.
Specifically, \g is now the temperature profile \mq corresponding to a given
$\mp_\dynamic$. Since \mq is an $\np \times \ns$ matrix, following
\rref{multiple-dimensions}, \g is viewed as an $\np \ns$-element row vector, in
which case each coefficient $\hat{\g}_{\multiindex}$ in
\eref{spectral-decomposition} is also such a vector. The projection in
\eref{coefficient-evaluation} and, consequently, \aref{surrogate-construction}
should be interpreted as follows: $\v{v}$ is an $\nq \times \np \ns$ matrix, and
the $i$th row of this matrix is the temperature profile computed at the $i$th
quadrature point and reshaped into a row vector. Similarly, $\hat{\v{v}}$ is an
$\nc \times \np \ns$ matrix, and the $i$th row of this matrix is the $i$th
coefficient $\hat{\g}_{\multiindex_i}$ of the spectral decomposition in
\eref{spectral-decomposition} (recall that a fixed ordering is assumed to be
imposed on the multi-indices). Keeping the above in mind, a call to
\aref{surrogate-construction} should be made such that Algorithm~G points at an
auxiliary routine which receives \vu, forwards it to \aref{temperature-solution}
along with $\mp_\dynamic$, and returns the resulting temperature profile to
\aref{surrogate-construction}. The constructed expansion can now be
postprocessed as needed; see \sref{postprocessing}.

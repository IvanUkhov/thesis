In order to give a better intuition about our solutions, we shall accompany the
development of our framework with the development of a concrete
example/application, which will eventually be utilized for the quantitative
evaluation of the framework given in \sref{experimental-results}. To this end,
we have decided to focus on two process parameters, which are arguably the most
crucial ones, namely, the effective channel length $L$ and the gate-oxide
thickness $T$. In this example,
\[
  \vu = (L, T): \Omega \to \real^2,
\]
which is to be discussed in detail shortly. Regarding reliability, we shall
address the thermal-cycling fatigue as it is naturally connected with dynamic
steady-state temperature analysis that we develop.

Before we proceed to the construction of surrogate models, let us first refine
our definition of $\vu = (\u_i)_{i = 1}^\nu$. Each $\u_i$ is a characteristic of
a single transistor---consider, for instance, the effective channel
length---and, therefore, each device in the electrical circuit at hand
potentially has a different value of this parameter since, in general, the
variability due to process variation is not uniform. Consequently, each $\u_i$
can be viewed as a random process $\u_i: \Omega \times \real^2 \to \real$
defined on a two-dimensional plane. Since this work is system-level oriented, we
model each processing element with one variable for each such random process.
More specifically, we let $\u_{ij} = \u_i(\cdot, \v{r}_j)$ be the random
variable representing uncertain parameter $i$ at processing element $j$ where
$\v{r}_j \in \real^2$ stands for the spatial location of the center of the
processing element. Therefore, we redefine the parameterization \vu of the
problem at hand as
\[
  \vu = (\u_i)_{i = 1}^{\nu \np}: \Omega \to \real^{\nu \np}
\]
such that there is a one-to-one correspondence between $\u_i$ for $i =
\range{1}{\nu \np}$ and $\u_{ij}$ for $i = \range{1}{\nu}$ and $j =
\range{1}{\np}$. For instance, in our illustrative application with two process
parameters, the total number of stochastic dimensions is $2 \np$.

\begin{remark}
Some authors prefer to split the variability of a process parameter at a spatial
location into several parts such as wafer-to-wafer, die-to-die, and within-die;
see, for instance, \cite{juan2012}. However, from the mathematical point of
view, it is sufficient to consider just one random variable per location which
is adequately correlated with the other locations of interest.
\end{remark}

A description of \vu given by the designer is an input to our analysis. A proper
(complete, unambiguous) way to describe a set of random variables is to specify
their joint probability distribution function. In practice, however, such
exhaustive information is often unavailable, in particular, due to the high
dimensionality in the presence of prominent dependencies inherent to the
considered problem. A more realistic assumption is the knowledge of the marginal
distributions and correlation matrix of \vu. Denote by $\{ F_{\u_i} \}_{i =
1}^{\nu \np}$ and $\correlation{\vu} \in \real^{\nu \np \times \nu \np}$ the
marginal distribution functions and correlation matrix of the uncertain
parameters \vu in \eref{uncertain-parameters}, respectively. Note that the
number of distinct marginals is only \nu since \np components of \vu correspond
to the same uncertain parameter.

\subsection{Probability Transformation}

Let us turn to the illustrative application. Recall that we exemplify our
framework considering the effective channel length and gate-oxide thickness with
the notation given in \eref{application-uncertain-parameters}. Both parameters
correspond to Euclidean distances; they take values on bounded intervals of the
positive part of the real line. With this in mind, we model the two process
parameters using the four-parametric family of beta distributions:
\[
  \u_i \sim F_{\u_i} = \mathrm{Beta}(a_i, b_i, c_i, d_i)
\]
where $i = 1, 2, \dots, 2 \np$, $a_i$ and $b_i$ control the shape of the
distributions, and $[c_i, d_i]$ correspond to their supports. Without loss of
generality, we let the two considered process parameters be independent of each
other, and the correlations among those elements of \vu that correspond to the
same process parameter be given by the following correlation function:
\[
  k(\v{r}_i, \v{r}_j) = w \; k_\SE(\v{r}_i, \v{r}_j) + (1 - w) k_\OU(\v{r}_i, \v{r}_j)
\]
where $\v{r}_i \in \real^2$ is the center of the $i$th processing element
relative to the center of the die. The correlation function is a composition of
two kernels:
\begin{align*}
  & k_\SE(\v{r}_i, \v{r}_j) = \exp\left(- \frac{\norm{\v{r}_i - \v{r}_j}^2}{\ell_\SE^2} \right) \text{ and} \\
  & k_\OU(\v{r}_i, \v{r}_j) = \exp\left(- \frac{\absolute{\norm{\v{r}_i} - \norm{\v{r}_j}}}{\ell_\OU} \right),
\end{align*}
which are known as the squared-exponential and Ornstein--Uhlenbeck kernels,
respectively. In the above formulae, $w \in [0, 1]$ is a weight coefficient
balancing the kernels; $\ell_\SE$ and $\ell_\OU > 0$ are so-called length-scale
parameters; and $\norm{\cdot}$ stands for the Euclidean norm in $\real^2$. The
choice of these two kernels is guided by the observations of the correlation
patterns induced by the fabrication process: $k_\SE$ imposes similarities
between those spatial locations that are close to each other, and $k_\OU$
imposes similarities between those locations that are at the same distance from
the center of the die; see, for instance, \cite{friedberg2005} for additional
details. The length-scale parameters $\ell_\SE$ and $\ell_\OU$ control the
extend of these similarities, that is the range wherein the influence of one
point on another is significant.

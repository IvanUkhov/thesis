In order to give a better intuition about the approach to reliability analysis
and optimization presented in \sref{chaos-reliability-analysis} and
\sref{chaos-reliability-optimization}, we consider a concrete application,
meaning that we specify the uncertain parameters and carry out the corresponding
computations. This application is also utilized for the quantitative evaluation
of our technique given in the next section, \sref{chaos-reliability-results}.

\subsection{Problem Formulation}

We focus on two crucial process parameters, namely, on the effective channel
length and gate-oxide thickness. Consequently, each processing element is
ascribed two random variables corresponding to the two process parameters, which
means that $\nu = 2 \np$ in this example; see also \sref{chaos-problem}.

\begin{remark}
The variability of a process parameter at a spatial location can be split into
several parts such as inter-lot, inter-wafer, inter-die, and intra-die; see, for
instance, \cite{juan2012}. However, from the mathematical point of view, it is
sufficient to consider just one random variable per location with an adequate
distribution and adequate correlations with the other locations of interest.
\end{remark}

A description of \vu is an input to our analysis. A proper way to describe a set
of random variables is to specify their joint probability distribution function.
In practice, however, such exhaustive information is often unavailable, in
particular, due to the high dimensionality in the presence of prominent
dependencies inherent to the considered problem. A more realistic assumption,
which we make here, is the knowledge of the marginal distributions and
correlation matrix of \vu. Denote by $\{ F_i \}_{i = 1}^\nu$ and
$\correlation{\vu} \in \real^{\nu \times \nu}$ the marginal distribution
functions and correlation matrix of the uncertain parameters \vu in,
respectively. Note that the number of distinct marginals is only two since \np
components of \vu correspond to the same process parameter.

\subsection{Probability Transformation}

Let us turn to the illustrative application. Recall that we exemplify our
framework considering the effective channel length and gate-oxide thickness with
the notation given in \eref{application-uncertain-parameters}. Both parameters
correspond to Euclidean distances; they take values on bounded intervals of the
positive part of the real line. With this in mind, we model the two process
parameters using the four-parametric family of beta distributions:
\[
  \u_i \sim F_{\u_i} = \mathrm{Beta}(a_i, b_i, c_i, d_i)
\]
where $i = 1, 2, \dots, 2 \np$, $a_i$ and $b_i$ control the shape of the
distributions, and $[c_i, d_i]$ correspond to their supports. Without loss of
generality, we let the two considered process parameters be independent of each
other, and the correlations among those elements of \vu that correspond to the
same process parameter be given by the following correlation function:
\[
  k(\v{r}_i, \v{r}_j) = w \; k_\SE(\v{r}_i, \v{r}_j) + (1 - w) k_\OU(\v{r}_i, \v{r}_j)
\]
where $\v{r}_i \in \real^2$ is the center of the $i$th processing element
relative to the center of the die. The correlation function is a composition of
two kernels:
\begin{align*}
  & k_\SE(\v{r}_i, \v{r}_j) = \exp\left(- \frac{\norm{\v{r}_i - \v{r}_j}^2}{\ell_\SE^2} \right) \text{ and} \\
  & k_\OU(\v{r}_i, \v{r}_j) = \exp\left(- \frac{\absolute{\norm{\v{r}_i} - \norm{\v{r}_j}}}{\ell_\OU} \right),
\end{align*}
which are known as the squared-exponential and Ornstein--Uhlenbeck kernels,
respectively. In the above formulae, $w \in [0, 1]$ is a weight coefficient
balancing the kernels; $\ell_\SE$ and $\ell_\OU > 0$ are so-called length-scale
parameters; and $\norm{\cdot}$ stands for the Euclidean norm in $\real^2$. The
choice of these two kernels is guided by the observations of the correlation
patterns induced by the fabrication process: $k_\SE$ imposes similarities
between those spatial locations that are close to each other, and $k_\OU$
imposes similarities between those locations that are at the same distance from
the center of the die; see, for instance, \cite{friedberg2005} for additional
details. The length-scale parameters $\ell_\SE$ and $\ell_\OU$ control the
extend of these similarities, that is the range wherein the influence of one
point on another is significant.

\subsection{Surrogate Construction}

Since we are interested in integration with respect to the standard Gaussian
measure over $\real^\nz$, we shall rely on the Gauss--Hermite family of
quadrature rules \cite{maitre2010}, which is a subset of a broader family known
as Gaussian quadratures.

For the Gauss--Hermite quadrature rules in one dimension, we have that $\nq =
\lq + 1$ and the precision is $2 \nq - 1$ \cite{heiss2008} or, equivalently, $2
\lq + 1$, which is a remarkable property of Gaussian quadratures. The resulting
sparse grid is exact for polynomials with a total order up to $2 \lq + 1$, which
is analogous to integration in one dimension.

\subsection{Dynamic Steady-State Analysis}

To give a concrete example, for a dual-core system ($\np = 2$) with one
independent random variable ($\nz = 1$), a second-level expansion ($\lc = 2$) of
the temperature profile with 100 time steps ($\ns = 100$) can be written as
follows (see \eref{spectral-decomposition}):
\[
  \g \approx \hat{\g}_0 + \hat{\g}_1 \z + \hat{\g}_2(\z^2 - 1),
\]
which is a polynomial in \z, and each coefficient is a vector with $\np \ns =
200$ elements. Then, for any outcome $\z \equiv \z(\o)$, the corresponding
temperature profile \mq can be evaluated by plugging in \z into the above
equation and reshaping the result into an $\np \times \ns = 2 \times 100$
matrix. In this case, the three rows of $\hat{\v{v}}$ are $\hat{\g}_0$,
$\hat{\g}_1$, and $\hat{\g}_2$; the first one is also a flattened version of the
expected value of $\mq$ as shown in \eref{probabilistic-moments}.

\subsection{Reliability Optimization}

Assume that the structure of the reliability model is the one shown in
\eref{reliability-model} where each individual reliability function $R_i$ is the
one given in \eref{weibull-reliability} with its own $\eta_i$ and $\beta_i$ for
$i = \range{1}{\np}$. During each iteration, the temperature of processing
element $i$ exhibits \nk{i} cycles. Each cycle generally has different
characteristics and, therefore, causes different damage. This aspect is taken
into account by adjusting $\eta_i$ as shown in \eref{thermal-cycling-scale}. The
shape parameter $\beta_i$ is known to be indifferent to temperature. For
simplicity, let us also assume that $\beta_i$ does not depend on process
parameters, and that $\beta_i = \beta$ for $i = \range{1}{\np}$.

Under the above assumptions, \rref{weibull-homogeneity} applies, and the
lifetime $T: \Omega \to \real$ of the system has a Weibull distribution as
follows:
\[
  T | \vg \sim \mathrm{Weibull}(\eta, \beta)
\]
where $\eta$ is as in \rref{weibull-homogeneity} combined with
\eref{thermal-cycling-scale}. The high-level parameters of the above model are
two parameters: $\eta$ and $\beta$, among which only $\eta$ is uncertain.
Thus, $\vg = (\eta)$ in this particular scenario; $\beta$ is given implicitly.

Now we apply the technique in \sref{uncertainty-analysis} to this quantity. In
this case, Algorithm~G in \aref{surrogate-construction} is an auxiliary function
that makes a call to \aref{temperature-solution}, processes the resulting
temperature profile as it was described earlier in this subsection, and returns
$\eta$ computed according to the formula in \eref{compound-weibull-eta}.
Consequently, we obtain a light polynomial surrogate of the parameterization of
the reliability model, which can be then studied from various perspectives. The
example for a dual-core system given at the end of \sref{temperature-solution}
can be considered in this context as well with the only change that the
dimensionality of the polynomial coefficients would be two here (since $\eta \in
\real^\np$ and $\np = 2$).

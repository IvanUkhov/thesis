In numerical integration, the integral of a function $g: \real \to \real$
\[
  I = \int_\real g(x) \d x
\]
is approximated by a summation of the function's values computed at prescribed
points $\{ \hat{x}_i \}_{i = 1}^\nq \subset \real$ and multiplied by prescribed
weights $\{ w_i \}_{i = 1}^\nq \subset \real$, which we denote as the following
operator:
\[
  I \approx \quadrature{1}{\lq}{g} = \sum_{i = 1}^\nq g(\hat{x}_i) w_i.
\]
Such a pair of a set of points and a set of weights is called a quadrature rule.
In the notation $\quadrature{1}{\lq}$, the superscript $1$ indicates that it is
a one-dimensional rule, and the subscript $\lq \in \natural$ gives the level of
the rule, which is its index in the corresponding family of rules with
increasing precision. \emph{Precision} refers to the maximum total order of
polynomials that the rule integrates exactly \cite{heiss2008}.

Similarly, in multiple dimensions, the integral of a function $g: \real^n \to
\real$
\[
  I = \int_{\real^n} g(\v{x}) \d \v{x}
\]
is approximated as
\begin{equation} \elab{quadrature-summation}
  I \approx \quadrature{n}{\lq}{g} = \sum_{i = 1}^\nq g(\hv{x}_i) w_i
\end{equation}
where $\{ \hv{x}_i \}_{i = 1}^\nq \subset \real^n$ and $\{ w_i \}_{i = 1}^\nq
\subset \real$. The number of points or weights \nq depends on $n$ and \lq,
which we occasionally emphasize by writing $\nq(n, \lq)$.

The foundation of an $n$-dimensional rule $\quadrature{n}{\lq}$ is a set of
one-dimensional counterparts of various levels. The most straightforward
construction of such a rule is the tensor product of $n$ copies of
$\quadrature{1}{\lq}$ as follows:
\begin{equation} \elab{integration-tensor}
  \quadrature{n}{\lq} = \bigotimes_{i = 1}^n \quadrature{1}{\lq},
\end{equation}
which is referred to as the full tensor product. In this case,
\[
  \nq(n, \lq) = \nq(1, \lq)^n,
\]
that is, the growth of the number of points is exponential. In low dimensions,
the grows is manageable; however, in high dimensions, the situation changes
dramatically as the number of points produced by this approach can easily
explode. For instance \cite{heiss2008}, if a one-dimensional rule has only 4
points, that is, $\nq = 4$, then in 10 stochastic dimensions, that is, $n = 10$,
the number of multivariate points becomes $\nq = 4^{10} = 1~048~576$, which is
not affordable. Moreover, it can be shown that most of the points obtained in
this way do not contribute to the asymptotic accuracy and, hence, are a waste of
time. In particular, if the integrand under consideration is a polynomial whose
total order is constrained according to a certain strategy, the full tensor
product cannot take this information into account. Consequently, a different
construction technique should be utilized in the case of high-dimensional
problems.

An alternative construction is the Smolyak algorithm \cite{smolyak1963}; see
also \cite{eldred2008, heiss2008, maitre2010}. The algorithm combines a range of
one-dimensional rules in such a way that $\quadrature{n}{\lq}$ is tailored to be
exact only for a specific polynomial subspace, which allows for a significantly
reduction of the number of points. For instance, in the example given earlier,
the number of points would only be $\nq = 1~581$, which would entail a drastic
saving of the computation time. Define $\Delta_0 = \quadrature{1}{0}$ and
$\Delta_i = \quadrature{1}{i} - \quadrature{1}{i - 1}$ for $i \geq 1$. Then
Smolyak's approximating formula is
\begin{equation} \elab{integration-smolyak}
  \quadrature{n}{\lq} = \bigoplus_{\multiindex \in \multiindices{n}{\lq}} \Delta_{i_1} \otimes \cdots \otimes \Delta_{i_n}.
\end{equation}
In the original (isotropic) formulation of the Smolyak algorithm,
\begin{equation} \elab{index-total-order-isotropic}
  \multiindices{n}{\lq} = \left\{ \multiindex \in \natural^n: \norm[1]{\multiindex} \leq \lq \right\}
\end{equation}
where $\norm[1]{\cdot}$ stands for the Manhattan norm. The multi-index set
$\multiindices{n}{\lq}$ is called a total-order multi-index set since it
constrains polynomials by their total order \cite{eldred2008, beck2011}, and its
cardinality can be calculated as follows:
\begin{equation} \elab{index-total-order-isotropic-length}
  \nq(n, \lq) = \cardinality{\multiindices{n}{\lq}} = {n + \lq \choose n} = \frac{(n + \lq)!}{n! \, \lq!}.
\end{equation}
It can be seen that the construction in \eref{integration-smolyak} is a
summation of cherry-picked tensor products of one-dimensional rules. This
formula is well suited for grasping the structure of the resulting sparse grids;
more implementation-oriented versions of the Smolyak algorithm can be found in
the literature. Note also that, although we use the same notation in
\eref{integration-tensor} and \eref{integration-smolyak}, the two constructions
are generally different; the latter reduces to the former if
\[
  \multiindices{n}{\lq} = \left\{ \multiindex \in \natural^n: \max_{k = 1}^n i_k \leq \lq \right\}.
\]

Consider now the following more general integral:
\[
  I = \int_{\real^n} g(\v{x}) \d F(\v{x}).
\]
Here $g$ is integrated with respect to a measure $F: \real^n \to \real$
\cite{durrett2010} that does not necessarily correspond to the usual Lebesgue
measure used in the earlier examples. If $F$ is absolutely continuous, the
integrand can be rewritten as
\[
  I = \int_{\real^n} g(\v{x}) f(\v{x}) \d \v{x}
\]
where $g$ is weighted by the derivative $f: \real^n \to \real$ of $F$ and
integrated with respect to the Lebesgue measure. Since integrating in this way
is a very frequent operation, there are families of quadrature rules that are
designed to automatically take this aspect into account for the most common
scenarios.

We would like to highlight one broad family of quadrature rules known as
Gaussian quadratures. In one dimension, the precision of a Gaussian quadrature
with \nq points is $2 \nq - 1$ \cite{heiss2008} or, equivalently, $2 \lq + 1$
since $\nq = \lq + 1$ in this case. In other words, a Gaussian quadrature with
\nq points is exact for polynomials with orders up to $2 \nq - 1$ or $2 \lq +
1$, which is a remarkable property of Gaussian quadratures that makes such rules
especially efficient.

Lastly, it is worth noting that quadrature rules are generally precomputed and
tabulated since they do not depend the integrand; see, for instance,
\cite{burkardt}.

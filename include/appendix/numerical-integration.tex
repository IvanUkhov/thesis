The integral of a function $g: \real \to \real$
\[
  I = \int_\real g(x) \d x
\]
is approximated by a summation of the function's values computed at prescribed
points $\{ \hat{x}_i \in \real \}_{i = 1}^\nq$ and multiplied by prescribed
weights $\{ w_i \in \real \}_{i = 1}^\nq$, which we denote as follows:
\[
  I \approx \quadrature{1}{\lq}{g} = \sum_{i = 1}^\nq g(\hat{x}_i) w_i.
\]
Such a pair of a set of points and a set of weights is called a quadrature rule.
In the notation $\quadrature{1}{\lq}$, the superscript $1$ indicates that it is
a one-dimensional rule, and the subscript \lq gives the accuracy level of the
rule, which is the index of the rule in the corresponding family of quadrature
rules with increasing accuracy. In the one-dimensional case, \lq corresponds to
the maximum order of polynomials that the rule integrates exactly
\cite{heiss2008}. Similarly, in multiple dimensions, the integral of a function
$g: \real^n \to \real$
\[
  I = \int_{\real^n} g(\v{x}) \d \v{x}
\]
is approximated by
\begin{equation} \elab{quadrature-summation}
  I \approx \quadrature{n}{\lq}{g} = \sum_{i = 1}^\nq g(\hat{\v{x}}_i) w_i.
\end{equation}
An appropriate $n$-dimensional quadrature rule $\quadrature{n}{\lq}$ is
typically formed by computing the tensor product of $n$ potentially distinct
one-dimensional quadrature rules. The number of terms \nq in summations is
dictated by both $n$ and \lq. Lastly, it is worth noting that quadrature rules
are generally precomputed and tabulated since they do not depend the integrand;
see, for instance, \cite{burkardt}.

Consider now the following more general integral:
\[
  I = \int_{\real^n} g(\v{x}) \d F(\v{x}).
\]
Here $g$ is integrated with respect to a measure $F: \real^n \to \real$
\cite{durrett2010} that does not necessarily correspond to the usual Lebesgue
measure used in the earlier examples. If $F$ is absolutely continuous, the
integrand can be rewritten as
\[
  I = \int_{\real^n} g(\v{x}) f(\v{x}) \d \v{x}
\]
where $g$ is weighted by the derivative $f: \real^n \to \real$ of $F$ and
integrated with respect to the Lebesgue measure. Since integrating in such a way
is a very frequent operation, there are families of quadrature rules that are
designed to automatically take this aspect into account for the most common
scenarios.

Let us highlight one broad class of rules known as Gaussian quadratures. The
accuracy level \lq of a one-dimensional Gaussian quadrature with \nq points is
$2 \nq - 1$, that is, the rule integrates exactly polynomials of an order up to
$2 \nq - 1$. This feature makes such quadratures especially efficient
\cite{heiss2008}.

There is one more and arguably the most crucial aspect of numerical integration
that we ought to discuss: the algorithm used to construct multidimensional
quadrature rules. In low dimensions, the construction can be based on the direct
tensor product of one-dimensional rules. However, in high dimensions, the
situation changes dramatically as the number of points produced by this approach
can easily explode. For instance \cite{heiss2008}, if a one-dimensional rule has
only 4 points, that is, $\nq = 4$, then in 10 stochastic dimensions, that is, $n
= 10$, the number of multivariate points becomes $\nq = 4^{10} = 1~048~576$,
which is not affordable. Moreover, it can be shown that most of the points
obtained in this way do not contribute to the asymptotic accuracy and, hence,
are a waste of time. In order to tackle this problem, one resides to sparse
integration grids constructed via the Smolyak algorithm \cite{burkardt,
eldred2008, heiss2008}. The algorithm preserves the accuracy of the underlying
one-dimensional rules for complete polynomials while significantly reducing the
number of points. For instance, in the example given earlier, the number of
points computed by the algorithm is only $\nq = 1~581$, which implies a drastic
saving of the computation time.

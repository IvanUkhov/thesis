Any symmetric matrix $\mx \in \real^{n \times n}$ admits the eigendecomposition
\cite{press2007}, which, in this particular case, takes the following form:
\begin{equation} \elab{eigendecomposition}
  \mx = \m{U} \m{\Lambda} \transpose{\m{U}}
\end{equation}
where $\m{U} \in \real^{n \times n}$ is an orthogonal matrix of the eigenvectors
of \mx, and
\[
  \m{\Lambda} = \diagonal{\lambda_1}{\lambda_n} = \left(
    \begin{array}{lll}
      \lambda_1 & \cdots & 0         \\
      \cdots    & \cdots & \cdots    \\
      0         & \cdots & \lambda_n
    \end{array}
  \right) \in \real^{n \times n}
\]
is a diagonal matrix of the eigenvalues of \mx. If, in addition, \mx is positive
semi-definite, the eigenvalues $\{ \lambda_i \}_{i = 1}^n$ are non-negative.

Vectors are denoted by bold lowercase letters. An $n$-dimensional vector \vx in
a vector space such as $\real^n$ can be specified in a number of equivalent
ways, depending on what is important to emphasize in a particular context, as
follows:
\begin{align*}
  & \vx = (x_i), \\
  & \vx = (x_i)_{i = 1}^n, \text{ and} \\
  & \vx = (\range{x_1}{x_n}).
\end{align*}
Matrices are denoted by bold uppercase letters. An $n_1 \times n_2$ matrix in a
vector space such as $\real^{n_1 \times n_2}$ can also be defined in several
ways as follows:
\begin{align*}
  & \m{X} = (x_{ij}), \\
  & \m{X} = (x_{ij})_{i = 1, j = 1}^{n_1, n_2}, \text{ and} \\
  & \m{X} = \left(
    \begin{array}{lll}
      x_{1 1}   & \cdots & x_{1 n_2}   \\
      \cdots    & \cdots & \cdots      \\
      x_{n_1 1} & \cdots & x_{n_1 n_2}
    \end{array}
  \right).
\end{align*}

Any symmetric matrix $\m{X} \in \real^{n \times n}$ admits the
eigendecomposition \cite{press2007}, which, in this particular case, takes the
following form:
\begin{equation} \elab{eigendecomposition}
  \m{X} = \m{U} \m{\Lambda} \transpose{\m{U}}
\end{equation}
where $\m{U} \in \real^{n \times n}$ is an orthogonal matrix of the eigenvectors
of $\m{X}$, and
\[
  \m{\Lambda} = \diagonal{\lambda_1}{\lambda_n} = \left(
    \begin{array}{lll}
      \lambda_1 & \cdots & 0         \\
      \cdots    & \cdots & \cdots    \\
      0         & \cdots & \lambda_n
    \end{array}
  \right) \in \real^{n \times n}
\]
is a diagonal matrix of the eigenvalues of $\m{X}$.

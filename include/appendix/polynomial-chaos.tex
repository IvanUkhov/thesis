Let $\L{2}(\Omega, \F, \probability)$ be as in \eref{square-integrable-space}
and $G \subset \L{2}(\Omega, \F, \probability)$ be the Gaussian Hilbert space
\cite{janson1997} spanned by $n$ mutually independent standard Gaussian random
variables, which we denote by $\vx: \Omega \to \real^n$. Since the variables are
independent and standard, they form an orthonormal basis in $G$, and $G$ is
$n$-dimensional. The variables induce a probability measure on $\real^n$, and
the corresponding distribution function $F$ is standard Gaussian given by
\[
  \d F(\v{y}) = (2 \pi)^{-\frac{n}{2}} \exp\left(-\frac{\norm[2]{\v{y}}^2}{2}\right) \d \v{y}.
\]
Then the inner product in the vector space of functions over $G$ is defined as
\begin{equation} \elab{chaos-inner-product}
  \innerproduct{g}{h} = \int_{\real^n} g(\v{y}) h(\v{y}) \d F(\v{y})
\end{equation}
for any $g$ and $h$ in this space, and the norm is defined as
\[
  \norm[2]{g} = \sqrt{\innerproduct{g}{g}}.
\]

Let $\Psi_\lc(G)$ be the space of $n$-variate polynomials over $G$ such that the
total order of each polynomial is at most $\lc \in \natural$. The space
$\Psi_\lc(G)$ can be constructed as a span of $n$-variate Hermite polynomials
\cite{eldred2008, maitre2010}
\[
  \Psi_\lc(G) = \hull{\left\{ \psi_{\vi}(\v{y}): \vi \in \multiindices{n}{\lc}, \v{y} \in G \right\}}
\]
where $\vi = (i_k) \in \natural^n$ is a multi-index; $\multiindices{n}{\lc}$ is
the one in \eref{index-total-order-isotropic}; and
\[
  \psi_{\vi}(\v{y}) = \prod_{k = 1}^n \psi_{i_k}(y_k).
\]
In the above formulae, $\psi_{i_k}: \real \to \real$ is a one-dimensional
Hermite polynomial of order $i_k$, which is assumed to be normalized. It is
important to note that the basis polynomials $\{ \psi_{\vi} \}$ are orthonormal
with respect to $F$, which means that
\begin{equation} \elab{chaos-orthogonality}
  \innerproduct{\psi_{\vi}}{\psi_{\vj}} = \delta_{\vi \vj}
\end{equation}
for any two multi-indices $\vi = (i_k) \in \natural^n$ and $\vj = (j_k) \in
\natural^n$ where
\[
  \delta_{\vi \vj} = \prod_{k = 1}^n \delta_{i_k j_k},
\]
and $\delta_{i_k j_k}$ is the Kronecker delta. In particular, the polynomials
are also centered with respect to $F$, and, therefore, the inner product in
\eref{chaos-inner-product} applied to two polynomials yields their correlation,
which is zero due to the orthogonality. In addition, the inner product applied
to a polynomial with itself yields the variance of that polynomial, which is
unity due to the normality.

Define $H_0 = \Psi_0(G)$ (the space of constants) and, for $i \geq 1$,
\[
  H_i = \Psi_i(G) \, \cap \, \Psi_{i - 1}(G)^\perp.
\]
The vector spaces $\{ H_i \}_{i = 0}^\infty$ are mutually orthogonal, closed
subspaces of $\L{2}(\Omega, \F, \probability)$. Since our scope of interest is
restricted to functions of \vx, \F is assumed to be generated by \vx. Then, by
the Cameron--Martin theorem,
\[
  \L{2}(\Omega, \F, \probability) = \bigoplus_{i = 0}^\infty H_i,
\]
which is known as the Wiener chaos decomposition.

The Wiener chaos decomposition implies that any $\g \in \L{2}(\Omega, \F,
\probability)$ admits an expansion with respect to the polynomial basis, which
is denoted by
\[
  \g = \sum_{\vi \in \natural^n} \hat{g}_{\vi} \psi_{\vi}
\]
where the equality should be understood in mean square. The coefficients $\{
\hat{g}_{\vi} \}$ of this spectral decomposition are found by multiplying both
sides of the equation by $\psi_{\vi}$ and making use of
\eref{chaos-orthogonality}, which results in
\begin{equation} \elab{chaos-projection}
  \hat{g}_{\vi} = \innerproduct{\g}{\psi_{\vi}}
\end{equation}
This operation is referred to as a spectral projection.

In practice, the infinite expansion is truncated, which we denote as
\[
  \g \approx \chaos{n}{\lc}{\g} = \sum_{\vi \in \multiindices{n}{\lc}} \hat{g}_{\vi} \psi_{\vi}.
\]
In the notation $\chaos{n}{\lc}$, the superscript $n$ indicates the
dimensionality of the vector space, and the subscript \lc indicates the level of
the truncated expansion. This truncated expansion converges in mean square to \g
as $\lc \to \infty$, that is,
\[
  \g = \lim_{\lc \to \infty} \chaos{n}{\lc}{\g}.
\]

The Wiener chaos decomposition is often referred to as the classical \acf{PC}
decomposition, and it can be generalized to other types of probability
distributions than Gaussian. Many popular distributions directly correspond to
certain families of orthogonal polynomials, which can be found in the Askey
scheme of orthogonal polynomials. A distribution that does not have such a
correspondence can be transformed into one of those that do using such
techniques as the one shown in \sref{probability-transformation}. Another
solutions is to construct a custom polynomial basis using the Gram--Schmidt
process. Lastly, let us note that the machinery of \ac{PC} expansions is
applicable to discrete distributions as well. The interested reader is referred
to \cite{xiu2010} for further discussions.

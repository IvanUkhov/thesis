Due to the inherent complexity, uncertainty-quantification problems are
typically viewed as approximation problems: one usually constructs a
computationally efficient surrogate for the system under consideration and then
studies this light representation instead of the original system. One way to
construct such a surrogate is \ac{PC} \cite{xiu2010}. The technique decomposes
stochastic quantities into infinite series of mutually orthogonal polynomials of
random variables. Such series are especially attractive from the post-processing
perspective as they are nothing more than the familiar polynomials; therefore,
\ac{PC} expansions are easy to interpret and easy to evaluate.

Let now $\{ \psi_i: \real^n \to \real \}_{i = 1}^\infty$ be a set of $n$-variate
polynomials. The polynomials span a weighted Hilbert space with the inner
product
\begin{equation} \elab{polynomial-inner-product}
  \innerproduct{\psi_i}{\psi_j}
  = \int_{\real^n} \psi_i(\v{x}) \psi_j(\v{x}) dF(\v{x})
  = \int_{\real^n} \psi_i(\v{x}) \psi_j(\v{x}) f(\v{x}) d\v{x}
\end{equation}
where $F$ and $f$ are a measure and a weight function, respectively, similarly
to \sref{numerical-integration}. Polynomials $\psi_i$ and $\psi_j$ are called
orthogonal if
\begin{equation} \elab{polynomial-orthogonality}
  \innerproduct{\psi_i}{\psi_j} = \innerproduct{\psi_i}{\psi_i} \delta_{ij}
\end{equation}
where $\delta_{ij}$ is the Kronecker delta function. The set of polynomials
constitutes a basis of the weighted space if all the polynomials are mutually
orthogonal.

Consider now a probability space $(\Omega, \mathcal{F}, \probability)$, defined
in \sref{probability-theory}, and let a function $g: \real^n \to \real$ depend
on a random vector $\v{x}: \Omega \to \real^n$ with mutually independent
components. Assume also that $g$, when viewed as a random variable $g: \Omega
\to \real$, belongs to $\L{2}(\Omega, \mathcal{F}, \probability)$. Lastly, let
$\{ \psi_i: \real^n \to \real \}_{i = 1}^\infty$ be a basis whose polynomials
are mutually orthogonal with respect to the probability distribution of $\v{x}$,
which means that $F$ and $f$ in \eref{polynomial-inner-product} are the \ac{CDF}
and \ac{PDF} of $\v{x}$, respectively. Then the \ac{PC} expansion of $g$ as a
function of $\v{x}$ is
\[
  g(\v{x}) = \sum_{i = 1}^\infty \hat{g}_i \psi_i(\v{x}).
\]
The coefficients $\{ \hat{g}_i \}_{i = 1}^\infty$ of the expansion are found by
multiplying both side of the equation by $\psi_i$ and making use of
\eref{polynomial-orthogonality}, which results in
\begin{equation} \elab{chaos-projection}
  \hat{g}_i = \frac{\innerproduct{g}{\psi_i}}{\innerproduct{\psi_i}{\psi_i}}
\end{equation}
for $i = \range{1}{\infty}$. This operation is referred to as a spectral
projection. For practical calculations, the expansion has be truncated, which we
denote as
\begin{equation} \elab{chaos-expansion}
  g(\v{x}) \approx \chaos{\nz}{\lc}{g}(\v{x}) = \sum_{i = 1}^\nc \hat{g}_i \psi_i(\v{x}).
\end{equation}
In the notation $\chaos{\nz}{\lc}$, the superscript \nz indicates the
dimensionality of the problem, and the subscript \lc indicates the accuracy
level of the expansion. The two parameters dictate the number of terms \nc,
which is
\begin{equation} \elab{chaos-length-total-order}
  \nc = { \lc + \nz \choose \nz } = \frac{(\lc + \nz)!}{\lc! \, \nz!}
\end{equation}
and corresponds to the total-order polynomial space \cite{eldred2008, beck2011}.

It is worth noting that, assuming that the polynomials are centered, the inner
product in \eref{polynomial-inner-product} corresponds to the covariance of two
polynomials, and orthogonality corresponds to the absence of correlation. In
addition, the inner product $\innerproduct{\psi_i}{\psi_i}$ in
\eref{polynomial-orthogonality} and \eref{chaos-projection} corresponds to the
variance of a polynomial.

Many of the most popular probability distributions directly correspond to
certain families of orthogonal polynomials, which can be found in the Askey
scheme of orthogonal polynomials \cite{xiu2010}. A distribution that does not
have such a correspondence can be transformed into one of those that do using
such techniques as the one shown in \sref{probability-transformation}. Another
solutions is to construct a custom polynomial basis using the Gram--Schmidt
process. Lastly, let us note that the machinery of \ac{PC} expansions is
applicable to discrete distributions as well. The interested reader is referred
to \cite{xiu2010} for further discussions.

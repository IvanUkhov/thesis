In numerical integration, the integral of a function $\f: \real \to \real$
\[
  I = \int_\real \f(\x) \d \x
\]
is approximated as
\[
  I \approx \quadrature{1}{i}{\f}
  = \sum_{j \in \tensorindex{1}{i}} \f(\x_{ij}) w_{ij},
\]
which is a summation of the function's values computed at prescribed points and
multiplied by prescribed weights. Such a pair of a set of points and a set of
weights is called a quadrature. In the notation $\quadrature{1}{i}$, the
superscript $1$ indicates that it is a one-dimensional quadrature, and the
subscript $i \in \natural$ indicates the level of the quadrature. In the above
formula,
\begin{align*}
  & \X^1_i = \{ \x_{ij}: j \in \tensorindex{1}{i} \} \subset \real \text{ and} \\
  & \W^1_i = \{ w_{ij}: j \in \tensorindex{1}{i} \} \subset \real
\end{align*}
are the points and weights of the quadrature, respectively, and
$\tensorindex{1}{i} \subset \natural$ is an index set, whose cardinality is
denoted by
\[
  n_i = \cardinality{\tensorindex{1}{i}}.
\]

The level $i$ is the quadrature's index in the corresponding family of
quadratures with increasing precision. \emph{Precision} refers to the maximum
total order of polynomials that the quadrature integrates exactly
\cite{heiss2008}. For instance, in the case of Gaussian quadratures, the
precision of a quadrature with $n_i$ points is $2 n_i - 1$ \cite{heiss2008}. In
other words, a Gaussian quadrature with $n_i$ points is exact for polynomials
with orders up to $2 n_i - 1$, which is a remarkable property of Gaussian
quadratures that makes such quadratures especially efficient.

Different families of quadratures can have different relations between $n_i$ and
$i$. Even in the scope of the same family, the integration grid can be made to
grow differently with respect to $i$ in order to attain certain properties such
as being nested. For our purposes, it is sufficient to mention one type of
growth: the so-called slow linear growth. In this case, $n_i = i + 1$. Assuming
the slow linear growth, the previous paragraph can be rephrased by stating that
a Gaussian quadrature with $n_i$ points is exact for polynomials for orders up
to $2 i + 1$.

In multiple dimensions, the integral of a function $\f: \real^n \to \real$
\[
  I = \int_{\real^n} \f(\vx) \d \vx
\]
is approximated as
\[
  I \approx \quadrature{n}{\vi}{\f}
  = \sum_{\vj \in \tensorindex{n}{\vi}} \f(\vx_{\vi \vj}) w_{\vi \vj}
\]
where $\vi = (i_k) \in \natural^n$ and $\vj = (j_k) \in \natural^n$ are
(multi-)indices. Further,
\begin{align*}
  & \X^n_{\vi} = \{ \vx_{\vi \vj}: \vj \in \tensorindex{n}{\vi} \} \subset \real^n \text{ and} \\
  & \W^n_{\vi} = \{ w_{\vi \vj}: \vj \in \tensorindex{n}{\vi} \} \subset \real
\end{align*}
and the points and weights, respectively, and $\tensorindex{n}{\vi} \subset
\natural^n$ is an index set. The cardinality of $\tensorindex{n}{\vi}$
dependents on both $n$ and \vi and is denoted by $n_{\vi}$.

The foundation of an $n$-dimensional quadrature $\quadrature{n}{\vi}$ is a set
of one-dimensional counterparts of various levels. The most straightforward
construction of such a quadrature is the full tensor product
\begin{equation} \elab{quadrature-tensor}
  \quadrature{n}{\vi} = \bigotimes_{k = 1}^n \quadrature{1}{i_k},
\end{equation}
in which case
\begin{align}
  & \tensorindex{n}{\vi}
  = \tensorindex{1}{i_1} \times \cdots \times \tensorindex{1}{i_n} \text{ and} \elab{quadrature-tensor-index} \\
  & n_{\vi}
  = \cardinality{\tensorindex{n}{\vi}}
  = \prod_{k = 1}^n \cardinality{\tensorindex{n}{i_k}}
  = \prod_{k = 1}^n n_{i_k}. \nonumber
\end{align}
It can be seen that, in this case, the growth of the number of points with
respect to the number of dimensions is exponential. In low dimensions, the grows
is manageable; however, in high dimensions, the situation changes dramatically
as the number of points produced by this approach can easily explode.

To give an example \cite{heiss2008}, suppose that each one-dimensional
quadrature has only four points, that is, $n_i = 4$. Then, in 10 stochastic
dimensions, that is, when $n = 10$, the number of multivariate points becomes
$n_{\vi} = 4^{10} = 1~048~576$, which is by far not affordable. Moreover, it can
be shown that most of the points obtained by the full tensor produce do not
contribute to the asymptotic accuracy and, therefore, are a waste of the
computation time. In particular, if the integrand under consideration is a
polynomial whose total order is constrained according to a certain strategy, the
full tensor product cannot take this information into account. Consequently, a
different construction technique should be utilized in the case of
high-dimensional integration problems.

An alternative construction is the Smolyak algorithm \cite{smolyak1963}; see
also \cite{klimke2006, eldred2008, heiss2008, maitre2010}. The algorithm is a
central technique in the field of not only integration but also interpolation;
the latter is elaborated on in \xref{sparse-interpolation}. In the context of
integration, the algorithm combines a range of one-dimensional quadratures in
such a way that the resulting grid is tailored to be exact only for a certain
polynomial subspace. Such grids are called sparse grids, and they allow for a
significant reduction of the number of points and, thus, the subsequent work.
For instance, in the example given earlier, the number of points would only be
$n_{\vi} = 1~581$, which is a drastic saving of the computation time.

To begin with, define
\begin{align*}
  & \quadrature{1}{-1} = 0, \\
  & \Delta\quadrature{1}{i} = \quadrature{1}{i} - \quadrature{1}{i - 1}, \text{ and} \\
  & \Delta\quadrature{n}{\vi} = \bigotimes_{k = 1}^n \Delta\quadrature{1}{i_k}.
\end{align*}
Then Smolyak's formula of level \lq is as follows:
\begin{equation} \elab{quadrature-sparse}
  \quadrature{n}{\lq} = \bigoplus_{\vi \in \sparseindex{n}{\lq}} \Delta\quadrature{n}{\vi}.
\end{equation}
In the original (isotropic) formulation of the Smolyak algorithm,
\begin{equation} \elab{index-total-order-isotropic}
  \sparseindex{n}{\lq} = \left\{ \vi: \vi \in \natural^n, \norm[1]{\vi} \leq \lq \right\}
\end{equation}
where $\norm[1]{\cdot}$ stands for the Manhattan norm. The index set
$\sparseindex{n}{\lq}$ is called a total-order index set \cite{eldred2008,
beck2011}, and its cardinality can be calculated as follows:
\begin{equation} \elab{index-total-order-isotropic-length}
  \cardinality{\sparseindex{n}{\lq}} = {n + \lq \choose n} = \frac{(n + \lq)!}{n! \, \lq!}.
\end{equation}
It can be seen that the construction in \eref{quadrature-sparse} is a summation
of a number of cherry-picked tensor products of one-dimensional quadratures;
this formula is well suited for grasping the structure of the resulting sparse
integration grid. Note also that \eref{quadrature-sparse} reduces to
\eref{quadrature-tensor} if we let
\[
  \sparseindex{n}{\lq} = \left\{ \vi: \vi \in \natural^n, \norm[\infty]{\vi} \leq \lq \right\}.
\]

Using \eref{quadrature-sparse}, the integral in question is approximated as
\begin{equation} \elab{quadrature-summation}
  I \approx \quadrature{n}{\lq}{\f}
  = \sum_{j \in \tensorindex{n}{\lq}} \f(\vx_j) w_j.
\end{equation}
For convenience, the points and weights of $\quadrature{n}{\lq}$ are indexed
using a single one-dimensional set $\tensorindex{n}{\lq} \subset \natural$ whose
cardinality is denoted by \nq. Note that, even though the level \lq is not
indicated in the notation of points and weights, it should be understood that
they are generally different for different levels.

Lastly, consider the following more general integral:
\[
  I = \int_{\real^n} \f(\vx) \d F(\vx).
\]
In this case, \f is integrated with respect to a measure $F: \real^n \to \real$
\cite{durrett2010} that does not necessarily correspond to the usual Lebesgue
measure, which is used in the earlier examples. Since integrating with respect
to a certain measure is a very frequent operation, there are families of
quadratures that are designed to automatically take this aspect into account in
the most common scenarios. For instance, the Gauss--Hermite family is suitable
for integrating with respect to the standard Gaussian measure, which can be seen
in \eref{gaussian-measure}.

It is also worth emphasizing that quadratures are generally precomputed and
tabulated since they do not depend the function being integrated.

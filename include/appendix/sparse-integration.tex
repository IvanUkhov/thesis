In numerical integration, the integral of a function $\f: \real \to \real$
\[
  I = \int_\real \f(\x) \d \x
\]
is approximated as
\[
  I \approx \quadrature{1}{i}{\f} = \sum_{j \in \tensorindex{1}{i}} \f(\x_{ij}) w_{ij},
\]
which is a summation of the function's values computed at prescribed points and
multiplied by prescribed weights. Such a pair of a set of points and a set of
weights is called a quadrature rule. In the notation $\quadrature{1}{i}$, the
superscript $1$ indicates that it is a one-dimensional rule, and the subscript
$i \in \natural$ gives the level of the rule, which is its index in the
corresponding family of rules with increasing precision. \emph{Precision} refers
to the maximum total order of polynomials that the rule integrates exactly
\cite{heiss2008}. In the above formula,
\begin{align*}
  & \X^1_i = \{ \x_{ij}: j \in \tensorindex{1}{i} \} \subset \real \text{ and} \\
  & \W^1_i = \{ w_{ij}: j \in \tensorindex{1}{i} \} \subset \real
\end{align*}
are the points and weights, respectively, and
\[
  \tensorindex{1}{i} = \{ j - 1 \}_{j = 1}^{n_i} \subset \natural
\]
is a set indexing the first two sets where $n_i$ denotes its cardinality.

Similarly, in multiple dimensions, the integral of a function $\f: \real^n \to
\real$
\[
  I = \int_{\real^n} \f(\vx) \d \vx
\]
is approximated as
\[
  I \approx \quadrature{n}{\vi}{\f} = \sum_{\vj \in \tensorindex{n}{\vi}} \f(\vx_{\vi \vj}) w_{\vi \vj}
\]
where $\vi = (i_k) \in \natural^n$ and $\vj = (j_k) \in \natural^n$ are
(multi-)indices. Further,
\begin{align*}
  & \X^n_{\vi} = \{ \vx_{\vi \vj}: \vj \in \tensorindex{n}{\vi} \} \subset \real^n \text{ and} \\
  & \W^n_{\vi} = \{ w_{\vi \vj}: \vj \in \tensorindex{n}{\vi} \} \subset \real
\end{align*}
and the points and weights, respectively, and $\tensorindex{n}{\vi} \subset
\natural^n$ is an index set whose cardinality dependents on both $n$ and \vi and
is denoted by $n_{\vi}$.

The foundation of an $n$-dimensional rule $\quadrature{n}{\vi}$ is a set of
one-dimensional counterparts of various levels. The most straightforward
construction of such a rule is the full tensor product denoted by
\begin{equation} \elab{quadrature-tensor}
  \quadrature{n}{\vi} = \bigotimes_{k = 1}^n \quadrature{1}{i_k},
\end{equation}
in which case
\begin{align}
  & \tensorindex{n}{\vi}
  = \tensorindex{1}{i_1} \times \cdots \times \tensorindex{1}{i_n} \text{ and} \elab{quadrature-tensor-index} \\
  & n_{\vi}
  = \cardinality{\tensorindex{n}{\vi}}
  = \prod_{k = 1}^n \cardinality{\tensorindex{n}{i_k}}
  = \prod_{k = 1}^n n_{i_k}. \nonumber
\end{align}
It can be seen that, in this case, the growth of the number of points with
respect to the number of dimensions is exponential. In low dimensions, the grows
is manageable; however, in high dimensions, the situation changes dramatically
as the number of points produced by this approach can easily explode. For
instance \cite{heiss2008}, if each one-dimensional rules has only 4 points, that
is, $n_i = 4$, then in 10 stochastic dimensions, that is, $n = 10$, the number
of multivariate points becomes $n_{\vi} = 4^{10} = 1~048~576$, which is by far
not affordable. Moreover, it can be shown that most of the points obtained in
this way do not contribute to the asymptotic accuracy and, therefore, are a
waste of time. In particular, if the integrand under consideration is a
polynomial whose total order is constrained according to a certain strategy, the
full tensor product cannot take this information into account. Consequently, a
different construction technique should be utilized in the case of
high-dimensional integration problems.

An alternative construction is the Smolyak algorithm \cite{smolyak1963}; see
also \cite{klimke2006, eldred2008, heiss2008, maitre2010}. The algorithm is a
central technique in the field of not only integration but also interpolation;
the latter is elaborated on in \sref{sparse-interpolation}. In the context of
integration, the algorithm combines a range of one-dimensional rules in such a
way that the resulting grid is tailored to be exact only for a certain
polynomial subspace. Such grids are called sparse grids, and they allow for a
significantly reduction of the number of points and, thus, of the subsequent
work. For instance, in the example given earlier, the number of points would
only be $n_{\vi} = 1~581$, which is a drastic saving of the computation time.

To begin with, define
\begin{align*}
  & \quadrature{1}{-1} = 0, \\
  & \Delta\quadrature{1}{i} = \quadrature{1}{i} - \quadrature{1}{i - 1}, \text{ and} \\
  & \Delta\quadrature{n}{\vi} = \bigotimes_{k = 1}^n \Delta\quadrature{1}{i_k}.
\end{align*}
Then Smolyak's formula of level \lq is as follows:
\begin{equation} \elab{quadrature-sparse}
  \quadrature{n}{\lq} = \bigoplus_{\vi \in \sparseindex{n}{\lq}} \Delta\quadrature{n}{\vi},
\end{equation}
and the corresponding approximation is written as
\begin{equation} \elab{quadrature-summation}
  I \approx \quadrature{n}{\lq}{\f} = \sum_{\vi \in \sparseindex{n}{\lq}} \f(\vx_{\vi}) w_{\vi}.
\end{equation}
Note that, even though the level \lq is not indicated in the notation
$\vx_{\vi}$ and $w_{\vi}$, points and weights that belong to different levels
are generally different. In the original (isotropic) formulation of the Smolyak
algorithm,
\begin{equation} \elab{index-total-order-isotropic}
  \sparseindex{n}{\lq} = \left\{ \vi: \vi \in \natural^n, \norm[1]{\vi} \leq \lq \right\}
\end{equation}
where $\norm[1]{\cdot}$ stands for the Manhattan norm. The index set
$\sparseindex{n}{\lq}$ is called a total-order index set since it constrains
polynomials by their total order \cite{eldred2008, beck2011}, and its
cardinality can be calculated as follows:
\begin{equation} \elab{index-total-order-isotropic-length}
  \nq = \cardinality{\sparseindex{n}{\lq}} = {n + \lq \choose n} = \frac{(n + \lq)!}{n! \, \lq!}.
\end{equation}
It can be seen that the construction in \eref{quadrature-sparse} is a summation
of cherry-picked tensor products of one-dimensional quadrature rules; this
formula is well suited for grasping the structure of the resulting sparse
integration grid. Note also that \eref{quadrature-sparse} reduces to
\eref{quadrature-tensor} if we let
\[
  \sparseindex{n}{\lq} = \left\{ \vi: \vi \in \natural^n, \max_{k = 1}^n i_k \leq \lq \right\}.
\]

Consider now the following more general integral:
\[
  I = \int_{\real^n} \f(\vx) \d F(\vx).
\]
In this case, \f is integrated with respect to a measure $F: \real^n \to \real$
\cite{durrett2010} that does not necessarily correspond to the usual Lebesgue
measure, which is used in the earlier examples. Since integrating in this way is
a very frequent operation, there are families of quadrature rules that are
designed to automatically take this aspect into account for the most common
scenarios.

We would like to highlight one broad family of quadrature rules known as
Gaussian quadratures. In one dimension, the precision of a Gaussian quadrature
with \nq points is $2 \nq - 1$ \cite{heiss2008} or, equivalently, $2 \lq + 1$
since $\nq = \lq + 1$ in this case. In other words, a Gaussian quadrature with
\nq points is exact for polynomials with orders up to $2 \nq - 1$ or $2 \lq +
1$, which is a remarkable property of Gaussian quadratures that makes such rules
especially efficient.

Lastly, it is worth noting that quadrature rules are generally precomputed and
tabulated since they do not depend the integrand; see, for instance,
\cite{burkardt}.

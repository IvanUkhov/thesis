Let \x be an uncertain parameter that we would like to characterize. To this
end, the following information is at our disposal: \one~a set of observations
$Y$ of a quantity \y that depends on \x; \two~a data model \f that describes the
relation between \x and \y, which is denoted by $\y = \f(\x)$; and \three~prior
knowledge (or beliefs) about \x. A natural solution is Bayes' theorem
\cite{gelman2013}
\begin{equation} \elab{bayes-theorem}
  p(\x | Y) \propto p(Y | \x) p(\x)
\end{equation}
where $p$ denotes a \ac{PDF}. In this context, $p(Y | \x)$ is known as the
likelihood function; it accommodates the data model \f and yields the relative
likelihood of observing the data set $Y$ given the parameter \x, that is, given
a particular assignment of the uncertain parameter \x. The distribution that
corresponds to $p(\x)$ is known as the prior (distribution) of \x; it represents
the knowledge about \x that is available prior to any observations. The
distribution that corresponds to $p(\x | Y)$ is known as the posterior
(distribution) of \x; it yields the relative likelihood of the parameter \x
given the observations gathered in $Y$ and taking into consideration the prior
knowledge about \x. The posterior is an exhaustive solution to our problem:
having constructed $p(\x | Y)$, the needed statistics about \x can be trivially
estimated by drawing samples from this posterior.

In practice, the posterior distribution is unlikely to belong to any of the
common families of probability distributions, which is in part due to the data
model involved in the likelihood function, and, therefore, the sampling
procedure is not straightforward. In order to circumvent this difficulty, one
usually relies on Markov chain Monte Carlo sampling \cite{gelman2013}. In this
case, an ergodic Markov chain with the stationary distribution equal to the
target posterior distribution is constructed and then utilized for exploring the
probability space.

A popular technique in this regard is the Metropolis--Hastings algorithm
\cite{gelman2013} where the chain is constructed via sampling from a
computationally convenient distribution known as the proposal (distribution).
Each sample drawn from the proposal is passed through the posterior in order to
calculates its posterior probability, which is then used to decide whether the
sample should be accepted or rejected. A rejection means that the sequence of
samples advances using the last accepted sample---as if it was drawn once again.
The acceptance strategy of the algorithm pushes the produced chain of samples
toward regions of high posterior probability, which results in an adequate
approximation of the target posterior distribution after a sufficient number of
steps, depending on the starting point of the chain and the efficiency of the
moves.

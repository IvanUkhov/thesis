Let \u be a set of unknown parameters that we would like to know. In order to
find them, the following information is at our disposal: \one~a set of
observations $H$ of a quantity \h that is related to \u; \two~a data model that
describes the relation between \u and \h; and \three~prior beliefs about what \u
should be. A natural solution is Bayes' theorem \cite{gelman2013}, which is as
follows:
\begin{equation} \elab{bayes-theorem}
  p(\u | H) \propto p(H | \u) p(\u).
\end{equation}
In this case, $p$ denotes a \ac{PDF}; $p(H | \u)$ is known as the likelihood
function, which accommodates the data model and yields the probability of
observing the data set $H$ given the parameters \u; $p(\u)$ is called the prior
of \u, which represents our knowledge about \u prior to any observations; and
$p(\u | H)$ is known as the posterior of \u given the data set $H$. The last is
an exhaustive solution to our problem: having constructed $p(\u | H)$, all the
needed information about \u can be trivially estimated by drawing samples from
this posterior.

The posterior distribution does not typically belong to any of the common
families of probability distributions, which is primarily due to the data model
involved in the likelihood function, and, therefore, the sampling procedure is
not straightforward. In order to tackle the difficulty, one usually relies on
Markov chain Monte Carlo sampling \cite{gelman2013}. In this case, an ergodic
Markov chain with the stationary distribution equal to the target posterior
distribution is constructed and then utilized for exploring the probability
space. A popular technique in this regard is the Metropolis--Hastings algorithm
where the chain is constructed via sampling from a computationally convenient
distribution known as the proposal distribution. Each sample drawn from the
proposal is passed through the posterior in order to calculates its posterior
probability, which is then used to decide whether the sample should be accepted
or rejected. A rejection means that the sequence of samples advances using the
last accepted sample as if it was drawn once again. The acceptance strategy of
the algorithm pushes the produced chain of samples toward regions of high
posterior probability, which, after a sufficient number of steps, depending on
the starting point of the chain and the efficiency of the moves, results in an
adequate approximation of the target posterior distribution.

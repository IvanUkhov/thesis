In this section, we turn our attention to temperature-aware reliability
analysis.

Let $(\Omega, \F, \probability)$ be a probability space as defined in
\xref{probability-theory} and let $\life: \Omega \to \real$ be a random variable
that represents the lifetime of the system at hand. The lifetime is the time
until the system experiences a fault after which it no longer meets certain
requirements. Let also $F(\cdot | \vg)$ be the distribution function of
\life---which gives the probability of failure before a certain time
moment---where \vg is a vector of parameters. The expectation
$\expectation{\life}$ is called the \ac{MTTF}. Lastly, the complementary
distribution function of \life is
\[
  R(t | \vg) = 1 - F(t | \vg),
\]
which, in the context of reliability analysis, gives the probability of survival
up to a certain time moment and is called the reliability function of the
system.

The lifetime \life of the system is a function of the lifetimes of the \np
processing elements that the system is composed of. Denote these individual
lifetimes by $\{ \life_i: \Omega \to \real \}_{i = 1}^\np$. Each $\life_i$ is
characterized by a physical model of wear \cite{jedec2016} that describes the
stress that processing element $i$ is exposed to. Each $\life_i$ is also
assigned an individual $R_i(\cdot | \vg_i)$ modeling the failures due to this
stress.

The structure of $R(\cdot | \vg)$ with respect to $\{ R_i(\cdot | \vg_i) \}_{i =
1}^\np$ is problem specific, and it can be especially diverse in the context of
fault-tolerant systems. Therefore, $R(\cdot | \vg)$ is to be devised by the
designer of the system under consideration. To give an example, suppose that the
failure of any of the \np processing elements makes the whole system fail, and
that $\{ \life_i \}_{i = 1}^\np$ are conditionally independent given the
parameters gathered in \vg. In this scenario,
\begin{equation} \elab{reliability-model}
  \begin{split}
    & \life = \min_{i = 1}^\np \life_i \text{ and} \\
    & R(t | \vg) = \prod_{i = 1}^\np R_i(t | \vg_i).
  \end{split}
\end{equation}

\subsection{Periodic Thermal Stress}

In this subsection, we present a model that is frequently used for
characterizing the lifetime $\life_i$ of a single processing element when it is
exposed to repetitive temperature-induced stress \cite{huang2009b, xiang2010}.
The considered scenario is that the system at hand is experiencing a periodic
workload with a certain period, which is denoted by \period. The corresponding
temperature profile is then a dynamic steady-state temperature profile, which is
to be discussed in \sref{dynamic-steady-state-analysis}.

The lifetime $\life_i$ is assumed to have a Weibull distribution as follows:
\[
  \life_i | \vg_i \sim \mathrm{Weibull}(\scale_i, \shape_i)
\]
where $\scale_i$ and $\shape_i$ are called the scale and shape parameters,
respectively, and $\vg_i = (\scale_i, \shape_i)$. The distribution function of
$\life_i$ is
\begin{equation} \elab{weibull-distribution}
  F_i(t | \vg_i) = 1 - \exp\left(-\left(\frac{t}{\scale_i}\right)^{\shape_i}\right);
\end{equation}
the reliability function of the processing element is
\begin{equation} \elab{weibull-reliability}
  R_i(t | \vg_i) = 1 - F_i(t | \vg_i) = \exp\left(-\left(\frac{t}{\scale_i}\right)^{\shape_i}\right);
\end{equation}
and the corresponding \ac{MTTF} is
\begin{equation} \elab{weibull-expectation}
  \mean_i = \expectation{\life_i} = \scale_i \: \Gamma\left(1 + \frac{1}{\shape_i}\right)
\end{equation}
where $\Gamma$ stands for the gamma function.

\begin{remark} \rlab{weibull-homogeneity}
The Weibull model has a special case: assuming that $\beta_i = \beta$ for $i =
\range{1}{\np}$, the reliability function in \eref{reliability-model} belongs to
a Weibull distribution whose shape parameter is $\beta$ and scale parameter is
given by
\[
  \scale = \left(\sum\left(\frac{1}{\eta_i}\right)^\beta\right)^{-\frac{1}{\beta}}.
\]
\end{remark}

It is natural to expect that the usage conditions and, hence, the corresponding
stress change over the period \period. Then the distribution of $\life_i$ should
reflect this aspect, which is not prominent in \eref{weibull-distribution}. In
order to take it into account, the period is split into \nk{i} time intervals
$\{ \dt_{ij} \}_{j = 1}^{\nk{i}}$ so that the conditions that are relevant to
the model stay unchanged within each interval. Let
\[
  \life_{ij} | \vg_{ij} \sim \mathrm{Weibull}(\scale_{ij}, \shape_{ij})
\]
be the time to failure that the processing element would have if interval $j$
was the only interval present and denote the corresponding \ac{MTTF} by
$\mean_{ij}$.

In the case of temperature-induced failures, we have that $\shape_{ij} =
\shape_i$ for $j = \range{1}{\nk{i}}$, that is, the shape parameters are all
equal. The reason is that, unlike the scale parameters, the shape parameters are
independent of the operating temperature \cite{chang2006}. As shown in
\cite{xiang2010}, in this scenario, the reliability function of the processing
element can still be approximated by means of \eref{weibull-reliability} with
the following scale parameter:
\[
  \scale_i = \frac{\sum_{j = 1}^{\nk{i}} \dt_{ij}}{\sum_{j = 1}^{\nk{i}} \frac{\Delta t_{ij}}{\scale_{ij}}}.
\]
Applying \eref{weibull-expectation} at the level of individual intervals,
$\scale_i$ is rewritten as
\[
  \scale_i = \frac{\sum_{j = 1}^{\nk{i}} \dt_{ij}}{\Gamma\left(1 + \frac{1}{\shape_i}\right) \sum_{j = 1}^{\nk{i}} \frac{\Delta t_{ij}}{\mean_{ij}}}.
\]
Note now that the model in \eref{weibull-reliability} becomes fully specified as
soon as $\dt_{ij}$ and $\mean_{ij}$ are identified for $j = \range{1}{\nk{i}}$.
This part of the model depends on the particular failure mechanism considered,
which we discuss next.

\subsection{Thermal-Cycling Fatigue}
\slab{thermal-cycling-fatigue}

Let us tailor the above Weibull model to the thermal-cycling fatigue
\cite{jedec2016}. The fatigue is of particular interest to us due to its
prominent dependence on temperature fluctuations: apart from the average and
maximum values, the frequencies and amplitudes of temperature oscillations
matter in this case.

The time intervals with constant relevant conditions correspond to thermal
cycles. In order to detect them in a given temperature curve, the curve is first
analyzed using a peak-detection algorithm in order to extract a set of extrema.
The rainflow counting method \cite{xiang2010} is then applied to these extrema.
The result is a set of \nk{i} thermal cycles. Each detected cycle is
characterized by a number of properties including its duration, which is the
desired $\dt_{ij}$. Regarding the corresponding $\mean_{ij}$, it can be
expressed as follows:
\[
  \mean_{ij} = \nk{ij} \dt_{ij}
\]
where \nk{ij} stands for the mean number of such cycles to failure.

\begin{remark}
A cycle detected by the rainflow counting method does not have to be formed by
adjacent extrema; cycles can overlap. This feature makes the counting method
very efficient at mitigating overestimation. A cycle can be a half cycle,
meaning that only an upward or downward swing is present in the time series,
which is assumed to be adequately taken into account.
\end{remark}

The number \nk{ij} is estimated using a modified version of the Coffin--Manson
equation with the Arrhenius term as follows \cite{xiang2010, jedec2016}:
\begin{equation} \elab{thermal-cycling-mean-cycles}
  \nk{ij} = a_i (\Delta \q_{ij} - \Delta \q_{0, ij})^{-b_i} \exp\left(\frac{c_i}{k \q_{\maximum, ij}}\right)
\end{equation}
where $a_i$, $b_i$ (called the Coffin--Manson exponent), and $c_i$ (called the
activation energy) are empirically determined constants; $k$ is the Boltzmann
constant; $\Delta \q_{ij}$ is the excursion of the cycle in question; $\Delta
\q_{0, ij}$ is the portion of the temperature excursion that resides in the
elastic region, which does not cause damage; and $\q_{\maximum, ij}$ is the
maximum temperature during the cycle.

The reliability model of a single processing element is now fully specified. The
reliability function is the one in \eref{weibull-reliability} with
\begin{equation} \elab{thermal-cycling-scale}
  \scale_i = \frac{\sum_{j = 1}^{\nk{i}} \dt_{ij}}{\Gamma\left(1 + \frac{1}{\shape_i}\right) \sum_{j = 1}^{\nk{i}} \frac{1}{\nk{ij}}}
\end{equation}
where \nk{ij} is given in \eref{thermal-cycling-mean-cycles}. Using
\eref{weibull-expectation}, the \ac{MTTF} of the processing element is as
follows:
\begin{equation} \elab{thermal-cycling-mean-time}
  \mean_i = \frac{\sum_{j = 1}^{\nk{i}} \dt_{ij}}{\sum_{j = 1}^{\nk{i}} \frac{1}{\nk{ij}}}.
\end{equation}

In conclusion, it is worth emphasizing that the reliability model requires
detailed information about the thermal cycles that the processing element is
exposed to, which is the topic of \cref{certainty-development} and, in
particular, \sref{dynamic-steady-state-analysis} where dynamic steady-state
temperature analysis is discussed in detail.

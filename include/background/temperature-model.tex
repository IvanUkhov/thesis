System-level temperature analysis of electronic systems is chiefly based on the
duality between the transfer of heat and the transfer of electric charge
\cite{kreith2000}. The core idea is to construct a thermal \up{RC} circuit for
the system under consideration \cite{skadron2003}. Such a circuit is a
collection of thermal nodes that are characterized by thermal capacitances and
connected with each other via thermal resistances. The circuit is to capture the
relevant physical structure and thermal properties of the system and, thereby,
model its thermal dynamics.

Similar to the identification of processing elements described in
\sref{system-model}, a thermal \up{RC} circuit can be constructed at different
levels of granularity, which, in this case, refers primarily to the number of
thermal nodes denoted by \nn and their placement. It should be duly noted that
the chosen level of granularity impacts on the accuracy of the subsequent
temperature calculations.

\inputfigure{thermal-circuit}
In order to give a better intuition about thermal \up{RC} circuits,
\fref{thermal-circuit} depicts a simplified example of such a circuit
constructed for a dual-core chip equipped with a thermal package composed of a
thermal interface material, heat spreader, and heat sink. Given \np processing
elements, each one is represented by one thermal node, and the thermal interface
material, heat spreader, and heat sink are captured by \np, $\np + 4$, and $\np
+ 8$ thermal nodes, respectively. Consequently, $\nn = 4 \times \np + 12$. In
this particular example, since $\np = 2$, $\nn = 20$. The construction principle
described in this paragraph is the one presented in \cite{huang2008}, and it is
also the one utilized throughout the thesis.

Suppose now that an adequate thermal \up{RC} circuit has been constructed for
the considered electronic system. Regardless of the structure of the circuit,
the thermal dynamics of the circuit and, therefore, the electronic system itself
are governed by the following system of \nn differential and \np algebraic
equations, which is based on Fourier's law \cite{fourier2009}:
\begin{subnumcases}{\elab{temperature-model-original}}
  \m{C} \frac{\d \tvs(t)}{\d t} + \m{G} \tvs(t) = \tm{B} \vp(t) \elab{temperature-differential-original} \\
  \vq(t) = \transpose{\tm{B}} \tvs(t) + \vq_\ambient \elab{temperature-algebraic-original}
\end{subnumcases}
where $\vq \in \real^\np$, $\vq_\ambient \in \real^\np$, and $\vp \in \real^\np$
are the operating temperature, ambient temperature, and power consumption of the
processing elements, respectively. Further, $\tvs \in \real^\nn$ represents the
state of the thermal nodes, and $\m{C} \in \real^{\nn \times \nn}$ and $\m{G}
\in \real^{\nn \times \nn}$ are a diagonal matrix of the thermal capacitance and
a symmetric and positive definite matrix of the thermal conductance,
respectively. Lastly, $\tm{B} \in \real^{\nn \times \np}$ is a mapping between
the processing elements and the thermal nodes, which, without loss of
generality, is assumed to be a rectangular diagonal matrix whose diagonal
elements are equal to unity.

In general, the differential part in \eref{temperature-differential-original} is
nonlinear due to \vp since we do not make any assumptions about its structure;
see also the discussion in \sref{power-model}. Therefore, there is no
closed-form solution to the system.

Lastly, let us note that temperature analysis of electronic systems is generally
considered to be a computationally expensive operation. The long computation
times can be severely limiting, especially when the analysis is to be performed
many times for such purposes as design-space exploration. Consequently, there is
always a need for developing more efficient techniques.

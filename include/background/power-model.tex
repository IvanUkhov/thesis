The total power consumption of an electrical circuit consists of two main
components: dynamic and static. Dynamic power is the result of the actual useful
work done by the circuit. By contract, static power is the result of various
parasitic currents that cannot be entirely eliminated. The total power consumed
by \np processing elements at time $t$ is then written as
\begin{equation} \elab{power-model}
  \vp(t) = \vp_\dynamic(t) + \vp_\static(t)
\end{equation}
where $\vp_\dynamic \in \real^\np$ and $\vp_\static \in \real^\np$ are the
aforementioned dynamic and static components of the total dissipation of power,
respectively.

Dynamic power is modeled as follows:
\[
  \p_\dynamic = C f V^2
\]
where $C$ is the effective switched capacitance, $f$ is the frequency, and $V$
is the supply voltage of the circuit. Static power is due chiefly to the leakage
current \cite{chandrakasan2000, srivastava2010, juan2011, juan2012}, especially
in modern \up{CMOS} transistors. Leakage power and, by extension, static power
are modeled as follows \cite{liao2005}:
\begin{equation} \elab{static-power}
  \p_\static = \n{g} V I_0 \left(a \, \q^2 e^{\frac{\alpha V + \beta}{\q}} + b \, e^{\gamma V + \delta}\right)
\end{equation}
where \q is the current temperature, $V$ is the current supply voltage, $I_0$ is
the average leakage current at the reference temperature and reference supply
voltage, and \n{g} is the number of gates in the circuit. The quantities $a$,
$b$, $\alpha$, $\beta$, $\gamma$, and $\delta$ are technology-dependent
constants, which can be found in \cite{liao2005}. There are two aspects to note
with respect to the equations given above.

First, the dynamic power component does not depend on the operating temperature,
whereas the static one does. The dependence of static (and hence total) power on
temperature is positive and strong: a higher level of heat dissipation leads to
a considerably higher level of power consumption. On the other hand, as more
power is consumed, more heat is dissipated. Consequently, power and temperature
constantly instigate each other; they are interdependent. This phenomenon cannot
be neglected, since it leads to a higher level of power (energy) consumption
than expected and a higher level of heat dissipation than expected and, in the
worst case, could cause a thermal runaway.

Second, even though it is not explicitly spelled out, static (and hence total)
power depends on a number of process parameters, including the effective channel
length and gate oxide thickness. In particular, the effective channel length has
one of the strongest impacts on the leakage current \cite{juan2012}.

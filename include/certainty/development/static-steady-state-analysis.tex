The goal of static steady-state analysis is to calculate the temperature of the
thermal equilibrium that the system reaches when its power consumption stays at
a certain constant level \vp for a sufficiently long period of time. In this
case, there are no dynamics, which means that the derivative in
\eref{temperature-differential} is zero. Then the static steady-state
temperature \vq can simply be calculated as
\[
  \vs = -\m{A}^{-1} \m{B} \vp
\]
followed by \eref{temperature-algebraic} where $\m{A}^{-1}$ is computed using
\eref{matrix-inverse}. In this case, \vp and \vq can be viewed as a degenerate
power profile \mp and a degenerate temperature profile \mq, respectively, in the
sense that $\ns = 1$.

The static steady-state temperature produced by static steady-state analysis is
frequently used as a computationally cheap approximation of the thermal behavior
of the system. For instance, given a power profile \mp with $\ns > 1$, one might
simply compute the average power vector \vp, undertake static steady-state
analysis, and utilize the resulting static \vq as a guide inside a
temperature-aware design-space-exploration procedure. However, this
approximation is of limited applicability. It assumes that the system under
consideration functions at one constant temperature at each spatial location,
which does not hold in the majority of cases since, in reality, power
consumption readily changes.

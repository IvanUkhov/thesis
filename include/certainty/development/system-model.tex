Consider a generic computer system that consists of \np heterogeneous processing
elements and is equipped with a thermal package. A processing element is any
active component of the system, that is, any component that consumes power and
dissipates heat. The thermal package is the cooling equipment of the system
including any passive components, that is, any components that do not consume
power. Processing elements can be identified at different levels of granularity.
For instance, a processing element can correspond to a system on a chip, central
processing unit, arithmetic logic unit, cache memory, or communication bus. In
the majority of the discussions in the thesis, processing elements are treated
as processing units or cores of a multiprocessor system.

The system is characterized by power profiles and temperature profiles, which
are discrete representations of the system's power consumption and heat
dissipation, respectively. A power profile is an $\np \times \ns$ matrix
\[
  \mp
  = \left(\range{\vp_1}{\vp_\ns}\right)
  = \left(\range{\vp(t_1)}{\vp(t_\ns)}\right) \in \real^{\np \times \ns}
\]
that contains \ns samples $\{ \vp_i \}_{i = 1}^\ns \subset \real^\np$ of the
power consumption of the \np processing elements taken at \ns time moments $\{
t_i \}_{i = 1}^\ns$ with a certain sampling interval \dt such that $\dt = t_i -
t_{i - 1}$ for $i = \range{2}{\ns}$. The number \ns is called the number of
steps. Similarly, a temperature profile is an $\np \times \ns$ matrix
\begin{equation} \elab{temperature-profile}
  \mq
  = \left(\range{\vq_1}{\vq_\ns}\right)
  = \left(\range{\vq(t_1)}{\vq(t_\ns)}\right) \in \real^{\np \times \ns}
\end{equation}
that captures the heat dissipation of the processing elements over a number of
equidistant time moments. Let us note that, even though the exact time frame and
sampling interval of a power or a temperature profile are not included in the
above notation, they are typically understood from the context.

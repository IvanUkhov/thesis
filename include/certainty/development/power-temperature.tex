So far, the interdependence between power and temperature, which is introduced
in \sref{power-model}, has been ignored. In order to properly take it into
account, several approaches can be employed as follows.

The first approach is to calculate the power and temperature profiles several
times in turns until the latter converges. In this case, we obtain a series of
pairs of a power profile and a temperature profile
\[
  \set{(\mp_i, \mq_i)}{i = 1, 2, \dots}.
\]
For each new temperature profile $\mq_i$---which is computed by performing
transient, static steady-state, or dynamic steady-state analysis as usual---a
new power profile $\mp_i$ is obtained by recalculating the static component of
power and adding it to the dynamic one, which we write as follows:
\[
  \mp_i = \mp_\dynamic + \mp_{\static, i}(\mq_{i - 1})
\]
where $\mp_{\static, i}$ is computed using $\mq_{i - 1}$. The process continues
until a stopping condition is satisfied, such as when the difference between two
successive temperature profiles $\mq_{i - 1}$ and $\mq_i$ drops below a
predefined threshold.

\inputalgorithm{dream-dynamic-steady-solution-iterative}
In the case of dynamic steady-state analysis, which is discussed in
\sref{dynamic-steady-analysis}, the pseudocode of the aforementioned iterative
procedure is listed in \aref{dream-dynamic-steady-solution-iterative}. The two
main steps of this algorithm are as follows.

Line~3: The power profile \mp is updated using the current temperature profile
\mq. In the first iteration, the ambient temperature is used for this purpose.

Line~4: The temperature profile \mq is updated using the current power profile
\mp. This task is delegated to \aref{dream-dynamic-steady-solution} discussed in
\sref{dynamic-steady-solution}.

In order to give a better sense of the convergence rate of the iterative
procedure, let us give an example. According to our experience, it typically
takes 4--7 iterations until the uniform norm of the difference between two
successive dynamic steady-state temperature profiles becomes smaller than
\celsius{0.5}, which is considered to be sufficiently accurate for many
applications.

In the case of transient analysis, the iterative procedure can also be done on a
step-by-step basis, even with only one iteration. Specifically, at each step of
the iterative process in \eref{transient-recurrence}, one can simply calculate
the static power component at the current temperature, add it to the dynamic
one, and use the result for evaluating the next temperature. Hence, the
recurrence becomes
\[
  \vs_i = \m{E} \vs_{i - 1} + \m{F} (\vp_{\dynamic, i} + \vp_{\static, i}(\vq_{i - 1}))
\]
for $i = \range{1}{\ns}$ where, as indicated, $\vp_{\static, i}$ is computed
using $\vq_{i - 1}$.

The second approach is to use a linear approximation of static power, which can
be formalized as follows:
\[
  \vp_\static(t) = \hm{A} \vq(t) + \hv{b}
\]
where $\hm{A} \in \real^{\nn \times \nn}$ and $\hv{b} \in \real^\nn$ are a
diagonal matrix and a vector to be fitted. It can be shown that the linear
approximation keeps the original system of differential equations in
\eref{temperature-differential-original} almost intact. More precisely, it
becomes
\[
  \m{C} \frac{\d \tvs(t)}{\d t} + \hm{G} \tvs(t) = \tm{B} \hv{p}(t)
\]
where
\begin{align*}
  & \hm{G} = \m{G} - \tm{B} \hm{A} \transpose{\tm{B}} \text{ and} \\
  & \hv{p}(t) = \vp_\dynamic(t) + \hm{A} \vq_\ambient + \hv{b},
\end{align*}
and $\hm{G}$ has the same properties as $\m{G}$. Therefore, all the solutions
that have been derived in this chapter remain perfectly valid. Moreover, as
shown in \cite{liu2007}, despite its simplicity, the approximation is relatively
accurate and, if needed, can be improved by considering multiple linear
segments.

In order to evaluate the linearization, we consider a number of hypothetical
platforms with 2--32 processing elements and undertake dynamic steady-state
analysis. The experiment is performed with both the linear approximation and the
full exponential model in \eref{static-power}. For the former, the model is
fitted using least squares \cite{press2007} targeted at the range
\celsius{40}--\celsius{80}. For the latter, we use the iterative approach
described earlier. The results show that the \ac{NRMSE} is bounded by 2\%,
indicating that the linearization is adequately accurate.

It is important to note that, regardless of the approach utilized, power and
temperature are analyzed simultaneously, as they are interdependent. One obtains
not only a temperature profile but also the corresponding static or total power
profile, and all profiles take account of the power-temperature interplay.

Lastly, let us mention that static power is not considered in the experiments
given in \sref{dynamic-steady-results}. However, if it was taken into account
using the linearization, the computation times would remain unchanged, and, if
the iterative model was utilized, the computation times would increase
proportionally for all the techniques, which would not affect any of the
conclusions.

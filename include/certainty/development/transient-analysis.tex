The goal of transient analysis is to compute the temperature profile \mq that
corresponds to a given power profile \mp by solving
\eref{temperature-model-original}. Traditionally, this operation is undertaken
numerically \cite{skadron2003}; however, we are interested in obtaining and
working with an analytical solution to the system since such a solution has many
advantages as we shall see throughout the thesis.

\subsection{\priortitle}

Let us discuss an analytical approach to solving the system of differential
equations given in \eref{temperature-differential-original}. To begin with, the
system is rewritten as follows:
\[
  \frac{\d \tvs(t)}{\d t} = \tm{A} \tvs(t) + \m{C}^{-1} \tm{B} \vp(t)
\]
where
\[
  \tm{A} = -\m{C}^{-1} \m{G}.
\]
Now we apply a solution technique taken from the family of exponential
integrators, which have good stability properties; refer to \cite{hochbruck2010}
for an overview. Multiplying both sides of the above equation by $e^{- \tm{A}
t}$ and noting that
\[
  e^{-\tm{A} t} \frac{\d \tvs(t)}{\d t} = \frac{\d e^{-\tm{A} t} \tvs(t)}{\d t} + e^{-\tm{A} t} \tm{A} \tvs(t),
\]
we obtain
\[
  \tvs(t) = e^{\tm{A} t} \tvs(0) +  e^{\tm{A} t} \int_0^t e^{-\tm{A} \tau} \m{C}^{-1} \tm{B} \vp(\tau) \d \tau,
\]
which is the solution at time $t$ starting from the initial condition $\tvs(0)$
at time 0. Suppose now that the power consumption of the processing elements
does not change over time; hence, $\vp(t) = \vp$. In this case, the system is a
system of linear differential equations that has the following analytical
solution:
\begin{equation} \elab{transient-state-solution-original}
  \tvs(t) = e^{\tm{A} t} \tvs(0) + \tm{A}^{-1} (e^{\tm{A} t} - \m{I}) \m{C}^{-1} \tm{B} \vp
\end{equation}
where $\m{I}$ is the identity matrix.

Assume now that the sampling interval \dt of \mp is sufficiently small so that
the power consumption in the interval $[t_i, t_{i + 1})$ can reasonably be
approximated by a constant equal to $\vp_i = \vp(t_i)$. Then the corresponding
\mq can be found by applying the following recurrence derived from
\eref{transient-state-solution-original}:
\begin{equation} \elab{transient-state-recurrence-original}
  \tvs_{i} = \tm{E} \tvs_{i - 1} + \tm{F} \vp_i
\end{equation}
for $i = \range{1}{\ns}$ where
\begin{align*}
  & \tvs_0 = \v{0} = (\range{0}{0}), \\
  & \tm{E} = e^{\tm{A} \dt}, \text{ and} \\
  & \tm{F} = \tm{A}^{-1}(e^{\tm{A} \dt} - \m{I})\m{C}^{-1} \tm{B}.
\end{align*}
It should be noted that, in order to obtain the actual \mq, the recurrence
should be followed by \eref{temperature-algebraic-original}, which involves two
trivial algebraic operations.

Similar to the observation made in \cite{thiele2011}, our experience shows that
the approach to transient analysis described above provides a significant
performance improvement compared to iterative solutions to systems of ordinary
differential equations such as the fourth-order Runge--Kutta method
\cite{press2007}. However, there is still room for improvement as we discuss in
what follows.

\subsection{\solutiontitle}
\slab{transient-state-solution}

Even though the matrices $\tm{E}$ and $\tm{F}$ have to be computed only once,
they necessitate two computationally problematic operations: the matrix
exponential and matrix inverse involving $\tm{A} \in \real^{\nn \times \nn}$,
which is a generic matrix. It is preferable to have a symmetric matrix $\m{A}
\in \real^{\nn \times \nn}$ when these operations are concerned since such a
matrix admits the eigendecomposition, which can be seen in
\eref{eigendecomposition}. Having computed such a decomposition, the calculation
of the matrix exponential and matrix inverse becomes trivial as follows:
\begin{align}
  & e^{\m{A} \dt}
  = \m{U} e^{\m{\Lambda} \dt} \transpose{\m{U}}
  = \m{U} \: \diagonal{e^{\lambda_1 \dt}}{e^{\lambda_\nn \dt}} \transpose{\m{U}} \text{ and} \elab{matrix-exponential} \\
  & \m{A}^{-1}
  = \m{U} \m{\Lambda}^{-1} \transpose{\m{U}}
  = \m{U} \: \diagonal{\lambda_1^{-1}}{\lambda_\nn^{-1}} \transpose{\m{U}}. \elab{matrix-inverse}
\end{align}

In order to obtain such an $\m{A}$, we propose to perform an auxiliary
transformation. Recall first that the conductance matrix $\m{G}$ is a symmetric
matrix, which, intuitively, is due to the fact that, if node $i$ is connected to
node $j$ with a certain conductance, node $j$ is also connected to node $i$ with
the same conductance; see \fref{thermal-circuit}. However, $\tm{A} = -\m{C}^{-1}
\m{G}$ does not have this property. The desired symmetry can be kept intact
using the following substitution:
\[
  \begin{split}
    & \vs(t) = \m{C}^{\frac{1}{2}} \tvs(t) \text{ and} \\
    & \m{A} = -\m{C}^{-\frac{1}{2}} \m{G} \: \m{C}^{-\frac{1}{2}}
  \end{split}
\]
where $\m{A}$ is symmetric since
\[
  \transpose{\m{A}}
  = -\transpose{(\m{C}^{-\frac{1}{2}} \m{G} \m{C}^{-\frac{1}{2}})}
  = -\transpose{(\m{C}^{-\frac{1}{2}})} \transpose{\m{G}} \transpose{(\m{C}^{-\frac{1}{2}})}
  = -\m{C}^{-\frac{1}{2}} \m{G} \m{C}^{-\frac{1}{2}}
  = \m{A}.
\]
Consequently, \eref{temperature-model-original} is rewritten as follows:
\begin{subnumcases}{\elab{temperature-model}}
  \frac{\d \vs(t)}{\d t} = \m{A} \vs(t) + \m{B} \vp(t) \elab{temperature-differential} \\
  \vq(t) = \transpose{\m{B}} \vs(t) + \vq_\ambient \elab{temperature-algebraic}
\end{subnumcases}
where
\[
  \m{B} = \m{C}^{-\frac{1}{2}} \tm{B}.
\]
Similarly, the solution in \eref{transient-state-solution-original} becomes
\[
  \vs(t) = e^{\m{A} t} \vs_0 + \m{A}^{-1} (e^{\m{A} t} - \m{I}) \m{B} \v{p},
\]
and the recurrence in \eref{transient-state-recurrence-original} becomes
\begin{equation} \elab{transient-state-recurrence}
  \vs_i = \m{E} \vs_{i - 1} + \m{F} \vp_i
\end{equation}
for $i = \range{1}{\ns}$ where
\begin{align*}
  & \vs_0 = \v{0}, \\
  & \m{E} = e^{\m{A} \dt}, \text{ and} \\
  & \m{F} = \m{A}^{-1} \left( e^{\m{A} \dt} - \m{I} \right) \m{B}.
\end{align*}
Using the eigendecomposition in \eref{eigendecomposition}, the last equation can
be efficiently computed in the following way:
\[
  \m{F} = \m{U} \: \diagonal{\frac{e^{\lambda_1 \dt} - 1}{\lambda_1}}{\frac{e^{\lambda_\nn \dt} - 1}{\lambda_\nn}} \transpose{\m{U}} \m{B}.
\]
As before, the recurrence in \eref{transient-state-recurrence} should be
followed by \eref{temperature-algebraic} in order to obtain \mq. The performed
auxiliary transformation is helpful not only for transient analysis but also in
other contexts as we shall see later on.

Let us note at this point that there have been attempts to simplify the
temperature model by making additional assumptions in order to reduce the size
of the circuit and, thereby, speed up the solution process. For instance, the
techniques proposed in \cite{bao2010, rai2011} are targeting single-core
platforms, and the approach described in \cite{rao2009} is aimed at homogeneous
platforms and applications where the execution times of individual tasks are
comparable with the thermal time constant of the thermal package, which is in
the order of 100~s. Such techniques can be combined with what we present in this
thesis as long as $\m{C}$ stays a diagonal matrix, and $\m{G}$ stays a symmetric
and positive definite matrix.

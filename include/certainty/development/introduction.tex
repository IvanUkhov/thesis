Although there are many aspects that are of significance to the design of an
electronic system, our interest revolves primarily around those that are related
to power and temperature, which is due to the following reason. Power
consumption and heat dissipation are of great importance. Power translates to
energy, and energy to hours of battery life and to electricity bills.
Temperature, on the other hand, is one of the major causes of permanent damage
\cite{jedec2016}, which necessitates adequate cooling equipment and, thereby,
escalates product expenses \cite{chaudhry2015}. The situation is complicated
even further by the interdependence between power and temperature, which is
discussed in \sref{power-model} and \sref{power-temperature-interplay}: higher
power leads to higher temperature, and higher temperature to higher power
\cite{liu2007}. Therefore, taking power and temperature into consideration is
key to achieving effectiveness, efficiency, and robustness.

In what follows, we proceed to temperature analysis; it is important to realize,
however, that power analysis is inseparable from temperature analysis due to the
aforementioned power-temperature interplay. Hence, power analysis is always
implied whenever we discuss temperature analysis, and vice versa.

There are two types of temperature analysis: \one~transient analysis and
\two~steady-state analysis. The first delivers the temperatures that the system
traverses in an arbitrary time interval starting from an arbitrary initial
condition when it is exposed to an arbitrary, both spatially and temporally,
power distribution. The second delivers the temperature that the system attains
and retains once it has reached a thermal equilibrium under a power distribution
that is spatially arbitrary but temporally constant or repeating.

Steady-state analysis can be further classified into two categories: \one~static
analysis and \two~dynamic analysis. The first is concerned with temporally
constant power distributions, that is, with those that do not change over time.
In this case, the resulting temperature distributions do not change over time
either. The second is concerned with temporary arbitrary power distributions
that are periodic, that is, with those that repeat with certain periods. In this
case, the resulting temperature distributions change over time with the same
periods as the corresponding repeating power distributions. The considered
scenario is that the system at hand is exposed to a periodic workload or to such
a workload that can be approximated as periodic. Prominent examples that have
such periodic behaviors are various multimedia applications.

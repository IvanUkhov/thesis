In this section, we illustrate the importance of temperature analysis for
design-space exploration and, more specifically, for reliability optimization.

Among the commonly considered failure mechanisms (see \sref{aging-variation}),
thermal cycling has arguably the most prominent dependence on temperature: as
noted earlier, not only the average and maximum temperature but also the
amplitude and frequency of temperature oscillations have a huge impact on the
lifetime of the system. With this in mind, we focus our attention on mitigating
the damage caused by thermal cycling. To this end, relying on our solution to
dynamic steady-state analysis presented in \sref{dynamic-steady-solution}, we
develop a thermal-cycling-aware technique for scheduling periodic applications.
Note that, by exploring the design space in order to find configurations that
reduce the wear of the system, we implicitly address aging uncertainty, which is
introduced in \sref{aging-variation}. However, this approach to aging
uncertainty is suboptimal, which is discussed further and addressed in
\sref{chaos-reliability-analysis}.

\subsection{\motivationtitle}

\inputfigure{dream-motivation-graph}
\inputfigure{dream-motivation-cycling}
Consider a periodic application with six tasks denoted by T1--T6 and a
heterogeneous platform with two processing elements denoted by PE1 and PE2. The
application's task graph is given in \fref{dream-motivation-graph} along with
the execution times of the tasks for both PE1 and PE2. The period of the
application is 60~ms.

The first alternative schedule and the resulting dynamic steady-state
temperature profile are shown on the left-hand side of
\fref{dream-motivation-cycling} where the height of a task represents its power
consumption. Note that the mapping of the tasks onto the processing elements is
treated as a part of the schedule. It can be seen that PE1 is experiencing three
thermal cycles as illustrated by the blue curve, that is, there are three
intervals where temperature starts from a certain value, reaches an extremum,
and then comes back. If we move T6 to PE2, the number of cycles decreases to
two, which can be seen in the middle of \fref{dream-motivation-cycling}. If we
also swap T2 and T4, the number of cycles that PE1 undergoes drops to one, which
is depicted on the right-hand side of \fref{dream-motivation-cycling}. According
to the reliability model described in \sref{thermal-cycling-fatigue}, these two
changes improve the lifetime of the electronic system by around 45\% and 55\%,
respectively, relative to the initial schedule, which can be seen on the
left-hand side of \fref{dream-motivation-cycling}.

The above motivational example shows that, when exploring the design space, it
is important to take into consideration the number of temperature fluctuations
as well as their characteristics. In order to acquire this information, dynamic
steady-state temperature analysis has to be undertaken.

\subsection{\problemtitle}
\slab{dream-optimization-problem}

In addition to the description given in \sref{system-model}, the considered
platform with \np processing elements is to execute a periodic application with
\nt tasks. The application is given as a directed acyclic graph whose vertices
and edges correspond to the tasks and to data dependencies between these tasks,
respectively; see \fref{dream-motivation-graph}. Each pair of a task and a
processing element is characterized by an execution time and a power
consumption, which are the characteristics that the task exhibits when it is
assigned to the processing element.

The optimization objective is to find a schedule that maximizes the lifetime of
the system under certain constraints. Formally, the objective is
\begin{equation} \elab{thermal-cycling-objective}
  \max_{\schedule} \min_{i = 1}^\np \mean_i(\schedule)
\end{equation}
such that
\begin{align} \elab{thermal-cycling-constraints}
  \begin{split}
    & \period(\schedule) \leq \period_\maximum \text{ and} \\
    & Q(\schedule) \leq \q_\maximum.
  \end{split}
\end{align}
In the above formulae, \schedule denotes a schedule, which specifies the
starting times of the tasks as well as their mapping onto the processing
elements; $\mean_i$ is the \ac{MTTF} of processing element $i$, which is
computed according to the reliability model presented in
\sref{thermal-cycling-fatigue} and can be seen in
\eref{thermal-cycling-mean-time}; \period stands for the end-to-end delay of the
application; and
\[
  Q(\schedule) = \norm[\infty]{\mq(\schedule)}
\]
where \mq is the dynamic steady-state temperature profile of the platform, and
$\norm[\infty]{\cdot}$ denotes the uniform norm. The first constraint in
\eref{thermal-cycling-constraints} enforces a deadline $\period_\maximum$ on the
schedule while the second constraint enforces a maximum temperature
$\q_\maximum$ on the resulting temperature profile \mq.

\subsection{\solutiontitle}

The application is scheduled by means of a static cyclic scheduler following the
list scheduling policy \cite{adam1974}. The input to the scheduler is a vector
mapping the tasks onto the processing elements and a vector assigning priorities
to the tasks. The scheduler produces a vector prescribing starting times to the
tasks, and this vector together with the given mapping constitute a schedule
\schedule.

The optimization procedure is undertaken via a genetic algorithm
\cite{schmitz2004}. In this paradigm inspired by biology, a population of
chromosomes, which represent candidate solutions, is evolved through a number of
generations in order to produce a chromosome with the best possible fitness,
that is, a solution that maximizes the objective function. The fitness of a
chromosome is calculated in accordance with \eref{thermal-cycling-objective};
more specifically, it is based on
\begin{equation} \elab{thermal-cycling-fitness}
  \min_{i = 1}^\np \mean_i(\schedule).
\end{equation}

Each chromosome contains $2 \nt$ genes---twice the number of tasks---and it can
be viewed as two concatenated vectors with \nt elements each. The first vector
encodes a mapping of the tasks onto the processing elements, and the other one a
set of priorities of the tasks. The usage of this information is to be discussed
shortly. The population contains $4 \nt$ chromosomes. In each generation, a
number of chromosomes are chosen for breeding by a tournament selection with the
number of competitors proportional to the population size. The chosen
chromosomes undergo a two-point crossover with probability 0.8 and a uniform
mutation with probability 0.01. The evolution mechanism follows the elitism
model where the best chromosome always survives. The stopping condition is the
absence of improvement within 200 successive generations.

The fitness of a chromosome is evaluated in a number of steps. First, the
chromosome is decoded, and the mapping of the tasks onto the processing elements
as well as the priorities of the tasks are fed to the scheduler. The scheduler
produces a schedule \schedule. If the schedule violates the deadline constraint
given in \eref{thermal-cycling-constraints}, the chromosome is penalized
proportionally to the amount of violation and is not further processed.
Otherwise, based on the parameters of the processing elements and tasks, a power
profile \mp is constructed, and the corresponding temperature profile \mq is
computed using our technique presented in \sref{dynamic-steady-solution}. If the
temperature profile violates the temperature constraint given in
\eref{thermal-cycling-constraints}, the chromosome is penalized proportionally
to the amount of violation and is not further processed. Otherwise, the
\ac{MTTF} of each processing element is estimated according to
\eref{thermal-cycling-mean-time}, and the fitness of the chromosome is set as
shown in \eref{thermal-cycling-fitness}.

\subsection{\resultstitle}

In this subsection, we evaluate our reliability optimization. To this end, we
first consider a set of synthetic applications and then study a real-life one.
The general experimental setup is the same as the one described in
\sref{dynamic-steady-results}. All the configuration files used in the
experiments are available online at \cite{eslab2012}.

Heterogeneous platforms and periodic applications are randomly generated by
virtue of \up{TGFF} \cite{dick1998} in such a way that the execution times of
tasks are uniformly distributed between 1 and 10~ms, and that the static power
accounts for around 40\% of the total power consumption \cite{liu2007}. The area
of a processing element is set to 4~\power{mm}{2}. The modeling of the static
power is based on the linear approximation discussed in
\sref{power-temperature}. The maximum temperature $\q_\maximum$ in
\eref{thermal-cycling-constraints} is set to \celsius{100}. In
\eref{thermal-cycling-mean-cycles}, the Coffin--Manson exponent $b_i$ is set to
6, the activation energy $c_i$ to 0.5, and the elastic region $\Delta \q_{0,
ij}$ to 0 \cite{jedec2016}; the value of the coefficient $a_i$ in
\eref{thermal-cycling-mean-cycles} is irrelevant since we are concerned with
relative improvements, which is to be explicated shortly.

The initial population of chromosomes is created partially randomly and
partially based on a temperature-aware heuristic proposed in \cite{xie2006},
which we refer to as the initial solution. The heuristic relies on spatial
temperature variations and tries to minimize the maximum temperature while
satisfying real-time constraints. This initial solution is also used to decide
on the deadline constraint $\period_\maximum$ in
\eref{thermal-cycling-constraints}: it is set to the duration of the initial
schedule extended by 5\%. Furthermore, the initial solution serves as a baseline
for evaluating the performance of the solutions delivered by our optimization.

\inputtable{dream-optimization-elements}
In the first set of experiments, we change the number of processing elements \np
while keeping the number of tasks \nt per processing element constant and equal
to 20. For each problem, we generate 20 random task graphs and compute the
average change in the \ac{MTTF} relative to the initial solution. In addition,
we calculate the average change in energy consumption. The obtained results are
reported in \tref{dream-optimization-elements}, which also shows the average
time taken by the optimization procedure. It can be seen that the
reliability-aware optimization increases the \ac{MTTF} by a factor of 13--40.
Even for large problems---such as those with 16 processing elements executing
320 tasks---a feasible schedule that significantly improves the lifetime of the
system can be found in an affordable time. Moreover, as shown in
\tref{dream-optimization-elements}, the performed optimization does not impact
much the energy efficiency of the system.

\inputtable{dream-optimization-tasks}
In the second set of experiments, we keep the quad-core platform and vary the
number of tasks \nt of the application. As before, for each problem, we generate
20 random task graphs and monitor the changes in the \ac{MTTF} and energy
consumption relative to the initial solution. The results can be seen in
\tref{dream-optimization-tasks}. The observations are similar to those made with
respect to \tref{dream-optimization-elements}.

The experiments show that our optimization is able to effectively increase the
\ac{MTTF} of the system at hand. The efficiency is due to the fast and accurate
approach to dynamic steady-state temperature analysis presented in
\sref{dynamic-steady-solution}, which is at the heart of the optimization
procedure. Due to its speed, the technique allows a large portion of the design
space to be explored.

\inputtable{dream-optimization-solutions}
In order to illustrate the above aspect, let us replace, inside the optimization
framework, our solution with \one~the one based on iterative transient analysis
with HotSpot and \two~the one based on static steady-state analysis; see
\sref{dynamic-steady-prior}. The goal is to compare the results in
\tref{dream-optimization-tasks} with the results produced by the two alternative
methods when they are given the same amount of time as the one taken by our
method. For each problem, the optimized \ac{MTTF} produced by either of the two
approximate methods is re-evaluated by our exact method. The results are
summarized in \tref{dream-optimization-solutions} where
\ac{MTTF}\textsubscript{a} and \ac{MTTF}\textsubscript{b} stand for the two
alternative methods, respectively. It can be seen that, for example, the
lifetime of the platform running 160 tasks can be extended by a factor of 18
using our technique whereas the best solutions found by the other two techniques
within the same time frame are only around 2--5 times better than the initial
solution. The reason for the poor performance of iterative transient analysis
with HotSpot is the excessively long execution time of the calculation of
dynamic steady-state temperature profiles, which means that this method allows
for a very shallow exploration of the design space. In the case of static
steady-state analysis, the reason is different: the method is fast but also very
inaccurate as discussed and illustrated in \sref{dynamic-steady-prior}. The
inaccuracy drives the optimization toward solutions that turn out to be of low
quality.

The experiments show that the impact of the optimization on energy consumption
is insignificant. This is not surprising: the optimization searches toward
low-temperature solutions, which are also implicitly the low-leakage ones. In
order to explore this further, let us perform a multiobjective optimization
along the dimensions of energy and reliability, and let us use \up{NSGA-II}
\cite{deb2002}. The resulting Pareto front averaged over 20 applications with 80
tasks deployed on a quad-core platform is displayed in
\fref{dream-optimization-multiobjective}. It can be observed that the variation
of the energy change is less than 2\%. This means that the solutions optimized
for the \ac{MTTF} have an energy consumption that is almost identical to the one
of the solutions optimized for energy. At the same time, the difference along
the \ac{MTTF} dimension is huge. This means that, by ignoring the reliability
aspect, the designer may end up with a significantly decreased \ac{MTTF} without
any significant gain in terms of energy consumption.

\inputfigure{dream-optimization-multiobjective}
Finally, we consider a real-life application, namely an \up{MPEG-2} decoder
\cite{ffmpeg}, which is assumed to be deployed on a dual-core platform. The
decoder is analyzed and split into 34 tasks. The parameters of each task are
obtained through a system-level simulation using \up{MPARM} \cite{benini2005}.
The deadline is set to 40~ms assuming 25 frames per second. The solution found
by the proposed method improves the lifetime of the system by a factor of 23.59
with a 5\% energy saving compared to the initial solution. The solutions found
by undertaking the same optimization via the two alternative methods mentioned
earlier are only 5.37 and 11.5 times better, respectively, than the initial one.

To conclude, the experimental results have demonstrated the superiority of the
proposed approach to dynamic steady-state temperature analysis in the context of
reliability optimization compared to the state of the art.

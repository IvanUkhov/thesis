Probabilistic analysis of computer systems is an extensive and diverse area,
which is expanding with an accelerating pace. The rapid growth is instigated by
the fact that computer systems naturally become more sophisticated and refined,
and that they penetrate deeper into everyday life. Hence, the impact of
uncertainty inevitably becomes more prominent and entails more severe
consequences, necessitating an adequate treatment. Therefore, the designer of a
computer system is obliged to account for the presence of uncertainty.

In order to account for uncertainty, one has to quantify it first. As discussed
in \cref{design-uncertainty-analog}, in this setting, the designer is typically
interested in evaluating a certain quantity whose complete knowledge would be
highly profitable for the design at hand but cannot be attained since the
quantity depends a number of parameters that are inherently uncertain at design
time. For instance, such a quantity could be the maximum temperature of a system
whose thermal behavior depends on the runtime workload, which is unknown at
design time.

Uncertainty quantification \cite{maitre2010} is a broad area. The techniques for
uncertainty quantification can deliver radically different pieces of information
about the quantity under consideration. It can be seen in the previous chapters
that we are primarily interested in probability distributions rather than, for
instance, corner cases, and we follow this trend in this chapter as well.
Designing for the worst case leads often to a poor solution as the system under
consideration might easily end up being too conservative, overdesigned
\cite{quinton2012}.

As we note in \sref{chaos-prior}, when probability distributions and uncertainty
quantification in general are concerned, sampling methods are of great use. The
classical \acf{MC} sampling, quasi-\ac{MC} sampling, and Latin hypercube
sampling are examples of such methods \cite{asmussen2007}. Compared to other
techniques for probabilistic analysis, they are straightforward to apply. The
system at hand is treated as an opaque object, and one only has to evaluate this
object a number of times in order to start to draw conclusions about the
system's behavior. The major problem with sampling, however, is in sampling: one
should be able to obtain sufficient many realizations of the quantity of
interest in order to accurately estimate the needed statistics about this
quantity \cite{diaz-emparanza2002}. When the subject under analysis is
computationally expensive to evaluate, sampling methods are rendered slow and
often unfeasible.

Similar to \cref{design-uncertainty-analog}, here we propose a design-time
system-level framework for the analysis of computer systems that are dependent
on uncertain parameters. At it is the case with sampling methods, our technique
treats the system at hand as a ``black box'' and, therefore, is straightforward
to apply since no handcrafting is required, and existing codes need no change.
Hence, the quantities that the framework is able to tackle are diverse. Examples
include those concerned with timing-, power-, and temperature-related
characteristics of elaborate applications running on heterogeneous platforms.

In contrast to \cref{design-uncertainty-analog} and sampling methods, the
framework presented in this chapter explores and exploits the nature of the
problem---that is, the way the quantity of interest depends on the uncertain
parameters---by exercising the aforementioned ``black box'' at a set of points
chosen adaptively. The adaptivity that we leverage is hybrid \cite{jakeman2012}:
it tries to pick up both global (that is, on the level of individual dimensions
\cite{klimke2006}) and, more importantly, local (that is, on the level on
individual points \cite{ma2009}) variations. This means that the framework is
able to benefit from any particularities that might be present in the stochastic
space, that is, in the space of the uncertain parameters.

The remainder of the chapter is organized as follows. The adaptivity is the
capital feature of our technique, and we motivate and illustrate it in the next
section, \sref{interpolant-example}. The problem formulation is given in
\sref{interpolant-problem}. In \sref{interpolant-prior}, the prior work is
discussed. Our solution is outlined in \sref{interpolant-solution} and then
detailed in \sref{interpolant-transformation} and
\sref{interpolant-construction}. The experimental results are given in
\sref{interpolant-results}. Finally, \sref{interpolant-conclusion} concludes the
chapter.

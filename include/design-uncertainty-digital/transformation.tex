At Stage~2 of the framework, we are to change the parameterization of the
problem from the random vector $\vu: \Omega \to \real^\nu$ to an auxiliary
random vector $\vz: \Omega \to \real^\nz$ such that \one~the support of the
\ac{PDF} of \vz is the unit hypercube $[0, 1]^\nz$, and \two~$\nz \leq \nu$ has
the smallest value that is needed to retain the desired level of accuracy. The
first is standardization, which is done primarily for convenience. The second is
model order reduction, which identifies and eliminates excessive complexity and,
hence, speeds up the subsequent solution process. The overall transformation is
denoted by
\[
  \vu = \transform{\vz}
\]
where $\transform: [0, 1]^\nz \to \real^\nu$. The quantity of interest \g can
now be computed as
\[
  \g(\vu) = (\g \circ \transform)(\vz) = \g(\transform(\vz)).
\]

The attentive reader might already have a suitable candidate for $\transform$;
it is the one in \eref{probability-transformation} used throughout the thesis,
which is to be discussed in \sref{interpolant-application}.

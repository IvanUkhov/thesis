In this section, we evaluate the performance of our framework. All the
experiments are conducted on a \up{GNU}/Linux machine equipped with 16
processors Intel Xeon E5520 2.27~\up{GH}z and 24~\up{GB} of \up{RAM}. All the
source code, configuration files, and input data used in the experiments are
available online at \cite{eslab2017}.

We address $3 \times 2 \times 3 = 18$ uncertainty-quantification problems by
considering three platform sizes \np, two application sizes \nt, and three
quantities of interest \g. Specifically, \np is in $\{ 2, 4, 8 \}$, \nt is in
$\{ 10, 20 \}$, and \g is the end-to-end delay, total energy consumption, or
maximum temperature, which are defined in \eref{interpolant-delay},
\eref{interpolant-energy}, and \eref{interpolant-temperature}, respectively. At
this point, it might be helpful to recall the illustrative application depicted
in \fref{interpolant-application}.

Each problem is configured as follows. A platform with \np processing elements
and an application with \nt tasks are generated randomly by \up{TGFF}
\cite{dick1998}. In particular, the tool generates a table for each processing
element that specifies certain properties of the tasks when they are mapped to
that processing element. Namely, each table assigns two numbers to each task: a
reference execution time---which is chosen uniformly between 10 and 50~ms---and
a power consumption---which is chosen uniformly between 5 and 25~W. The
application is scheduled using a list scheduler \cite{adam1974}. The mapping of
the application is fixed and obtained by scheduling the tasks based on their
reference execution times and assigning them to the earliest available
processing elements.

Similar to the previous chapters, the construction of thermal \up{RC} circuits,
which are needed for temperature analysis, is delegated to HotSpot
\cite{skadron2003}. The floorplan of each platform is a regular grid where each
processing element occupies $2 \times 2$~mm\textsuperscript{2} on the die. The
modeling of the static power is based on a linear fit to a data set of
\up{SPICE} simulations of a series of \up{CMOS} invertors. The sampling interval
of power and temperature profiles is equal to 1~ms.

The uncertain parameters \vu of the system are the execution times of the tasks,
and their marginal distributions and correlation matrix are as described in
\sref{interpolant-formulation}; all other parameters are deterministic.
Regarding the marginal distributions---which are beta distributions as shown in
\eref{beta-distribution}---the left $c$ and right $d$ endpoints of their
supports are set to 80\% and 120\%, respectively, of the reference execution
times generated by \up{TGFF} as described earlier. The parameter $a$ and $b$ are
set to two and five, respectively, for all tasks, which skews the distributions
toward the left endpoint. The model-order-reduction parameter $\eta$ in
\eref{model-order-reduction} is set to 0.9, which results in $\nz = 2$ and $\nz
= 3$ independent variables for applications with $\nt = 10$ and $\nt = 20$
tasks, respectively.

The configuration of the interpolation algorithm---including the collocation
nodes, basis functions, and adaptation strategy with stopping
conditions---follows closely the description given in
\sref{interpolant-construction}. The parameters \error{a}, \error{r}, and
\error{s} are around 10\textsuperscript{3}, 10\textsuperscript{2}, and
10\textsuperscript{4}, respectively, depending on the problem.

The performance of our framework with respect to each problem is assessed as
follows. First, as in \cref{design-uncertainty-analog}, the true probability
distribution of the quantity in question \g is estimated by sampling \g directly
and extensively. Direct sampling means that there is no any intermediate
representation or any intermediate model order reduction involved. Second, we
construct an interpolant for \g and estimate \g's distribution by sampling this
interpolant, which is discussed in \sref{interpolant-construction} and
\sref{interpolant-processing}. In both cases, we draw 10\textsuperscript{5}
samples; let us remind, however, that, unlike the cost of direct sampling, the
cost of sampling the interpolant is practically negligible. Third, we perform
another round of direct sampling. This time we draw as many samples as many
times the calculation of \g is invoked during our interpolation process. In each
of the three cases, sampling is undertaken in accordance with a Sobol sequence,
which is a quasi-random low-discrepancy sequence featuring much better
convergence properties than those of the classical \ac{MC} sampling
\cite{joe2008}. As a result, we obtain three estimates of \g's distribution:
reference (the one considered true), proposed (the one based on interpolation),
and direct (the one equal in terms of the number of \g's evaluations to the
proposed solution). The last two are compared with the first one. For comparing
the proximity between two distributions, we use the well-known
Kolmogorov--Smirnov (\up{KS}) statistic \cite{rao2002}, which is the supremum
over the distance (pointwise) between two empirical distribution functions and,
hence, is a rather unforgiving error indicator.

\hiddensubsection{Discussion}

The results of all 18 uncertainty-quantification problems are given in
\fref{results} as a 6-by-3 grid of plots, one plot per problem. The three
columns correspond to the three metrics at hand: the end-to-end delay (left),
total energy (middle), and maximum temperature (right). The three pairs of rows
correspond to the three platform sizes: 2 (top), 4 (middle), and 8 (bottom)
processing elements. The rows alternate between the two application sizes: 10
(odd) and 20 (even) tasks.

The horizontal axis of each plot shows the number of points, that is,
evaluations of the metric \g, and the vertical one shows the \up{KS} statistic
on a logarithmic scale. Each plot has two lines. The solid line represents our
technique. The circles on this line correspond to the steps of the interpolation
process given in \eref{approximation}. They show how the \up{KS} statistic
computed with respect to the reference solution changes as the interpolation
process takes steps (and increases the number of collocation nodes) until the
stopping condition is satisfied (\sref{adaptivity}). Note that only a subset of
the actual steps is displayed in order to make the figure legible. Synchronously
with the solid line (that is, for the same numbers of $\g$'s evaluations), the
dashed line shows the error of direct sampling, which, as before, is computed
with respect to the reference solution.

Let us first describe one particular problem shown in \fref{results}. Consider,
for instance, the one labeled with $\bigstar$. It can be seen that, at the very
beginning, our solution and the solution of direct sampling are poor. The
\up{KS} statistic tells us that there are substantial mismatches between the
estimates and the reference solution. However, as the interpolant is being
adaptively refined, our solution approaches rapidly the reference one and, by
the end of the interpolation process, leaves the solution of na\"{i}ve sampling
approximately an order of magnitude behind.

Studying \fref{results}, one can make a number of observations. First and
foremost, our interpolation-powered approach (solid lines) to probabilistic
analysis outperforms direct sampling (dashed lines) in all the cases. This means
that, given a fixed budget of the computation time---the probability
distributions delivered by our framework are much closer to the true ones than
those delivered by sampling $\g$ directly, despite the fact that the latter
relies on Sobol sequences, which are a sophisticated sampling strategy. Since
direct sampling methods try to cover the probability space impartially,
\fref{results} is a salient illustration of the difference between being
adaptive and nonadaptive.

It can also be seen in \fref{results} that, as the number of evaluations
increases, the solutions computed by our technique approach the exact ones. The
error of our framework decreases generally steeper than the one of direct
sampling. The decrease, however, tends to plateau toward the end of the
interpolation process (when the stopping condition is satisfied). This behavior
can be explained by the following two reasons. First, the algorithm has been
instructed to satiate certain accuracy requirements ($\epsilon_a$, $\epsilon_r$,
and $\epsilon_s$), and it reasonably does not do more than what has been
requested. Second, since the model-order reduction mechanism is enabled in the
case of interpolation, the metric being interpolated is not \g, strictly
speaking; it is a lower-dimensional representation of \g, which already implies
an information loss. Therefore, there is a limit on the accuracy that can be
achieved, which depends on the amount of reduction.

The message of the above observations is that the designer of an electronic
system can benefit substantially in terms of accuracy per computation time by
switching from direct sampling to the proposed technique. If the designer's
current workhorse is the classical \up{MC} sampling, the switch might lead to
even more dramatic savings than those shown in \fref{results}. Needless to
mention that the gain is especially prominent in situations where the analysis
needs to be performed many times such as when it resides in a design-space
exploration loop.

\begin{remark}
The wall-clock time taken by the experiments is not reported in this paper
because this time is irrelevant: since the evaluation of \g is time consuming
(see \sref{problem}), the number of \g's evaluations is the most apposite
expense indicator. For the curious reader, however, let us give an example by
considering the problem labeled with $\clubsuit$ in \fref{results}. Obtaining a
reference solution with $10^5$ simulations in parallel on 16 processors took us
around two hours. Constructing an interpolant with 383 collocation nodes took
around 30 seconds (this is also the time of direct sampling with 383 simulations
of \g). Evaluating the interpolant $10^5$ times took less than a second. The
relative computation cost of sampling an interpolant readily diminishes as the
complexity of \g increases; contrast it with direct sampling, whose cost grows
proportional to \g's evaluation time.
\end{remark}

\hiddensubsection{Real-life Example}

Last but not least, we investigate the viability of deploying the proposed
framework in a real environment. It means that we need to couple the framework
with a battle-proven simulator, which is used in both academia and industry, and
let it simulate a real application running on a real platform. Before we
proceed, we would like to remind that all the implementation and configuration
details including the infrastructure developed for this example can be found at
\cite{sources}.

The scenario that we consider is the same as the one depicted in \fref{example}
except for the fact that an industrial-standard simulator is put in place of the
``black box'' on the left side, and that the metric of interest \g is now the
total energy. Unlike the previous examples, there is no true solution to compare
with due to the prohibitive expense of the simulator, which is exactly why our
framework is needed in such cases.

The simulator of choice is the well-known and widely used combination of Sniper
\cite{carlson2011} and McPAT \cite{li2009}. The architecture that we simulate is
Intel's Nehalem-based Gainestown series. Sniper is distributed with a
configuration file for this architecture, and we use it without any changes. The
platform is configured to have three \up{CPU}s sharing one \up{L3} cache.

The application that has been chosen for simulation is \up{VIPS}, which is an
image-processing piece of software taken from the \up{PARSEC} benchmark suite
\cite{bienia2011}. In this scenario, \up{VIPS} applies a fixed set of operations
to a given image. The width and height of the image to process are considered as
the uncertain parameters \vu (see \sref{problem}), which are assumed to be
distributed uniformly within certain ranges.

The real-life deployment has fulfilled our expectations. The interpolation
process successfully finished and delivered a surrogate after 78 invocations of
the simulator. Each invocation took 40 minutes on average. The probability
distribution of the total energy was then estimated by sampling the constructed
surrogate $10^5$ times. These many samples would take around 6 months to obtain
on our machine if we sampled the simulator directly in parallel on 16
processors; using the proposed technique, the whole procedure took approximately
9 hours.

In this chapter, we have developed a framework for system-level analysis of
computer systems whose runtime behaviors depend on uncertain parameters. The
proposed approach thrives on hierarchical interpolation guided by an advanced
adaptation strategy, which makes the proposed framework general and suitable for
studying various quantities that are of interest to the designer. Examples of
such quantities include the end-to-end delay, total energy consumption, and
maximum temperature of the system under consideration. Moreover, the hybrid
adaptivity that the framework features makes it particularly suited for problems
with idiosyncratic behaviors and steep response surfaces, which arise frequently
in computer systems due to their digital nature.

Given a means of evaluating the quantity of interest for a given outcome of the
uncertain parameters and a description of the probability distribution of the
parameters, the framework prescribes the steps that need to be taken in order to
computationally efficiently characterize the quantity from the probabilistic
perspective. More concretely, the framework delivers a light generative
representation that allows for a straightforward calculation of the probability
distribution of the quantity of interest and other accompanying statistics.

The performance of our technique has been evaluated by addressing a number of
problems that often arise in computer-system design. The results delivered by
our approach have been compared with ones produced by an advanced sampling
technique with a large number of samples, and the comparison has shown that, for
a fixed budget of evaluations of the quantity of interest, the framework
achieves higher accuracy compared to direct simulations. Our technique has also
been applied to a real-life problem, which has confirmed that the deployment of
the framework in a real-life context is straightforward.

Finally, we would like to note that, even though the framework has been
exemplified by considering random execution times and three specific quantities
of interest, it is general and can be applied in many other settings.

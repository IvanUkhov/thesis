Process variation is an acute concern of electronic-system designs
\cite{chandrakasan2000, srivastava2010}. A crucial implication of process
variation is that it renders the key parameters of a technological
process---such as the effective channel length, gate oxide thickness, and
threshold voltage---as uncertain quantities. Therefore, the same workload
applied to two seemingly identical dies can lead to two different power profiles
and, thus, to two different temperature profiles since the power consumption and
heat dissipation depend on the aforementioned quantities. This concern is
especially urgent due to the interdependence between the static power and
temperature \cite{liu2007, srivastava2010}, which is discussed in
\sref{power-temperature-interplay}. As it is the case with any other type of
uncertainty in computer systems, the uncertainty due to process variation leads
to performance degradation and faults of various magnitudes, and, therefore,
process variation should be adequately analyzed as the foremost step toward
efficient and robust products.

An important problem in this regard is the characterization of the on-wafer
distribution of a quantity of interest that is deteriorated by process
variation, based on measurements. The problem belongs to the class of inverse
problems since the measured data can be seen as an output of the system at hand,
and the desired quantity as an input. Such an inverse problem is addressed here.

Our goal is to characterize arbitrary process parameters with high accuracy and
at low costs. The goal is accomplished by measuring auxiliary quantities that
are more convenient and less expensive to work with and employing statistics in
order to infer the desired parameters from the measurements. More specifically,
we propose a novel approach to the quantification of process variation based on
indirect, incomplete, and noisy measurements. Moreover, we develop and implement
a solid framework around the proposed idea and perform a thorough study of
various aspects of our technique.

The remainder of the chapter is structured as follows. A motivational example is
given in \sref{inference-example}. In \sref{inference-problem}, we formulate the
problem that we address and the requirements to a potential solution. The
studies relevant in this context are outlined in \sref{inference-prior}. The
prior solution that we propose is presented in \sref{inference-solution}, and
the corresponding experimental results are reported and discussed in
\sref{inference-results}. \sref{inference-conclusion} concludes this chapter.



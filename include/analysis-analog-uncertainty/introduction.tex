Process variation is an acute concern of electronic-system designs
\cite{chandrakasan2000, srivastava2010}. A crucial implication of process
variation is that it renders the key parameters of a technological
process---such as the effective channel length and gate oxide thickness---as
uncertain quantities. Therefore, the same workload applied to two seemingly
identical dies can lead to two different power profiles and, consequently, to
two different temperature profiles since the power consumption and heat
dissipation depend on the aforementioned quantities, which is primarily due to
the static component of power discussed in \sref{interdependence}. As it is the
case with any other type of uncertainty in computer systems, the uncertainty due
to process variation leads to performance degradation and faults of various
magnitudes, and, therefore, process variation should be adequately analyzed as
the foremost step toward efficient and robust products.

An important problem in this regard is the characterization of the on-wafer
distribution of a quantity of interest that is deteriorated by process
variation, based on measurements. The problem belongs to the class of inverse
problems since the measured data can be seen as an output of the system at hand,
and the desired quantity as an input. Such an inverse problem is addressed here.

Our goal is to characterize arbitrary process parameters with high accuracy and
at low costs. The goal is accomplished by measuring auxiliary quantities that
are more convenient and less expensive to work with and employing statistics in
order to infer the desired parameters from the measurements. More specifically,
we propose a novel approach to the quantification of process variation based on
indirect, incomplete, and noisy measurements. Moreover, we develop and implement
a solid framework around the proposed idea and perform a thorough study of
various aspects of our technique.

There are a number of related studies that we would like to highlight. Bayesian
inference is utilized in \cite{zhang2010} for identifying the optimal set of
locations on the wafer where the parameter under consideration should be
measured in order to characterize it with the maximal accuracy. The
expectation-maximization algorithm is considered in \cite{reda2009} in order to
estimate missing test measurements. In \cite{paek2012}, the authors consider an
inverse problem focused on the inference of the power dissipation based on
transient temperature maps using Markov random fields. Another temperature-based
characterization of power is developed in \cite{mesa-martinez2007} where a
genetic algorithm is employed for the reconstruction of the power model. It
should be noted that the procedures in \cite{zhang2010, reda2009} operate on
direct measurements, meaning that the output is the same quantity as the one
being measured. In particular, these procedures rely heavily on the availability
of adequate test structures on the dies and are practical only for secondary
quantities affected by process variation such as delays and currents, but not
for the primary ones such as various geometrical properties. Hence, they often
lead to excessive costs and have a limited range of applications. The approaches
in \cite{paek2012, mesa-martinez2007}, on the other hand, concentrating on the
power dissipation of a single die, are not concerned with process variation.

The remainder of the chapter is structured as follows. A motivational example is
given in \sref{inference-example}. In \sref{inference-problem}, we formulate the
problem and the requirements to a potential solution. The solution that we
propose is presented in \sref{inference-solution}, and the corresponding
experimental results are reported and discussed in \sref{inference-results}.
\sref{inference-conclusion} concludes this chapter.
